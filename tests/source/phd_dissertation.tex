% !TeX program = lualatex

%! TeX root = phd_dissertation.tex

%\pdfvariable suppressoptionalinfo 512\relax
\documentclass[11pt,lof]{puthesis}

% packages
\usepackage{amsmath}
\usepackage{amssymb}
\usepackage[amsmath,thmmarks,noconfig]{ntheorem}
\usepackage{mathtools}
\usepackage{multirow}
\usepackage{pgfplots}
\usepackage{graphicx}
\usepackage{enumitem}
\usepackage{subcaption}
\usepackage{titlesec}
\usepackage{stackengine}
\usepackage{scalerel}
\usepackage{microtype}
\usepackage[boxruled,linesnumbered,commentsnumbered,procnumbered]{algorithm2e}
\usepackage[longnamesfirst]{natbib}
\usepackage[hypertexnames=false,hidelinks]{hyperref}
\usepackage[norefs,nocites]{refcheck}
\usepackage[defaultlines=3,all]{nowidow}
\usepackage{float}

% settings
\pgfplotsset{compat=1.9}
\setcitestyle{round}
\captionsetup[subfigure]{justification=centering}
\def\arraystretch{1.3}
\renewcommand{\descriptionlabel}[1]{\hspace{\labelsep}\textit{#1}}

% tables numbered as figures
\def\table{\def\figurename{Table}\figure}
\let\endtable\endfigure
\renewcommand\listfigurename{List of Figures and Tables}

% arxiv
\newcommand{\arxiv}[1]{\href{https://arxiv.org/abs/#1}{\texttt{arXiv:#1}}}

% github
\newcommand{\github}[1]{\href{https://github.com/#1}{\texttt{github.com/#1}}}

% blackboard
\renewcommand{\P}{\ensuremath{\mathbb{P}}}
\newcommand{\N}{\ensuremath{\mathbb{N}}}
\newcommand{\R}{\ensuremath{\mathbb{R}}}
\newcommand{\E}{\ensuremath{\mathbb{E}}}
\newcommand{\Q}{\ensuremath{\mathbb{Q}}}
\newcommand{\I}{\ensuremath{\mathbb{I}}}
\newcommand{\Z}{\ensuremath{\mathbb{Z}}}

% roman
\newcommand{\rF}{\ensuremath{\mathrm{F}}}
\newcommand{\rH}{\ensuremath{\mathrm{H}}}
\newcommand{\rL}{\ensuremath{\mathrm{L}}}
\newcommand{\rk}{\ensuremath{\mathrm{k}}}
\newcommand{\rd}{\ensuremath{\mathrm{d}}}
\newcommand{\comp}{\ensuremath{\mathrm{c}}}
\newcommand{\TV}{\mathrm{TV}}

% bold
\newcommand{\bW}{\ensuremath{\mathbf{W}}}
\newcommand{\bY}{\ensuremath{\mathbf{Y}}}
\newcommand{\bX}{\ensuremath{\mathbf{X}}}
\newcommand{\bT}{\ensuremath{\mathbf{T}}}
\newcommand{\bA}{\ensuremath{\mathbf{A}}}
\newcommand{\bV}{\ensuremath{\mathbf{V}}}

% calligraphic
\newcommand{\cH}{\ensuremath{\mathcal{H}}}
\newcommand{\cF}{\ensuremath{\mathcal{F}}}
\newcommand{\cN}{\ensuremath{\mathcal{N}}}
\newcommand{\cX}{\ensuremath{\mathcal{X}}}
\newcommand{\cG}{\ensuremath{\mathcal{G}}}
\newcommand{\cW}{\ensuremath{\mathcal{W}}}
\newcommand{\cB}{\ensuremath{\mathcal{B}}}
\newcommand{\cS}{\ensuremath{\mathcal{S}}}
\newcommand{\cT}{\ensuremath{\mathcal{T}}}
\newcommand{\cV}{\ensuremath{\mathcal{V}}}
\newcommand{\cE}{\ensuremath{\mathcal{E}}}
\newcommand{\cU}{\ensuremath{\mathcal{U}}}
\newcommand{\cR}{\ensuremath{\mathcal{R}}}
\newcommand{\cA}{\ensuremath{\mathcal{A}}}
\newcommand{\cC}{\ensuremath{\mathcal{C}}}
\newcommand{\cM}{\ensuremath{\mathcal{M}}}
\newcommand{\cD}{\ensuremath{\mathcal{D}}}
\newcommand{\cP}{\ensuremath{\mathcal{P}}}
\newcommand{\cI}{\ensuremath{\mathcal{I}}}
\newcommand{\cY}{\ensuremath{\mathcal{Y}}}

% sans serif
\newcommand{\T}{\ensuremath{\mathsf{T}}}

% symbols
\newcommand{\vvvert}{{\vert\kern-0.25ex\vert\kern-0.25ex\vert}}
\newcommand{\bigvvvert}{{\big\vert\kern-0.35ex\big\vert\kern-0.35ex\big\vert}}
\newcommand{\Bigvvvert}{{\Big\vert\kern-0.3ex\Big\vert\kern-0.3ex\Big\vert}}
\newcommand{\bigsetminus}{\mathbin{\big\backslash}}
\newcommand{\Bigsetminus}{\mathbin{\Big\backslash}}
\newcommand{\dprime}{\ensuremath{\prime\prime}}
\newcommand{\tprime}{\ensuremath{\prime\prime\prime}}
\newcommand{\objective}{\ensuremath{\mathrm{obj}}}
\newcommand{\Dl}{\ensuremath{D_{\textup{lo}}}}
\newcommand{\Du}{\ensuremath{D_{\textup{up}}}}

% floor of beta
\newcommand{\flbeta}{{\ThisStyle{%
\ensurestackMath{\stackengine{-0.5\LMpt}{\SavedStyle \beta}%
{\SavedStyle {\rule{3.7\LMpt}{0.3\LMpt}}}
{U}{c}{F}{F}{S}}\vphantom{\beta}}}}

% operators
\DeclareMathOperator{\Var}{Var}
\DeclareMathOperator{\Cov}{Cov}
\DeclareMathOperator{\AIMSE}{AIMSE}
\DeclareMathOperator{\LOOCV}{LOOCV}
\DeclareMathOperator{\symconv}{symconv}
\DeclareMathOperator{\GCV}{GCV}
\DeclareMathOperator{\Unif}{Unif}
\DeclareMathOperator*{\logistic}{logistic}
\DeclareMathOperator{\Bias}{Bias}
\DeclareMathOperator{\Env}{Env}
\DeclareMathOperator*{\esssup}{ess\,sup}
\DeclareMathOperator{\Ber}{Ber}
\DeclareMathOperator{\KL}{KL}
\DeclareMathOperator{\Gam}{Gam}
\DeclareMathOperator{\Yule}{Yule}
\DeclareMathOperator{\rank}{rank}
\DeclareMathOperator{\Exp}{Exp}
\DeclareMathOperator{\Bin}{Bin}
\DeclareMathOperator{\Tr}{Tr}
\DeclareMathOperator{\Leb}{Leb}
\DeclareMathOperator*{\argmin}{arg\,min}
\DeclareMathOperator*{\minimize}{minimize:}
\DeclareMathOperator*{\subjectto}{subject\ to:}
\DeclareMathOperator{\ROT}{ROT}
\newcommand{\diff}[1]{\,\mathrm{d}#1}

% theorem environments
\renewtheoremstyle{break}{%
\item[\rlap{\vbox{\hbox{\hskip\labelsep \bfseries\upshape ##1\ %
##2}\hbox{\strut}}}]%
}{%
\item[\rlap{\vbox{\hbox{\hskip\labelsep \bfseries\upshape ##1\ %
##2\ \normalfont (##3)}\hbox{\strut}}}]%
}
\theoremstyle{break}
\theorempreskip{7mm}
\newtheorem{theorem}{Theorem}[section]
\newtheorem{lemma}{Lemma}[section]
\newtheorem{assumption}{Assumption}[section]
\newtheorem{corollary}{Corollary}[section]
\newtheorem{proposition}{Proposition}[section]
\newtheorem{definition}{Definition}[section]
\newtheorem{remark}{Remark}[section]

% proof environments
\let\proof\relax
\newtheoremstyle{proof}{%
\item[\rlap{\vbox{\hbox{\hskip\labelsep \bfseries\upshape ##1\ %
}\hbox{\strut}}}]%
}{%
\item[\rlap{\vbox{\hbox{\hskip\labelsep \bfseries\upshape ##1\ %
\normalfont (##3)}\hbox{\strut}}}]%
}
\theoremstyle{proof}
\theorembodyfont{\upshape}
\theorempreskip{7mm}
\theoremsymbol{\ensuremath{\square}}
\newtheorem{proof}{Proof}
\AtBeginEnvironment{proof}{\setcounter{proofparagraphcounter}{0}}%

% proof paragraphs
\titleformat{\paragraph}[hang]{\bfseries\upshape}{}{0pt}{}[]
\titlespacing*{\paragraph}{0pt}{6pt}{0pt}
\newcounter{proofparagraphcounter}
\newcommand{\proofparagraph}[1]{
\refstepcounter{proofparagraphcounter}%
\paragraph{Part \theproofparagraphcounter : #1}}%

% inline roman lists
\newlist{inlineroman}{enumerate*}{1}
\setlist[inlineroman]{afterlabel=~,label=(\roman*)}

% algorithms
\DontPrintSemicolon%
\makeatletter%
\renewcommand{\SetKwInOut}[2]{%
\sbox\algocf@inoutbox{\KwSty{#2}\algocf@typo:}%
\expandafter\ifx\csname InOutSizeDefined\endcsname\relax%
\newcommand\InOutSizeDefined{}%
\setlength{\inoutsize}{\wd\algocf@inoutbox}%
\sbox\algocf@inoutbox{%
\parbox[t]{\inoutsize}%
{\KwSty{#2}\algocf@typo:\hfill}~%
}%
\setlength{\inoutindent}{\wd\algocf@inoutbox}%
\else%
\ifdim\wd\algocf@inoutbox>\inoutsize%
\setlength{\inoutsize}{\wd\algocf@inoutbox}%
\sbox\algocf@inoutbox{%
\parbox[t]{\inoutsize}%
{\KwSty{#2}\algocf@typo:\hfill}~%
}%
\setlength{\inoutindent}{\wd\algocf@inoutbox}%
\fi%
\fi%
\algocf@newcommand{#1}[1]{%
\ifthenelse{\boolean{algocf@inoutnumbered}}{\relax}{\everypar={\relax}}{%
\let\\\algocf@newinout\hangindent=\inoutindent\hangafter=1\parbox[t]%
{\inoutsize}{\KwSty{#2}%
\algocf@typo:\hfill}~##1\par%
}%
\algocf@linesnumbered%
}%
}%
\makeatother%
\SetKwInOut{Input}{Input}%
\SetKwInOut{Output}{Output}%
\setlength{\algomargin}{2em}%

\author{William George Underwood}
\adviser{Matias Damian Cattaneo}
\title{Estimation and Inference in \\ Modern Nonparametric Statistics}

\abstract{

% 350 words max

Nonparametric methods are central to modern statistics, enabling data analysis
with minimal assumptions in a wide range of scenarios. While contemporary
procedures such as random forests and kernel methods are popular due to their
performance and flexibility, their statistical properties are often less well
understood. The availability of sound inferential techniques is vital in the
sciences, allowing researchers to quantify uncertainty in their models. We
develop methodology for robust and practical statistical estimation and
inference in some modern nonparametric settings involving complex estimators
and nontraditional data.

We begin in the regression setting by studying the Mondrian random forest, a
variant in which the partitions are drawn from a Mondrian process. We present a
comprehensive analysis of the statistical properties of Mondrian random
forests, including a central limit theorem for the estimated regression
function and a characterization of the bias. We show how to conduct feasible
and valid nonparametric inference by constructing confidence intervals, and
further provide a debiasing procedure that enables minimax-optimal estimation
rates for smooth function classes in arbitrary dimension.

Next, we turn our attention to nonparametric kernel density estimation with
dependent dyadic network data. We present results for minimax-optimal
estimation, including a novel lower bound for the dyadic uniform convergence
rate, and develop methodology for uniform inference via confidence bands and
counterfactual analysis. Our methods are based on strong approximations and are
designed to be adaptive to potential dyadic degeneracy. We give empirical
results with simulated and real-world economic trade data.

Finally, we develop some new probabilistic results with applications to
nonparametric statistics. Coupling has become a popular approach for
distributional analysis in recent years, and Yurinskii's method stands out for
its wide applicability and explicit formulation. We present a generalization of
Yurinskii's coupling, treating approximate martingale data under weaker
conditions than previously imposed. We allow for Gaussian mixture coupling
distributions, and a third-order method permits faster rates in certain
situations. We showcase our results with applications to factor models and
martingale empirical processes, as well as
nonparametric partitioning-based and
local polynomial regression procedures.
}
\acknowledgments{

I am extremely fortunate to have been surrounded by many truly wonderful people over the course of my career, and without their support this dissertation would not have been possible. While it is impossible for me to identify every one of them individually, I would like to mention a few names in particular to recognize those who have been especially important to me during the last few years.

Firstly, I would like to express my utmost gratitude to my Ph.D.\ adviser, Matias Cattaneo. Working with Matias has been genuinely inspirational for me, and I could not have asked for a more rewarding start to my journey as a researcher. From the very beginning, he has guided me expertly through my studies, providing hands-on assistance when required while also allowing me the independence necessary to develop as an academic. I hope that, during the four years we have worked together, I have acquired just a fraction of his formidable mathematical intuition, keen attention to detail, boundless creativity, and inimitable pedagogical skill. Alongside his role as my adviser, Matias has been above all a friend, who has been in equal measure inspiring, insightful, dedicated, understanding, and kind.

Secondly, I would like to thank all of the faculty members at Princeton and beyond who have acted as my collaborators and mentors, without whom none of my work could have been realized. In particular, I express my gratitude to my tireless Ph.D.\ committee members and letter writers Jianqing Fan and Jason Klusowski, my coauthors Yingjie Feng and Ricardo Masini, my dissertation reader Boris Hanin, my teachers Amir Ali Ahmadi, Ramon van Handel, Mikl{\'o}s R{\'a}cz, and Mykhaylo Shkolnikov, my colleagues Sanjeev Kulkarni and Roc{\'i}o Titiunik, and my former supervisor Mihai Cucuringu. I am also thankful for the staff members at Princeton who have been perpetually helpful, and I would like to identify Kim Lupinacci in particular; her assistance in all things administrative has been invaluable.

I am grateful to my fellow graduate students in the ORFE department for their technical expertise and generosity with their time, and for making Sherrerd Hall such a vibrant and exciting space, especially Jose Avilez, Pier Beneventano, Ben Budway, Rajita Chandak, Abraar Chaudhry, Stefan Clarke, Giulia Crippa, G{\"o}k{\c{c}}e Dayan{\i}kl{\i}, Nicolas Garcia, Felix Hoefer, Erica Lai, Jackie Lok, Maya Mutic, Dan Rigobon, Till Saenger, Rajiv Sambharya, Boris Shigida, Igor Silin, Giang Truong, and Rae Yu. Our regular social events made a contribution to my well-being which is difficult to overstate. My thanks extend also to the students I taught, as well as to my group of senior thesis undergraduates, for their commitment, patience, and responsiveness.

More broadly, I would like to thank all of my friends, near and far, for their unfailing support and reliability, and for helping to create so many of my treasured memories. In particular, Ole Agersnap, James Ashford, Christian Baehr, Chris Bambic, Kevin Beeson, James Broadhead, Alex Cox, Reece Edmends, Robin Franklin, Greg Henderson, Bonnie Ko, Grace Matthews, Dan Mead, Ben Musachio, Jacob Neis, Monika Papayova, Will Pedrick, Oliver Philcox, Nandita Rao, Alex Rice, Edward Rowe, David Snyder, Titi Sodimu, Nikitas Tampakis, and Anita Zhang. Thank you to the Princeton Chapel Choir for being such a wonderful community of musicians and a source of close friends, and to our directors, Nicole Aldrich and Penna Rose, and organist Eric Plutz.

Lastly, yet most importantly, I want to thank my family for their unwavering support throughout my studies. My visits back home have been a source of joy throughout my long and often challenging Ph.D., and I cherish every moment I have spent with my parents, sister, grandparents, and extended family.
}

\begin{document}


\chapter{Introduction}

% nonparametric estimation is common
Nonparametric estimation procedures are at the heart of many contemporary
theoretical and methodological topics within the fields of statistics, data
science, and machine learning. Where classical parametric techniques impose
specific distributional and structural assumptions when modeling statistical
problems, nonparametric methods instead take a more flexible approach,
typically positing only high-level restrictions such as moment conditions,
independence criteria, and smoothness assumptions. Examples of such procedures
abound in modern data science and machine learning, encompassing histograms,
kernel estimators, smoothing splines, decision trees, nearest neighbor methods,
random forests, neural networks, and many more.

% nonparametric estimation is good
The benefits of the nonparametric framework are clear: statistical procedures
can be formulated in cases where the stringent assumptions of parametric models
are untestable, demonstrably violated, or simply unreasonable. As a
consequence, the resulting methods often inherit desirable robustness
properties against various forms of misspecification or misuse. The class of
problems that can be formulated is correspondingly larger: arbitrary
distributions and relationships can be characterized and estimated in a
principled manner.

% nonparametric estimation is hard
Nonetheless, these attractive properties do come at a price. In particular, as
its name suggests, the nonparametric approach forgoes the ability to reduce a
complex statistical problem to that of estimating a fixed, finite number of
parameters. Rather, nonparametric procedures typically involve making
inferences about a growing number of parameters simultaneously, as witnessed in
high-dimensional regimes, or even directly handling infinite-dimensional
objects such as entire regression or density functions. As a consequence,
nonparametric estimators are usually less efficient than their correctly
specified parametric counterparts, when they are available; rates of
convergence tend to be slower, and confidence sets more conservative. Another
challenge is that theoretical mathematical analyses of nonparametric estimators
are often significantly more demanding than those required for low-dimensional
parametric settings, necessitating tools from contemporary developments in
high-dimensional concentration phenomena, coupling and strong approximation
theory, empirical processes, mathematical optimization, and stochastic
calculus.

% nonparametric inference
In addition to providing accurate point estimates of unknown (possibly
high-dimensional or infinite-dimensional) quantities of interest, modern
nonparametric procedures are also expected to come equipped with methodologies
for conducting statistical inference. The availability of such inferential
techniques is paramount, with contemporary nonparametric methods forming a
ubiquitous component of modern data science tool kits. Valid uncertainty
quantification is essential for hypothesis testing, error bar construction,
assessing statistical significance, and performing power analyses. Inference is
a central concept in classical statistics, and despite the rapid recent
development of theory for modern nonparametric estimators, their applicability
to statistical inference is in certain cases rather less well studied;
theoretically sound and practically implementable inference procedures are
sometimes absent in the literature.

% complex data
In any statistical modeling problem, the selection and application of an
estimator must naturally be tailored to the available data. Today, much of the
data produced and analyzed does not necessarily fit neatly into the classical
framework of independent and identically distributed samples, and instead might
consist of time series, stochastic processes, networks, or high-dimensional or
functional data, to name just a few. Therefore, it is important to understand
how nonparametric methods might be adapted to correctly handle these data
types, maintaining fast estimation rates and valid techniques for statistical
inference. The technical challenges associated with such an endeavor are
non-trivial; many standard techniques are ineffective in the presence of
dependent or infinite-dimensional data, for example. As such, the development
of new mathematical results in probability theory plays an important role in
the comprehensive treatment of nonparametric statistics with complex data.

\section*{Overview of the dissertation}

% what we do
This dissertation presents a selection of topics relating to nonparametric
estimation and inference, and the associated technical mathematical tools.

% mondrian
Chapter~\ref{ch:mondrian}, titled ``Inference with Mondrian Random Forests,''
is based on the work of \citet{cattaneo2023inference}.
% what are random forests
Random forests are popular ensembling-based methods for classification and
regression, which are well known for their good performance, flexibility,
robustness, and efficiency. The majority of random forest models share the
following common framework for producing estimates of a classification or
regression function using covariates and a response variable. Firstly, the
covariate space is partitioned in some algorithmic manner, possibly using a
source of external randomness. Secondly, a local estimator of the
classification or regression function is fitted to the responses in each cell
separately, yielding a tree estimator. Finally, this process is repeated with
many different partitions, and the resulting tree estimators are averaged to
produce a random forest.

% why are there variants
Many different variants of random forests have been proposed in recent years,
typically with the aim of improving their statistical or computational
properties, or simplifying their construction in order to permit a more
detailed theoretical analysis.
% mondrian random forests
One interesting such example is that of the Mondrian random forest, in which
the underlying partitions (or trees) are constructed independently of the data.
Naturally, this restriction rules out many classical random forest models,
which exhibit a complex and data-dependent partitioning scheme. Instead, trees
are sampled from a canonical stochastic process known as the Mondrian process,
which endows the resulting tree and forest estimators with various agreeable
features.

% what we do
We study the estimation and inference properties of Mondrian random forests in
the nonparametric regression setting. In particular, we establish a novel
central limit theorem for the estimates made by a Mondrian random forest which,
when combined with a characterization of the bias and a consistent variance
estimator, allows one to perform asymptotically valid statistical inference,
such as constructing confidence intervals, on the unknown regression function.
We also provide a debiasing procedure for Mondrian random forests, which allows
them to achieve minimax-optimal estimation rates with H{\"o}lder smooth
regression functions, for any smoothness parameter and in arbitrary dimension.

% kernel
Chapter~\ref{ch:kernel}, titled ``Dyadic Kernel Density Estimators,'' is based
on the work of \citet{cattaneo2024uniform}. Network data plays an important role
in statistics, econometrics, and many other data science disciplines, providing
a natural framework for modeling relationships between units, be they people,
financial institutions, proteins, or economic entities. Of prominent interest
is the task of performing statistical estimation and inference with data
sampled from the edges of such networks, known as dyadic data. The archetypal
lack of independence between edges in a network renders many classical
statistical tools unsuited for direct application. As such, researchers must
appeal to techniques tailored to dyadic data in order to accurately capture the
complex structure present in the network.

% broad scope
We focus on nonparametric estimation and inference with dyadic data, and in
particular we seek methods that are robust in the sense that our results should
hold uniformly across the support of the data. Such uniformity guarantees allow
for statistical inference in a broader range of settings, including
specification testing and distributional counterfactual analysis. We
specifically consider the problem of uniformly estimating a dyadic density
function, focusing on kernel estimators taking the form of dyadic empirical
processes.

% main contributions
Our main contributions include the minimax-optimal uniform convergence rate of
the dyadic kernel density estimator, along with strong approximation results
for the associated standardized and Studentized $t$-processes. A consistent
variance estimator enables the construction of feasible uniform confidence
bands for the unknown density function. We showcase the broad applicability of
our results by developing novel counterfactual density estimation and inference
methodology for dyadic data, which can be used for causal inference and program
evaluation.
% why it is difficult
A crucial feature of dyadic distributions is that they may be ``degenerate'' at
certain points in the support of the data, a property that makes our analysis
somewhat delicate. Nonetheless, our methods for uniform inference remain robust
to the potential presence of such points.
% applications
For implementation purposes, we discuss inference procedures based on positive
semi-definite covariance estimators, mean squared error optimal bandwidth
selectors, and robust bias correction. We illustrate the empirical performance
of our methods in simulations and with real-world trade data, for which we make
comparisons between observed and counterfactual trade distributions in
different years. Our technical results on strong approximations and maximal
inequalities are of potential independent interest.

% yurinskii
Finally, Chapter~\ref{ch:yurinskii}, titled ``Yurinskii's Coupling for
Martingales,'' is based on the work of \citet{cattaneo2022yurinskii}.
Yurinskii's coupling is a popular theoretical tool for non-asymptotic
distributional analysis in mathematical statistics and applied probability.
Coupling theory, also known as strong approximation, provides an alternative
framework to the more classical weak convergence approach to statistical
analysis. Rather than merely approximating the distribution of a random
variable, strong approximation techniques construct a sequence of random
variables which are close almost surely or in probability, often with
finite-sample guarantees.

% what is it used for
Coupling allows distributional analysis in settings where weak convergence
fails, including many applications to nonparametric or high-dimensional
statistics; it is a key technical component in the main strong approximation
results of our Chapter~\ref{ch:kernel}. The Yurinskii method specifically
offers a Gaussian coupling with an explicit error bound under easily verified
conditions; originally stated in $\ell^2$-norm for sums of independent random
vectors, it has recently been extended both to the $\ell^p$-norm, for $1 \leq p
\leq \infty$, and to vector-valued martingales in $\ell^2$-norm, under some
strong conditions.

% what we do
We present as our main result a Yurinskii coupling for approximate martingales
in $\ell^p$-norm, under substantially weaker conditions than previously
imposed. Our formulation allows the coupling variable to follow a
general Gaussian mixture distribution, and we provide a novel third-order
coupling method which gives tighter approximations in certain situations. We
specialize our main result to mixingales, martingales, and independent data,
and derive uniform Gaussian mixture strong approximations for martingale
empirical processes. Applications to nonparametric partitioning-based and local
polynomial regression procedures are provided.

% appendices
Supplementary materials for Chapters~\ref{ch:mondrian}, \ref{ch:kernel}, and
\ref{ch:yurinskii} are provided in Appendices~\ref{app:mondrian},
\ref{app:kernel}, and \ref{app:yurinskii} respectively. These contain detailed
proofs of the main results, additional technical contributions, and further
discussion.

\chapter[Inference with Mondrian Random Forests]%
{Inference with \\ Mondrian Random Forests}
\label{ch:mondrian}

% abstract
Random forests are popular methods for classification and regression, and many
different variants have been proposed in recent years. One interesting example
is the Mondrian random forest, in which the underlying trees are constructed
according to a Mondrian process. In this chapter we give a central limit theorem
for the estimates made by a Mondrian random forest in the regression setting.
When combined with a bias characterization and a consistent variance estimator,
this allows one to perform asymptotically valid statistical inference, such as
constructing confidence intervals, on the unknown regression function. We also
provide a debiasing procedure for Mondrian random forests which allows them to
achieve minimax-optimal estimation rates with $\beta$-H{\"o}lder regression
functions, for all $\beta$ and in arbitrary dimension, assuming appropriate
parameter tuning.

\section{Introduction}

Random forests, first introduced by \citet{breiman2001random}, are a workhorse
in modern machine learning for classification and regression tasks.
Their desirable traits include computational efficiency (via parallelization
and greedy heuristics) in big data settings, simplicity of configuration and
amenability to tuning parameter selection, ability to adapt to latent structure
in high-dimensional data sets, and flexibility in handling mixed data types.
Random forests have achieved great empirical successes in many fields of study,
including healthcare, finance, online commerce, text analysis, bioinformatics,
image classification, and ecology.

Since Breiman introduced random forests over twenty years ago, the study of
their statistical properties remains an active area of research: see
\citet{scornet2015consistency}, \citet{chi2022asymptotic},
\citet{klusowski2024large}, and references therein, for a sample of recent
developments. Many fundamental questions about Breiman's random forests remain
unanswered, owing in part to the subtle ingredients present in the estimation
procedure which make standard analytical tools ineffective. These technical
difficulties stem from the way the constituent trees greedily partition the
covariate space, utilizing both the covariate and response data. This creates
complicated dependencies on the data which are often exceedingly hard to
untangle without overly stringent assumptions, thereby hampering theoretical
progress.

To address the aforementioned technical challenges while retaining the
phenomenology of Breiman's random forests, a variety of stylized versions of
random forest procedures have been proposed and studied in the literature.
These include centered random forests
\citep{biau2012analysis,arnould2023interpolation} and median random forests
\citep{duroux2018impact,arnould2023interpolation}. Each tree in a centered
random forest is constructed by first choosing a covariate uniformly at random
and then splitting the cell at the midpoint along the direction of the chosen
covariate. Median random forests operate in a similar way, but involve the
covariate data by splitting at the empirical median along the direction of the
randomly chosen covariate. Known as purely random forests, these procedures
simplify Breiman's original---albeit more data-adaptive---version by growing
trees that partition the covariate space in a way that is statistically
independent of the response data.

Yet another variant of random forests, Mondrian random forests
\citep{lakshminarayanan2014mondrian}, have received significant attention in
the statistics and machine learning communities in recent years
\citep{ma2020isolation, mourtada2020minimax, scillitoe2021uncertainty,
mourtada2021amf, vicuna2021reducing, gao2022towards, oreilly2022stochastic}.
Like
other purely random forest variants, Mondrian random forests offer a simplified
modification of Breiman's original proposal in which the partition is generated
independently of the data and according to a canonical stochastic process known
as the Mondrian process \citep{roy2008mondrian}. The Mondrian process takes a
single parameter $\lambda > 0$ known as the ``lifetime'' and enjoys various
mathematical properties. These probabilistic
features allow Mondrian random forests to be
fitted in an online manner as well as being subject to a rigorous statistical
analysis, while also retaining some of the appealing features of other
more traditional random forest methods.

This chapter studies the statistical properties of Mondrian random forests. We
focus on this purely random forest variant not only because of its importance
in the development of random forest theory in general, but also because the
Mondrian process is, to date, the only known recursive tree mechanism involving
randomization, pure or data-dependent, for which the resulting random forest is
minimax-optimal for point estimation over a class of smooth regression
functions in arbitrary dimension \citep{mourtada2020minimax}. In fact, when the
covariate dimension exceeds one, the aforementioned centered and median random
forests are both minimax-\emph{suboptimal}, due to their large biases, over the
class of Lipschitz smooth regression functions \citep{klusowski2021sharp}. It
is therefore natural to focus our study of inference for random forests on
versions that at the very least exhibit competitive bias and variance, as this
will have important implications for the trade-off between precision and
confidence.

Despite their recent popularity, relatively little is known about the formal
statistical properties of Mondrian random forests. Focusing on nonparametric
regression, \citet{mourtada2020minimax} recently showed that Mondrian forests
containing just a single tree (called a Mondrian tree) can be minimax-optimal
in integrated mean squared error whenever the regression function is
$\beta$-H{\"o}lder continuous for some $\beta \in (0, 1]$. The authors also
showed that, when appropriately tuned, large Mondrian random forests can be
similarly minimax-optimal for $\beta \in (0, 2]$, while the constituent trees
cannot. See also \citet{oreilly2022stochastic} for analogous results for more
general
Mondrian tree and forest constructions. These results formally demonstrate the
value of ensembling with random forests from a point estimation perspective. No
results are currently available in the literature for statistical inference
using Mondrian random forests.

This chapter contributes to the literature on the foundational statistical
properties of Mondrian random forest regression estimation with two main
results. Firstly, we give a central limit theorem for the classical Mondrian
random forest point estimator, and propose valid large-sample inference
procedures employing a consistent standard error estimator. We establish this
result by deploying a martingale central limit theorem
\citep[Theorem~3.2]{hall1980martingale} because we need to handle delicate
probabilistic features of the Mondrian random forest estimator. In particular,
we deal with the existence of Mondrian cells which are ``too small'' and lead
to a reduced effective (local) sample size for some trees in the forest. Such
pathological cells are in fact typical in Mondrian random forests and
complicate the probability limits of certain sample averages; in fact, small
Mondrian random forests (or indeed single Mondrian trees) remain random even
in the limit due to the lack of ensembling. The presence of small cells
renders inapplicable prior distributional approximation results for
partitioning-based estimators in the literature
\citep{huang2003local,cattaneo2020large}, since the commonly required
quasi-uniformity assumption on the underlying partitioning scheme is violated
by cells generated using the Mondrian process. We circumvent this
technical challenge by establishing new theoretical results for Mondrian
partitions and their associated Mondrian trees and forests, which may be of
independent interest.

The second main contribution of the chapter is to propose a debiasing approach
for the Mondrian random forest point estimator. We accomplish this by first
precisely characterizing the probability limit of the large sample conditional
bias, and then applying a debiasing procedure based on the generalized
jackknife \citep{schucany1977improvement}. We thus exhibit a Mondrian random
forest variant which is minimax-optimal in pointwise mean squared error when
the regression function is $\beta$-H{\"o}lder for any $\beta > 0$. Our method
works by generating an ensemble of Mondrian random forests carefully chosen to
have smaller misspecification bias when extra smoothness is available,
resulting in minimax optimality even for $\beta > 2$. This result complements
\citet{mourtada2020minimax} by demonstrating the existence of a class of
Mondrian random forests that can efficiently exploit the additional smoothness
of the unknown regression function for minimax-optimal point estimation. Our
proposed debiasing procedure is also useful when conducting statistical
inference because it provides a principled method for ensuring that the bias is
negligible relative to the standard deviation of the estimator. More
specifically, we use our debiasing approach to construct valid inference
procedures based on robust bias correction
\citep{calonico2018effect,calonico2022coverage}.

This chapter is structured as follows. In Section~\ref{sec:mondrian_setup} we
introduce the Mondrian process and give our assumptions on the data generating
process, using a H{\"o}lder smoothness condition on the regression function to
control the bias of various estimators. We define the Mondrian random forest
estimator and present our assumptions on its lifetime parameter and the number
of trees. We give our notation for the following sections in this chapter.

Section~\ref{sec:mondrian_inference} presents our first set of main results,
beginning with a central limit theorem for the centered Mondrian random forest
estimator (Theorem~\ref{thm:mondrian_clt}), in which we characterize the
limiting
variance. Theorem~\ref{thm:mondrian_bias} complements this result by precisely
calculating the limiting bias of the estimator, with the aim of subsequently
applying a debiasing procedure. To enable valid feasible statistical inference,
we provide a consistent variance estimator in
Theorem~\ref{thm:mondrian_variance_estimation} and briefly discuss implications
for
lifetime parameter selection.

In Section~\ref{sec:mondrian_overview_proofs} we provide a brief overview of
the proofs
of these first main results. We focus on the technical innovations and general
strategic approach, giving some insight into the challenges involved, and refer
the reader to Section~\ref{sec:mondrian_app_proofs} for detailed proofs.

In Section~\ref{sec:mondrian_debiased} we define debiased Mondrian random
forests, a
collection of estimators based on linear combinations of Mondrian random
forests with varying lifetime parameters. These parameters are carefully chosen
to annihilate leading terms in our bias characterization, yielding an estimator
with provably superior bias properties
(Theorem~\ref{thm:mondrian_bias_debiased}). In
Theorem~\ref{thm:mondrian_clt_debiased}
we verify that a central limit theorem continues to hold for the debiased
Mondrian random forest. We again state the limiting variance, discuss the
implications for the lifetime parameter, and provide a consistent variance
estimator (Theorem~\ref{thm:mondrian_variance_estimation_debiased}) for
constructing
confidence intervals (Theorem~\ref{thm:mondrian_confidence_debiased}). As a
final
corollary of the improved bias properties, we demonstrate in
Theorem~\ref{thm:mondrian_minimax} that the debiased Mondrian random forest
estimator is minimax-optimal in pointwise mean squared error for all
$\beta > 0$, provided that $\beta$ is known a priori.

Section~\ref{sec:mondrian_parameter_selection} discusses tuning parameter
selection,
beginning with a data-driven approach to selecting the crucial lifetime
parameter using polynomial estimation, alongside other practical suggestions
including generalized cross-validation.
We also give advice on choosing the number of trees, and other parameters
associated with the debiasing procedure.

In Section~\ref{sec:mondrian_weather} we present an illustrative example
application of our proposed methodology for estimation and inference in the
setting of weather forecasting in Australia. We demonstrate the use of
our debiased Mondrian random forest estimator and our
generalized cross-validation procedure for lifetime parameter selection,
as well as the construction of point estimates and confidence intervals.

Concluding remarks are given in Section~\ref{sec:mondrian_conclusion}, while
Appendix~\ref{app:mondrian} contains all the mathematical proofs of our
theoretical contributions, along with some other technical
probabilistic results on the Mondrian process which may be of interest.

\subsection{Notation}

We write $\|\cdot\|_2$ for the usual Euclidean $\ell^2$-norm on $\R^d$. The
natural numbers are $\N = \{0, 1, 2, \ldots \}$. We use $a \wedge b$ for the
minimum and $a \vee b$ for the maximum of two real numbers. For a set $A$, we
use $A^{\comp}$ for the complement whenever the background space is clear from
context. We use $C$ to denote a positive constant whose value may change from
line to line. For non-negative sequences $a_n$ and $b_n$, write
$a_n \lesssim b_n$ or $a_n = O(b_n)$ to indicate that $a_n / b_n$ is bounded
for $n\geq 1$. Write $a_n \ll b_n$ or $a_n = o(b_n)$ if $a_n / b_n \to 0$. If
$a_n \lesssim b_n \lesssim a_n$, write $a_n \asymp b_n$. For random
non-negative sequences $A_n$ and $B_n$, similarly write $A_n \lesssim_\P B_n$
or $A_n = O_\P(B_n)$ if $A_n / B_n$ is bounded in probability,
and $A_n = o_\P(B_n)$ if $A_n / B_n \to 0$ in probability. Convergence of
random variables $X_n$ in distribution to a law $\P$ is denoted by
$X_n \rightsquigarrow \P$.

\section{Setup}
\label{sec:mondrian_setup}

When using a Mondrian random forest, there are two sources of randomness. The
first is of course the data, and here we consider the nonparametric regression
setting with $d$-dimensional covariates. The second source is a collection of
independent trees drawn from a Mondrian process, which we define in the
subsequent section, using a specified lifetime parameter.

\subsection{The Mondrian process}
\label{sec:mondrian_process}

The Mondrian process was introduced by \citet{roy2008mondrian} and offers a
canonical method for generating random rectangular partitions, which can be
used as the trees for a random forest
\citep{lakshminarayanan2014mondrian,lakshminarayanan2016mondrian}. For
the reader's convenience, we give a brief description of this process here; see
\citet[Section~3]{mourtada2020minimax} for a more complete definition.

For a fixed dimension $d$ and lifetime parameter $\lambda > 0$, the Mondrian
process is a stochastic process taking values in the set of finite rectangular
partitions of $[0,1]^d$. For a rectangle
$D = \prod_{j=1}^d [a_j, b_j] \subseteq [0,1]^d$,
we denote the side aligned with dimension $j$ by $D_j = [a_j, b_j]$, write
$D_j^- = a_j$ and $D_j^+ = b_j$ for its left and right endpoints respectively,
and use $|D_j| = D_j^+ - D_j^-$ for its length. The volume of $D$ is
$|D| = \prod_{j=1}^{d} |D_j|$ and its linear dimension (or half-perimeter) is
$|D|_1 = \sum_{j=1}^{d} |D_j|$.

To sample a partition $T$ from the Mondrian process
$\cM \big( [0,1]^d, \lambda \big)$ we start at time $t=0$ with the trivial
partition of $[0,1]^d$ which has no splits. We then repeatedly apply the
following procedure to each cell $D$ in the partition. Let $t_D$ be the time at
which the cell was formed, and sample $E_D \sim \Exp \left( |D|_1 \right)$. If
$t_D + E_D \leq \lambda$, then we split $D$. This is done by first selecting a
split dimension $J$ with $\P(J=j) = |D_j| / |D|_1$, and then sampling a split
location $S_J \sim \Unif\big[D_J^-, D_J^+\big]$. The cell $D$ splits into the
two new cells $\{x \in D : x_J \leq S_J\}$ and $\{x \in D : x_J > S_J\}$, each
with formation time $t_D + E_D$. The final outcome is the partition $T$
consisting of the cells $D$ which were not split because $t_D + E_D > \lambda$.
The cell in $T$ containing a point $x \in [0,1]^d$ is written $T(x)$.
Figure~\ref{fig:mondrian_process} shows typical realizations of
$T \sim \cM\big( [0,1]^d, \lambda \big)$ for $d=2$ and with different lifetime
parameters $\lambda$.
%
\begin{figure}[t]
\centering
%
\begin{subfigure}{0.32\textwidth}
\centering
%\includegraphics[scale=0.64]{graphics/plot_mondrian_process_1.pdf}
\caption{$\lambda = 3$}
\end{subfigure}
%
\begin{subfigure}{0.32\textwidth}
\centering
%\includegraphics[scale=0.64]{graphics/plot_mondrian_process_2.pdf}
\caption{$\lambda = 10$}
\end{subfigure}
%
\begin{subfigure}{0.32\textwidth}
\centering
%\includegraphics[scale=0.64]{graphics/plot_mondrian_process_3.pdf}
\caption{$\lambda = 30$}
\end{subfigure}
%
\caption[The Mondrian process]{
The Mondrian process $T \sim \cM \big( [0,1]^d, \lambda \big)$ with
$d=2$ and lifetime parameters $\lambda$.}
\label{fig:mondrian_process}
\end{figure}

\subsection{Data generation}

Throughout this chapter, we assume that the data satisfies
Assumption~\ref{ass:mondrian_data}. We begin with a definition of H{\"o}lder
continuity which will be used for controlling the bias of various estimators.

\begin{definition}[H{\"o}lder continuity]%

Take $\beta > 0$ and define $\flbeta$ to be the largest integer which is
strictly less than $\beta$. We say a function $g: [0,1]^d \to \R$ is
$\beta$-H{\"o}lder continuous and write $g \in \cH^\beta$ if $g$ is $\flbeta$
times differentiable and
$\max_{|\nu| = \flbeta}
\left| \partial^\nu g(x) - \partial^{\nu} g(x') \right|
\leq C \|x-x'\|_2^{\beta - \flbeta}$
for some constant $C > 0$ and all $x, x' \in [0,1]^d$. Here, $\nu \in \N^d$
is a multi-index with $|\nu| = \sum_{j=1}^d \nu_j$ and
$\partial^{\nu} g(x) = \partial^{|\nu|} g(x) \big/
\prod_{j=1}^d \partial x_j^{\nu_j}$. We say $g$ is Lipschitz if $g \in \cH^1$.

\end{definition}

\begin{assumption}[Data generation]%
\label{ass:mondrian_data}

Fix $d \geq 1$ and let $(X_i, Y_i)$ be i.i.d.\ samples from a distribution on
$\R^d \times \R$, writing $\bX = (X_1, \ldots, X_n)$ and
$\bY = (Y_1, \ldots, Y_n)$. Suppose $X_i$ has a Lebesgue density function
$f(x)$ on $[0,1]^d$ which is bounded away from zero and satisfies
$f \in \cH^\beta$ for some $\beta \geq 1$. Suppose $\E[Y_i^2 \mid X_i]$ is
bounded, let $\mu(X_i) = \E[Y_i \mid X_i]$, and assume $\mu \in \cH^\beta$.
Write $\varepsilon_i = Y_i - \mu(X_i)$ and assume
$\sigma^2(X_i) = \E[\varepsilon_i^2 \mid X_i]$
is Lipschitz and bounded away from zero.

\end{assumption}

Some comments are in order surrounding Assumption~\ref{ass:mondrian_data}. The
requirement that the covariate density $f(x)$ be strictly positive on all of
$[0,1]^d$ may seem strong, particularly when $d$ is moderately large. However,
since our theory is presented pointwise in $x$, it is sufficient for this to
hold only on some neighborhood of $x$. To see this, note that continuity
implies the density is positive on some hypercube containing $x$. Upon
rescaling the covariates, we can map this hypercube onto $[0,1]^d$. The same
argument of course holds for the H{\"o}lder smoothness assumptions and the
upper and lower bounds on the conditional variance function.

\subsection{Mondrian random forests}
\label{sec:mondrian_forests}

We define the basic Mondrian random forest estimator
\eqref{eq:mondrian_estimator} as in \citet{lakshminarayanan2014mondrian} and
\citet{mourtada2020minimax}, and will later extend it to a debiased version in
Section~\ref{sec:mondrian_debiased}. For a lifetime parameter $\lambda > 0$ and
forest
size $B \geq 1$, let $\bT = (T_1, \ldots, T_B)$ be a Mondrian forest where
$T_b \sim \cM\big([0,1]^d, \lambda\big)$ are i.i.d.\ Mondrian trees
which are independent of the data. For $x \in [0,1]^d$, write
$N_b(x) = \sum_{i=1}^{n} \I \left\{ X_i \in T_b(x) \right\}$ for the number of
samples in $T_b(x)$, with $\I$ denoting an indicator function. Then the
Mondrian random forest estimator of $\mu(x)$ is
%
\begin{equation}
\label{eq:mondrian_estimator}
\hat\mu(x) = \frac{1}{B} \sum_{b=1}^B
\frac{\sum_{i=1}^n Y_i \, \I\big\{ X_i \in T_b(x) \big\}} {N_b(x)}.
\end{equation}
%
If there are no samples $X_i$ in $T_b(x)$ then $N_b(x) = 0$, so we define
$0/0 = 0$ (see Section~\ref{sec:mondrian_app_proofs} for details). To ensure the
bias and variance of the Mondrian random forest estimator converge to zero (see
Section~\ref{sec:mondrian_inference}), and to avoid boundary issues, we impose
some basic conditions on $x$, $\lambda$, and $B$ in
Assumption~\ref{ass:mondrian_estimator}.

\begin{assumption}[Mondrian random forest estimator]%
\label{ass:mondrian_estimator}
%
Suppose $x \in (0,1)^d$ is an interior point of the support of $X_i$,
$\frac{\lambda^d}{n} \to 0$,
$\log \lambda \asymp \log n$,
and $B \asymp n^{\xi}$ for some $\xi \in (0, 1)$,
which may depend on the dimension $d$ and smoothness $\beta$.
%
\end{assumption}

Assumption~\ref{ass:mondrian_estimator} implies that the size of the forest $B$
grows
with $n$. For the purpose of mitigating the computational burden, we suggest
the sub-linear polynomial growth $B \asymp n^{\xi}$, satisfying the conditions
imposed in our main results. Large forests usually do not present computational
challenges in practice as the ensemble estimator is easily parallelizable over
the trees. We emphasize places where this ``large forest'' condition is
important to our theory as they arise throughout the chapter.

\section{Inference with Mondrian random forests}%
\label{sec:mondrian_inference}

We begin with a bias--variance decomposition for the Mondrian random
forest estimator:
%
\begin{align}
\nonumber
\hat\mu(x) - \mu(x)
&=
\Big( \hat\mu(x) - \E \big[ \hat \mu(x) \mid \bX, \bT \big]\Big)
+ \Big( \E \big[ \hat \mu(x) \mid \bX, \bT \big] - \mu(x)\Big) \\
&=
\nonumber
\frac{1}{B} \sum_{b=1}^B
\frac{\sum_{i=1}^n \varepsilon_i \, \I\big\{ X_i \in T_b(x) \big\}}
{N_b(x)} \\
\label{eq:mondrian_bias_variance}
&\quad+
\frac{1}{B} \sum_{b=1}^B
\frac{\sum_{i=1}^n \big(\mu(X_i) - \mu(x)\big) \,
\I\big\{ X_i \in T_b(x) \big\}} {N_b(x)}.
\end{align}
%
Our approach to inference is summarized as follows. Firstly, we provide a
central limit theorem (weak convergence to a Gaussian) for the first
``variance'' term in \eqref{eq:mondrian_bias_variance}. Secondly, we precisely
compute
the probability limit of the second ``bias'' term. By ensuring that the
standard deviation dominates the bias, a corresponding
central limit theorem holds for the Mondrian random forest. With an appropriate
estimator for the limiting variance, we establish procedures for valid and
feasible statistical inference on the unknown regression function $\mu(x)$.

We begin with the aforementioned central limit theorem, which forms the core of
our methodology for performing statistical inference. Before stating our main
result, we highlight some of the challenges involved. At first glance, the
summands in the first term in \eqref{eq:mondrian_bias_variance} seem to be
independent
over $1 \leq i \leq n$, conditional on the forest $\bT$, depending only on
$X_i$ and $\varepsilon_i$. However, the $N_b(x)$ appearing in the denominator
depends on all $X_i$ simultaneously, violating this independence assumption and
rendering classical central limit theorems inapplicable. A natural preliminary
attempt to resolve this issue is to observe that
%
\begin{equation*}
N_b(x)= \sum_{i=1}^{n} \I\big\{X_i \in T_b(x)\big\}
\approx n \, \P \big( X_i \in T_b(x) \mid T_b \big)
\approx n f(x) |T_b(x)|
\end{equation*}
%
with high probability. One could attempt to use this by approximating the
estimator with an average of i.i.d.\ random variables, or by employing a
central limit theorem conditional on $\bX$ and $\bT$. However, such an approach
fails because $\E \left[ \frac{1}{|T_b(x)|^2} \right] = \infty$; the possible
existence of small cells causes the law of the inverse cell volume to have
heavy tails. For similar reasons, attempts to directly establish a central
limit theorem based on $2 + \delta$ moments, such as the Lyapunov central limit
theorem, are ineffective.

We circumvent these problems by directly analyzing
$\frac{\I\{N_b(x) \geq 1\}}{N_b(x)}$. We establish concentration properties for
this non-linear function of $X_i$ via the Efron--Stein inequality
\citep[Section 3.1]{boucheron2013concentration} along with a sequence of
somewhat delicate preliminary lemmas regarding inverse moments of truncated
(conditional) binomial random variables. In particular, we show that
$\E \left[ \frac{\I \{N_b(x) \geq 1\}}{N_b(x)} \right]
\lesssim \frac{\lambda^d}{n}$ and
$\E \left[ \frac{\I \{N_b(x) \geq 1\}}{N_b(x)^2} \right]
\lesssim \frac{\lambda^{2d} \log n}{n^2}$.
Asymptotic normality is then established using a central limit theorem for
martingale difference sequences \citep[Theorem~3.2]{hall1980martingale} with
respect to an appropriate filtration.
Section~\ref{sec:mondrian_overview_proofs} gives
an overview our proof strategy in which we further discuss the underlying
challenges, while Section~\ref{sec:mondrian_app_proofs} gives all the technical
details.

\subsection{Central limit theorem}
\label{sec:mondrian_clt}

Theorem~\ref{thm:mondrian_clt} gives our first main result.

\begin{theorem}[Central limit theorem for the centered
Mondrian random forest estimator]%
\label{thm:mondrian_clt}
%
Suppose Assumptions~\ref{ass:mondrian_data} and \ref{ass:mondrian_estimator}
hold, and further assume that
$\E[Y_i^4 \mid X_i ]$ is bounded almost surely
and $\frac{\lambda^d \log n}{n} \to 0$. Then
%
\begin{align*}
\sqrt{\frac{n}{\lambda^d}}
\Big( \hat \mu(x) - \E \big[ \hat \mu(x) \mid \bX, \bT \big] \Big)
&\rightsquigarrow \cN\big(0, \Sigma(x)\big)
& &\text{where}
&\Sigma(x) &=
\frac{\sigma^2(x)}{f(x)} \left( \frac{4 - 4 \log 2}{3 } \right)^d.
\end{align*}
\end{theorem}

The condition of $B \to \infty$ is crucial, ensuring sufficient ``mixing'' of
different Mondrian cells to escape the heavy-tailed phenomenon detailed in the
preceding discussion. For concreteness, the large forest condition allows us to
deal with expressions such as
$\E \left[ \frac{1}{|T_b(x)| |T_{b'}(x)|} \right]
= \E \left[ \frac{1}{|T_b(x)|} \right] \E \left[ \frac{1}{|T_{b'}(x)|} \right]
\approx \lambda^{2d} < \infty$
where $b \neq b'$, by independence of the trees, rather than the ``no
ensembling'' single tree analog
$\E \left[ \frac{1}{|T_b(x)|^2} \right] = \infty$.

We take this opportunity to contrast Mondrian random forests with more
classical kernel-based smoothing methods. The lifetime $\lambda$ plays a
similar role to the inverse bandwidth in determining the effective sample size
$n / \lambda^d$, and thus the associated rate of convergence. However, due to
the Mondrian process construction, some cells are typically ``too small''
(equivalent to an insufficiently large bandwidth) to give an appropriate
effective sample size. Similarly, classical methods based on non-random
partitioning such as spline estimators \citep{huang2003local,cattaneo2020large}
typically impose a quasi-uniformity assumption to ensure all the cells are of
comparable size, a property which does not hold for the Mondrian process (not
even with probability approaching one).

\subsection*{Bias characterization}

We turn to the second term in \eqref{eq:mondrian_bias_variance}, which captures
the bias
of the Mondrian random forest estimator conditional on the covariates $\bX$ and
the forest $\bT$. As such, it is a random quantity which, as we will
demonstrate, converges in probability. We precisely characterize the limiting
non-random bias, including high-degree polynomials in $\lambda$ which for now
may seem ignorable. Indeed the magnitude of the bias is determined by its
leading term, typically of order $1/\lambda^2$ whenever $\beta \geq 2$, and
this suffices for ensuring a negligible contribution from the bias with an
appropriate choice of lifetime parameter. However, the advantage of specifying
higher-order bias terms is made apparent in Section~\ref{sec:mondrian_debiased}
when we
construct a debiased Mondrian random forest estimator. There, we target and
annihilate the higher-order terms in order to furnish superior estimation and
inference properties.
Theorem~\ref{thm:mondrian_bias} gives our main result on
the bias of the Mondrian random forest estimator.

\begin{theorem}[Bias of the Mondrian random forest estimator]%
\label{thm:mondrian_bias}
%
Suppose Assumptions~\ref{ass:mondrian_data} and \ref{ass:mondrian_estimator}
hold.
Then for each $1 \leq r \leq \lfloor \flbeta / 2 \rfloor$ there exists
$B_r(x) \in \R$, which is a function only of
the derivatives of $f$ and $\mu$ at $x$ up to order $2r$, with
%
\begin{equation*}
\E \left[ \hat \mu(x) \mid \bX, \bT \right]
= \mu(x)
+ \sum_{r=1}^{\lfloor \flbeta / 2 \rfloor}
\frac{B_r(x)}{\lambda^{2r}}
+ O_\P \left(
\frac{1}{\lambda^\beta}
+ \frac{1}{\lambda \sqrt B}
+ \frac{\log n}{\lambda} \sqrt{\frac{\lambda^d}{n}}
\right).
\end{equation*}
%
Whenever $\beta > 2$ the leading bias is the quadratic term
%
\begin{equation*}
\frac{B_1(x)}{\lambda^2}
=
\frac{1}{2 \lambda^2}
\sum_{j=1}^d \frac{\partial^2 \mu(x)}{\partial x_j^2}
+ \frac{1}{2 \lambda^2}
\frac{1}{f(x)}
\sum_{j=1}^{d} \frac{\partial \mu(x)}{\partial x_j}
\frac{\partial f(x)}{\partial x_j}.
\end{equation*}
%
If $X_i \sim \Unif\big([0,1]^d\big)$ then $f(x) = 1$,
and using multi-index notation we have
%
\begin{equation*}
\frac{B_r(x)}{\lambda^{2r}}
= \frac{1}{\lambda^{2r}} \sum_{|\nu|=r} \partial^{2 \nu} \mu(x)
\prod_{j=1}^d \frac{1}{\nu_j + 1}.
\end{equation*}
%
\end{theorem}

In Theorem~\ref{thm:mondrian_bias} we give some explicit examples of
calculating the
limiting bias if $\beta > 2$ or when $X_i$ are uniformly distributed. The
general form of $B_r(x)$ is provided in Section~\ref{sec:mondrian_app_proofs}
but
is somewhat unwieldy except in specific situations. Nonetheless the most
important properties are that $B_r(x)$ are non-random and do not depend on the
lifetime $\lambda$, crucial facts for our debiasing procedure given in
Section~\ref{sec:mondrian_debiased}. If the forest size $B$ does not diverge to
infinity
then we suffer the first-order bias term $\frac{1}{\lambda \sqrt B}$. This
phenomenon was explained by \citet{mourtada2020minimax}, who noted that it
allows single Mondrian trees to achieve minimax optimality only when
$\beta \in (0, 1]$. Large forests remove this first-order bias
and are optimal for all $\beta \in (0, 2]$.

Using Theorem~\ref{thm:mondrian_clt} and Theorem~\ref{thm:mondrian_bias}
together,
along with an appropriate choice of lifetime parameter $\lambda$,
gives a central limit theorem for the Mondrian random forest estimator
which can be used, for example, to build confidence intervals
for the unknown regression function $\mu(x)$
whenever the bias shrinks faster than the standard deviation.
In general this will require
$\frac{1}{\lambda^2} + \frac{1}{\lambda^\beta} + \frac{1}{\lambda \sqrt B}
\ll \sqrt{\frac{\lambda^d}{n}}$,
which can be satisfied by imposing the restrictions
$\lambda \gg n^{\frac{1}{d + 2(2 \wedge \beta)}}$
and $B \gg n^{\frac{2(2 \wedge \beta) - 2}{d + 2(2 \wedge \beta)}}$
on the lifetime $\lambda$ and forest size $B$.
If instead we aim for optimal point estimation,
then balancing the bias and standard deviation requires
$\frac{1}{\lambda^2} + \frac{1}{\lambda^\beta} + \frac{1}{\lambda \sqrt B}
\asymp \sqrt{\frac{\lambda^d}{n}}$,
which can be satisfied by
$\lambda \asymp n^{\frac{1}{d + 2(2 \wedge \beta)}}$
and $B \gtrsim n^{\frac{2(2 \wedge \beta) - 2}{d + 2(2 \wedge \beta)}}$.
Such a choice of $\lambda$ gives the convergence rate
$n^{\frac{-(2 \wedge \beta)}{d + 2(2 \wedge \beta)}}$
which is the minimax-optimal rate of convergence \citep{stone1982optimal}
for $\beta$-H{\"o}lder functions with $\beta \in (0,2]$
as shown by \citet[Theorem~2]{mourtada2020minimax}.
In Section~\ref{sec:mondrian_debiased} we will show how the Mondrian random
forest
estimator can be debiased, giving both weaker lifetime conditions for inference
and also improved rates of convergence, under additional smoothness assumptions.

\subsection*{Variance estimation}

The limiting variance $\Sigma(x)$ from the resulting central limit theorem
depends on the unknown quantities $\sigma^2(x)$ and $f(x)$.
To conduct feasible inference, we must therefore first estimate
$\Sigma(x)$. To this end, define
%
\begin{align}
\label{eq:mondrian_sigma2_hat}
\hat\sigma^2(x)
&=
\frac{1}{B} \sum_{b=1}^{B} \sum_{i=1}^n
\frac{\big(Y_i - \hat \mu(x)\big)^2 \, \I\{X_i \in T_b(x)\}} {N_b(x)}, \\
\nonumber
\hat\Sigma(x)
&=
\hat\sigma^2(x) \frac{n}{\lambda^d} \sum_{i=1}^n
\left( \frac{1}{B} \sum_{b=1}^B \frac{\I\{X_i \in T_b(x)\}}{N_b(x)} \right)^2.
\end{align}
%
In Theorem~\ref{thm:mondrian_variance_estimation} we show that this
estimator is consistent, and establish its rate of convergence.
%
\begin{theorem}[Variance estimation]%
\label{thm:mondrian_variance_estimation}
Grant Assumptions~\ref{ass:mondrian_data} and \ref{ass:mondrian_estimator},
and
suppose $\E[Y_i^4 \mid X_i ]$ is bounded almost surely. Then
%
\begin{align*}
\hat\Sigma(x)
= \Sigma(x)
+ O_\P \left(
\frac{(\log n)^{d+1}}{\lambda}
+ \frac{1}{\sqrt B} + \sqrt{\frac{\lambda^d \log n}{n}}
\right).
\end{align*}

\end{theorem}

\subsection{Confidence intervals}

Theorem~\ref{thm:mondrian_confidence} shows how to construct valid confidence
intervals
for the regression function $\mu(x)$ under the lifetime and forest size
assumptions previously discussed. For details on feasible and practical
selection of the lifetime parameter $\lambda$, see
Section~\ref{sec:mondrian_parameter_selection}.
%
\begin{theorem}[Feasible confidence intervals using a Mondrian random forest]%
\label{thm:mondrian_confidence}
%
Suppose that Assumptions~\ref{ass:mondrian_data} and
\ref{ass:mondrian_estimator} hold,
$\E[Y_i^4 \mid X_i ]$ is bounded almost surely,
and $\frac{\lambda^d \log n}{n} \to 0$. Assume that
$\lambda \gg n^{\frac{1}{d + 2(2 \wedge \beta)}}$
and $B \gg n^{\frac{2 (2 \wedge \beta) - 2}{d + 2 (2 \wedge \beta)}}$.
For a confidence level $\alpha \in (0, 1)$,
let $q_{1 - \alpha / 2}$ be the normal quantile satisfying
$\P \left( \cN(0, 1) \leq q_{1 - \alpha / 2} \right) = 1 - \alpha / 2$. Then
%
\begin{align*}
\P \left(
\mu(x) \in
\left[
\hat \mu(x)
- \sqrt{\frac{\lambda^d}{n}} \hat \Sigma(x)^{1/2}
q_{1 - \alpha / 2}, \
\hat \mu(x)
+ \sqrt{\frac{\lambda^d}{n}} \hat \Sigma(x)^{1/2}
q_{1 - \alpha / 2}
\right]
\right)
\to
1 - \alpha.
\end{align*}

\end{theorem}

When coupled with an appropriate lifetime selection method,
Theorem~\ref{thm:mondrian_confidence} gives a fully feasible procedure for
uncertainty
quantification in Mondrian random forests. Our procedure requires no adjustment
of the original Mondrian random forest estimator beyond ensuring that the bias
is negligible, and in particular does not rely on sample splitting. The
construction of confidence intervals is just one corollary of the weak
convergence result given in Theorem~\ref{thm:mondrian_clt}, and follows
immediately from Slutsky's theorem
\citep[Chapter~7]{pollard2002user}
with a consistent variance estimator. Other applications
include hypothesis testing on the value of $\mu(x)$ at a design point $x$ by
inversion of the confidence interval, as well as parametric specification
testing by comparison with a $\sqrt{n}$-consistent parametric regression
estimator. The construction of simultaneous confidence intervals for finitely
many points $x_1, \ldots, x_D$ can be accomplished either using standard
multiple testing corrections or by first establishing a multivariate central
limit theorem using the Cram{\'e}r--Wold device
\citep[Chapter~8]{pollard2002user}
and formulating a consistent multivariate variance estimator.

\section{Overview of proof strategy}%
\label{sec:mondrian_overview_proofs}

This section provides some insight into the general approach we use to
establish the main results in the preceding sections. We focus on the technical
innovations forming the core of our arguments, and refer the reader to
Section~\ref{sec:mondrian_app_proofs} for detailed proofs, including those for
the
debiased estimator discussed in the upcoming
Section~\ref{sec:mondrian_debiased}.

\subsection*{Preliminary results}

The starting point for our proofs is a characterization of the exact
distribution of the shape of a Mondrian cell $T(x)$. This property is a direct
consequence of the fact that the restriction of a Mondrian process to a subcell
remains Mondrian \citep[Fact~2]{mourtada2020minimax}. We have
%
\begin{align*}
|T(x)_j|
&= \left( \frac{E_{j1}}{\lambda} \wedge x_j \right)
+ \left( \frac{E_{j2}}{\lambda} \wedge (1-x_j) \right)
\end{align*}
%
for all $1 \leq j \leq d$, recalling that $T(x)_j$ is the side of the cell
$T(x)$ aligned with axis $j$, and where $E_{j1}$ and $E_{j2}$ are mutually
independent $\Exp(1)$ random variables. Our assumptions that $x \in (0,1)$ and
$\lambda \to \infty$ make the boundary terms $x_j$ and $1-x_j$
eventually ignorable so
%
\begin{align*}
|T(x)_j| &= \frac{E_{j1} + E_{j2}}{\lambda}
\end{align*}
%
with high probability. Controlling the size of the largest cell in the forest
containing $x$ is now straightforward with a union bound, exploiting the sharp
tail decay of the exponential distribution, and thus
%
\begin{align*}
\max_{1 \leq b \leq B} \max_{1 \leq j \leq d} |T_b(x)_j|
\lesssim_\P \frac{\log B}{\lambda}.
\end{align*}
%
This shows that up to logarithmic terms, none of the cells in the forest at $x$
are significantly larger than average, ensuring that the Mondrian random forest
estimator is localized around $x$ on the scale of $1/\lambda$, an important
property for the upcoming bias characterization.

Having provided upper bounds for the sizes of Mondrian cells, we also must
establish some lower bounds in order to quantify the ``small cell'' phenomenon
mentioned previously. The first step towards this is to bound the first two
moments of the truncated inverse Mondrian cell volume; we show that
%
\begin{align*}
\E\left[ 1 \wedge \frac{1}{n |T(x)|} \right]
&\asymp \frac{\lambda^d}{n}
&&\text{and}
&\frac{\lambda^{2d}}{n^2}
&\lesssim
\E\left[ 1 \wedge \frac{1}{n^2 |T(x)|^2} \right]
\lesssim \frac{\lambda^{2d} \log n}{n^2}.
\end{align*}
%
These bounds are computed directly using the exact distribution of $|T(x)|$.
Note that $\E\left[ \frac{1}{|T(x)|^2} \right] = \infty$ because
$\frac{1}{E_{j1} + E_{j2}}$ has only $2 - \delta$ finite moments, so the
truncation is crucial here. Since we nearly have two moments, this
truncation is at the expense of only a logarithmic term. Nonetheless, third and
higher truncated moments will not enjoy such tight bounds, demonstrating both
the fragility of this result and the inadequacy of tools such as the Lyapunov
central limit theorem which require $2 + \delta$ moments.

To conclude this investigation into the small cell phenomenon, we apply the
previous bounds to ensure that the empirical effective sample sizes
$N_b(x) = \sum_{i=1}^{n} \I \left\{ X_i \in T_b(x) \right\}$ are approximately
of the order $n / \lambda^d$ in an appropriate sense; we demonstrate that
%
\begin{align*}
\E\left[ \frac{\I\{N_b(x) \geq 1\}}{N_b(x)} \right]
&\lesssim \frac{\lambda^d}{n}
&&\text{and}
&\E\left[ \frac{\I\{N_b(x) \geq 1\}}{N_b(x)^2} \right]
&\lesssim \frac{\lambda^{2d} \log n}{n^2},
\end{align*}
%
as well as similar bounds for mixed terms such as
%
$\E \left[
\frac{\I\{N_b(x) \geq 1\}}{N_b(x)}
\frac{\I\{N_{b'}(x) \geq 1\}}{N_{b'}(x)}
\right]
\lesssim \frac{\lambda^{2d}}{n^2}$
%
when $b \neq b'$, which arise from covariance terms across multiple trees. The
proof of this result is involved and technical, and proceeds by induction. The
idea is to construct a class of subcells by taking all possible intersections
of the cells in $T_b$ and $T_{b'}$ (we show two trees here for clarity; there
may be more) and noting that each $N_b(x)$ is the sum of the number of points
in each such refined cell intersected with $T_b(x)$. We then swap out each
refined cell one at a time and replace the number of data points it contains
with its volume multiplied by $n f(x)$, showing that the expectation on the
left hand side does not increase too much using a moment bound for inverse
binomial random variables based on Bernstein's inequality. By induction and
independence of the trees, eventually the problem is reduced to computing
moments of truncated inverse Mondrian cell volumes, as above.

\subsection*{Central limit theorem}

To prove our main central limit theorem result
(Theorem~\ref{thm:mondrian_clt}), we use
the martingale central limit theorem given by
\citet[Theorem~3.2]{hall1980martingale}. For each $1 \leq i \leq n$ define
$\cH_{n i}$ to be the filtration generated by $\bT$, $\bX$, and
$(\varepsilon_j : 1 \leq j \leq i)$, noting that
$\cH_{n i} \subseteq \cH_{(n+1)i}$ because $B$ increases as $n$ increases.
Define the $\cH_{n i}$-measurable and square integrable variables
%
\begin{align*}
S_i(x) &=
\sqrt{\frac{n}{\lambda^d}} \frac{1}{B} \sum_{b=1}^B
\frac{\I \{X_i \in T_b(x)\} \varepsilon_i} {N_{b}(x)},
\end{align*}
%
which satisfy the martingale difference property
$\E [ S_i(x) \mid \cH_{n i} ] = 0$. Further,
%
\begin{align*}
\sqrt{\frac{n}{\lambda^d}}
\big(
\hat\mu(x)
- \E\left[
\hat\mu(x) \mid \bX, \bT
\right]
\big)
= \sum_{i=1}^n S_i(x).
\end{align*}
%
To establish weak convergence to $\cN\big(0, \Sigma(x)\big)$,
it suffices to check that $\max_i |S_i(x)| \to 0$ in probability,
$\E\left[\max_i S_i(x)^2\right] \lesssim 1$,
and $\sum_i S_i(x)^2 \to \Sigma(x)$ in probability.
Checking the first two of these is straightforward given the denominator moment
bounds derived above. For the third condition, we demonstrate that
$\sum_i S_i(x)^2$ concentrates by checking its variance is vanishing. To do
this, first observe that $S_i(x)^2$ is the square of a sum over the $B$ trees.
Expanding this square, we see that the diagonal terms (where $b = b'$) provide
a negligible contribution due to the large forest assumption. For the other
terms, we apply the law of total variance and the moment bounds detailed
earlier. Here, it is crucial that $b \neq b'$ in order to exploit the
independence of the trees and avoid having to control any higher moments. The
law of total variance requires that we bound
%
\begin{align*}
\Var \left[
\E \left[
\sum_{i=1}^n \sum_{b=1}^B \sum_{b' \neq b}
\frac{\I\{X_i \in T_b(x) \cap T_{b'}(x)\} \varepsilon_i^2}
{N_{b}(x) N_{b'}(x)} \Bigm| \bX, \bY
\right]
\right],
\end{align*}
%
which is the variance of a non-linear function of the i.i.d.\ variables
$(X_i, \varepsilon_i)$, and so we apply the Efron--Stein inequality.
The important insight here is that replacing a sample
$(X_i, \varepsilon_i)$ with an independent copy
$(\tilde X_i, \tilde \varepsilon_i)$ can change the value of
$N_b(x)$ by at most one. Further, this can happen only on the event
$\{ X_i \in T_{b}(x) \} \cup \{ \tilde X_i \in T_{b}(x) \}$,
which occurs with probability on the order $1/\lambda^d$
(the expected cell volume).

The final part of the central limit theorem proof is to calculate the limiting
variance $\Sigma(x)$. The penultimate step showed that we must have
%
\begin{align*}
\Sigma(x)
&= \lim_{n \to \infty} \sum_{i=1}^n \E \left[S_i(x)^2 \right]
= \lim_{n \to \infty}
\frac{n^2}{\lambda^d} \,
\E \left[
\frac{\I\{X_i \in T_b(x) \cap T_{b'}(x)\} \varepsilon_i^2}
{N_{b}(x) N_{b'}(x)}
\right],
\end{align*}
%
assuming the limit exists, so it remains to check this and calculate the limit.
It is a straightforward but tedious exercise to verify that each term can be
replaced with its conditional expectation given $T_b$ and $T_{b'}$, using some
further properties of the binomial and exponential distributions. This yields
%
\begin{align*}
\Sigma(x)
&=
\frac{\sigma^2(x)}{f(x)}
\lim_{\lambda \to \infty}
\frac{1}{\lambda^d}
\E \left[
\frac{|T_{b}(x) \cap T_{b'}(x)|}
{|T_{b}(x)| \, |T_{b'}(x)|}
\right]
= \frac{\sigma^2(x)}{f(x)}
\E \left[
\frac{(E_{1} \wedge E'_{1}) + (E_{2} \wedge E'_{2})}
{(E_{1} + E_{2}) (E'_{1} + E'_{2})}
\right]^d
\end{align*}
%
where $E_1$, $E_2$, $E'_1$, and $E'_2$ are independent $\Exp(1)$,
by the cell shape distribution and independence of the trees. This final
expectation is calculated by integration, using various incomplete gamma
function identities.

\subsection*{Bias characterization}

Our second substantial technical result is the bias characterization
given as Theorem~\ref{thm:mondrian_bias}, in which we precisely
characterize the probability limit of the conditional bias
%
\begin{align*}
\E \left[ \hat \mu(x) \mid \bX, \bT \right]
- \mu(x)
&=
\frac{1}{B} \sum_{b=1}^B
\sum_{i=1}^n \big( \mu(X_i) - \mu(x) \big)
\frac{\I\{X_i \in T_b(x)\}}{N_b(x)}.
\end{align*}
%
The first step is to pass to the ``infinite forest''
limit by taking an expectation conditional on $\bX$, or equivalently
marginalizing over $\bT$, applying the conditional Markov inequality
to see
%
\begin{align*}
\big|
\E \left[ \hat \mu(x) \mid \bX, \bT \right]
- \E \left[ \hat \mu(x) \mid \bX \right]
\big|
&\lesssim_\P
\frac{1}{\lambda \sqrt B}.
\end{align*}
%
While this may seem a crude approximation, it is already known that fixed-size
Mondrian forests have suboptimal bias properties when compared to forests with
a diverging number of trees. In fact, the error $\frac{1}{\lambda \sqrt B}$
exactly accounts for the first-order bias of individual Mondrian trees noted by
\citet{mourtada2020minimax}.

Next we show that $\E \left[ \hat \mu(x) \mid \bX \right]$ converges in
probability to its expectation, again using the Efron--Stein theorem for this
non-linear function of the i.i.d.\ variables $X_i$. The Lipschitz property of
$\mu$ and the upper bound on the maximum cell size give
$|\mu(X_i) - \mu(x)| \lesssim \max_{1 \leq j \leq d} |T_b(x)_j|
\lesssim_\P \frac{\log B}{\lambda}$
whenever $X_i \in T_b(x)$,
so we combine this with moment bounds for the denominator $N_b(x)$ to see
%
\begin{align*}
\left|
\E \left[ \hat \mu(x) \mid \bX \right]
- \E \left[ \hat \mu(x) \right]
\right|
\lesssim_\P
\frac{\log n}{\lambda} \sqrt{\frac{\lambda^d}{n}}.
\end{align*}

The next step is to approximate the resulting non-random bias
$\E \left[ \hat \mu(x) \right] - \mu(x)$ as a polynomial in $1/\lambda$.
To this end, we firstly apply a concentration-type result for the binomial
distribution to deduce that
%
\begin{align*}
\E \left[ \frac{\I\{N_b(x) \geq 1\}}{N_b(x)} \Bigm| \bT \right]
\approx \frac{1}{n \int_{T_b(x)} f(s) \diff s}
\end{align*}
%
in an appropriate sense, and hence,
by conditioning on $\bT$ and $\bX$ without $X_i$, we write
%
\begin{align}
\label{eq:mondrian_bias_ratio}
\E \left[ \hat \mu(x) \right] - \mu(x)
&\approx
\E \left[
\frac{\int_{T_b(x)} (\mu(s) - \mu(x)) f(s) \diff s}
{\int_{T_b(x)} f(s) \diff s}
\right].
\end{align}
%
Next we apply the multivariate version of Taylor's theorem to the integrands in
both the numerator and the denominator in \eqref{eq:mondrian_bias_ratio}, and
then apply
the Maclaurin series of $\frac{1}{1+x}$ and the multinomial theorem to recover
a single polynomial in $1/\lambda$. The error term is on the order of
$1/\lambda^\beta$ and depends on the smoothness of $\mu$ and $f$, and the
polynomial coefficients are given by various expectations involving exponential
random variables. The final step is to verify using symmetry of Mondrian cells
that all the odd monomial coefficients are zero, and to calculate some explicit
examples of the form of the limiting bias.

\section{Debiased Mondrian random forests}%
\label{sec:mondrian_debiased}

In this section we give our next main contribution, proposing a variant of the
Mondrian random forest estimator which corrects for higher-order bias with an
approach based on generalized jackknifing \citep{schucany1977improvement}. This
estimator retains the basic form of a Mondrian random forest estimator in the
sense that it is a linear combination of Mondrian tree estimators, but in this
section we allow for non-identical linear coefficients, some of which may be
negative, and for differing lifetime parameters across the trees. Since the
basic Mondrian random forest estimator is a special case of this more general
debiased version, we will discuss only the latter throughout the rest of the
chapter.

We use the explicit form of the bias given in Theorem~\ref{thm:mondrian_bias} to
construct a debiased version of the Mondrian forest estimator. Let $J \geq 0$
be the bias correction order. As such, with $J=0$ we retain the original
Mondrian forest estimator, with $J=1$ we remove second-order bias, and with
$J = \lfloor\flbeta / 2 \rfloor$ we remove bias terms up to and including order
$2 \lfloor\flbeta / 2 \rfloor$, giving the maximum possible bias reduction
achievable in the H{\"o}lder class $\cH^\beta$. As such, only bias terms of
order $1/\lambda^\beta$ will remain.

For $0 \leq r \leq J$ let $\hat \mu_r(x)$ be a Mondrian forest estimator
based on the trees $T_{b r} \sim \cM\big([0,1]^d, \lambda_r \big)$
for $1 \leq b \leq B$, where $\lambda_r = a_r \lambda$ for some $a_r > 0$
and $\lambda > 0$. Write $\bT$ to denote the collection of all the trees,
and suppose they are mutually independent. We find values of $a_r$ along with
coefficients $\omega_r$ in order to annihilate the leading $J$ bias terms of
the debiased Mondrian random forest estimator
%
\begin{align}
\label{eq:mondrian_debiased}
\hat \mu_\rd(x)
&= \sum_{r=0}^J \omega_r \hat \mu_r(x)
= \sum_{r=0}^{J} \omega_r
\frac{1}{B} \sum_{b=1}^B
\frac{\sum_{i=1}^n Y_i \, \I\big\{ X_i \in T_{r b}(x) \big\}} {N_{r b}(x)}.
\end{align}
%
This ensemble estimator retains the ``forest'' structure of the original
estimators, but with varying lifetime parameters $\lambda_r$ and coefficients
$\omega_r$. Thus by Theorem~\ref{thm:mondrian_bias} we want to solve
%
\begin{align*}
\sum_{r=0}^{J} \omega_r
\left( \mu(x) + \sum_{s=1}^{J} \frac{B_{s}(x)}{a_r^{2s} \lambda^{2s}} \right)
&= \mu(x)
\end{align*}
%
for all $\lambda$, or equivalently the system of linear equations
$\sum_{r=0}^{J} \omega_r = 1$
and $\sum_{r=0}^{J} \omega_r a_r^{-2s} = 0$ for each $1 \leq s \leq J$.
We solve these as follows. Define the $(J+1) \times (J+1)$ Vandermonde matrix
$A_{r s} = a_{r-1}^{2-2s}$,
and let $\omega = (\omega_0, \ldots, \omega_J)^\T \in \R^{J+1}$
and $e_0 = (1, 0, \ldots, 0)^\T \in \R^{J+1}$.
Then a solution for the debiasing coefficients is given by
$\omega = A^{-1} e_0$ whenever $A$ is non-singular.
In practice we can take $a_r$ to be a fixed geometric or arithmetic sequence
to ensure this is the case, appealing to the Vandermonde determinant formula:
$\det A = \prod_{0 \leq r < s \leq J} (a_r^{-2} - a_s^{-2})
\neq 0$ whenever $a_r$ are distinct. For example, we could set
$a_r = (1 + \gamma)^r$ or $a_r = 1 + \gamma r$ for some $\gamma > 0$.
Because we assume $\beta$, and therefore the choice of $J$, do not
depend on $n$, there is no need to quantify
the invertibility of $A$ by, for example, bounding its eigenvalues
away from zero as a function of $J$.

\subsection{Central limit theorem}

In Theorem~\ref{thm:mondrian_clt_debiased}, we verify that a central
limit theorem holds for the debiased
random forest estimator $\hat\mu_\rd(x)$ and give its limiting variance.
The strategy and challenges associated with proving
Theorem~\ref{thm:mondrian_clt_debiased} are identical to those discussed earlier
surrounding Theorem~\ref{thm:mondrian_clt}. In fact in
Section~\ref{sec:mondrian_app_proofs}
we provide a direct proof only for Theorem~\ref{thm:mondrian_clt_debiased}
and deduce Theorem~\ref{thm:mondrian_clt} as a special case.

\begin{theorem}[Central limit theorem for the
debiased Mondrian random forest estimator]%
\label{thm:mondrian_clt_debiased}
%
Suppose Assumptions~\ref{ass:mondrian_data} and \ref{ass:mondrian_estimator}
hold,
$\E[Y_i^4 \mid X_i ]$ is bounded,
and $\frac{\lambda^d \log n}{n} \to 0$. Then
%
\begin{align*}
\sqrt{\frac{n}{\lambda^d}}
\Big(
\hat \mu_\rd(x)
- \E \big[ \hat \mu_\rd(x) \mid \bX, \bT \big]
\Big)
&\rightsquigarrow
\cN\big(0, \Sigma_\rd(x)\big)
\end{align*}
%
where, with $\ell_{r r'} = \frac{2 a_r}{3} \left( 1 - \frac{a_{r}}{a_{r'}}
\log\left(\frac{a_{r'}}{a_{r}} + 1\right) \right)$,
the limiting variance is
%
\begin{align*}
\Sigma_\rd(x)
&=
\frac{\sigma^2(x)}{f(x)}
\sum_{r=0}^{J} \sum_{r'=0}^{J} \omega_r \omega_{r'}
\left( \ell_{r r'} + \ell_{r' r} \right)^d.
\end{align*}
%
\end{theorem}

It is easy to verify that in the case of no debiasing we have
$J=0$ and $a_0 = \omega_0 = 1$, yielding
$\Sigma_\rd(x) = \Sigma(x)$, and recovering Theorem~\ref{thm:mondrian_clt}.

\subsection*{Bias characterization}

In Theorem~\ref{thm:mondrian_bias_debiased} we verify that this debiasing
procedure does indeed annihilate the desired bias terms, and its proof is a
consequence of Theorem~\ref{thm:mondrian_bias} and the construction of the
debiased Mondrian random forest estimator $\hat\mu_\rd(x)$.

\begin{theorem}[Bias of the debiased Mondrian random forest estimator]%
\label{thm:mondrian_bias_debiased}
Grant Assumptions~\ref{ass:mondrian_data} and \ref{ass:mondrian_estimator}.
In the notation of Theorem~\ref{thm:mondrian_bias} with
$\bar\omega = \sum_{r=0}^J \omega_r a_r^{-2J - 2}$,
%
\begin{align*}
\E \big[ \hat \mu_\rd(x) \mid \bX, \bT \big]
&= \mu(x) + \I\{2J+2 < \beta \}
\frac{\bar\omega B_{J+1}(x)}{\lambda^{2J + 2}} \\
&\quad+
O_\P \left(
\frac{1}{\lambda^{2J + 4}}
+ \frac{1}{\lambda^\beta}
+ \frac{1}{\lambda \sqrt B}
+ \frac{\log n}{\lambda} \sqrt{\frac{\lambda^d}{n}}
\right).
\end{align*}
%
\end{theorem}

Theorem~\ref{thm:mondrian_bias_debiased} has the following consequence:
the leading bias term is characterized in terms of
$B_{J+1}(x)$ whenever $J < \beta/2 - 1$,
or equivalently $J < \lfloor \flbeta/2 \rfloor$,
that is, the debiasing order
$J$ does not exhaust the H{\"o}lder smoothness $\beta$.
If this condition does not hold, then the estimator is
fully debiased, and the resulting leading bias
term is bounded above by $1/\lambda^\beta$ up to constants,
but its form is left unspecified.

\subsection*{Variance estimation}

As before, we propose a variance estimator in order to conduct feasible
inference and show that it is consistent.
With $\hat\sigma^2(x)$ as in \eqref{eq:mondrian_sigma2_hat}
in Section~\ref{sec:mondrian_inference}, define the estimator
%
\begin{align}
\label{eq:mondrian_debiased_variance_estimator}
\hat\Sigma_\rd(x)
&=
\hat\sigma^2(x)
\frac{n}{\lambda^d}
\sum_{i=1}^n
\left(
\sum_{r=0}^J
\omega_r
\frac{1}{B}
\sum_{b=1}^B
\frac{\I\{X_i \in T_{r b}(x)\}}
{N_{r b}(x)}
\right)^2.
\end{align}
%
\begin{theorem}[Variance estimation]%
\label{thm:mondrian_variance_estimation_debiased}
Grant Assumptions~\ref{ass:mondrian_data} and \ref{ass:mondrian_estimator},
and
suppose $\E[Y_i^4 \mid X_i ]$ is bounded almost surely. Then
%
\begin{align*}
\hat\Sigma_\rd(x)
= \Sigma_\rd(x)
+ O_\P \left(
\frac{(\log n)^{d+1}}{\lambda}
+ \frac{1}{\sqrt B}
+ \sqrt{\frac{\lambda^d \log n}{n}}
\right).
\end{align*}
%
\end{theorem}

\subsection{Confidence intervals}

In analogy to Section~\ref{sec:mondrian_inference},
we now demonstrate the construction of feasible valid confidence
intervals using the debiased Mondrian random forest estimator
in Theorem~\ref{thm:mondrian_confidence_debiased}.
Once again we must ensure that the bias
(now significantly reduced due to our debiasing procedure)
is negligible when compared to the standard deviation
(which is of the same order as before).
We assume for simplicity that the estimator has been fully
debiased by setting $J \geq \lfloor \flbeta / 2\rfloor$
to yield a leading bias of order $1/\lambda^\beta$,
but intermediate ``partially debiased'' versions can easily
be provided, with leading bias terms of order
$1/\lambda^{\beta \wedge (2J+2)}$ in general.
We thus require
$\frac{1}{\lambda^\beta} + \frac{1}{\lambda \sqrt B}
\ll \sqrt{\frac{\lambda^d}{n}}$,
which can be satisfied by imposing the restrictions
$\lambda \gg n^{\frac{1}{d + 2 \beta}}$
and $B \gg n^{\frac{2\beta - 2}{d + 2\beta}}$
on the lifetime parameter $\lambda$
and forest size $B$.

\begin{theorem}[Feasible confidence intervals using a
debiased Mondrian random forest]%
\label{thm:mondrian_confidence_debiased}
%
Suppose Assumptions~\ref{ass:mondrian_data} and \ref{ass:mondrian_estimator}
hold,
$\E[Y_i^4 \mid X_i ]$ is bounded,
and $\frac{\lambda^d \log n}{n} \to 0$.
Fix $J \geq \lfloor \flbeta / 2 \rfloor$ and assume that
$\lambda \gg n^{\frac{1}{d + 2 \beta}}$
and $B \gg n^{\frac{2 \beta - 2}{d + 2 \beta}}$.
For a confidence level $\alpha \in (0, 1)$,
let $q_{1 - \alpha / 2}$ be as in Theorem~\ref{thm:mondrian_confidence}. Then
%
\begin{align*}
\P \left(
\mu(x) \in
\left[
\hat \mu_\rd(x)
- \sqrt{\frac{\lambda^d}{n}} \hat \Sigma_\rd(x)^{1/2}
q_{1 - \alpha / 2}, \
\hat \mu_\rd(x)
+ \sqrt{\frac{\lambda^d}{n}} \hat \Sigma_\rd(x)^{1/2}
q_{1 - \alpha / 2}
\right]
\right)
\to
1 - \alpha.
\end{align*}

\end{theorem}

One important benefit of our debiasing technique is made clear in
Theorem~\ref{thm:mondrian_confidence_debiased}: the restrictions imposed on the
lifetime
parameter $\lambda$ are substantially relaxed, especially in smooth classes
with large $\beta$. As well as the high-level of benefit of relaxed conditions,
this is also useful for practical selection of appropriate lifetimes for
estimation and inference respectively; see
Section~\ref{sec:mondrian_parameter_selection} for more details. Nonetheless,
such
improvements do not come without concession. The limiting variance
$\Sigma_\rd(x)$ of the debiased estimator is larger than that of the unbiased
version (the extent of this increase depends on the choice of the debiasing
parameters $a_r$), leading to wider confidence intervals and larger estimation
error in small samples despite the theoretical asymptotic improvements.

\subsection{Minimax optimality}

Our final result Theorem~\ref{thm:mondrian_minimax} shows that,
when using an appropriate sequence of lifetime parameters $\lambda$,
the debiased Mondrian random forest estimator
achieves, up to constants, the minimax-optimal rate of convergence
for estimating a regression function $\mu \in \cH^\beta$
in $d$ dimensions \citep{stone1982optimal}.
This result holds for all $d \geq 1$ and all $\beta > 0$,
complementing a previous result established only for $\beta \in (0, 2]$
by \citet{mourtada2020minimax}.
%
\begin{theorem}[Minimax optimality of the debiased
Mondrian random forest estimator]%
\label{thm:mondrian_minimax}
Grant Assumptions~\ref{ass:mondrian_data} and \ref{ass:mondrian_estimator},
and let $J \geq \lfloor \flbeta / 2 \rfloor$,
$\lambda \asymp n^{\frac{1}{d + 2 \beta}}$, and
$B \gtrsim n^{\frac{2 \beta - 2}{d + 2 \beta}}$. Then
%
\begin{align*}
\E \left[
\big( \hat \mu_\rd(x) - \mu(x) \big)^2
\right]^{1/2}
&\lesssim
\sqrt{\frac{\lambda^d}{n}}
+ \frac{1}{\lambda^\beta}
+ \frac{1}{\lambda \sqrt B}
\lesssim
n^{-\frac{\beta}{d + 2 \beta}}.
\end{align*}
%
\end{theorem}

The sequence of lifetime parameters $\lambda$ required in
Theorem~\ref{thm:mondrian_minimax} are chosen to balance the bias and standard
deviation bounds implied by Theorem~\ref{thm:mondrian_bias_debiased} and
Theorem~\ref{thm:mondrian_clt_debiased} respectively, in order to minimize the
pointwise
mean squared error. While selecting an optimal debiasing order $J$ needs only
knowledge of an upper bound on the smoothness $\beta$, choosing an optimal
sequence of $\lambda$ values does assume that $\beta$ is known a priori. The
problem of adapting to $\beta$ from data is challenging and beyond the scope of
this chapter; we provide some practical advice for tuning parameter
selection in Section~\ref{sec:mondrian_parameter_selection}.

Theorem~\ref{thm:mondrian_minimax} complements the minimaxity results proven by
\citet{mourtada2020minimax} for Mondrian trees (with $\beta \leq 1$) and for
Mondrian random forests (with $\beta \leq 2$), with one modification: our
version is stated in pointwise rather than integrated mean squared error. This
is because our debiasing procedure is designed to handle interior smoothing
bias and so does not provide any correction for boundary bias. We leave
the development of such boundary corrections to future work, but constructions
similar to higher-order boundary-correcting kernels should be possible. If the
region of integration is a compact set in the interior of $[0,1]^d$, then we do
obtain an optimal integrated mean squared error bound: if $\delta \in (0, 1/2)$
is fixed then under the same conditions as Theorem~\ref{thm:mondrian_minimax},
%
\begin{align*}
\E \left[
\int_{[\delta, 1-\delta]^d}
\big(
\hat \mu_\rd(x)
- \mu(x)
\big)^2
\diff x
\right]^{1/2}
&\lesssim
\sqrt{\frac{\lambda^d}{n}}
+ \frac{1}{\lambda^\beta}
+ \frac{1}{\lambda \sqrt B}
\lesssim
n^{-\frac{\beta}{d + 2 \beta}}.
\end{align*}

The debiased Mondrian random forest estimator defined in
\eqref{eq:mondrian_debiased} is
a linear combination of Mondrian random forests, and as such contains both a
sum over $0 \leq r \leq J$, representing the debiasing procedure, and a sum
over $1 \leq b \leq B$, representing the forest averaging. We have thus far
been interpreting this estimator as a debiased version of the standard Mondrian
random forest given in \eqref{eq:mondrian_estimator}, but it is
equally valid to swap the order of these sums. This gives rise to an
alternative point of view: we replace each Mondrian random tree with a
``debiased'' version, and then take a forest of such modified trees. This
perspective is more in line with existing techniques for constructing
randomized ensembles, where the outermost operation represents a $B$-fold
average of randomized base learners, not necessarily locally constant decision
trees, each of which has a small bias component \citep{caruana2004ensemble,
zhou2019deep, friedberg2020local}.

\section{Tuning parameter selection}%
\label{sec:mondrian_parameter_selection}

We discuss various procedures for selecting the parameters involved in fitting
a debiased Mondrian random forest; namely the base lifetime parameter
$\lambda$, the number of trees in each forest $B$, the bias correction order
$J$, and the debiasing scale parameters $a_r$ for $0 \leq r \leq J$.

\subsection{Selecting the base lifetime parameter
\texorpdfstring{$\lambda$}{lambda}}%
\label{sec:mondrian_lifetime_selection}

The most important parameter is the base Mondrian lifetime parameter $\lambda$,
which plays the role of a complexity parameter and thus governs the overall
bias--variance trade-off of the estimator. Correct tuning of $\lambda$ is
especially important in two main respects:
%
firstly, in order to use the central limit theorem established in
Theorem~\ref{thm:mondrian_clt_debiased}, we must have that the bias converges
to zero,
requiring $\lambda \gg n^{\frac{1}{d + 2\beta}}$.
%
Secondly, the minimax optimality result of Theorem~\ref{thm:mondrian_minimax}
is valid only in the regime $\lambda \asymp n^{\frac{1}{d + 2\beta}}$, and thus
requires careful determination in the more realistic finite-sample setting. For
clarity, in this section we use the notation $\hat\mu_\rd(x; \lambda, J)$ for
the debiased Mondrian random forest with lifetime $\lambda$ and debiasing order
$J$ as in \eqref{eq:mondrian_debiased}.
Similarly, write $\hat\Sigma_\rd(x; \lambda, J)$ for the associated
variance estimator given in \eqref{eq:mondrian_debiased_variance_estimator}.

For minimax-optimal point estimation when $\beta$ is known,
choose any sequence $\lambda \asymp n^{\frac{1}{d + 2\beta}}$
and use $\hat\mu_\rd(x; \lambda, J)$ with $J = \lfloor \flbeta / 2 \rfloor$,
following the theory given in Theorem~\ref{thm:mondrian_minimax}.
For an explicit example of how to choose the lifetime, one can instead use
$\hat\mu_\rd\big(x; \hat\lambda_{\AIMSE}(J-1), J-1\big)$
so that the leading bias is explicitly characterized by
Theorem~\ref{thm:mondrian_bias_debiased},
and with $\hat\lambda_{\AIMSE}(J-1)$ as defined below.
This is no longer minimax-optimal as $J-1 < J$
does not satisfy the conditions of Theorem~\ref{thm:mondrian_minimax}.

For performing inference, a more careful procedure is required;
we suggest the following method assuming $\beta > 2$.
Set $J = \lfloor \flbeta / 2 \rfloor$ as before,
and use $\hat\mu_\rd\big(x; \hat\lambda_{\AIMSE}(J-1), J\big)$
and $\hat\Sigma_\rd\big(x; \hat\lambda_{\AIMSE}(J-1), J\big)$
to construct a confidence interval.
The reasoning for this is that we select a lifetime tailored for a more biased
estimator than we actually use. This results in an inflated lifetime estimate,
guaranteeing the resulting bias is negligible when it is plugged into the fully
debiased estimator. This approach to tuning parameter selection and debiasing
for valid nonparametric inference corresponds to an application of robust bias
correction \citep{calonico2018effect,calonico2022coverage},
where the point estimator is bias-corrected
and the robust standard error estimator incorporates the additional
sampling variability introduced by the bias correction.
This leads to a more refined distributional approximation
but does not necessarily exhaust the underlying
smoothness of the regression function.
An alternative inference approach based on Lepskii's method
\citep{lepskii1992asymptotically,birge2001alternative}
could be developed with the latter goal in mind.

It remains to propose a concrete method for computing $\hat\lambda_{\AIMSE}(J)$
in the finite-sample setting; we suggest two such procedures based on plug-in
selection with polynomial estimation and cross-validation respectively,
building on classical ideas from the nonparametric
smoothing literature \citep{fan2020statistical}.

\subsubsection*{Lifetime selection with polynomial estimation}

Firstly, suppose $X_i \sim \Unif\big([0,1]^d\big)$
and that the leading bias of $\hat\mu_\rd(x)$ is well approximated by an
additively separable function so that,
writing $\partial^{2 J + 2}_j \mu(x)$
for $\partial^{2 J + 2}_j \mu(x) / \partial x_j^{2 J + 2}$,
%
\begin{align*}
\frac{\bar \omega B_{J+1}(x)}{\lambda^{2 J + 2}}
&\approx
\frac{1}{\lambda^{2 J + 2}}
\frac{\bar \omega }{J + 2}
\sum_{j=1}^d
\partial^{2 J + 2}_j \mu(x).
\end{align*}
%
Now suppose the model is homoscedastic so $\sigma^2(x) = \sigma^2$ and
the limiting variance of $\hat\mu_\rd$ is
%
\begin{align*}
\frac{\lambda^d}{n}
\Sigma_\rd(x)
&=
\frac{\lambda^d \sigma^2}{n}
\sum_{r=0}^{J}
\sum_{r'=0}^{J}
\omega_r
\omega_{r'}
\left( \ell_{r r'} + \ell_{r' r} \right)^d.
\end{align*}
%
The asymptotic integrated mean squared error (AIMSE) is
%
\begin{align*}
\AIMSE(\lambda, J)
&=
\frac{1}{\lambda^{4 J + 4}}
\frac{\bar \omega^2}{(J + 2)^2}
\int_{[0,1]^d}
\left(
\sum_{j=1}^d
\partial^{2 J + 2}_j \mu(x)
\right)^2
\diff x \\
&\quad+
\frac{\lambda^d \sigma^2}{n}
\sum_{r=0}^{J}
\sum_{r'=0}^{J}
\omega_r
\omega_{r'}
\left( \ell_{r r'} + \ell_{r' r} \right)^d.
\end{align*}
%
Minimizing over $\lambda > 0$ yields the AIMSE-optimal lifetime parameter
%
\begin{align*}
\lambda_{\AIMSE}(J)
&=
\left(
\frac{
\frac{(4 J + 4) \bar \omega^2}{(J + 2)^2}
n \int_{[0,1]^d}
\left(
\sum_{j=1}^d
\partial^{2 J + 2}_j \mu(x)
\right)^2
\diff x
}{
d \sigma^2
\sum_{r=0}^{J}
\sum_{r'=0}^{J}
\omega_r
\omega_{r'}
\left( \ell_{r r'} + \ell_{r' r} \right)^d
}
\right)^{\frac{1}{4 J + 4 + d}}.
\end{align*}
%
An estimator of $\lambda_{\AIMSE}(J)$ is therefore given by
%
\begin{align*}
\hat\lambda_{\AIMSE}(J)
&=
\left(
\frac{
\frac{(4 J + 4) \bar \omega^2}{(J + 2)^2}
\sum_{i=1}^n
\left(
\sum_{j=1}^d
\partial^{2 J + 2}_j \hat\mu(X_i)
\right)^2
}{
d \hat\sigma^2
\sum_{r=0}^{J}
\sum_{r'=0}^{J}
\omega_r
\omega_{r'}
\left( \ell_{r r'} + \ell_{r' r} \right)^d
}
\right)^{\frac{1}{4 J + 4 + d}}
\end{align*}
%
for some preliminary estimators
$\partial^{2 J + 2}_j \hat\mu(x)$ and $\hat\sigma^2$.
These can be obtained by fitting a global polynomial regression
to the data of order $2 J + 4$ without interaction terms.
To do this, define the $n \times ((2 J + 4)d + 1)$ design matrix $P$ with rows
%
\begin{align*}
P_i = \big(
1, X_{i1}, X_{i1}^2, \ldots, X_{i1}^{2 J + 4},
X_{i2}, X_{i2}^2, \ldots, X_{i2}^{2 J + 4},
\ldots,
X_{id}, X_{id}^2, \ldots, X_{id}^{2 J + 4}
\big),
\end{align*}
%
and let
%
$P_x = \big(
1, x_{1}, x_{1}^2, \ldots, x_{1}^{2 J + 4},
x_{2}, x_{2}^2, \ldots, x_{2}^{2 J + 4},
\ldots,
x_{d}, x_{d}^2, \ldots, x_{d}^{2 J + 4}
\big).
$
%
Then we define the derivative estimator as
%
\begin{align*}
\partial^{2 J + 2}_j \hat\mu(x)
&=
\partial^{2 J + 2}_j P_x
\big( P^\T P \big)^{-1}
P^\T \bY \\
&=
(2J + 2)!
\left(
0_{1 + (j-1)(2 J + 4) + (2J + 1)},
1, x_j, x_j^2 / 2,
0_{(d-j)(2 J + 4)}
\right)
\big( P^\T P \big)^{-1}
P^\T \bY,
\end{align*}
%
and the variance estimator $\hat\sigma^2$ is
based on the residual sum of squared errors of this model:
%
\begin{align*}
\hat\sigma^2
&=
\frac{1}{n - (2J + 4)d - 1}
\big(
\bY^\T \bY
- \bY^\T P \big( P^\T P \big)^{-1} P^\T \bY
\big).
\end{align*}

\subsubsection*{Lifetime selection with cross-validation}

As an alternative to the analytic plug-in methods described above, one can use
a cross-validation approach. While leave-one-out cross-validation (LOOCV) can
be applied directly \citep{fan2020statistical},
the linear smoother structure of the (debiased) Mondrian
random forest estimator allows a computationally simpler formulation. Writing
$\hat\mu_\rd^{-i}(x)$ for a debiased Mondrian random forest estimator fitted
without the $i$th data sample, it is easy to show that
%
\begin{align*}
\LOOCV(\lambda, J)
&=
\frac{1}{n}
\sum_{i=1}^{n}
\left( Y_i - \hat\mu_\rd^{-i}(X_i) \right)^2 \\
&=
\frac{1}{n}
\sum_{i=1}^{n}
\left(
\sum_{r=0}^{J}
\omega_r
\frac{1}{B}
\sum_{b=1}^{B}
\frac{1}{1 - 1/N_{r b}(X_i)}
\left( Y_i -
\sum_{j=1}^{n}
\frac{ Y_j \I \left\{ X_j \in T_{r b}(X_i) \right\}}
{N_{r b}(X_i)}
\right)
\right)^{2},
\end{align*}
%
avoiding refitting the model leaving each sample out in turn.
Supposing $X_i \sim \Unif\big([0,1]^d\big)$ and
replacing $1/N_{r b}(X_i)$ with their average expectation
$ \frac{1}{J+1} \sum_{r=0}^{J} \E \left[ 1/N_{r b}(X_i) \right]
\approx \bar a^d \lambda^d / n$
where $\bar a^d = \frac{1}{J+1} \sum_{r=0}^{J} a_r^d$
gives the generalized cross-validation (GCV) formula
%
\begin{align}
\label{eq:mondrian_gcv}
\GCV(\lambda, J)
&=
\frac{1}{n}
\sum_{i=1}^{n}
\left(
\frac{Y_i - \hat\mu_\rd(X_i)}
{1 - \bar a^d \lambda^d / n}
\right)^2.
\end{align}
%
The lifetime can then be selected by computing
either $\hat\lambda_{\LOOCV} \in \argmin_\lambda \LOOCV(\lambda, J)$
or $\hat\lambda_{\GCV} \in \argmin_\lambda \GCV(\lambda, J)$.
See Section~\ref{sec:mondrian_weather} for a practical illustration.

\subsection{Choosing the other parameters}

\subsubsection*{The number \texorpdfstring{$B$}{B} of trees in each forest}%

If no debiasing is applied, we suggest
$B = \sqrt{n}$ to satisfy
Theorem~\ref{thm:mondrian_confidence}.
If debiasing is used then we recommend
$B = n^{\frac{2J-1}{2J}}$, consistent with
Theorem~\ref{thm:mondrian_confidence_debiased}
and Theorem~\ref{thm:mondrian_minimax}.

\subsubsection*{The debiasing order \texorpdfstring{$J$}{J}}%

When debiasing a Mondrian random forest, one must decide
how many orders of bias to remove. This requires some
oracle knowledge of the H{\"o}lder smoothness of $\mu$ and $f$, which is
difficult to estimate statistically. As such, we recommend
removing only the first one or two bias terms, taking $J \in \{0,1,2\}$ to
avoid overly inflating the variance of the estimator.

\subsubsection*{The debiasing coefficients \texorpdfstring{$a_r$}{ar}}%

As in Section~\ref{sec:mondrian_debiased}, we take $a_r$ to be a fixed
geometric or arithmetic sequence. For example, one could set
$a_r = (1+\gamma)^r$ or $a_r = 1 + \gamma r$ for some $\gamma > 0$.
We suggest taking $a_r = 1.05^r$.

\section{Illustrative example: weather forecasting}%
\label{sec:mondrian_weather}

To demonstrate our methodology for estimation and inference with Mondrian random
forests, we consider a simple application
to a weather forecasting problem. We emphasize that the main aim of this
section is to provide intuition and understanding for how a Mondrian random
forest may be used in practice, and we refrain from an in-depth analysis of the
specific results obtained. Indeed, our assumption of i.i.d.\ data is
certainly violated with weather data, due to the time-series
structure of sequential observations.
Nonetheless, we use data from the \citet{bureau2017daily}, containing daily
weather information from 2007--2017, at 49 different
locations across Australia, with $n = 125\,927$ samples.

\begin{figure}[b!]
\centering
\begin{subfigure}{0.49\textwidth}
\centering
%\includegraphics[scale=0.64]{graphics/weather_data.png}%
\end{subfigure}
\begin{subfigure}{0.49\textwidth}
\centering
%\includegraphics[scale=0.64]{graphics/weather_data_filled_partition.png}%
\end{subfigure}
\caption[Australian weather forecasting data]{
Australian weather forecasting data. Left: colors indicate the response
variable of dry (orange) or wet (blue) on the following
day. Right: the data is overlaid with a Mondrian random tree,
fitted with a lifetime of $\lambda = 5$
selected by generalized cross-validation. Cell colors represent the response
proportions.}
\label{fig:mondrian_weather_data}
\end{figure}

We consider the classification problem of predicting whether or not it will
rain on the following day using two covariates: the percentage relative
humidity, and the pressure in mbar, both at 3pm on the current day. For the
purpose of framing this as a nonparametric regression problem, we consider
estimating the probability of rain as the regression function by setting
$Y_i = 1$ if there is rain on the following day and $Y_i = 0$ otherwise.
Outliers with pressure less than 985\,mbar or more than 1040\,mbar are removed
to justify the assertion in Assumption~\ref{ass:mondrian_data} that the density
of the covariates should be bounded away from zero, and the features are
linearly scaled to provide normalized samples
$(X_i, Y_i) \in [0, 1]^2 \times \{0, 1\}$.
We fit a Mondrian random forest to the data as defined in
Section~\ref{sec:mondrian_forests}, selecting the lifetime parameter with the
generalized cross-validation (GCV) method detailed in
Section~\ref{sec:mondrian_lifetime_selection}.

Figure~\ref{fig:mondrian_weather_data} plots the
data, using colors to indicate the response values, and illustrates how a
single Mondrian tree is fitted by sampling from an independent Mondrian process
and then computing local averages (equivalent to response proportions in this
special setting with binary outcomes) within each cell. The general pattern of
rain being predicted by high humidity and low pressure is apparent, with the
preliminary tree estimator taking the form of a step function on axis-aligned
rectangles. This illustrates the first-order bias of Mondrian random trees
discussed in Section~\ref{sec:mondrian_clt}, with the piecewise constant
estimator providing a poor approximation for the smooth true regression
function.

\begin{figure}[b!]
\centering
\begin{subfigure}{0.49\textwidth}
\centering
%\includegraphics[scale=0.64]{graphics/weather_forest_2.png}%
\end{subfigure}
\begin{subfigure}{0.49\textwidth}
\centering
%\includegraphics[scale=0.64]{graphics/weather_forest_design.png}%
\end{subfigure}
\caption[Fitting Mondrian random forests to the Australian weather data]{
Fitting Mondrian random forests to the Australian weather data.
Left: with $B=2$ trees, individual cells are clearly visible and the step
function persists. Right: with $B=40$ trees, the estimate is much smoother
as the individual tree estimates average out.
Three design points are identified for further analysis.}
\label{fig:mondrian_weather_forest}
\end{figure}

Figure~\ref{fig:mondrian_weather_forest} adds more trees to the estimator,
demonstrating the effect of increasing the forest size first to $B=2$
and then to $B=40$.
As more trees are included in the Mondrian random forest,
the regression estimate $\hat \mu(x)$ becomes smoother and therefore also
enjoys improved bias properties as shown in
Theorem~\ref{thm:mondrian_bias}, assuming a correct model specification.
We also choose three specific design points in the
(humidity, pressure) covariate space,
namely (20\%, 1020\,mbar), (70\%, 1000\,mbar), and (80\%, 990\,mbar),
at which to conduct inference
by constructing confidence intervals. See Table~\ref{tab:mondrian_weather_ci}
for the results.

\begin{figure}[b!]
\centering
\begin{subfigure}{0.49\textwidth}
\centering
%\includegraphics[scale=0.64]{graphics/weather_gcv.png}%
\end{subfigure}
\begin{subfigure}{0.49\textwidth}
\centering
%\includegraphics[scale=0.64]{graphics/weather_debiased_forest_design.png}%
\end{subfigure}
\caption[Cross-validation and debiasing for the Australian weather data]{
Left: mean squared error and generalized cross-validation scores
for Mondrian random forests with the Australian weather data.
Right: a debiased Mondrian random forest with $B=20$, giving $40$ trees
in total. Three design points are identified for further analysis.}
\label{fig:mondrian_weather_gcv}
\end{figure}

In Figure~\ref{fig:mondrian_weather_gcv} we show the mean squared error and GCV
scores
computed using \eqref{eq:mondrian_gcv} with $B=400$ trees
for several candidate lifetime parameters $\lambda$. As
expected, the mean squared error decreases monotonically
as $\lambda$ increases and the model
overfits, but the GCV score is minimized at a value which appropriately
balances the bias and variance; we take $\lambda = 5$.
We then fit a debiased Mondrian forest
with bias correction order $J = 1$ as described in
Section~\ref{sec:mondrian_debiased}, using $B=20$ trees at each debiasing level
$r \in \{0, 1\}$ for a total of $40$ trees.
We continue to use the same lifetime parameter
$\lambda = 5$ selected through GCV without debiasing, following the approach
recommended in Section~\ref{sec:mondrian_lifetime_selection} to ensure valid
inference
through negligible bias.
The resulting debiased Mondrian random forest estimate is noticeably
less smooth than the version without bias correction.
This is expected due to both the inflated variance resulting from the debiasing
procedure, and the undersmoothing enacted by selecting a lifetime parameter
using GCV on the original estimator without debiasing.

\begin{table}[t]
\centering
\begin{tabular}{|c|c|c|c|c|c|c|}
\hline
\multirow{2}{*}{Point}
& \multirow{2}{*}{Humidity}
& \multirow{2}{*}{Pressure}
& \multicolumn{2}{|c|}{No debiasing, $J=0$}
& \multicolumn{2}{|c|}{Debiasing, $J=1$} \\
\cline{4-7}
& &
& $\hat\mu(x)$ & 95\% CI
& $\hat\mu(x)$ & 95\% CI \\
\hline
$1$ & $20\%$ & $1020\,\textrm{mbar}$ &
$\phantom{0}4.2\%$ &
$3.9\%$ -- $4.5\%$ &
$\phantom{0}2.0\%$ &
$1.6\%$ -- $2.4\%$ \\
$2$ & $70\%$ & $1000\,\textrm{mbar}$ &
$52.6\%$ &
$51.7\%$ -- $53.6\%$ &
$59.8\%$ &
$57.8\%$ -- $61.9\%$ \\
$3$ & $80\%$ & $\phantom{1}990\,\textrm{mbar}$ &
$78.1\%$ &
$75.0\%$ -- $81.2\%$ &
$93.2\%$ &
$86.7\%$ -- $99.6\%$ \\
\hline
\end{tabular}
\caption[Results for the Australian weather data]{
Results for the Australian weather data
at three specified design points.}
\label{tab:mondrian_weather_ci}
\end{table}

Table~\ref{tab:mondrian_weather_ci} presents numerical results for estimation
and
inference at the three specified design points. We first give the outcomes
without debiasing, using a Mondrian random forest with $B = 400$ trees and
$\lambda = 5$ selected by GCV. We then show the results with a first-order
($J=1$) debiased Mondrian random forest using $B = 200$ (again a total of
$400$ trees) and the same value of $\lambda = 5$. The predicted chance of rain
$\hat\mu(x)$ is found to vary substantially across different covariate values,
and the resulting confidence intervals (CI) are generally narrow due to the
large sample size and moderate lifetime parameter. The forest with debiasing
exhibits more extreme predictions away from $50\%$ and wider confidence
intervals in general, in line with the illustration in
Figure~\ref{fig:mondrian_weather_gcv}. Interestingly, the confidence intervals
for the
non-debiased and debiased estimators do not intersect, indicating that the
original estimator is severely biased, and providing further justification for
our modified debiased random forest estimator.

\section{Conclusion}%
\label{sec:mondrian_conclusion}

We gave a central limit theorem for the Mondrian random forest estimator
and showed how to perform statistical inference on an unknown nonparametric
regression function. We introduced debiased versions of the Mondrian random
forest, and demonstrated their advantages
for statistical inference and minimax-optimal estimation. We discussed
tuning parameter selection, enabling a fully feasible and practical methodology.
An application to weather forecasting was presented
as an illustrative example. Implementations of this chapter's methodology and
empirical results are provided by a Julia
package at \github{wgunderwood/MondrianForests.jl}.
This work is based on \citet{cattaneo2023inference}, and has been
presented by Underwood at the University of Illinois Statistics Seminar (2024),
the University of Michigan Statistics Seminar (2024), and the University of
Pittsburgh Statistics Seminar (2024).

\chapter{Dyadic Kernel Density Estimators}
\label{ch:kernel}

% abstract
Dyadic data is often encountered when quantities of interest are associated
with the edges of a network. As such, it plays an important role in statistics,
econometrics, and many other data science disciplines. We consider the problem
of uniformly estimating a dyadic Lebesgue density function, focusing on
nonparametric kernel-based estimators taking the form of dyadic empirical
processes. The main contributions of this chapter
include the minimax-optimal uniform
convergence rate of the dyadic kernel density estimator, along with strong
approximation results for the associated standardized and Studentized
$t$-processes. A consistent variance estimator enables the construction of
valid and feasible uniform confidence bands for the unknown density function.
We showcase the broad applicability of our results by developing novel
counterfactual density estimation and inference methodology for dyadic data,
which can be used for causal inference and program evaluation. A crucial
feature of dyadic distributions is that they may be ``degenerate'' at certain
points in the support of the data, a property making our analysis somewhat
delicate. Nonetheless our methods for uniform inference remain robust to the
potential presence of such points. For implementation purposes, we discuss
inference procedures based on positive semi-definite covariance estimators,
mean squared error optimal bandwidth selectors, and robust bias correction
techniques. We illustrate the empirical finite-sample performance of our
methods both in simulations and with real-world trade data, for which we make
comparisons between observed and counterfactual trade distributions in
different years. Our technical results concerning strong approximations and
maximal inequalities are of potential independent interest.

\section{Introduction}
\label{sec:kernel_introduction}

Dyadic data, also known as graphon data, plays an important role in the
statistical, social, behavioral, and biomedical sciences. In network settings,
this type of dependent data captures interactions between the units of study,
and its analysis is of interest in statistics \citep{kolaczyk2009statistical},
economics \citep{graham2020network}, psychology \citep{kenny2020dyadic}, public
health \citep{luke2007network}, and many other data science disciplines. For
$n \geq 2$, a dyadic data set contains $\frac{1}{2}n(n-1)$ observed real-valued
random variables
%
\begin{align*}
\bW_n = (W_{i j}:1\leq i<j \leq n),
\quad\qquad W_{i j}
&= W(A_i,A_j,V_{i j}),
\end{align*}
%
where $W$ is an unknown function, $\bA_n=(A_{i}:1\leq i \leq n)$ are
independent and identically distributed (i.i.d.)\ latent random variables, and
$\bV_n=(V_{i j}:1\leq i<j \leq n)$ are i.i.d.\ latent random variables
independent of $\bA_n$. A natural interpretation of this data is as a complete
undirected network on $n$ vertices, with the latent variable $A_i$ associated
with node $i$ and the observed variable $W_{i j}$ associated with the edge
between nodes $i$ and $j$. The data generating process above is justified
without loss of generality by the celebrated Aldous--Hoover representation
theorem for exchangeable arrays
\citep{aldous1981representations,hoover1979relations}.

Various distributional features of dyadic data are of interest in applications.
Most of the statistical literature focuses on parametric analysis, almost
exclusively considering moments of (transformations of) the identically
distributed $W_{i j}$. See \citet{davezies2021exchangeable},
\citet{gao2021minimax}, and \citet{matsushita2021jackknife} for
contemporary contributions and overviews. More recently, however, a few
nonparametric procedures for dyadic data have been proposed in the literature
\citep{graham2021minimax,graham2024kernel}.

With the aim of estimating functions associated with $W_{i j}$
using nonparametric kernel methods, we investigate the statistical
properties of the class of local stochastic processes
%
\begin{align}\label{eq:kernel_estimator}
w \mapsto \hat{f}_W(w)
= \frac{2}{n(n-1)} \sum_{i=1}^{n-1} \sum_{j=i+1}^n k_h(W_{i j},w),
\end{align}
%
where $k_h(s,w)$ is a kernel function that can change with the $n$-varying
bandwidth parameter $h=h(n)$ and the evaluation point $w \in \cW\subseteq \R$.
For each $w\in\cW$ and with an appropriate choice of the kernel function
(e.g.\ $k_h(s,w)=K((s-w)/h)/h$ for an interior point $w$ of $\cW$ and a
fixed symmetric integrable kernel function $K$), the statistic $\hat{f}_W(w)$
becomes a kernel density estimator for the Lebesgue density function
$f_W(w) = \E\big[f_{W \mid AA}(w \mid A_i,A_j)\big]$, where
$f_{W \mid AA}(w \mid A_i,A_j)$ denotes the conditional Lebesgue density of
$W_{i j}$ given $A_i$ and $A_j$. Setting $k_h(s,w)=K((s-w)/h)/h$,
\citet{graham2024kernel} recently introduced the dyadic point estimator
$\hat{f}_W(w)$ and studied its large sample properties pointwise in
$w\in\cW=\mathbb{R}$, while \citet{chiang2020empirical} established its rate of
convergence uniformly in $w\in\cW$ for a compact interval $\cW$ strictly
contained in the support of the dyadic data $W_{i j}$.
\citet{chiang2022inference} obtained a distributional approximation for the
supremum statistic $\sup_{w\in\cW}\big|\hat{f}_W(w)\big|$ over a finite
collection $\cW$ of design points. More generally, as we discuss below, the
estimand $f_W(w)$ is useful in different applications because it forms the
basis for counterfactual distributional analysis
(Section~\ref{sec:kernel_counterfactual}) and other nonparametric and
semiparametric
methods (Section~\ref{sec:kernel_future}). While we assume throughout
that the network is complete, our approach generalizes in a straightforward way
to networks with missing edges, as in Section~\ref{sec:kernel_trade_data}.
This can be
seen by setting $W_{i j} = -\infty$ whenever the edge $\{i, j\}$ is not
present, so that the law of $W_{i j}$ is a mixture between a continuous
distribution and a point mass at $-\infty$. We apply our methodology to
recover the continuous component of this distribution, following
\citet{chiang2022inference}.

We contribute to the emerging literature on nonparametric smoothing methods for
dyadic data with two main technical results. Firstly, we derive the minimax
rate of uniform convergence for density estimation with dyadic data, and show
that the estimator $\hat{f}_W$ in \eqref{eq:kernel_estimator} is
minimax-optimal under appropriate conditions. Secondly, we present a set of
uniform distributional approximation results for the \emph{entire} stochastic
process $\big(\hat{f}_W(w):w\in\cW\big)$. Furthermore, we illustrate the
usefulness of our main results with two distinct substantive statistical
applications:
%
\begin{inlineroman}
\item
confidence bands for $f_W$ (Section~\ref{sec:kernel_implementation}), and
\item
estimation and inference for counterfactual
dyadic distributions (Section~\ref{sec:kernel_counterfactual}).
\end{inlineroman}
%
Our main results also lay the foundation for studying the uniform
distributional properties of other nonparametric and semiparametric tests and
estimators based on dyadic data (Section~\ref{sec:kernel_future}). Importantly,
our
inference results cannot be deduced from the existing U-statistic, empirical
process and U-process theory available in the literature
\citep{van1996weak,gine2021mathematical} because, as explained in detail below,
$\hat{f}_W(w)$ is not a standard U-statistic, nor is
(a suitable rescaling of) the stochastic process
$\hat{f}_W$ Donsker in general, and the underlying dyadic data $\bW_n$ exhibits
statistical dependence due to its network structure.

Section~\ref{sec:kernel_setup} outlines the setup and presents the main
assumptions imposed throughout this chapter. We demonstrate in
Theorem~\ref{thm:kernel_bias} how the smoothing bias of the dyadic kernel
density estimator can be controlled, and then discuss a Hoeffding-type
decomposition of the U-statistic-like $\hat{f}_W$
in Lemma~\ref{lem:kernel_hoeffding}. This is more general than
the standard Hoeffding decomposition for second-order U-statistics due to the
intrinsic dyadic data structure. In particular, \eqref{eq:kernel_hoeffding}
shows that $\hat{f}_W(w)$ decomposes into a sum of the four terms $B_n(w)$,
$L_n(w)$, $E_n(w)$, and $Q_n(w)$, where $E_n(w)$ is not present in the classical
second-order U-statistic theory. The first term $B_n(w)$ captures the usual
smoothing bias, the second term $L_n(w)$ is akin to the H{\'a}jek projection
for second-order U-statistics, the third term $E_n(w)$ is a mean-zero double
average of conditionally independent terms, and the fourth term $Q_n(w)$ is a
negligible totally degenerate second-order U-process. The leading stochastic
fluctuations of the process $\hat{f}_W$ are captured by $L_n$ and $E_n$, both
of which are known to be asymptotically distributed as Gaussian random
variables pointwise in $w\in\cW$ \citep{graham2024kernel}. However, the
H{\'a}jek projection term $L_n$ will often be ``degenerate'' at some or
possibly all evaluation points $w\in\cW$.
The three possible types of degeneracy are detailed in
Lemma~\ref{lem:kernel_trichotomy},
and we establish bounds in probability for each term in the Hoeffding-type
decomposition in Lemma~\ref{lem:kernel_uniform_concentration}.
We give an example of a simple family of dyadic distributions
exhibiting all three degeneracy types.

Section~\ref{sec:kernel_point_estimation} studies minimax convergence rates for
point
estimation of $f_W$ uniformly over $\cW$ and gives precise conditions under
which the estimator $\hat{f}_W$ is minimax-optimal. Firstly, in
Theorem~\ref{thm:kernel_uniform_consistency} we establish the uniform rate of
convergence of $\hat{f}_W$ for $f_W$. This result improves upon the recent
paper of \citet{chiang2020empirical} by allowing for compactly supported dyadic
data and generic kernel-like functions $k_h$ (including boundary-adaptive
kernels), while also explicitly accounting for possible degeneracy of the
H\'{a}jek projection term $L_n$ at some or possibly all points $w\in\cW$.
Secondly, in Theorem~\ref{thm:kernel_minimax} we derive the minimax uniform
convergence rate for estimating $f_W$, again allowing for possible degeneracy,
and verify that it is achieved by $\hat f_W$. This result appears to be new to
the literature, complementing recent work on parametric moment estimation using
graphon data \citep{gao2021minimax} and on nonparametric kernel-based
regression using dyadic data \citep{graham2021minimax}.

Section~\ref{sec:kernel_inference} presents a distributional analysis of the
stochastic process $\hat{f}_W$ uniformly in $w \in \cW$. Because the
$t$-process based on $\hat{f}_W$ is
not asymptotically tight in general, it does not converge weakly in the space
of uniformly bounded real functions supported on $\cW$ and equipped with the
uniform norm \citep{van1996weak}, and hence is non-Donsker. To circumvent this
problem, we employ strong approximation methods to characterize its
distributional properties. Up to the smoothing bias term $B_n$ and the
negligible term $Q_n$, it suffices to consider the stochastic process
$w \mapsto L_n(w)+E_n(w)$. Since $L_n$ can be degenerate at some or possibly all
points $w\in\cW$, and also because under some bandwidth choices both $L_n$ and
$E_n$ can be of comparable order, it is crucial to analyze the joint
distributional properties of $L_n$ and $E_n$. To do so, we employ a carefully
crafted conditioning approach where we first establish an unconditional strong
approximation for $L_n$ and a conditional-on-$\bA_n$ strong approximation for
$E_n$. We then combine these to obtain a strong approximation for $L_n+E_n$.

The stochastic process $L_n$ is an empirical process indexed by an $n$-varying
class of functions depending only on the i.i.d.\ random variables $\bA_n$. Thus
we use the celebrated Hungarian construction \citep{komlos1975approximation},
building on ideas in \citet{gine2004kernel} and \citet{gine2010confidence}. The
resulting rate of strong approximation is optimal, and follows from a generic
strong approximation result of potential independent interest given in
Section~\ref{sec:kernel_app_technical}. Our main result for $L_n$ is given as
Lemma~\ref{lem:kernel_strong_approx_Ln}, and makes explicit the potential
presence of
degenerate points.

The stochastic process $E_n$ is an empirical process depending on the dyadic
variables $W_{i j}$ and indexed by an $n$-varying class of functions. When
conditioning on $\bA_n$, the variables $W_{i j}$ are independent but not
necessarily identically distributed (i.n.i.d.), and thus we establish a
conditional-on-$\bA_n$ strong approximation for $E_n$ based on the Yurinskii
coupling \citep{yurinskii1978error}, leveraging a refinement obtained by
\citet*[Lemma~38]{belloni2019conditional}. This result follows from a generic
strong approximation result which gives a novel rate of strong approximation
for (local) empirical processes based on i.n.i.d. data, given in
Section~\ref{sec:kernel_app_technical}.
Lemma~\ref{lem:kernel_conditional_strong_approx_En} gives our conditional strong
approximation for $E_n$.

Once the unconditional strong approximation for $L_n$ and the
conditional-on-$\bA_n$ strong approximation for $E_n$ are established, we show
how to properly ``glue'' them together to deduce a final unconditional strong
approximation for $L_n+E_n$ and hence also for $\hat{f}_W$ and its associated
$t$-process. This final step requires some additional technical work. Firstly,
building on our conditional strong approximation for $E_n$, we establish an
unconditional strong approximation for $E_n$ in
Lemma~\ref{lem:kernel_unconditional_strong_approx_En}.
We then employ a generalization
of the celebrated Vorob'ev--Berkes--Philipp theorem \citep{dudley1999uniform},
given in given in Section~\ref{sec:kernel_app_technical}, to deduce a
\emph{joint}
strong approximation for $(L_n,E_n)$ and, in particular, for $L_n+E_n$. Thus we
obtain our main result in Theorem~\ref{thm:kernel_strong_approx_Tn},
which establishes
a valid strong approximation for the $t$-process associated with $\hat{f}_W$.
This uniform inference result complements the recent contribution of
\citet{davezies2021exchangeable}, which is not applicable here as
the $t$-process is non-Donsker.

We illustrate the applicability of our strong approximation results for
$\hat{f}_W$ and its associated $t$-process by constructing valid standardized
uniform confidence bands for the unknown density function $f_W$
in Theorem~\ref{thm:kernel_infeasible_ucb}. Instead of
relying on extreme value theory \citep*[as in][]{gine2004kernel}, we employ
anti-concentration methods, following \citet{chernozhukov2014anti}. This
illustration improves on the recent work of \citet{chiang2022inference}, which
obtained simultaneous confidence intervals for the dyadic density $f_W$ based
on a high-dimensional central limit theorem over rectangles, following prior
work by \citet{chernozhukov2017central}. The distributional
approximation therein is applied to the H\'{a}jek projection term $L_n$ only,
whereas our main construction leading to
Theorem~\ref{thm:kernel_strong_approx_Tn}
gives a strong approximation for the entire U-process-like $\hat{f}_W$ and its
associated $t$-process, uniformly on $\cW$. As a consequence, our uniform
inference theory is robust to potential unknown degeneracies in $L_n$ by virtue
of our strong approximation for $L_n+E_n$ and the use of proper standardization,
delivering a ``rate-adaptive'' inference procedure. Our result appears to be
the first to provide confidence bands that are valid uniformly over $w \in \cW$
rather than merely over a finite collection of design points. Moreover, they
provide distributional approximations for the whole $t$-statistic process,
which can be useful in applications where functionals other than the supremum
are of interest.

Section~\ref{sec:kernel_implementation} addresses outstanding issues of
implementation. Firstly, we discuss estimation of the covariance function of
the Gaussian process underlying our strong approximation results. We present
two estimators, one based on a plug-in method, and the other on a
positive semi-definite regularization thereof \citep{laurent2005semidefinite}.
We derive the uniform convergence rates for both estimators in
Lemma~\ref{lem:kernel_sdp}, which we then use to justify Studentization
of $\hat{f}_W$
and a feasible simulation-based approximation of the infeasible Gaussian
process underlying our strong approximation results. Secondly, we discuss
integrated mean squared error (IMSE) bandwidth selection and provide a simple
rule-of-thumb implementation for applications
\citep{wand1994kernel,simonoff1996smoothing}. Thirdly, we provide feasible,
valid uniform inference methods for $f_W$ by employing robust bias correction
\citep{calonico2018effect, calonico2022coverage}.
Algorithm~\ref{alg:kernel_method}
summarizes our entire feasible uniform inference methodology.

Section~\ref{sec:kernel_simulations} reports empirical evidence for our proposed
feasible robust bias-corrected confidence bands for $f_W$. We use simulations
to show that these confidence bands are robust to potential unknown degenerate
points in the underlying dyadic distribution.

Section~\ref{sec:kernel_counterfactual} presents novel results for
counterfactual
dyadic density estimation and inference, offering an application of our general
theory to a substantive problem in statistics and other data science
disciplines. Counterfactual distributions are important for causal inference
and policy evaluation
\citep{dinardo1996distribution,chernozhukov2013inference}, and in the context
of network data, such analysis can be used to answer empirical questions such
as ``what would the international trade distribution have been if
the gross domestic product (GDP) of the countries had remained the same as in a
previous year?'' We formally show how our theory for kernel-based dyadic
estimators can be used to infer the counterfactual density function of dyadic
data had some monadic covariates followed a different distribution. We propose
a two-step semiparametric reweighting approach in which we first estimate the
Radon--Nikodym derivative between the observed and counterfactual covariate
distributions using a simple parametric estimator, and then use this to
construct a weighted dyadic kernel density estimator. We present uniform
consistency, strong approximation, and feasible inference results for this
dyadic counterfactual density estimator. Finally, we illustrate our
methods with a real dyadic data set recording bilateral trade between
countries from 1995 to 2005, using GDP as a covariate for the
counterfactual analysis.

Section~\ref{sec:kernel_future} discusses further statistical applications
of our main
results, including dyadic density testing and nonparametric and
semiparametric dyadic regression. Section~\ref{sec:kernel_conclusion} concludes.
Appendix~\ref{app:kernel} includes other technical and methodological results,
proofs, and additional details omitted here to conserve space.
Section~\ref{sec:kernel_app_technical} may be of independent interest,
containing
two generic strong approximation theorems for empirical processes, a
generalized Vorob'ev--Berkes--Philipp theorem, and a maximal inequality for
i.n.i.d.\ random variables.

\subsection{Notation}

The total variation norm of a
real-valued function $g$ of a single real variable is written as
$\|g\|_\TV = \sup_{n \geq 1} \sup_{x_1 \leq \cdots \leq x_n}
\sum_{i=1}^{n-1} |g(x_{i+1}) - g(x_i)|$.
For an integer $m\geq 0$, denote by $\mathcal{C}^m(\mathcal{X})$
the space of all functions from $\R$ to $\R$
which are $m$ times continuously differentiable on
a subset $\mathcal{X} \subseteq \R$.
For $C>0$, define the H\"{o}lder class with smoothness parameter
$\beta > 0$ to be
$\cH^\beta_C(\cX) =
\big\{
g \in \cC^{\flbeta}(\cX) \! : \!
\max_{1 \leq r \leq \flbeta}
\big| g^{(r)}(x) \big| \leq C,
\big| g^{(\flbeta)}(x) - g^{(\flbeta)}(x') \big|
\leq C |x-x'|^{\beta - \flbeta},
\forall x, x' \in \cX
\big\}$,
where $\flbeta$ denotes the largest integer which is strictly less than $\beta$.
Note that $\cH^1_C(\cX)$ is the class of $C$-Lipschitz functions on $\cX$.
For $a \in \R$ and $b \geq 0$, we write $[a \pm b]$ for the interval
$[a-b, a+b]$. For non-negative sequences $a_n$ and $b_n$, write
$a_n \lesssim b_n$ or $a_n = O(b_n)$ to indicate that
$a_n / b_n$ is bounded for $n\geq 1$.
Write $a_n \ll b_n$ or $a_n = o(b_n)$ if $a_n / b_n \to 0$.
If $a_n \lesssim b_n \lesssim a_n$, write $a_n \asymp b_n$.
For random non-negative sequences $A_n$ and $B_n$, write
$A_n \lesssim_\P B_n$ or $A_n = O_\P(B_n)$ if
$A_n / B_n$ is bounded in probability.
Write $A_n = o_\P(B_n)$ if $A_n / B_n \to 0$ in probability.
For $a,b \in \R$, define $a\wedge b=\min\{a,b\}$ and $a \vee b = \max\{a,b\}$.

\section{Setup}\label{sec:kernel_setup}

We impose the following two assumptions throughout this chapter,
which concern firstly the dyadic data generating process, and
secondly the choice of kernel and bandwidth sequence.

%
\begin{assumption}[Data generation]
\label{ass:kernel_data}
%
% A and V variables
Let $\bA_n = (A_i: 1 \leq i \leq n)$ be i.i.d.\ random variables supported on
$\cA \subseteq \R$ and let $\bV_n = (V_{i j}: 1 \leq i < j \leq n)$ be
i.i.d.\ random variables with a Lebesgue density $f_V$ on $\R$, with $\bA_n$
independent of $\bV_n$.
%
% W variables
Let $W_{i j} = W(A_i, A_j, V_{i j})$ and
$\bW_n = (W_{i j}: 1 \leq i < j \leq n)$, where $W$ is an unknown real-valued
function which is symmetric in its first two arguments.
%
Let $\cW \subseteq \R$ be a compact interval with positive Lebesgue measure
$\Leb(\cW)$. The conditional distribution of $W_{i j}$ given $A_i$ and $A_j$
admits a Lebesgue density $f_{W \mid AA}(w \mid A_i, A_j)$.
For $C_\rH > 0$ and $\beta \geq 1$, take $f_W \in \cH^\beta_{C_\rH}(\cW)$
where $f_{W}(w) = \E\left[f_{W \mid AA}(w \mid A_i,A_j)\right]$ and
$f_{W \mid AA}(\cdot \mid a, a') \in \cH^1_{C_\rH}(\cW)$
for all $a,a' \in \cA$. Suppose
$\sup_{w \in \cW} \|f_{W \mid A}(w \mid \cdot\,)\|_\TV <\infty$ where
$f_{W \mid A}(w \mid a) = \E\left[f_{W \mid AA}(w \mid A_i,a)\right]$.
%
\end{assumption}

In Assumption~\ref{ass:kernel_data} we require the density $f_W$ be in a
$\beta$-smooth H\"older class of functions on the compact interval $\cW$.
H\"older classes are well established in the minimax estimation literature
\citep{stone1982optimal,gine2021mathematical},
with the smoothness parameter $\beta$ appearing
in the minimax-optimal rate of convergence. If the H\"older condition is
satisfied only piecewise, then our results remain valid provided that the
boundaries between the pieces are known and treated as boundary points.

If $W(a_1, a_2, v)$ is strictly monotonic and continuously differentiable in
its third argument, we can give the conditional density of $W_{i j}$ explicitly
using the usual change-of-variables formula: with $w=W(a_1,a_2,v)$, we have
$f_{W \mid AA}(w \mid a_1,a_2)
= f_V(v) \big|\partial W(a_1,a_2,v)/\partial v\big|^{-1}$.

\begin{assumption}[Kernels and bandwidth]
\label{ass:kernel_bandwidth}%
%
Let $h = h(n) > 0$ be a sequence of bandwidths satisfying $h \log n \to 0$
and $\frac{\log n}{n^2h} \to 0$. For each $w \in \cW$, let $k_h(\cdot, w)$ be
a real-valued function supported on $[w \pm h] \cap \cW$. For an integer
$p \geq 1$, let $k_h$ belong to a family of boundary bias-corrected kernels
of order $p$, i.e.,
%
\begin{align*}
\int_{\cW}
(s-w)^r k_h(s,w) \diff{s}
\quad
\begin{cases}
\begin{alignedat}{2}
&= 1 &\qquad &\text{for all } w \in \cW \text{ if }\, r = 0, \\
&= 0 & &\text{for all } w \in \cW \text{ if }\, 1 \leq r \leq p-1, \\
&\neq 0 & &\text{for some } w \in \cW \text{ if }\, r = p.
\end{alignedat}
\end{cases}
\end{align*}
%
Also, for $C_\rL > 0$,
suppose $k_h(s, \cdot) \in \cH^1_{C_\rL h^{-2}}(\cW)$
for all $s \in \cW$.
%
\end{assumption}

This assumption allows for all standard compactly supported and possibly
boundary-corrected kernel functions
\citep{wand1994kernel,simonoff1996smoothing}, constructed for example by taking
polynomials on a compact interval and solving a linear system for the
coefficients. Assumption~\ref{ass:kernel_bandwidth} implies
(see Lemma~\ref{lem:kernel_app_lipschitz_kernels_bounded}
in Appendix~\ref{app:kernel})
that if $h \leq 1$ then $k_h$ is uniformly bounded by
$C_\rk h^{-1}$ where $C_\rk \vcentcolon = 2 C_\rL + 1 + 1/\Leb(\cW)$.

\subsection{Bias characterization}
\label{sec:kernel_bias}

We begin by characterizing and bounding the bias
$B_n(w) = \E \big[ \hat f_W(w) \big] - f_W(w)$.
Theorem~\ref{thm:kernel_bias} is a standard result for the non-random smoothing
bias in kernel density estimation with higher-order kernels and boundary bias
correction, and does not rely on the dyadic structure.

\begin{theorem}[Bias bound]
\label{thm:kernel_bias}

Suppose that Assumptions \ref{ass:kernel_data} and \ref{ass:kernel_bandwidth}
hold. For $w \in \cW$ define the leading bias term as
%
\begin{align*}
b_p(w)
&=
\frac{f_W^{(p)}(w)}{p!}
\int_{\cW}
k_h(s,w)
\left(
\frac{s-w}{h}
\right)^p
\diff{s}.
\end{align*}
%
for $1 \leq p \leq \flbeta$. Then we have the following bias bounds.
%
\begin{enumerate}[label=(\roman*)]
\item If $p \leq \flbeta - 1$, then
$\sup_{w \in \cW} | B_n(w) - h^p b_p(w) |
\leq \frac{2 C_\rk C_\rH}{(p+1)!} h^{p+1}$.

\item If $p = \flbeta$, then
$\sup_{w \in \cW} | B_n(w) - h^p b_p(w) |
\leq \frac{2 C_\rk C_\rH}{\flbeta !} h^\beta$.

\item If $p \geq \flbeta+1$, then
$\sup_{w \in \cW} | B_n(w) |
\leq \frac{2 C_\rk C_\rH}{\flbeta !} h^\beta$.
\end{enumerate}
%
Noting that $\sup_{\cW} |b_p(w)| \leq 2 C_\rk C_\rH / p!$,
we deduce that for $h \leq 1$,
%
\begin{align*}
\sup_{w \in \cW} | B_n(w) |
\leq
\frac{4 C_\rk C_\rH}{(p \wedge \flbeta)!}
h^{p \wedge \beta}
\lesssim
h^{p \wedge \beta}.
\end{align*}

\end{theorem}

\subsection{Hoeffding-type decomposition and degeneracy}
\label{sec:kernel_degeneracy}

Our next step is to consider the stochastic part
$\hat f_W(w) - \E \big[ \hat f_W(w) \big]$
of the classical bias--variance decomposition. This term is akin to a
U-statistic and thus admits a Hoeffding-type decomposition, presented in
Lemma~\ref{lem:kernel_hoeffding}, which is a key element in our analysis.

\begin{lemma}[Hoeffding-type decomposition for $\hat f_W$]
\label{lem:kernel_hoeffding}

Suppose that Assumptions~\ref{ass:kernel_data} and~\ref{ass:kernel_bandwidth}
hold. Define the linear, quadratic, and error terms
%
\begin{align*}
L_n(w)
&=
\frac{2}{n} \sum_{i=1}^n l_i(w),
&Q_n(w) &= \frac{2}{n(n-1)} \sum_{i=1}^{n-1} \sum_{j=i+1}^{n} q_{i j}(w), \\
E_n(w) &= \frac{2}{n(n-1)} \sum_{i=1}^{n-1} \sum_{j=i+1}^{n} e_{i j}(w)
\end{align*}
%
respectively, where
%
\begin{align*}
l_i(w)
&=
\E\left[k_h(W_{i j},w) \mid A_i\right] - \E\left[k_h(W_{i j},w)\right], \\
q_{i j}(w)
&=
\E\left[k_h(W_{i j},w) \mid A_i, A_j\right]
- \E\left[k_h(W_{i j},w) \mid A_i\right]
- \E\left[k_h(W_{i j},w) \mid A_j\right]
+ \E\left[k_h(W_{i j},w)\right], \\
e_{i j}(w)
&=
k_h(W_{i j},w) - \E\left[k_h(W_{i j},w) \mid A_i, A_j\right].
\end{align*}
%
Then, recalling the bias term $B_n$ from Section~\ref{sec:kernel_bias},
we have the Hoeffding-type decomposition
%
\begin{align}
\label{eq:kernel_hoeffding}
\hat f_W(w) - f_W(w) = L_n(w) + Q_n(w) + E_n(w) + B_n(w).
\end{align}
%
The processes $L_n$, $Q_n$, and $E_n$ are mean-zero
with $\E\big[L_n(w)\big] = \E\big[Q_n(w)\big] = \E\big[E_n(w)\big] = 0$
for all $w \in \cW$. They are also orthogonal,
satisfying $\E\big[ L_n(w) Q_n(w') \big] = \E\big[ L_n(w) E_n(w') \big]
= \E\big[ Q_n(w) E_n(w') \big] = 0$ for all $w, w' \in \cW$.
%
\end{lemma}

The process $L_n$ is the H{\'a}jek projection of a U-process,
which can exhibit degeneracy if $\Var[L_n(w)] = 0$ at some
or all points $w \in \cW$. To characterize the different possible
degeneracy types in Lemma~\ref{lem:kernel_trichotomy},
we first introduce the following lower and upper degeneracy constants:
%
\begin{align*}
\Dl^2 := \inf_{w \in \cW} \Var\left[f_{W \mid A}(w \mid A_i)\right]
\qquad \text{ and } \qquad
\Du^2 := \sup_{w \in \cW} \Var\left[f_{W \mid A}(w \mid A_i)\right].
\end{align*}
%
\begin{lemma}[Trichotomy of degeneracy]%
\label{lem:kernel_trichotomy}%
%
Grant Assumptions~\ref{ass:kernel_data} and~\ref{ass:kernel_bandwidth}.
Then the type of degeneracy exhibited by $\hat f_W(w)$
is precisely one of the following three possibilities.
%
\begin{enumerate}[label=(\roman*)]

\item Total degeneracy:
$\Du = \Dl = 0$. Then $L_n(w) = 0$ for all $w \in \cW$ almost surely.

\item No degeneracy:
$\Dl > 0$. Then $\inf_{w \in \cW} \Var[L_n(w)] \geq \frac{2 \Dl}{n}$
for all large enough n.

\item Partial degeneracy:
$\Du > \Dl = 0$. There exists $w \in \cW$ with
$\Var\left[f_{W \mid A}(w \mid A_i)\right] = 0$;
such a point is labeled \emph{degenerate} and satisfies
$\Var[L_n(w)] \leq 64 C_\rk C_\rH C_\rd \frac{h}{n}$.
There is also a point $w' \in \cW$ with
$\Var\left[f_{W \mid A}(w' \mid A_i)\right] > 0$;
such a point is labeled \emph{non-degenerate} and satisfies
$\Var[L_n(w')] \geq
\frac{2}{n} \Var\left[f_{W \mid A}(w' \mid A_i)\right]$
for all large enough $n$.

\end{enumerate}

\end{lemma}

The following lemma describes the uniform stochastic order of the different
terms in the Hoeffding-type decomposition, explicitly accounting for potential
degeneracy.

\begin{lemma}[Uniform concentration]
\label{lem:kernel_uniform_concentration}

Suppose Assumptions \ref{ass:kernel_data} and
\ref{ass:kernel_bandwidth} hold. Then
%
\begin{align*}
\E\left[ \sup_{w \in \cW} |L_n(w)| \right]
&\lesssim \frac{\Du}{\sqrt n},
&\E\left[ \sup_{w \in \cW} |Q_n(w)| \right]
&\lesssim \frac{1}{n},
&\E\left[ \sup_{w \in \cW} |E_n(w)| \right]
&\lesssim \sqrt{\frac{\log n}{n^2h}}.
\end{align*}
\end{lemma}

Lemma~\ref{lem:kernel_uniform_concentration} captures the potential total
degeneracy
of $L_n$ by illustrating how if $\Du=0$ then $L_n=0$ everywhere on $\cW$ almost
surely. The following lemma captures the potential partial degeneracy of $L_n$,
where $\Du > \Dl = 0$. For $w,w' \in \cW$, define the covariance function
%
\begin{align*}
\Sigma_n(w,w') =
\E\Big[
\Big(
\hat f_W(w)
- \E\big[\hat f_W(w)\big]
\Big)
\Big(
\hat f_W(w')
- \E\big[\hat f_W(w')\big]
\Big)
\Big].
\end{align*}
%
\begin{lemma}[Variance bounds]
\label{lem:kernel_variance_bounds}
Suppose that Assumptions~\ref{ass:kernel_data} and~\ref{ass:kernel_bandwidth}
hold. Then for sufficiently large $n$,
%
\begin{align*}
\frac{\Dl^2}{n} + \frac{1}{n^2h}
\inf_{w \in \cW} f_W(w)
&\lesssim
\inf_{w \in \cW} \Sigma_n(w,w)
\leq
\sup_{w \in \cW} \Sigma_n(w,w)
\lesssim
\frac{\Du^2}{n} + \frac{1}{n^2h}.
\end{align*}
%
\end{lemma}

As a simple example of the different types of degeneracy, consider the family
of dyadic distributions $\P_{\pi}$ indexed by $\pi = (\pi_1, \pi_2, \pi_3)$
with $\sum_{i=1}^3 \pi_i = 1$ and $\pi_i \geq 0$, generated by
$W_{i j} = A_i A_j + V_{i j}$, where $A_i$ equals $-1$ with probability
$\pi_1$, equals $0$ with probability $\pi_2$ and equals $+1$ with probability
$\pi_3$, and $V_{i j}$ is standard Gaussian. This model induces a latent
``community structure'' where community membership is determined by the value
of $A_i$ for each node $i$, and the interaction outcome $W_{i j}$ is a function
only of the communities which $i$ and $j$ belong to and some idiosyncratic
noise. Unlike the stochastic block model \citep{kolaczyk2009statistical}, our
setup assumes that community membership has no impact on edge existence, as we
work with fully connected networks; see Section~\ref{sec:kernel_trade_data} for
a
discussion of how to handle missing edges in practice. Also note that the
parameter of interest in this chapter is the Lebesgue density of a continuous
random variable $W_{i j}$ rather than the probability of network edge
existence, which is the focus of the graphon estimation literature
\citep{gao2021minimax}.

In line with Assumption~\ref{ass:kernel_data}, $\bA_n$ and $\bV_n$ are i.i.d.\
sequences independent of each other. Then
$f_{W \mid AA}(w \mid A_i, A_j) = \phi(w - A_i A_j)$,\,
$f_{W \mid A}(w \mid A_i) = \pi_1 \phi(w + A_i) + \pi_2 \phi(w)
+ \pi_3 \phi(w - A_i)$, and
$f_W(w) = (\pi_1^2 + \pi_3^2) \phi(w-1) + \pi_2 (2 - \pi_2) \phi(w) + 2
\pi_1 \pi_3 \phi(w+1),$
where $\phi$ denotes the probability density function of the standard normal
distribution. Note that $f_W(w)$ is strictly positive for all $w \in \R$.
Consider the parameter choices:
%
\begin{enumerate}[label=(\roman*)]

\item $\pi = \left( \frac{1}{2}, 0, \frac{1}{2} \right)$:\quad
$\P_\pi$ is degenerate at all $w \in \R$,

\item $\pi = \left( \frac{1}{4}, 0, \frac{3}{4} \right)$:\quad
$\P_\pi$ is degenerate only at $w=0$,

\item $\pi = \left( \frac{1}{5}, \frac{1}{5}, \frac{3}{5} \right)$:\quad
$\P_\pi$ is non-degenerate for all $w \in \R$.

\end{enumerate}
%
Figure~\ref{fig:kernel_distribution} demonstrates these phenomena, plotting the
density $f_W$ and the standard deviation of the conditional
density $f_{W|A}$ over $\cW = [-2,2]$ for each choice of the parameter $\pi$.

The trichotomy of total/partial/no degeneracy is useful for understanding the
distributional properties of the dyadic kernel density estimator
$\hat{f}_W(w)$. Crucially, our need for uniformity in $w$ complicates the
simpler degeneracy/no degeneracy dichotomy observed previously in the
literature \citep{graham2024kernel}. From a pointwise-in-$w$
perspective, partial degeneracy causes no issues, while it is a fundamental
problem when conducting inference uniformly over $w \in \cW$. We develop
methods that are valid regardless of the presence of partial or total
degeneracy.

\begin{figure}[t]
\centering
%
\begin{subfigure}{0.32\textwidth}
\centering
%\includegraphics[scale=0.64]{graphics/distribution_plot_total.pdf}
\caption{Total degeneracy, \\
$\pi = \left( \frac{1}{2}, 0, \frac{1}{2} \right)$.}
\end{subfigure}
%
\begin{subfigure}{0.32\textwidth}
\centering
%\includegraphics[scale=0.64]{graphics/distribution_plot_partial.pdf}
\caption{Partial degeneracy, \\
$\pi = \left( \frac{1}{4}, 0, \frac{3}{4} \right)$.}
\end{subfigure}
%
\begin{subfigure}{0.32\textwidth}
\centering
%\includegraphics[scale=0.64]{graphics/distribution_plot_none.pdf}
\caption{No degeneracy, \\
$\pi = \left( \frac{1}{5}, \frac{1}{5}, \frac{3}{5} \right)$.}
\end{subfigure}
%
\caption[The family of distributions $\P_\pi$]{
Density $f_W$ and standard deviation
of $f_{W|A}$ for the family of distributions $\P_\pi$.}
%
\label{fig:kernel_distribution}
\end{figure}

\section{Point estimation results}
\label{sec:kernel_point_estimation}

Using the bias bound from Theorem~\ref{thm:kernel_bias} and
the concentration results from Lemma~\ref{lem:kernel_uniform_concentration},
the next theorem establishes an upper bound on the uniform convergence rate of
$\hat f_W$.
%
\begin{theorem}[Uniform convergence rate]%
\label{thm:kernel_uniform_consistency}%
Suppose that Assumptions \ref{ass:kernel_data} and
\ref{ass:kernel_bandwidth} hold. Then
%
\begin{align*}
\E\left[
\sup_{w \in \cW}
\big|\hat{f}_W(w) - f_W(w)\big|
\right]
\lesssim
h^{p\wedge\beta} + \frac{\Du}{\sqrt n} + \sqrt{\frac{\log n}{n^2h}}.
\end{align*}
\end{theorem}
%
The implicit constant in Theorem~\ref{thm:kernel_uniform_consistency} depends
only on
$\cW$, $\beta$, $C_\rH$, and the choice of kernel. We interpret this result in
light of the degeneracy trichotomy from Lemma~\ref{lem:kernel_trichotomy}.
These results generalize \citet*[Theorem~1]{chiang2020empirical}
by allowing for compactly supported data and more general kernels
$k_h(\cdot,w)$, enabling boundary-adaptive estimation.

%
\begin{enumerate}[label=(\roman*)]
\item Partial or no degeneracy: $\Du > 0$.
Any bandwidths satisfying
$n^{-1} \log n \lesssim h \lesssim n^{-\frac{1}{2(p\wedge\beta)}}$ yield
$\E\big[\sup_{w \in \cW}\big|\hat f_W(w)
- f_W(w)\big| \big] \lesssim \frac{1}{\sqrt n}$, the ``parametric''
bandwidth-independent rate noted by \citet{graham2024kernel}.

\item Total degeneracy: $\Du = 0$.
Minimizing the bound in Theorem~\ref{thm:kernel_uniform_consistency} with
$h \asymp \left( \frac{\log n}{n^2} \right)^{\frac{1}{2(p\wedge\beta)+1}}$
yields $\E\big[ \sup_{w \in \cW} \big|\hat f_W(w) - f_W(w)\big| \big]
\lesssim
\big(\frac{\log n}{n^2} \big)^{\frac{p\wedge\beta}{2(p\wedge\beta)+1}}$.
\end{enumerate}

\subsection{Minimax optimality}

We establish the minimax rate under the supremum norm for density estimation
with dyadic data. This implies minimax optimality of the kernel density
estimator $\hat f_W$, regardless of the degeneracy type of the dyadic
distribution.

\begin{theorem}[Uniform minimax optimality]
\label{thm:kernel_minimax}

Fix $\beta \geq 1$ and $C_\rH > 0$, and take $\cW$ a compact interval with
positive Lebesgue measure. Define $\cP = \cP(\cW, \beta, C_\rH)$ as the class
of dyadic distributions satisfying Assumption~\ref{ass:kernel_data}. Define
$\cP_\rd$ as the subclass of $\cP$ containing only those distributions which
are totally degenerate on $\cW$ in the sense that
$\sup_{w \in \cW} \Var\left[f_{W \mid A}(w \mid A_i)\right] = 0$. Then
%
\begin{align*}
\inf_{\tilde f_W}
\sup_{\P \in \cP}
\E_\P\left[
\sup_{w \in \cW}
\big| \tilde f_W(w) - f_W(w) \big|
\right]
&\asymp
\frac{1}{\sqrt n}, \\
\inf_{\tilde f_W}
\sup_{\P \in \cP_\rd}
\E_\P\left[
\sup_{w \in \cW}
\big| \tilde f_W(w) - f_W(w) \big|
\right]
&\asymp
\left( \frac{\log n}{n^2} \right)^{\frac{\beta}{2\beta+1}},
\end{align*}
%
where $\tilde f_W$ is any estimator depending only on the data
$\bW_n = (W_{i j}: 1 \leq i < j \leq n)$ distributed according to the dyadic
law $\P$. The constants in $\asymp$ depend only on
$\cW$, $\beta$, and $C_\rH$.

\end{theorem}

Theorem~\ref{thm:kernel_minimax} shows that the uniform convergence rate of
$n^{-1/2}$ obtained in Theorem~\ref{thm:kernel_uniform_consistency}
(coming from the $L_n$ term) is minimax-optimal in general.
When attention is restricted to totally degenerate dyadic distributions,
$\hat f_W$ also achieves the minimax rate of uniform convergence
(assuming a kernel of sufficiently high order $p \geq \beta$),
which is on the order of
$\left(\frac{\log n}{n^2}\right)^{\frac{\beta}{2\beta+1}}$ and
is determined by the bias $B_n$ and the leading variance term $E_n$ in
\eqref{eq:kernel_hoeffding}.

Combining Theorems
\ref{thm:kernel_uniform_consistency}~and~\ref{thm:kernel_minimax},
we conclude that $\hat{f}_W(w)$ achieves the minimax-optimal rate for uniformly
estimating $f_W(w)$ if $h \asymp \left( \frac{\log n}{n^2}
\right)^{\frac{1}{2\beta+1}}$ and a kernel of sufficiently high order
($p \geq \beta$) is used, whether or not there are any degenerate points in the
underlying data generating process. This result appears to be new to the
literature on nonparametric estimation with dyadic data. See
\citet{gao2021minimax} for a contemporaneous review.

\section{Distributional results}
\label{sec:kernel_inference}

We investigate the distributional properties of the
standardized $t$-statistic process
%
\begin{align*}
T_n(w) = \frac{\hat{f}_W(w) - f_W(w)}{\sqrt{\Sigma_n(w,w)}},
\end{align*}
%
which is not necessarily asymptotically tight. Therefore, to approximate the
distribution of the entire $t$-statistic process, as well as specific
functionals thereof, we rely on a novel strong approximation approach outlined
in this section. Our results can be used to perform valid uniform inference
irrespective of the degeneracy type.

This section is largely concerned with distributional properties and thus
frequently requires copies of stochastic processes. For succinctness of
notation, we will not differentiate between a process and its copy, but details
are available in Section~\ref{sec:kernel_app_technical}.

\subsection{Strong approximation}

By the Hoeffding-type decomposition \eqref{eq:kernel_hoeffding} and
Lemma~\ref{lem:kernel_uniform_concentration}, it suffices to consider the
distributional properties of the stochastic process $L_n + E_n$.
Our approach combines the K{\'o}mlos--Major--Tusn{\'a}dy (KMT) approximation
\citep{komlos1975approximation} to obtain a strong approximation of $L_n$ with
a Yurinskii approximation \citep{yurinskii1978error} to obtain a
\emph{conditional} (on $\bA_n$) strong approximation of $E_n$. The latter is
necessary because $E_n$ is akin to a local empirical process of i.n.i.d.\
random variables, conditional on $\bA_n$, and therefore the KMT approximation
is not applicable. These approximations are then combined to give a final
(unconditional) strong approximation for $L_n+E_n$, and thus for the
$t$-statistic process $T_n$.

The following lemma is an application of our generic KMT approximation result
for empirical processes, given in Section~\ref{sec:kernel_app_technical}, which
builds on earlier work by \citet{gine2004kernel} and \citet{gine2010confidence}
and may be of independent interest.

\begin{lemma}[Strong approximation of $L_n$]
\label{lem:kernel_strong_approx_Ln}
%
Suppose that Assumptions \ref{ass:kernel_data}~and~\ref{ass:kernel_bandwidth}
hold. For each $n$ there exists a mean-zero Gaussian process $Z^L_n$ indexed
on $\cW$ satisfying
$\E\big[ \sup_{w \in \cW} \big| \sqrt{n} L_n(w) - Z_n^L(w) \big| \big]
\lesssim \frac{\Du \log n}{\sqrt{n}}$, where
$\E[Z_n^L(w)Z_n^L(w')] = n\E[L_n(w)L_n(w')]$ for all $w, w' \in \cW$. The
process $Z_n^L$ is a function only of $\bA_n$ and some random noise
independent of $(\bA_n, \bV_n)$.
\end{lemma}

% donsker case
The strong approximation result in Lemma~\ref{lem:kernel_strong_approx_Ln}
would be
sufficient to develop valid and even optimal uniform inference procedures
whenever both $\Dl > 0$ (no degeneracy in $L_n$) and $n h \gg \log n$
($L_n$ is leading). In this special case, the recent Donsker-type results of
\citet{davezies2021exchangeable} can be applied to analyze the limiting
distribution of the stochastic process $\hat{f}_W$. Alternatively, again only
when $L_n$ is non-degenerate and leading, standard empirical process methods
could also be used. However, even in the special case when $\hat{f}_W(w)$ is
asymptotically Donsker, our result in Lemma~\ref{lem:kernel_strong_approx_Ln}
improves
upon the literature by providing a rate-optimal strong approximation for
$\hat{f}_W$ as opposed to only a weak convergence result. See Theorem
\ref{thm:kernel_infeasible_ucb} and the subsequent discussion below.

% however often non-donsker
More importantly, as illustrated above, it is common in the literature to find
dyadic distributions which exhibit partial or total degeneracy, making the
process $\hat{f}_W$ non-Donsker. Thus approximating only $L_n$ is in general
insufficient for valid uniform inference, and it is necessary to capture the
distributional properties of $E_n$ as well.
% we do better
The following lemma is an application of our strong approximation result for
empirical processes based on the Yurinskii approximation, which builds on a
refinement by \citet{belloni2019conditional}.

\begin{lemma}[Conditional strong approximation of $E_n$]
\label{lem:kernel_conditional_strong_approx_En}
%
Suppose Assumptions \ref{ass:kernel_data}~and~\ref{ass:kernel_bandwidth} hold
and take any $R_n \to \infty$. For each $n$ there exists $\tilde Z^E_n$
a mean-zero Gaussian process conditional on $\bA_n$ satisfying
$\sup_{w \in \cW}
\big| \sqrt{n^2h} E_n(w) - \tilde Z_n^E(w) \big|
\lesssim_\P \frac{(\log n)^{3/8} R_n}{n^{1/4}h^{3/8}}$,
where $\E[\tilde Z_n^E(w)\tilde Z_n^E(w')\bigm\vert \bA_n]
=n^2h\E[E_n(w)E_n(w')\bigm\vert \bA_n]$
for all $w, w' \in \cW$.
%
\end{lemma}

The process $\tilde Z_n^E$ is a Gaussian process conditional on $\bA_n$ but is
not in general a Gaussian process unconditionally. The following lemma
constructs an unconditional Gaussian process $Z_n^E$ that approximates
$\tilde Z_n^E$.

\begin{lemma}[Unconditional strong approximation of $E_n$]
\label{lem:kernel_unconditional_strong_approx_En}

Suppose that Assumptions \ref{ass:kernel_data} and
\ref{ass:kernel_bandwidth} hold. For each $n$ there exists
a mean-zero Gaussian process $Z^E_n$ satisfying
$\E\big[ \sup_{w \in \cW} \big|\tilde Z_n^E(w) - Z_n^E(w)\big| \big]
\lesssim \frac{(\log n)^{2/3}}{n^{1/6}}$,
where $Z_n^E$ is independent of $\bA_n$ and
$\E[Z_n^E(w)Z_n^E(w')]=\E[\tilde Z_n^E(w)\tilde Z_n^E(w')]
= n^2h \, \E[E_n(w)E_n(w')]$ for all $w, w' \in \cW$.
%
\end{lemma}

Combining Lemmas \ref{lem:kernel_conditional_strong_approx_En}
and~\ref{lem:kernel_unconditional_strong_approx_En}, we obtain
an unconditional strong
approximation for $E_n$. The resulting rate of approximation may not be
optimal, due to the Yurinskii coupling, but to the best of our knowledge it is
the first in the literature for the process $E_n$, and hence for $\hat{f}_W$
and its associated $t$-process in the context of dyadic data. The approximation
rate is sufficiently fast to allow for optimal bandwidth choices; see Section
\ref{sec:kernel_implementation} for more details. Strong approximation results
for
local empirical processes (e.g.\ \citealp{gine2010confidence}) are not
applicable here because the summands in the non-negligible $E_n$ are not
(conditionally) i.i.d. Likewise, neither standard empirical process and
U-process theory \citep{van1996weak,gine2021mathematical} nor the recent
results in \citet{davezies2021exchangeable} are applicable to the non-Donsker
process $E_n$.

The previous lemmas showed that $L_n$ is $\sqrt{n}$-consistent while $E_n$ is
$\sqrt{n^2h}$-consistent (pointwise in $w$), showcasing the importance of
careful standardization (cf.\ Studentization in
Section~\ref{sec:kernel_implementation}) for the purpose of rate adaptivity to
the
unknown degeneracy type. In other words, a challenge in conducting uniform
inference is that the finite-dimensional distributions of the stochastic
process $L_n+E_n$, and hence those of $\hat{f}_W$ and its associated
$t$-process $T_n$, may converge at different rates at different points
$w\in\cW$. The following theorem provides an (infeasible) inference procedure
which is fully adaptive to such potential unknown degeneracy.

\begin{theorem}[Strong approximation of $T_n$]
\label{thm:kernel_strong_approx_Tn}

Suppose that Assumptions~\ref{ass:kernel_data} and \ref{ass:kernel_bandwidth}
hold and $f_W(w) > 0$ on $\cW$, and take any $R_n \to \infty$. Then for each
$n$ there exists a centered Gaussian process $Z_n^{T}$ such that
%
\begin{align*}
&\sup_{w \in \cW} \left| T_n(w) - Z_n^{T}(w) \right|
\lesssim_\P \!
\frac{
n^{-1} \! \log n
+ n^{-5/4} h^{-7/8} (\log n)^{3/8} R_n
+ n^{-7/6} h^{-1/2} (\log n)^{2/3}
+ h^{p\wedge\beta}}
{\Dl/\sqrt{n} + 1/\sqrt{n^2h}},
\end{align*}
%
where $\E[Z_n^T(w)Z_n^T(w')] = \E[T_n(w)T_n(w')]$ for all $w,w' \in \cW$.
%
\end{theorem}

The first term in the numerator corresponds to the strong approximation for
$L_n$ in Lemma~\ref{lem:kernel_strong_approx_Ln} and the error introduced by
$Q_n$.
The second and third terms correspond to the conditional and unconditional
strong approximation errors for $E_n$ in Lemmas
\ref{lem:kernel_conditional_strong_approx_En} and
\ref{lem:kernel_unconditional_strong_approx_En}.
The fourth term is from
the smoothing bias result in Theorem~\ref{thm:kernel_bias}. The denominator is
the lower bound on the standard deviation $\Sigma_n(w,w)^{1/2}$ formulated in
Lemma~\ref{lem:kernel_variance_bounds}.

In the absence of degenerate points ($\Dl > 0$) and if $n h^{7/2}\gtrsim 1$,
Theorem~\ref{thm:kernel_strong_approx_Tn} offers a strong approximation of the
$t$-process at the rate $(\log n)/\sqrt{n}+\sqrt{n}h^{p\wedge\beta}$, which
matches the celebrated KMT approximation rate for i.i.d.\ data plus the
smoothing bias. Therefore, our novel $t$-process strong approximation can
achieve the optimal KMT rate for non-degenerate dyadic distributions provided
that $p\wedge\beta \geq 3.5$. This is achievable if a fourth-order
(boundary-adaptive) kernel is used and $f_W$ is sufficiently smooth.

In the presence of partial or total degeneracy ($\Dl =0$),
Theorem~\ref{thm:kernel_strong_approx_Tn} provides a strong approximation for
the
$t$-process at the rate
$\sqrt{h}\log n + n^{-1/4}h^{-3/8}(\log n)^{3/8} R_n + n^{-1/6}(\log n)^{2/3}
+ n h^{1/2+p\wedge\beta}$. If, for example, $n h^{p\wedge\beta}\lesssim 1$,
then our result can achieve a strong approximation rate of $n^{-1/7}$ up to
$\log n $ terms. Theorem~\ref{thm:kernel_strong_approx_Tn} appears to be the
first in the dyadic literature which is also robust to the presence of
degenerate points in the underlying dyadic distribution.

\subsection{Application: confidence bands}

Theorem~\ref{thm:kernel_infeasible_ucb} constructs standardized confidence
bands for
$f_W$ which are infeasible as they depend on the unknown population variance
$\Sigma_n$. In Section~\ref{sec:kernel_implementation} we will make this
inference
procedure feasible by proposing a valid estimator of the covariance function
$\Sigma_n$ for Studentization, as well as developing bandwidth selection and
robust bias correction methods. Before presenting our result on valid
infeasible uniform confidence bands, we first impose in
Assumption~\ref{ass:kernel_rates} some extra restrictions on the bandwidth
sequence,
which depend on the degeneracy type of the dyadic distribution, to ensure the
coverage rate converges.

\begin{assumption}[Rate restriction for uniform confidence bands]
\label{ass:kernel_rates}
Assume that one of the following holds:
%
\begin{enumerate}[label=(\roman*)]

\item
\label{it:kernel_rate_non}
No degeneracy ($\Dl > 0$):
$n^{-6/7} \log n \ll h \ll (n \log n)^{-\frac{1}{2(p \wedge \beta)}}$,

\item
\label{it:kernel_rate_degen}
Partial or total degeneracy ($\Dl = 0$):
$n^{-2/3} (\log n)^{7/3} \ll h
\ll (n^2 \log n)^{-\frac{1}{2(p \wedge \beta) + 1}}$.
\end{enumerate}
\end{assumption}

We now construct the infeasible uniform confidence bands.
For $\alpha \in (0,1)$, let $q_{1-\alpha}$ be the quantile satisfying
$ \P\left(\sup_{w \in \cW} \left| Z_n^T(w) \right| \leq q_{1-\alpha} \right)
= 1 - \alpha$.
The following result employs the anti-concentration idea due to
\citet{chernozhukov2014anti} to deduce valid standardized confidence bands,
where we approximate the quantile of the unknown finite sample distribution of
$\sup_{w\in\cW} |T_n(w)|$ by the quantile $q_{1-\alpha}$ of
$\sup_{w\in\cW}|Z_n^T(w)|$. This approach offers a better rate of convergence
than relying on extreme value theory for the distributional approximation,
hence improving the finite sample performance of the proposed confidence bands.

\begin{theorem}[Infeasible uniform confidence bands]
\label{thm:kernel_infeasible_ucb}

Suppose that Assumptions~\ref{ass:kernel_data},~\ref{ass:kernel_bandwidth},
and~\ref{ass:kernel_rates} hold and $f_W(w) > 0$ on $\cW$. Then
%
\begin{align*}
\P\left(
f_W(w) \in
\left[ \hat f_W(w) \pm q_{1-\alpha} \sqrt{\Sigma_n(w,w)} \, \right]
\, \textup{for all } w \in \cW
\right)
\to 1 - \alpha.
\end{align*}
%
\end{theorem}

By Theorem~\ref{thm:kernel_uniform_consistency}, the asymptotically optimal
choice of
bandwidth for uniform convergence is
$h \asymp ((\log n)/n^2)^{\frac{1}{2(p \wedge \beta)+1}}$.
As discussed in the next section, the approximate
IMSE-optimal bandwidth is $h \asymp (1/n^2)^{\frac{1}{2(p \wedge \beta)+1}}$.
Both bandwidth choices satisfy Assumption~\ref{ass:kernel_rates} only in the
case of
no degeneracy. The degenerate cases in
Assumption~\ref{ass:kernel_rates}\ref{it:kernel_rate_degen}, which require
$p \wedge \beta > 1$, exhibit behavior more similar to that of standard
nonparametric kernel-based estimation and so the aforementioned optimal
bandwidth choices will lead to a non-negligible smoothing bias in the
distributional approximation for $T_n$. Different approaches are available in
the literature to address this issue, including undersmoothing or ignoring the
bias \citep{hall2001bootstrapping}, bias correction \citep{hall1992effect},
robust bias correction \citep{calonico2018effect, calonico2022coverage}, and
Lepskii's method
\citep{lepskii1992asymptotically,birge2001alternative}, among others. In the
next section we develop a feasible uniform inference procedure, based on robust
bias correction methods, which amounts to first selecting an optimal bandwidth
for the point estimator $\hat{f}_W$ using a $p$th-order kernel, and then
correcting the bias of the point estimator while also adjusting the
standardization (Studentization) when forming the $t$-statistic $T_n$.

Importantly, regardless of the specific implementation details,
Theorem~\ref{thm:kernel_infeasible_ucb} shows that any bandwidth sequence $h$
satisfying both \ref{it:kernel_rate_non} and \ref{it:kernel_rate_degen}
in Assumption~\ref{ass:kernel_rates} leads to valid uniform inference which is
robust
and adaptive to the (unknown) degeneracy type.

\section{Implementation}
\label{sec:kernel_implementation}

We address outstanding implementation details to make our main uniform
inference results feasible. In Section~\ref{sec:kernel_covariance_estimation} we
propose a covariance estimator along with a modified version which is
guaranteed to be positive semi-definite. This allows for the construction of
fully feasible confidence bands in
Section~\ref{sec:kernel_feasible_confidence_bands}.
In Section~\ref{sec:kernel_bandwidth_selection} we discuss bandwidth selection
and
formalize our procedure for robust bias correction inference.

\subsection{Covariance function estimation}
\label{sec:kernel_covariance_estimation}

Define the following plug-in covariance function
estimator of $\Sigma_n$. For $w, w' \in \cW$,
let $S_i(w) = \frac{1}{n-1} \big( \sum_{j = 1}^{i-1} k_h(W_{j i}, w)
+ \sum_{j = i+1}^n k_h(W_{i j}, w) \big)$
estimate $\E[k_h(W_{i j},w) \mid A_i]$ and take
%
\begin{align*}
\hat \Sigma_n(w,w')
&= \frac{4}{n^2} \sum_{i=1}^n S_i(w) S_i(w')
- \frac{4}{n^2(n-1)^2} \sum_{i<j} k_h(W_{i j},w) k_h(W_{i j},w') \\
&\quad- \frac{4n-6}{n(n-1)} \hat f_W(w) \hat f_W(w').
\end{align*}
%
Though $\hat\Sigma_n(w,w')$ is consistent in an appropriate sense as shown in
Lemma~\ref{lem:kernel_sdp}, it is not necessarily positive semi-definite, even
in the limit. We therefore propose a modified covariance estimator which is
guaranteed to be positive semi-definite. Specifically, consider the following
optimization problem where $C_\rk$ and $C_\rL$ are as in
Section~\ref{sec:kernel_setup}.
%
\begin{equation}
\label{eq:kernel_sdp}
\begin{aligned}
\minimize
\qquad
& \sup_{w,w' \in \cW}
\left|
\frac{M(w,w') - \hat\Sigma_n(w,w')}
{\sqrt{\hat \Sigma_n(w,w) + \hat \Sigma_n(w',w')}}
\right|
\quad \textup{ over } M: \cW \times \cW \to \R
\\
\subjectto
\qquad
& M \textup{ is symmetric and positive semi-definite}, \\
& \big|M(w,w') - M(w, w'')\big|
\leq \frac{4}{n h^3}
C_\rk C_\rL
|w'-w''|
\textup{ for all }
w, w', w'' \in \cW.
\end{aligned}
\end{equation}

Denote by $\hat\Sigma_n^+$ any (approximately) optimal solution to
\eqref{eq:kernel_sdp}. The following lemma establishes uniform convergence rates
for both $\hat \Sigma_n$ and $\hat \Sigma_n^+$.
We then use $\hat \Sigma_n^+$ to construct feasible versions of $T_n$ and its
associated Gaussian approximation $Z_n^{T}$ defined in
Theorem~\ref{thm:kernel_strong_approx_Tn}.
%
\begin{lemma}[Consistency of $\hat \Sigma_n$ and $\hat \Sigma_n^+$]
\label{lem:kernel_sdp}
Suppose Assumptions~\ref{ass:kernel_data} and~\ref{ass:kernel_bandwidth} hold
and that $n h \gtrsim \log n$ and $f_W(w) > 0$ on $\cW$. Then
%
\begin{align*}
\sup_{w,w' \in \cW}
\left| \frac{\hat \Sigma_n(w,w') - \Sigma_n(w,w')}
{\sqrt{\Sigma_n(w,w) + \Sigma_n(w',w')}} \right|
&\lesssim_\P \frac{\sqrt{\log n}}{n}.
\end{align*}
%
The optimization problem \eqref{eq:kernel_sdp} is a semi-definite
program
\citep[SDP,][]{laurent2005semidefinite} and has an approximately optimal
solution $\hat\Sigma_n^+$ satisfying
%
\begin{align*}
\sup_{w,w' \in \cW} \left|
\frac{\hat \Sigma_n^+(w,w') - \Sigma_n(w,w')}
{\sqrt{\Sigma_n(w,w) + \Sigma_n(w',w')}} \right|
&\lesssim_\P \frac{\sqrt{\log n}}{n}.
\end{align*}
%
\end{lemma}

In practice we take $w, w' \in \cW_d$ where $\cW_d$ is a finite subset of
$\cW$, typically taken to be an equally-spaced grid. This yields
finite-dimensional covariance matrices, for which \eqref{eq:kernel_sdp} can be
solved
in polynomial time in $|\cW_d|$ using a general-purpose SDP solver
\citep[e.g.\ by interior point methods,][]{laurent2005semidefinite}.
The number of points in $\cW_d$ should be taken as large as is computationally
practical in order to generate confidence bands rather than merely simultaneous
confidence intervals. It is worth noting that the complexity of solving
\eqref{eq:kernel_sdp} does not depend on the number of vertices $n$, and so
does not
influence the ability of our methodology to handle large and possibly sparse
networks.

The bias-corrected variance estimator in
\citet[Section~3.2]{matsushita2021jackknife} takes a similar form to our
estimator
$\hat\Sigma_n$ but in the parametric setting, and is therefore also not
guaranteed to be positive semi-definite in finite samples. Our approach
addresses this issue, ensuring a positive semi-definite estimator
$\hat\Sigma_n^+$ is always available.

\subsection{Feasible confidence bands}
\label{sec:kernel_feasible_confidence_bands}

Given a choice of the kernel order $p$ and a bandwidth $h$, we construct a
valid confidence band that is implementable in practice. Define the Studentized
$t$-statistic process
%
\begin{align*}
\hat T_n(w) = \frac{\hat{f}_W(w) - f_W(w)}{\sqrt{\hat \Sigma_n^+(w,w)}}.
\end{align*}
%
Let $\hat Z_n^T(w)$ be a process which, conditional on the data $\bW_n$,
is mean-zero and Gaussian, whose conditional covariance structure is
$\E\big[ \hat Z_n^T(w) \hat Z_n^T(w') \bigm\vert \bW_n \big]
= \frac{\hat \Sigma_n^+(w,w')}
{\sqrt{\hat \Sigma_n^+(w,w) \hat \Sigma_n^+(w',w')}}$.
For $\alpha \in (0,1)$, let $\hat q_{1-\alpha}$ be the
conditional quantile satisfying
$\P\big(\sup_{w \in \cW} \big| \hat Z_n^T(w) \big| \leq \hat q_{1-\alpha}
\bigm\vert \bW_n \big) = 1 - \alpha$,
which is shown to be well defined in Section~\ref{sec:kernel_app_proofs}.

\begin{theorem}[Feasible uniform confidence bands]
\label{thm:kernel_ucb}

Suppose that Assumptions \ref{ass:kernel_data}, \ref{ass:kernel_bandwidth},
and \ref{ass:kernel_rates} hold and $f_W(w) > 0$ on $\cW$. Then
%
\begin{align*}
\P\left(
f_W(w) \in
\left[ \hat f_W(w) \pm \hat q_{1-\alpha}
\sqrt{\hat\Sigma_n^+(w,w)} \,\right]
\,\textup{for all } w \in \cW
\right) \to 1 - \alpha.
\end{align*}
%
\end{theorem}

Recently, \citet{chiang2022inference} derived high-dimensional central limit
theorems over rectangles for exchangeable arrays and applied them to construct
simultaneous confidence intervals for a sequence of design points. Their
inference procedure relies on the multiplier bootstrap, and their conditions
for valid inference depend on the number of design points considered. In
contrast, Theorem~\ref{thm:kernel_ucb} constructs a feasible uniform confidence
band over the entire domain of inference $\cW$ based on our strong
approximation results for the whole $t$-statistic process and the covariance
estimator $\hat\Sigma_n^+$. The required rate condition specified in
Assumption~\ref{ass:kernel_rates} does not depend on the number of design
points.
Furthermore, our proposed inference methods are robust to potential unknown
degenerate points in the underlying dyadic data generating process.

In practice, suprema over $\cW$ can be replaced by maxima over sufficiently
many design points in $\cW$. The conditional quantile $\hat q_{1-\alpha}$ can
be estimated by Monte Carlo simulation, resampling from the Gaussian process
defined by the law of $\hat Z_n^T \mid \bW_n$.

The bandwidth restrictions in Theorem~\ref{thm:kernel_ucb} are the same as
those for the infeasible version given in
Theorem~\ref{thm:kernel_infeasible_ucb},
namely those imposed in Assumption \ref{ass:kernel_rates}. This follows from
the rates
of convergence obtained in Lemma~\ref{lem:kernel_sdp}, coupled with some careful
technical work given in Section~\ref{sec:kernel_app_proofs} to handle the
potential
presence of degenerate points in $\Sigma_n$.

\subsection{Bandwidth selection and robust bias-corrected inference}
\label{sec:kernel_bandwidth_selection}

We give practical suggestions for selecting the bandwidth parameter $h$.
Let $\nu(w)$ be a non-negative real-valued function on $\cW$ and suppose we use
a kernel of order $p < \beta$ of the form $k_h(s,w) = K\big((s-w) / h\big)/h$.
The $\nu$-weighted asymptotic IMSE (AIMSE) is minimized by
%
\begin{align*}
h^*_{\AIMSE}
&=
\left(
\frac{p!(p-1)!
\Big(\int_\cW f_W(w) \nu(w) \diff{w}\Big)
\Big(\int_\R K(w)^2 \diff{w}\Big)}
{2 \Big(
\int_{\cW}
f_W^{(p)}(w)^2
\nu(w)
\diff{w}
\Big)
\Big(
\int_\R
w^p K(w)
\diff{w}
\Big)^2
}
\right)^{\frac{1}{2p+1}}
\left( \frac{n(n-1)}{2} \right)^{-\frac{1}{2p+1}}.
\end{align*}
%
This is akin to the AIMSE-optimal bandwidth choice for traditional monadic
kernel density estimation with a sample size of $\frac{1}{2}n(n-1)$. The choice
$h^*_{\AIMSE}$ is slightly undersmoothed (up to a polynomial $\log n$ factor)
relative to the uniform minimax-optimal bandwidth choice discussed in
Section~\ref{sec:kernel_point_estimation}, but it is easier to implement in
practice.

To implement the AIMSE-optimal bandwidth choice, we propose a simple
rule-of-thumb (ROT) approach based on Silverman's rule.
Suppose $p\wedge\beta=2$ and let $\hat\sigma^2$ and $\hat I$
be the sample variance and sample interquartile range respectively
of the data $\bW_n$. Then
$\hat{h}_{\ROT} = C(K) \big( \hat\sigma \wedge
\frac{\hat I}{1.349} \big) \big(\frac{n(n-1)}{2} \big)^{-1/5}$,
where we have $C(K)=2.576$ for the triangular kernel $K(w) = (1 - |w|) \vee 0$,
and $C(K)=2.435$ for the Epanechnikov kernel
$K(w) = \frac{3}{4}(1 - w^2) \vee 0$.

The AIMSE-optimal bandwidth selector $h^*_{\AIMSE}\asymp n^{-\frac{2}{2p+1}}$
and any of its feasible estimators only satisfy
Assumption~\ref{ass:kernel_rates} in
the case of no degeneracy ($\Dl>0$). Under partial or total degeneracy, such
bandwidths are not valid due to the usual leading smoothing (or
misspecification) bias of the distributional approximation. To circumvent this
problem and construct feasible uniform confidence bands for $f_W$, we employ
the following robust bias correction approach.

\begin{algorithm}[b!]
\caption{Feasible uniform confidence bands}
\label{alg:kernel_method}
\setstretch{1.5}

Choose a kernel $k_h$ of order $p \geq 2$ satisfying
Assumption~\ref{ass:kernel_bandwidth}. \\

Select a bandwidth $h \approx h^*_{\AIMSE}$ for $k_h$
as in Section~\ref{sec:kernel_bandwidth_selection},
perhaps using $h = \hat{h}_{\ROT}$. \\

Choose another kernel $k_h'$ of order $p'>p$ satisfying
Assumption~\ref{ass:kernel_bandwidth}.

For $d \geq 1$, choose a set of $d$ distinct evaluation points $\cW_d$. \\

For each $w \in \cW_d$, construct the density estimate $\hat f_W(w)$
using $k'_{h}$ as in Section~\ref{sec:kernel_introduction}. \\

For $w, w' \in \cW_d$, estimate the covariance $\hat \Sigma_n(w,w')$
using $k'_{h}$ as in Section~\ref{sec:kernel_covariance_estimation}. \\

Construct positive semi-definite
covariance estimate $\hat \Sigma_n^+$
as in Section~\ref{sec:kernel_covariance_estimation}. \\

For $B \geq 1$, let $(\hat Z_{n,r}^T: 1\leq r\leq B)$ be i.i.d.\
from $\hat{Z}_n^T$ as in Section~\ref{sec:kernel_feasible_confidence_bands}.
\\

For $\alpha \in (0,1)$, set
$\hat q_{1-\alpha} = \inf_{q \in \R}
\{ q : \# \{r: \max_{w\in\cW_d}|\hat Z_{n,r}^T(w)| \leq q \}
\geq B(1-\alpha) \}$. \\

Construct $ \big[\hat f_W(w) \pm
\hat q_{1-\alpha} \hat\Sigma_n^+(w,w)^{1/2} \big]$ for each $w \in \cW_d$.
%
\end{algorithm}

Firstly, estimate the bandwidth $h^*_{\AIMSE}\asymp n^{-\frac{2}{2p+1}}$ using a
kernel of order $p$, which leads to an AIMSE-optimal point estimator
$\hat{f}_W$ in an $L^2(\nu)$ sense. Then use this bandwidth and a kernel of
order $p' > p$ to construct the statistic $\hat T_n$ and the confidence band as
detailed in Section~\ref{sec:kernel_feasible_confidence_bands}. Importantly,
both
$\hat{f}_W$ and $\hat{\Sigma}^+_n$ are recomputed with the new higher-order
kernel. The change in centering is equivalent to a bias correction of the
original AIMSE-optimal point estimator, while the change in scale captures the
additional variability introduced by the bias correction itself. As shown
formally in \citet{calonico2018effect, calonico2022coverage} for the case of
kernel-based density
estimation with i.i.d.\ data, this approach leads to higher-order refinements
in the distributional approximation whenever additional smoothness is available
($p'\leq\beta$). In the present dyadic setting, this procedure is valid so long
as $n^{-2/3} (\log n)^{7/3} \ll n^{-\frac{2}{2p+1}}
\ll (n^2 \log n)^{-\frac{1}{2p' + 1}}$,
which is equivalent to $2 \leq p < p'$.
For concreteness, we recommend taking $p = 2$ and $p' = 4$,
and using the rule-of-thumb bandwidth choice $\hat{h}_{\ROT}$ defined above.
In particular, this approach automatically delivers a KMT-optimal
strong approximation whenever there are no degeneracies in the
underlying dyadic data generating process.
Our feasible robust bias correction method based on AIMSE-optimal dyadic
kernel density estimation for constructing uniform confidence bands
for $f_W$ is summarized in Algorithm~\ref{alg:kernel_method}.

\section{Simulations}
\label{sec:kernel_simulations}

We investigate the empirical finite-sample performance of the kernel density
estimator with dyadic data using simulations. The family of dyadic
distributions defined in Section~\ref{sec:kernel_degeneracy}, with its three
parameterizations, is used to generate data sets with different degeneracy
types.

We use two different boundary bias-corrected Epanechnikov kernels of orders
$p=2$ and $p=4$ respectively, on the inference domain $\cW = [-2,2]$. We select
an optimal bandwidth for $p=2$ as recommended in
Section~\ref{sec:kernel_bandwidth_selection}, using the rule-of-thumb with
$C(K) = 2.435$. The semi-definite program in
Section~\ref{sec:kernel_covariance_estimation} is solved with the MOSEK
interior point
optimizer \citep{mosek}, ensuring positive semi-definite covariance estimates.
Gaussian vectors are resampled $B = 10\,000$ times.

\begin{figure}[b!]
\centering
%
\begin{subfigure}{0.32\textwidth}
\centering
%\includegraphics[scale=0.64]{graphics/outcome_plot_total.pdf}
\caption{Total degeneracy, \\
$\pi = \left( \frac{1}{2}, 0, \frac{1}{2} \right)$.}
\end{subfigure}
%
\begin{subfigure}{0.32\textwidth}
\centering
%\includegraphics[scale=0.64]{graphics/outcome_plot_partial.pdf}
\caption{Partial degeneracy, \\
$\pi = \left( \frac{1}{4}, 0, \frac{3}{4} \right)$.}
\end{subfigure}
%
\begin{subfigure}{0.32\textwidth}
\centering
%\includegraphics[scale=0.64]{graphics/outcome_plot_none.pdf}
\caption{No degeneracy, \\
$\pi = \left( \frac{1}{5}, \frac{1}{5}, \frac{3}{5} \right)$.}
\end{subfigure}
%
\caption[Typical outcomes for different values of the parameter $\pi$]
{Typical outcomes for three different values of the parameter $\pi$.}
%
\label{fig:kernel_results}
%
\end{figure}

In Figure~\ref{fig:kernel_results} we plot a typical outcome for each of the
three
degeneracy types (total, partial, none), using the Epanechnikov kernel of order
$p=2$, with sample size $n=100$ (so $N=4950$ pairs of nodes) and with $d=100$
equally-spaced evaluation points. Each plot contains the true density function
$f_W$, the dyadic kernel density estimate $\hat f_W$ and two different
approximate $95\%$ confidence bands for $f_W$. The first is the uniform
confidence band (UCB) constructed using one of our main results,
Theorem~\ref{thm:kernel_ucb}. The second is a sequence of pointwise confidence
intervals (PCI) constructed by finding a confidence interval for each
evaluation point separately. We show only $10$ pointwise confidence intervals
for clarity. In general, the PCIs are too narrow as they fail to provide
simultaneous (uniform) coverage over the evaluation points. Note that under
partial degeneracy the confidence band narrows near the degenerate point
$w = 0$.

\begin{table}[b!]
\centering
\begin{tabular}{|c|c|c|c|c|cc|cc|}
\hline
\multirow{2}{*}{$ \pi $}
& \multirow{2}{*}{Degeneracy type}
& \multirow{2}{*}{$ \hat h_{\ROT} $}
& \multirow{2}{*}{$ p $}
& \multirow{2}{*}{RIMSE}
& \multicolumn{2}{|c|}{UCB}
& \multicolumn{2}{|c|}{PCI} \\
\cline{6-9}
& & & &
& CR & AW
& CR & AW \\
\hline
\multirow{2}{*}{$ \left(\frac{1}{2}, 0, \frac{1}{2}\right) $}
& \multirow{2}{*}{Total}
& \multirow{2}{*}{0.161}
& 2 & 0.00048 & 87.1\% & 0.0028 & 6.5\% & 0.0017 \\
& & & 4 & 0.00068 & 95.2\% & 0.0042 & 9.7\% & 0.0025 \\
\hline
\multirow{2}{*}{$ \left(\frac{1}{4}, 0, \frac{3}{4}\right) $}
& \multirow{2}{*}{Partial}
& \multirow{2}{*}{0.158}
& 2 & 0.00228 & 94.5\% & 0.0112 & 75.6\% & 0.0083 \\
& & & 4 & 0.00234 & 94.7\% & 0.0124 & 65.3\% & 0.0087 \\
\hline
\multirow{2}{*}{$ \left(\frac{1}{5}, \frac{1}{5}, \frac{3}{5}\right) $}
& \multirow{2}{*}{None}
& \multirow{2}{*}{0.145}
& 2 & 0.00201 & 94.2\% & 0.0106 & 73.4\% & 0.0077 \\
& & & 4 & 0.00202 & 95.6\% & 0.0117 & 64.3\% & 0.0080 \\
\hline
\end{tabular}
\caption[Numerical results for three values of the parameter $\pi$]{
Numerical results for three values of the parameter $\pi$.}
\label{tab:kernel_results}
\end{table}

Next, Table~\ref{tab:kernel_results} presents numerical results. For each
degeneracy
type (total, partial, none) and each kernel order ($p=2$, $p=4$), we run $2000$
repeats with sample size $n=3000$ (giving $N=4\,498\,500$ pairs of nodes) and
with $d=50$ equally-spaced evaluation points. We record the average
rule-of-thumb bandwidth $\hat{h}_{\ROT}$ and the average root integrated mean
squared error (RIMSE). For both the uniform confidence bands (UCB) and the
pointwise confidence intervals (PCI), we report the coverage rate (CR) and the
average width (AW).
%
The lower-order kernel ($p=2$) ignores the bias, leading to good RIMSE
performance and acceptable UCB coverage under partial or no degeneracy, but
gives invalid inference under total degeneracy. In contrast, the higher-order
kernel ($p=4$) provides robust bias correction and hence improves the coverage
of the UCB in every regime, particularly under total degeneracy, at the cost of
increasing both the RIMSE and the average widths of the confidence bands.
%
As expected, the pointwise (in $w\in\cW$) confidence intervals (PCIs) severely
undercover in every regime. Thus our simulation results show that the proposed
feasible inference methods based on robust bias correction and proper
Studentization deliver valid uniform inference which is robust to unknown
degenerate points in the underlying dyadic distribution.

\section{Counterfactual dyadic density estimation}
\label{sec:kernel_counterfactual}

To further showcase the applicability of our main results, we develop a kernel
density estimator for dyadic counterfactual distributions. The aim of such
counterfactual analysis is to estimate the distribution of an outcome variable
had some covariates followed a distribution different from the actual one, and
it is important in causal inference and program evaluation settings
\citep{dinardo1996distribution,chernozhukov2013inference}.

For each $r \in \{0,1\}$, let $\bW_n^r$, $\bA_n^r$, and $\bV_n^r$ be random
variables as defined in Assumption~\ref{ass:kernel_data} and
$\bX_n^r = (X_1^r, \ldots, X_n^r)$ be some covariates.
We assume that $(A_i^r, X_i^r)$ are independent over $1 \leq i \leq n$
and that $\bX_n^r$ is independent of $\bV_n^r$, that
$W_{i j}^r \mid X_i^r, X_j^r$ has a conditional Lebesgue density
$f_{W \mid XX}^r(\,\cdot \mid x_1, x_2) \in \cH^\beta_{C_\rH}(\cW)$,
that $X_i^r$ follows a distribution function $F_X^r$ on a common support $\cX$,
and that $(\bA_n^0, \bV_n^0, \bX_n^0)$
is independent of $(\bA_n^1, \bV_n^1, \bX_n^1)$.

We interpret $r$ as an index for two populations, labeled $0$ and $1$. The
counterfactual density of population $1$ had it followed the
same covariate distribution as population $0$ is
%
\begin{align*}
f_W^{1 \triangleright 0}(w)
&= \E\left[ f_{W \mid XX}^1\big(w \mid X_1^0, X_2^0\big) \right] \\
&= \int_{\cX} \int_{\cX} f_{W \mid XX}^{1}(w \mid x_1, x_2)
\psi(x_1) \psi(x_2) \diff F_X^{1}(x_1) \diff F_X^{1}(x_2),
\end{align*}
%
where $\psi(x) = \mathrm{d} F_X^0(x) / \mathrm{d} F_X^1(x)$ for $x \in \cX$
is a Radon--Nikodym derivative. If $X^0_i$ and $X^1_i$ have Lebesgue densities,
it is natural to consider a parametric model of the form
$\mathrm{d} F_X^{r}(x)=f_X^r(x;\theta)\diff x$
for some finite-dimensional parameter $\theta$.
Alternatively, if the covariates $X_n^r$ are discrete and have a positive
probability mass function $p_X^r(x)$ on a finite
support $\cX$, the object of interest becomes
$f_W^{1 \triangleright 0}(w)
= \sum_{x_1 \in \cX} \sum_{x_2 \in \cX}
f_{W \mid XX}^{1}(w \mid x_1, x_2) \psi(x_1) \psi(x_2)
p_X^{1}(x_1) p_X^{1}(x_2)$,
where $\psi(x) = p_X^0(x)/p_X^1(x)$ for $x \in \cX$.
We consider discrete covariates for simplicity,
and hence the counterfactual dyadic kernel density estimator is
%
\begin{align*}
\hat f_W^{\,1 \triangleright 0}(w)
&= \frac{2}{n(n-1)} \sum_{i=1}^{n-1} \sum_{j=i+1}^n
\hat \psi(X_i^1) \hat \psi(X_j^1) k_h(W_{i j}^1, w),
\end{align*}
%
where $\hat\psi(x) = \hat p_X^{\,0}(x) / \hat p_X^{\,1}(x)$ and
$\hat p_X^{\,r}(x) = \frac{1}{n}\sum_{i = 1}^n \I\{X_i^r = x\}$,
with $\I$ the indicator function.

Section~\ref{sec:kernel_app_main} provides technical details:
we show how an asymptotic linear representation for $\hat\psi(x)$ leads to a
Hoeffding-type decomposition of $\hat f_W^{\,1 \triangleright 0}(w)$,
which is then used to establish that $\hat f_W^{\,1 \triangleright 0}$
is uniformly consistent for $f_W^{\,1 \triangleright 0}(w)$
and also admits a Gaussian strong approximation, with the same rates of
convergence as for the standard density estimator. Furthermore, define the
covariance function of $\hat f_W^{\,1 \triangleright 0}(w)$ as
$\Sigma_n^{1 \triangleright 0}(w,w') = \Cov\big[
\hat f_W^{\,1 \triangleright 0}(w),
\hat f_W^{\,1 \triangleright 0}(w') \big]$,
which can be estimated as follows. First let
$\hat\kappa(X_i^0, X_i^1, x)
= \frac{\I\{X_i^0 = x\} - \hat p_X^0(x)}{\hat p_X^1(x)}
- \frac{\hat p_X^0(x)}{\hat p_X^1(x)} \frac{\I\{X_i^1 = x\} - \hat
p_X^1(x)}{\hat p_X^1(x)}$
be a plug-in estimate of the influence function for $\hat\psi(x)$
and define the leave-one-out conditional expectation estimators
$S_i^{1 \triangleright 0}(w)
= \frac{1}{n-1} \big( \sum_{j=1}^{i-1} k_h(W_{j i}^1,w) \hat\psi(X_j^1)
+ \sum_{j=i+1}^n k_h(W_{i j}^1,w) \hat\psi(X_j^1) \big)$
and $\tilde S_i^{1 \triangleright 0}(w)
= \frac{1}{n-1} \sum_{j=1}^n \I\{j \neq i\}
\hat\kappa(X_i^0, X_i^1, X_j^1) S_j^{1 \triangleright 0}(w)$.
Define the covariance estimator
%
\begin{align*}
\hat\Sigma_n^{1 \triangleright 0}(w,w')
&= \frac{4}{n^2} \sum_{i=1}^n
\big(
\hat\psi(X_i^1) S_i^{1 \triangleright 0}(w)
+ \tilde S_i^{1 \triangleright 0}(w)
\big)
\big(
\hat\psi(X_i^1) S_i^{1 \triangleright 0}(w')
+ \tilde S_i^{1 \triangleright 0}(w')
\big) \\
&\quad-
\frac{4}{n^3(n-1)}
\sum_{i<j} k_h(W_{i j}^1, w) k_h(W_{i j}^1, w')
\hat\psi(X_i^1)^2 \hat\psi(X_j^1)^2
- \frac{4}{n}
\hat f_W^{\,1 \triangleright 0}(w) \hat f_W^{\,1 \triangleright 0}(w').
\end{align*}
%
We use a positive semi-definite approximation to
$\hat\Sigma_n^{1 \triangleright 0}$, denoted by
$\hat\Sigma_n^{+, 1 \triangleright 0}$,
as in Section~\ref{sec:kernel_covariance_estimation}. To construct feasible
uniform
confidence bands, define a process $\hat Z_n^{T, 1 \triangleright 0}(w)$ which
is conditionally mean-zero and Gaussian given the data $\bW_n^1$, $\bX_n^0$, and
$\bX_n^1$, and whose conditional covariance structure is
$\E\big[\hat Z_n^{T, 1 \triangleright 0}(w)
\hat Z_n^{T, 1 \triangleright 0}(w')
\bigm| \bW_n^1, \bX_n^0, \bX_n^1 \big]
= \frac{\hat \Sigma_n^{+, 1 \triangleright 0}(w,w')}
{\sqrt{\hat \Sigma_n^{+, 1 \triangleright 0}(w,w)
\hat \Sigma_n^{+, 1 \triangleright 0}(w',w')}}$.
For $\alpha \in (0,1)$, define
$\hat q_{1-\alpha}^{\,1 \triangleright 0}$
as the quantile satisfying
$\P\big(\sup_{w \in \cW}\big| \hat Z_n^{T, 1 \triangleright 0}(w) \big|
\leq \hat q_{1-\alpha}^{\,1 \triangleright 0}
\bigm\vert \bW_n^1, \bX_n^0, \bX_n^1 \big)
= 1 - \alpha$.
Then if the covariance estimator is appropriately consistent,
%
\begin{align*}
\P\left(
f_W^{1 \triangleright 0}(w) \in
\left[
\hat f_W^{\,1 \triangleright 0}(w)
\pm \hat q^{\,1 \triangleright 0}_{1-\alpha}
\sqrt{\hat\Sigma_n^{+, 1 \triangleright 0}(w,w)}
\,\right]
\,\textup{for all } w \in \cW
\right) \to 1 - \alpha,
\end{align*}
%
giving feasible uniform inference methods, which are robust to unknown
degeneracies, for counterfactual distribution analysis in dyadic data settings.

\subsection{Application to trade data}
\label{sec:kernel_trade_data}

We illustrate the performance of our estimation and inference methods with a
real-world data set. We use international bilateral trade data from the
International Monetary Fund's Direction of Trade Statistics (DOTS), previously
analyzed by \citet{head2014gravity} and \citet{chiang2022inference}. This data
set contains information about the yearly trade flows among $n = 207$ economies
($N = 21\,321$ pairs), and we focus on the years $1995$, $2000$, and $2005$.

We define the \emph{trade volume} between countries $i$ and $j$ as the
logarithm of the sum of the trade flow (in billions of US dollars) from $i$ to
$j$ and the trade flow from $j$ to $i$. In each year several pairs of countries
did not trade directly, yielding trade flows of zero and hence a trade volume
of $-\infty$. We therefore assume that the distribution of trade volumes is a
mixture of a point mass at $-\infty$ and a Lebesgue density on $\R$. The local
nature of our estimator means that observations taking the value of $-\infty$
can simply be removed from the data set.
Table~\ref{tab:kernel_trade_network_stats}
gives summary statistics for these trade networks, and shows how the networks
become more connected over time, with edge density, average degree, and
clustering coefficient increasing.

\begin{table}[b!]
\centering
\begin{tabular}{|c|c|c|c|c|c|}
\hline
Year & Nodes & Edges & Edge density & Average degree
& Clustering coefficient \\
\hline
1995 & 207 & 11\,603 & 0.5442 & 112.1 & 0.7250 \\
2000 & 207 & 12\,528 & 0.5876 & 121.0 & 0.7674 \\
2005 & 207 & 12\,807 & 0.6007 & 123.7 & 0.7745 \\
\hline
\end{tabular}
\caption[Summary statistics for the DOTS trade networks]{
Summary statistics for the DOTS trade networks.}
\label{tab:kernel_trade_network_stats}
\end{table}

For counterfactual analysis we use the gross domestic product (GDP) of each
country as a covariate, using $10\%$-percentiles to group the values into $10$
different levels for ease of estimation. This allows for a comparison of the
observed distribution of trade at each year with, for example, the
counterfactual distribution of trade had the GDP distribution remained as it
was in $1995$. As such, we can measure how much of the change in trade
distribution is attributable to a shift in the GDP distribution.

To estimate the trade volume density function we use
Algorithm~\ref{alg:kernel_method}
with $d=100$ equally-spaced evaluation points in $[-10,10]$, using the
rule-of-thumb bandwidth selector $\hat h_{\ROT}$ from
Section~\ref{sec:kernel_bandwidth_selection} with $p=2$ and $C(K) = 2.435$. For
inference we use an Epanechnikov kernel of order $p=4$ and resample the
Gaussian process $B = 10\,000$ times. We also estimate the counterfactual trade
distributions in 2000 and 2005 respectively, replacing the GDP distribution
with that from 1995. For each year, Figure~\ref{fig:kernel_trade} plots the
real and
counterfactual density estimates along with their respective uniform confidence
bands (UCB) at the nominal coverage rate of $95\%$. Our empirical results show
that the counterfactual distribution drifts further from the truth in 2005
compared with 2000, indicating a shift in the GDP distribution.

\begin{figure}[t]
\centering
%
\begin{subfigure}{0.32\textwidth}
\centering
%\includegraphics[scale=0.64]{graphics/trade_plot_1995.pdf}
\caption{Year 1995, $\hat h_{\ROT} = 1.27$.}
\end{subfigure}
%
\begin{subfigure}{0.32\textwidth}
\centering
%\includegraphics[scale=0.64]{graphics/trade_plot_1995_2000.pdf}
\caption{Year 2000, $\hat h_{\ROT} = 1.31$.}
\end{subfigure}
%
\begin{subfigure}{0.32\textwidth}
\centering
%\includegraphics[scale=0.64]{graphics/trade_plot_1995_2005.pdf}
\caption{Year 2005, $\hat h_{\ROT} = 1.37$.}
\end{subfigure}
%
\caption[Histogram-based estimation and inference for the DOTS data]{
Real and counterfactual density estimates and confidence bands for
the DOTS data with histogram-based covariate estimation.}
%
\label{fig:kernel_trade}
%
\end{figure}

\begin{figure}[b!]
\centering
%
\begin{subfigure}{0.32\textwidth}
\centering
%\includegraphics[scale=0.64]{graphics/trade_gdp_1995.pdf}
\caption{Year 1995}
\end{subfigure}
%
\begin{subfigure}{0.32\textwidth}
\centering
%\includegraphics[scale=0.64]{graphics/trade_gdp_2000.pdf}
\caption{Year 2000}
\end{subfigure}
%
\begin{subfigure}{0.32\textwidth}
\centering
%\includegraphics[scale=0.64]{graphics/trade_gdp_2005.pdf}
\caption{Year 2005}
\end{subfigure}
%
\caption[Estimated GDP distributions for the DOTS data]{
Estimated GDP distributions for the DOTS data using histograms and
normal likelihood maximization.}
%
\label{fig:kernel_gdp}
%
\end{figure}

In Figure~\ref{fig:kernel_gdp} we illustrate how, in the preliminary step of the
counterfactual analysis, the distribution of log GDP is approximated using the
histogram estimators $\hat p_X^{\,0}$ and $\hat p_X^{\,1}$ defined in
Section~\ref{sec:kernel_counterfactual}. We also plot the density function of a
normal distribution, fitted using maximum likelihood estimation, and this seems
to capture the distribution of log GDP reasonably well. Such a parametric
approach to the preliminary step may be favored in cases where a choice of
model is clear or where the histogram estimators perform poorly.

To demonstrate the relative robustness of our counterfactual analysis to the
choice of preliminary estimation step, we provide results using a
parametric estimator of the distribution of GDP.
Figure~\ref{fig:kernel_trade_para}
repeats the procedure used for Figure~\ref{fig:kernel_trade}, but this time
replacing
the histogram estimators by parametric estimators of the log GDP based on
normal likelihood maximization. The point estimates are qualitatively similar,
with the counterfactual distribution drifting in the same direction over time.
The confidence bands are also similar, with the band based on the parametric
fit being slightly narrower in general. This could be due to the more stringent
model specification leading to less estimated variance in the fitted values.

\begin{figure}[t]
\centering
%
\begin{subfigure}{0.32\textwidth}
\centering
%\includegraphics[scale=0.64]{graphics/trade_plot_parametric_1995.pdf}
\caption{Year 1995, $\hat h_{\ROT} = 1.27$.}
\end{subfigure}
%
\begin{subfigure}{0.32\textwidth}
\centering
%\includegraphics[scale=0.64]{graphics/trade_plot_parametric_1995_2000.pdf}
\caption{Year 2000, $\hat h_{\ROT} = 1.31$.}
\end{subfigure}
%
\begin{subfigure}{0.32\textwidth}
\centering
%\includegraphics[scale=0.64]{graphics/trade_plot_parametric_1995_2005.pdf}
\caption{Year 2005, $\hat h_{\ROT} = 1.37$.}
\end{subfigure}
%
\caption[Parametric likelihood-based estimation and
inference for the DOTS data]{
Real and counterfactual density estimates and confidence bands for
the DOTS data with parametric covariate estimation.}
%
\label{fig:kernel_trade_para}
%
\end{figure}

\section{Other applications and future work}
\label{sec:kernel_future}

To emphasize the broad applicability of our methods to network science
problems, we present three application scenarios. The first concerns comparison
of networks \citep{kolaczyk2009statistical}, while the second and third involve
nonparametric and semiparametric dyadic regression respectively.

Firstly, consider the setting where there are two independent networks with
continuous dyadic covariates $\bW_n^0$ and $\bW_m^1$ respectively.
Practitioners may wish to test if these two dyadic distributions are the same,
that is, whether their density functions $f_W^0$ and $f_W^1$ are equal on their
common support $\cW \subseteq \R$. We present a family of hypothesis tests for
this scenario based on dyadic kernel density estimation. Let $\hat
f_W^{\,0}(w)$ and $\hat f_W^{\,1}(w)$ be the associated (bias-corrected) dyadic
kernel density estimators. Consider the test statistics $\tau_p$ for
$1 \leq p \leq \infty$ where
%
\begin{align}
\nonumber
\tau_p^p
&= \int_{-\infty}^{\infty}
\left| \hat f_W^{\,1}(w) - \hat f_W^{\,0}(w) \right|^p
\diff w
\ \text{ for } p < \infty, \\
\label{eq:kernel_hypothesis_test}
\tau_\infty
&= \sup_{w \in \cW} \left| \hat f_W^{\,1}(w) - \hat f_W^{\,0}(w) \right|.
\end{align}
%
Clearly, we should reject the null hypothesis that $f_W^0 = f_W^1$ whenever the
test statistic $\tau_p$ is sufficiently large. To estimate the critical value,
let $\hat\Sigma_n^{+,0}(w, w')$ and $\hat\Sigma_m^{+,1}(w, w')$ be the positive
semi-definite estimators defined in
Section~\ref{sec:kernel_covariance_estimation} and
let $\hat Z^0_n(w)$ and $\hat Z^1_m(w)$ be zero-mean Gaussian processes with
covariance structures $\hat\Sigma_n^{+,0}(w, w')$ and
$\hat\Sigma_m^{+,1}(w, w')$ respectively, which are independent conditional on
the data. Define the approximate null test statistic $\hat \tau_p$ by replacing
$\hat f_W^{\,0}(w)$ and $\hat f_W^{\,1}(w)$ with $\hat Z^0_n(w)$ and
$\hat Z^1_m(w)$ respectively in \eqref{eq:kernel_hypothesis_test}.
For a significance level
$\alpha \in (0,1)$, the critical value is $\hat C_\alpha$ where
%
$\P \big(
\hat \tau_p \geq \hat C_\alpha \bigm\vert \bW_n^0, \bW_n^1
\big) = \alpha$.
%
This is estimated by Monte Carlo simulation, resampling from the conditional
law of $\hat Z^0_n(w)$ and $\hat Z^1_m(w)$ and replacing integrals and suprema
by sums and maxima over a finite partition of $\cW$.

While our focus has been on density estimation with dyadic data,
our uniform dyadic estimation and inference results are readily applicable
to the settings of nonparametric and semiparametric dyadic regression.
For a second example, suppose $Y_{i j} = Y(X_i, X_j, A_i, A_j, V_{i j})$,
where only $\bX_n$ and $\bY_n$ are observed and
$\bV_n$ is independent of $(\bX_n, \bA_n)$,
with $\bX_n = (X_i : 1 \leq i \leq n)$,
$\bA_n = (A_i : 1 \leq i \leq n)$, $\bY_n = (Y_{i j}:1\leq i<j\leq n)$,
and $\bV_n = (V_{i j} : 1 \leq i < j \leq n)$.
A parameter of interest is the regression function
$\mu(x_1, x_2) = \E[Y_{i j} \mid X_i=x_1, X_j=x_2]$,
which can be used to analyze average or partial effects
of changing the node attributes $X_i$ and $X_j$ on the edge variable $Y_{i j}$.
This conditional expectation could be estimated using local polynomial methods:
suppose that $X_i$ takes values in $\R^m$ and
let $r(x_1, x_2)$ be a monomial basis up to degree
$\gamma \geq 0$ on $\R^m \times \R^m$. Then, for some bandwidth $h > 0$ and
a kernel function $k_h$ on $\R^m \times \R^m$,
the local polynomial regression estimator of $\mu(x_1, x_2)$ is
$\hat\mu(x_1, x_2) = e_1^\T \hat\beta(x_1, x_2)$ where
$e_1$ is the first standard unit vector in $\R^q$ for
$q=\binom{2m+\gamma}{\gamma}$ and
%
\begin{align}
\nonumber
\hat{\beta}(x_1, x_2)
&=
\argmin_{\beta \in \R^q}
\sum_{i=1}^{n-1} \sum_{j=i+1}^n
\left( Y_{i j} - r(X_i-x_1, X_j-x_2)^\T \beta \right)^2
k_h(X_i-x_1, X_j-x_2) \\
\label{eq:kernel_locpol}
&=
\left(
\sum_{i=1}^{n-1} \sum_{j=i+1}^n k_{i j} r_{i j} r_{i j}^\T
\right)^{-1}
\left(
\sum_{i=1}^{n-1} \sum_{j=i+1}^n k_{i j} r_{i j} Y_{i j}
\right),
\end{align}
%
with $k_{i j} = k_h(X_i-x_1, X_j-x_2)$ and $r_{i j} = r(X_i-x_1, X_j-x_2)$.
\citet{graham2021minimax} established pointwise distribution theory
for the special case of the dyadic Nadaraya--Watson kernel regression estimator
($\gamma=0$), but no uniform analogues have yet been given. It can be shown
that the ``denominator'' matrix in \eqref{eq:kernel_locpol} converges uniformly
to its
expectation, while the U-process-like ``numerator'' matrix can be handled the
same way as we analyzed $\hat f_W(w)$ in this chapter, through a Hoeffding-type
decomposition and strong approximation methods, along with standard bias
calculations. Such distributional approximation results can be used to
construct valid uniform confidence bands for the regression function
$\mu(x_1, x_2)$, as well as to conduct hypothesis testing for parametric
specifications or shape constraints.

As a third example, we consider applying our results to semiparametric
semi-linear regression problems. The dyadic semi-linear regression model is
$\E[Y_{i j} \mid W_{i j}, X_i, X_j] = \theta^\T W_{i j} + g(X_i, X_j)$
where $\theta$ is the finite-dimensional parameter of interest
and $g(X_i, X_j)$ is an unknown function of the covariates $(X_i, X_j)$.
Local polynomial (or other) methods can be used to estimate $\theta$ and $g$,
where the estimator of the nonparametric component $g$ takes a similar form to
\eqref{eq:kernel_locpol}, that is, a ratio of two kernel-based estimators as in
\eqref{eq:kernel_estimator}. Consequently, the strong approximation techniques
presented in this chapter can be appropriately modified to develop valid
uniform inference procedures for $g$ and
$\E[Y_{i j} \mid W_{i j}=w, X_i=x_1, X_j=x_2]$, as well as functionals thereof.

\section{Conclusion}
\label{sec:kernel_conclusion}

We studied the uniform estimation and inference properties of the dyadic kernel
density estimator $\hat{f}_W$ given in \eqref{eq:kernel_estimator}, which forms
a class of U-process-like estimators indexed by the $n$-varying kernel function
$k_h$ on $\cW$. We established uniform minimax-optimal point estimation results
and uniform distributional approximations for this estimator based on novel
strong approximation strategies. We then applied these results to derive valid
and feasible uniform confidence bands for the dyadic density estimand $f_W$,
and also developed a substantive application of our theory to counterfactual
dyadic density analysis. We gave some other statistical applications of our
methodology as well as potential avenues for future research. From a technical
perspective, Appendix~\ref{app:kernel} contains several generic results
concerning strong approximation methods and maximal inequalities for empirical
processes that may be of independent interest. Implementations of this
chapter's methodology, along with replication files for the empirical results,
are provided by a Julia package available at
\github{wgunderwood/DyadicKDE.jl}.
This work is based on \citet{cattaneo2024uniform},
and has been presented by Cattaneo at
the Columbia University Biostatistics Colloquium Seminar (2022)
and the Georgia Institute of Technology Statistics Seminar (2022),
by Feng at
the Renmin University Econometrics Seminar (2022),
the Xiamen University Symposium on Modern Statistics (2022),
the Peking University Econometrics Seminar (2023),
and the Asian Meeting of the Econometric Society
in East and Southeast Asia, Singapore (2023),
and by Underwood at the University of Illinois Statistics Seminar (2024),
the University of Michigan Statistics Seminar (2024), and the University of
Pittsburgh Statistics Seminar (2024).

\chapter[Yurinskii's Coupling for Martingales]%
{Yurinskii's Coupling \\ for Martingales}
\label{ch:yurinskii}

% abstract
Yurinskii's coupling is a popular theoretical tool for non-asymptotic
distributional analysis in mathematical statistics and applied probability,
offering a Gaussian strong approximation with an explicit error bound under
easily verified conditions. Originally stated in $\ell^2$-norm for sums of
independent random vectors, it has recently been extended both to the
$\ell^p$-norm, for $1 \leq p \leq \infty$, and to vector-valued martingales in
$\ell^2$-norm, under some strong conditions. We present as our main result a
Yurinskii coupling for approximate martingales in $\ell^p$-norm, under
substantially weaker conditions than those previously imposed. Our formulation
further allows for the coupling variable to follow a more general Gaussian
mixture distribution, and we provide a novel third-order coupling method which
gives tighter approximations in certain settings. We specialize our main result
to mixingales, martingales, and independent data, and derive uniform Gaussian
mixture strong approximations for martingale empirical processes. Substantive
applications of our theory to nonparametric partitioning-based and local
polynomial regression procedures are provided.

\section{Introduction}

Yurinskii's coupling \citep{yurinskii1978error} has proven to be an important
theoretical tool for developing non-asymptotic distributional approximations in
mathematical statistics and applied probability. For a sum $S$ of $n$
independent zero-mean $d$-dimensional random vectors, this coupling technique
constructs (on a suitably enlarged probability space) a zero-mean
$d$-dimensional Gaussian vector $T$ with the same covariance matrix as $S$ and
which is close to $S$ in probability, bounding the discrepancy $\|S-T\|$ as a
function of $n$, $d$, the choice of the norm, and some features of the
underlying distribution. See, for example, \citet[Chapter 10]{pollard2002user}
for a textbook introduction.

When compared to other coupling approaches, such as the celebrated Hungarian
construction \citep{komlos1975approximation} or Zaitsev's coupling
\citep{zaitsev1987estimates,zaitsev1987gaussian}, Yurinskii's approach stands
out for its simplicity, robustness, and wider applicability, while also
offering tighter couplings in some applications (see below for more discussion
and examples). These features have led many scholars to use Yurinskii's
coupling to study the distributional features of high-dimensional statistical
procedures in a variety of settings, often with the end goal of developing
uncertainty quantification or hypothesis testing methods. For example, in
recent years, Yurinskii's coupling has been used to construct Gaussian
approximations for the suprema of empirical processes
\citep{chernozhukov2014gaussian}; to establish distribution theory for
non-Donsker stochastic $t$-processes generated in nonparametric series
regression \citep{belloni2015some}; to prove distributional approximations for
high-dimensional $\ell^p$-norms \citep{biau2015high}; to develop distribution
theory for vector-valued martingales \citep{belloni2018high,li2020uniform}; to
derive a law of the iterated logarithm for stochastic gradient descent
optimization methods \citep{anastasiou2019normal}; to establish uniform
distributional results for nonparametric high-dimensional quantile processes
\citep{belloni2019conditional}; to develop distribution theory for non-Donsker
stochastic $t$-processes generated in partitioning-based series regression
\citep{cattaneo2020large}; to deduce Bernstein--von Mises theorems in
high-dimensional settings \citep{ray2021bernstein}; and to develop distribution
theory for non-Donsker U-processes based on dyadic network data
\citep{cattaneo2024uniform}. There are also many other early applications of
Yurinskii's coupling: \citet{dudley1983invariance} and \citet{dehling1983limit}
establish invariance principles for Banach space-valued random variables, and
\citet{lecam1988} and \citet{sheehy1992uniform} obtain uniform Donsker results
for empirical processes, to name just a few.

This chapter presents a new Yurinskii coupling which encompasses and improves
upon all of the results previously available in the literature, offering four
new features:
%
\begin{enumerate}[label=(\roman*),leftmargin=*]
\item
\label{it:yurinskii_contribution_approximate_martingale}
It applies to vector-valued \textit{approximate martingale} data.
\item
\label{it:yurinskii_contribution_gaussian_mixture}
It allows for a \textit{Gaussian mixture} coupling distribution.
\item
\label{it:yurinskii_contribution_degeneracy}
It imposes \textit{no restrictions on degeneracy} of the
data covariance matrix.
\item
\label{it:yurinskii_contribution_third_order}
It establishes a \textit{third-order} coupling to
improve the approximation in certain situations.
\end{enumerate}
%

Closest to our work are the unpublished manuscript by \citet{belloni2018high}
and the recent paper by \citet{li2020uniform}, which both investigated
distribution theory for martingale data using Yurinskii's coupling and related
methods. Specifically, \citet{li2020uniform} established a Gaussian
$\ell^2$-norm Yurinskii coupling for mixingales and martingales under the
assumption that the covariance structure has a minimum eigenvalue bounded away
from zero. As formally demonstrated in this chapter
(Section~\ref{sec:yurinskii_kde}),
such eigenvalue assumptions can be prohibitively strong in practically relevant
applications. In contrast, our Yurinskii coupling does not impose any
restrictions on covariance degeneracy
\ref{it:yurinskii_contribution_degeneracy}, in
addition to offering several other new features not present in
\citet{li2020uniform}, including
\ref{it:yurinskii_contribution_approximate_martingale},
\ref{it:yurinskii_contribution_gaussian_mixture},
\ref{it:yurinskii_contribution_third_order}, and
applicability to general $\ell^p$-norms. In addition, we correct a slight
technical inaccuracy in their proof relating to the derivation of bounds in
probability (Remark \ref{rem:yurinskii_coupling_bounds_probability}).
\citet{belloni2018high} did not establish a Yurinskii coupling for martingales,
but rather a central limit theorem for smooth functions of high-dimensional
martingales using the celebrated second-order Lindeberg method
\citep[see][and references therein]{chatterjee2006generalization}, explicitly
accounting for covariance degeneracy. As a consequence, their result could be
leveraged to deduce a Yurinskii coupling for martingales with additional,
non-trivial technical work (see Section~\ref{sec:yurinskii_app_proofs}
in Appendix~\ref{app:yurinskii} for details).
Nevertheless, a Yurinskii coupling derived from
\citet{belloni2018high} would not feature
\ref{it:yurinskii_contribution_approximate_martingale},
\ref{it:yurinskii_contribution_gaussian_mixture},
\ref{it:yurinskii_contribution_third_order}, or
general $\ell^p$-norms, as our results do. We discuss further the connections
between our work and the related literature in the upcoming sections, both when
introducing our main theoretical results and when presenting the examples and
statistical applications.

The most general coupling result of this chapter
(Theorem~\ref{thm:yurinskii_sa_dependent}) is presented in
Section~\ref{sec:yurinskii_main_results}, where we also specialize it to a
slightly
weaker yet more user-friendly formulation
(Proposition~\ref{pro:yurinskii_sa_simplified}). Our Yurinskii coupling for
approximate
martingales is a strict generalization of all previous Yurinskii couplings
available in the literature, offering a Gaussian mixture strong approximation
for approximate martingale vectors in $\ell^p$-norm, with an improved rate of
approximation when the third moments of the data are negligible, and with no
assumptions on the spectrum of the data covariance matrix. A key technical
innovation underlying the proof of Theorem~\ref{thm:yurinskii_sa_dependent} is
that we
explicitly account for the possibility that the minimum eigenvalue of the
variance may be zero, or its lower bound may be unknown, with the argument
proceeding using a carefully tailored regularization. Establishing a coupling
to a Gaussian mixture distribution is achieved by an appropriate conditioning
argument, leveraging a conditional version of Strassen's theorem established by
\citet{chen2020jackknife}, along with some related technical work detailed in
Section~\ref{sec:yurinskii_app_proofs}.
A third-order coupling is obtained via
a modification of a standard smoothing technique for Borel sets from classical
versions of Yurinskii's coupling, enabling improved approximation errors
whenever third moments are negligible.

In Proposition~\ref{pro:yurinskii_sa_simplified}, we explicitly tune the
parameters of
the aforementioned regularization to obtain a simpler, parameter-free version
of Yurinskii's coupling for approximate martingales, again offering Gaussian
mixture coupling distributions and an improved third-order approximation error.
This specialization of our main result takes an agnostic approach to potential
singularities in the data covariance matrix and, as such, may be improved in
specific applications where additional knowledge of the covariance structure is
available. Section~\ref{sec:yurinskii_main_results} also presents some further
refinements when additional structure is imposed, deriving Yurinskii couplings
for mixingales, martingales, and independent data as
Corollaries~\ref{cor:yurinskii_sa_mixingale},
\ref{cor:yurinskii_sa_martingale}, and
\ref{cor:yurinskii_sa_indep}, respectively. We take the opportunity to discuss
and correct
in Remark~\ref{rem:yurinskii_coupling_bounds_probability} a technical issue
which is
often neglected \citep{pollard2002user, li2020uniform} when using Yurinskii's
coupling to derive bounds in probability. Section~\ref{sec:yurinskii_factor}
presents a
stylized example portraying the relevance of our main technical results in the
context of canonical factor models, illustrating the importance of each of our
new Yurinskii coupling features
\ref{it:yurinskii_contribution_approximate_martingale}--%
\ref{it:yurinskii_contribution_third_order}.

Section~\ref{sec:yurinskii_emp_proc} considers a substantive application of our
main
results: strong approximation of martingale empirical processes. We begin with
the motivating example of canonical kernel density estimation, demonstrating
how Yurinskii's coupling can be applied, and showing in
Lemma~\ref{lem:yurinskii_kde_eigenvalue} why it is essential that we do not
place any
conditions on the minimum eigenvalue of the variance matrix
\ref{it:yurinskii_contribution_degeneracy}.
We then present a general-purpose strong
approximation for martingale empirical processes in
Proposition~\ref{pro:yurinskii_emp_proc}, combining classical results in the
empirical
process literature \citep{van1996weak} with our
Corollary~\ref{cor:yurinskii_sa_martingale}. This statement appears to be the
first of
its kind for martingale data, and when specialized to independent
(and not necessarily identically distributed) data, it is
shown to be superior to the best known comparable strong approximation result
available in the literature \citep{berthet2006revisiting}. Our improvement
comes from using Yurinskii's coupling for the $\ell^\infty$-norm, where
\citet{berthet2006revisiting} apply Zaitsev's coupling
\citep{zaitsev1987estimates, zaitsev1987gaussian} with the larger
$\ell^2$-norm.

Section~\ref{sec:yurinskii_nonparametric} further illustrates the applicability
of our
results through two examples in nonparametric regression estimation. Firstly,
we deduce a strong approximation for partitioning-based least squares series
estimators with time series data, applying
Corollary~\ref{cor:yurinskii_sa_martingale}
directly and additionally imposing only a mild mixing condition on the
regressors. We show that our Yurinskii coupling for martingale vectors delivers
the same distributional approximation rate as the best known result for
independent data, and discuss how this can be leveraged to yield a feasible
statistical inference procedure. We also show that if the residuals have
vanishing conditional third moment, an improved rate of Gaussian approximation
can be established. Secondly, we deduce a strong approximation for local
polynomial estimators with time series data,
using our result on martingale empirical processes
(Proposition~\ref{pro:yurinskii_emp_proc}) and again imposing a mixing
assumption.
Appealing to empirical process theory is essential here as, in contrast with
series estimators, local polynomials do not possess certain additive
separability properties. The bandwidth restrictions we require are relatively
mild, and, as far as we know, they have not been improved upon even with
independent data.

Section \ref{sec:yurinskii_conclusion} concludes the chapter.
All proofs are collected in
Appendix~\ref{app:yurinskii}, which also includes other technical lemmas
of potential independent interest, alongside some further results on
applications of our theory to deriving high-dimensional central limit theorems
for martingales in Section~\ref{sec:yurinskii_app_high_dim_clt}.

\subsection{Notation}

We write $\|x\|_p$ for $p\in[1,\infty]$ to denote the $\ell^p$-norm if $x$ is a
(possibly random) vector or the induced operator $\ell^p$--$\ell^p$-norm if $x$
is a matrix. For $X$ a real-valued random variable and an Orlicz function
$\psi$, we use $\vvvert X \vvvert_\psi$ to denote the Orlicz $\psi$-norm
\citep[Section~2.2]{van1996weak} and $\vvvert X \vvvert_p$
for the $L^p(\P)$-norm where
$p\in [1,\infty]$. For a matrix $M$, we write $\|M\|_{\max}$ for the
maximum absolute entry and $\|M\|_\rF$ for the Frobenius norm. We denote
positive semi-definiteness by $M \succeq 0$ and write $I_d$ for the $d \times
d$ identity matrix.

For scalar sequences $x_n$ and $y_n$, we write $x_n \lesssim y_n$ if there
exists a positive constant $C$ such that $|x_n| \leq C |y_n|$ for sufficiently
large $n$. We write $x_n \asymp y_n$ to indicate both $x_n \lesssim y_n$ and
$y_n \lesssim x_n$. Similarly, for random variables $X_n$ and $Y_n$, we write
$X_n \lesssim_\P Y_n$ if for every $\varepsilon > 0$ there exists a positive
constant $C$ such that $\P(|X_n| \leq C |Y_n|) \leq \varepsilon$, and write
$X_n \to_\P X$ for limits in probability. For real numbers $a$ and $b$ we use
$a \vee b = \max\{a,b\}$. We write $\kappa \in \N^d$ for a multi-index, where
$d \in \N = \{0, 1, 2, \ldots\}$, and define $|\kappa| = \sum_{j=1}^d \kappa_j$
and $x^\kappa = \prod_{j=1}^d x_j^{\kappa_j}$ for $x \in \R^d$,
and $\kappa! = \prod_{j=1}^{d} \kappa_j !$.

Since our results concern couplings, some statements must be made on a new or
enlarged probability space. We omit the details of this for clarity of
notation, but technicalities are handled by the Vorob'ev--Berkes--Philipp
Theorem~\citep[Theorem~1.1.10]{dudley1999uniform}.

\section{Main results}
\label{sec:yurinskii_main_results}

We begin with our most general result: an $\ell^p$-norm Yurinskii coupling of a
sum of vector-valued approximate martingale differences to a Gaussian
mixture-distributed random vector. The general result is presented in
Theorem~\ref{thm:yurinskii_sa_dependent}, while
Proposition~\ref{pro:yurinskii_sa_simplified} gives
a simplified and slightly weaker version which is easier to use in
applications. We then further specialize
Proposition~\ref{pro:yurinskii_sa_simplified} to
three scenarios with successively stronger assumptions, namely mixingales,
martingales, and independent data in
Corollaries~\ref{cor:yurinskii_sa_mixingale},
\ref{cor:yurinskii_sa_martingale}, and \ref{cor:yurinskii_sa_indep}
respectively. In each case we
allow for possibly random quadratic variations (cf.\ mixing convergence),
thereby establishing a Gaussian mixture coupling in the general setting. In
Remark~\ref{rem:yurinskii_coupling_bounds_probability} we comment on and
correct an often
overlooked technicality relating to the derivation of bounds in probability
from Yurinskii's coupling. As a first illustration of the power of our
generalized $\ell^p$-norm Yurinskii coupling, we present in
Section~\ref{sec:yurinskii_factor} a simple factor model example relating to
all three of the aforementioned scenarios.

\begin{theorem}[Strong approximation for vector-valued approximate martingales]
\label{thm:yurinskii_sa_dependent}

Take a complete probability space with a countably generated filtration
$\cH_0, \ldots, \cH_n$ for $n \geq 1$, supporting the $\R^d$-valued
square-integrable variables $X_1, \ldots, X_n$.
Let $S = \sum_{i=1}^n X_i$ and define
%
\begin{align*}
\tilde X_i
&= \sum_{r=1}^n \big(\E[X_{r} \mid \cH_{i}] - \E[X_{r} \mid \cH_{i-1}]\big)
& &\text{and}
&U &= \sum_{i=1}^{n} \big( X_i - \E[ X_i \mid \cH_n]
+ \E[ X_i \mid \cH_0 ] \big).
\end{align*}
%
Let $V_i = \Var[\tilde X_i \mid \cH_{i-1}]$ and
define $\Omega = \sum_{i=1}^n V_i - \Sigma$
where $\Sigma$ is an almost surely positive semi-definite $\cH_0$-measurable
$d \times d$ matrix. Then, for each $\eta > 0$ and $p \in [1,\infty]$,
there exists, on an enlarged probability space, an $\R^d$-valued random
vector $T$ with $T \mid \cH_0 \sim \cN(0, \Sigma)$ and
%
\begin{align}
\label{eq:yurinskii_sa_dependent}
\P\big(\|S-T\|_p > 6\eta\big)
&\leq
\inf_{t>0}
\left\{
2 \P\big( \|Z\|_p > t \big)
+ \min\left\{
\frac{\beta_{p,2} t^2}{\eta^3},
\frac{\beta_{p,3} t^3}{\eta^4}
+ \frac{\pi_3 t^3}{\eta^3}
\right\}
\right\} \nonumber \\
&\quad+
\inf_{M \succeq 0}
\Big\{ 2 \P\big(\Omega \npreceq M\big) + \delta_p(M,\eta)
+ \varepsilon_p(M, \eta)\Big\}
+\P\big(\|U\|_p>\eta\big),
\end{align}
%
where $Z, Z_1,\dots ,Z_n$ are i.i.d.\ standard Gaussian random variables on
$\R^d$ independent of $\cH_n$, the second infimum is taken over all positive
semi-definite $d \times d$ non-random matrices $M$,
%
\begin{align*}
\beta_{p,k}
&=
\sum_{i=1}^n \E\left[\| \tilde X_i \|^k_2 \| \tilde X_i \|_p
+ \|V_i^{1/2} Z_i \|^k_2 \|V_i^{1/2} Z_i \|_p \right],
&\pi_3
&=
\sum_{i=1}^{n}
\sum_{|\kappa| = 3}
\E \Big[ \big|
\E [ \tilde X_i^\kappa \mid \cH_{i-1} ]
\big| \Big]
\end{align*}
%
for $k \in \{2, 3\}$, with $\pi_3 = \infty$ if the associated
conditional expectation does not exist, and with
%
\begin{align*}
\delta_p(M,\eta)
&=
\P\left(
\big\|\big((\Sigma +M)^{1/2}- \Sigma^{1/2}\big) Z\big\|_p
\geq \eta
\right), \\
\varepsilon_p(M, \eta)
&=
\P\left(\big\| (M - \Omega)^{1/2} Z \big\|_p\geq \eta, \
\Omega \preceq M\right).
\end{align*}
\end{theorem}

This theorem offers four novel contributions to the literature on coupling
theory and strong approximation, as discussed in the introduction.
% approximate martingales
Firstly \ref{it:yurinskii_contribution_approximate_martingale}, it allows for
approximate
vector-valued martingales, with the variables $\tilde X_i$ forming martingale
differences with respect to $\cH_i$ by construction, and $U$ quantifying the
associated martingale approximation error. Such martingale approximation
techniques for sequences of dependent random vectors are well established and
have been used in a range of scenarios: see, for example,
\citet{wu2004martingale}, \citet{dedecker2007weak}, \citet{zhao2008martingale},
\citet{peligrad2010conditional}, \citet{atchade2014martingale},
\citet{cuny2014martingale}, \citet{magda2018martingale}, and references
therein. In Section~\ref{sec:yurinskii_mixingales} we demonstrate how this
approximation
can be established in practice by restricting our general theorem to the
special case of mixingales, while the upcoming example in
Section~\ref{sec:yurinskii_factor} provides an illustration in the context of
auto-regressive factor models.

% Gaussian mixture
Secondly \ref{it:yurinskii_contribution_gaussian_mixture},
Theorem~\ref{thm:yurinskii_sa_dependent} allows for the
resulting coupling variable $T$
to follow a multivariate Gaussian distribution only conditionally,
and thus we offer a useful analog of mixing convergence in the context
of strong approximation.
To be more precise, the random matrix $\sum_{i=1}^{n} V_i$
is the quadratic variation of the constructed martingale
$\sum_{i=1}^n \tilde X_i$, and we approximate it using the $\cH_0$-measurable
random matrix $\Sigma$. This yields the coupling variable
$T \mid \cH_0 \sim \cN(0, \Sigma)$, which can alternatively be written as
$T=\Sigma^{1/2} Z$ with $Z \sim \cN(0,I_d)$ independent of $\cH_0$.
The errors in this quadratic variation
approximation are accounted for by the terms
$\P(\Omega \npreceq M)$, $\delta_p(M, \eta)$, and $\varepsilon_p(M, \eta)$,
utilizing a regularization argument through the free matrix parameter $M$.
If a non-random $\Sigma$ is used, then $T$ is unconditionally Gaussian,
and one can take $\cH_0$ to be the trivial $\sigma$-algebra.
As demonstrated in our proof, our approach to establishing a
mixing approximation is different from naively taking an unconditional version
of Yurinskii's coupling and applying
it conditionally on $\cH_0$, which will not deliver the same coupling as in
Theorem~\ref{thm:yurinskii_sa_dependent} for a few reasons.
To begin with, we explicitly indicate in the
conditions of Theorem~\ref{thm:yurinskii_sa_dependent} where conditioning is
required.
Next, our error of approximation is given unconditionally,
involving only marginal expectations and probabilities.
Finally, we provide a rigorous account of the construction of the
conditionally Gaussian coupling variable $T$ via a conditional version
of Strassen's theorem \citep{chen2020jackknife}.
Section~\ref{sec:yurinskii_martingales}
illustrates how a strong approximation akin to
mixing convergence can arise when the data
forms an exact martingale, and Section~\ref{sec:yurinskii_factor} gives a
simple example
relating to factor modeling in statistics and data science.

% remove lower bound on minimum eigenvalue
As a third contribution to the literature
\ref{it:yurinskii_contribution_degeneracy}, and
of particular importance for applications,
Theorem~\ref{thm:yurinskii_sa_dependent} makes
no requirements on the minimum eigenvalue of the quadratic variation of the
approximating martingale sequence. Instead, our proof technique employs a
careful regularization scheme designed to account for any such exact or
approximate rank degeneracy in $\Sigma$. This capability is fundamental in some
applications, a fact which we illustrate in Section \ref{sec:yurinskii_kde} by
demonstrating the significant improvements in strong approximation errors
delivered by Theorem~\ref{thm:yurinskii_sa_dependent} relative to those
obtained using
prior results in the literature.

% matching third moments
Finally \ref{it:yurinskii_contribution_third_order},
Theorem~\ref{thm:yurinskii_sa_dependent} gives
a third-order strong approximation alongside the usual second-order
version considered in all prior literature.
More precisely, we observe that an analog of the term
$\beta_{p,2}$ is present in the
classical Yurinskii coupling and comes from a Lindeberg
telescoping sum argument,
replacing random variables by Gaussians with the same mean
and variance to match the first and second moments.
Whenever the third moments of $\tilde X_i$ are negligible
(quantified by $\pi_3$), this moment-matching argument can be extended to
third-order terms, giving a new term $\beta_{p,3}$.
In certain settings, such as when the data is symmetrically
distributed around zero, using $\beta_{p,3}$ rather than $\beta_{p,2}$
can give smaller approximation errors in the coupling given in
\eqref{eq:yurinskii_sa_dependent}.
Such a refinement can be viewed as a strong approximation counterpart
to classical Edgeworth expansion methods.
We illustrate this phenomenon in our
upcoming applications to nonparametric inference
(Section~\ref{sec:yurinskii_nonparametric}).

\subsection{User-friendly formulation of the main result}%

The result in Theorem~\ref{thm:yurinskii_sa_dependent} is given in a somewhat
implicit
manner, involving infima over the free parameters $t > 0$ and $M \succeq 0$,
and it is not clear how to compute these in general. In the upcoming
Proposition~\ref{pro:yurinskii_sa_simplified}, we set $M = \nu^2 I_d$ and
approximately
optimize over $t > 0$ and $\nu > 0$, resulting in a simplified and slightly
weaker version of our main general result. In specific applications, where
there is additional knowledge of the quadratic variation structure, other
choices of regularization schemes may be more appropriate. Nonetheless, the
choice $M = \nu^2 I_d$ leads to arguably the principal result of our work,
due to its simplicity and utility in statistical applications. For convenience,
define the functions $\phi_p : \N \to \R$ for $p \in [0, \infty]$,
%
\begin{align*}
\phi_p(d) =
\begin{cases}
\sqrt{pd^{2/p} } & \text{ if } p \in [1,\infty), \\
\sqrt{2\log 2d} & \text{ if } p =\infty,
\end{cases}
\end{align*}
%
which are related to tail probabilities
of the $\ell^p$-norm of a standard Gaussian.

\begin{proposition}[Simplified strong approximation
for approximate martingales]%
\label{pro:yurinskii_sa_simplified}

Assume the setup and notation of Theorem~\ref{thm:yurinskii_sa_dependent}.
For each $\eta > 0$ and $p \in [1,\infty]$,
there exists a random vector $T \mid \cH_0 \sim \cN(0, \Sigma)$ satisfying
%
\begin{align*}
\P\big(\|S-T\|_p > \eta\big)
&\leq
24 \left(
\frac{\beta_{p,2} \phi_p(d)^2}{\eta^3}
\right)^{1/3}
+ 17 \left(
\frac{\E \left[ \|\Omega\|_2 \right] \phi_p(d)^2}{\eta^2}
\right)^{1/3}
+\P\left(\|U\|_p>\frac{\eta}{6}\right).
\end{align*}
%
If further $\pi_3 = 0$ then
%
\begin{align*}
\P\big(\|S-T\|_p > \eta\big)
&\leq
24 \left(
\frac{\beta_{p,3} \phi_p(d)^3}{\eta^4}
\right)^{1/4}
+ 17 \left(
\frac{\E \left[ \|\Omega\|_2 \right] \phi_p(d)^2}{\eta^2}
\right)^{1/3}
+\P\left(\|U\|_p>\frac{\eta}{6}\right).
\end{align*}
%
\end{proposition}

Proposition~\ref{pro:yurinskii_sa_simplified} makes clear the potential benefit
of a
third-order coupling when $\pi_3 = 0$, as in this case the bound features
$\beta_{p,3}^{1/4}$ rather than $\beta_{p,2}^{1/3}$. If $\pi_3$ is small but
non-zero, an analogous result can easily be derived by adjusting the optimal
choices of $t$ and $\nu$, but we omit this for clarity of notation. In
applications (see Section~\ref{sec:yurinskii_series}), this reduction of the
exponent can
provide a significant improvement in terms of the dependence of the bound on
the sample size $n$, the dimension $d$, and other problem-specific quantities.
When using our results for strong approximation, it is usual to set
$p = \infty$ to bound the maximum discrepancy over the entries of a vector (to
construct uniform confidence sets, for example). In this setting, we have that
$\phi_\infty(d) = \sqrt{2 \log 2d}$ has a sub-Gaussian slow-growing dependence
on the dimension. The remaining term depends on $\E[\|\Omega\|_2]$ and requires
that the matrix $\Sigma$ be a good approximation of $\sum_{i=1}^{n} V_i$, while
remaining $\cH_0$-measurable. In some applications (such as factor modeling;
see Section~\ref{sec:yurinskii_factor}), it can be shown that the quadratic
variation
$\sum_{i=1}^n V_i$ remains random and $\cH_0$-measurable even in large samples,
giving a natural choice for $\Sigma$.

In the next few sections, we continue to refine
Proposition~\ref{pro:yurinskii_sa_simplified}, presenting a sequence of results
with
increasingly strict assumptions on the dependence structure of the data $X_i$.
These allow us to demonstrate the broad applicability of our main results,
providing more explicit bounds in settings which are likely to be of special
interest. In particular, we consider mixingales, martingales, and independent
data, comparing our derived results with those in the existing literature.

\subsection{Mixingales}
\label{sec:yurinskii_mixingales}

In our first refinement, we provide a natural method for bounding the
martingale approximation error term $U$. Suppose that $X_i$ form an
$\ell^p$-mixingale in $L^1(\P)$ in the sense that there exist non-negative
$c_1, \ldots, c_n$ and $\zeta_0, \ldots, \zeta_n$ such that for all
$1 \leq i \leq n$ and $0 \leq r \leq i$,
%
\begin{align}
\label{eq:yurinskii_mixingale_1}
\E \left[ \left\|
\E \left[ X_i \mid \cH_{i-r} \right]
\right\|_p \right]
&\leq
c_i \zeta_r,
\end{align}
%
and for all $1 \leq i \leq n$ and $0 \leq r \leq n-i$,
%
\begin{align}
\label{eq:yurinskii_mixingale_2}
\E \left[ \big\|
X_i - \E \big[ X_i \mid \cH_{i+r} \big]
\big\|_p \right]
&\leq
c_i \zeta_{r+1}.
\end{align}
%
These conditions are satisfied, for example, if $X_i$ are integrable strongly
$\alpha$-mixing random variables \citep{mcleish1975invariance}, or if $X_i$ are
generated by an auto-regressive or auto-regressive moving average process (see
Section~\ref{sec:yurinskii_factor}), among many other possibilities
\citep{bradley2005basic}. Then, in the notation of
Theorem~\ref{thm:yurinskii_sa_dependent}, we have by Markov's inequality that
%
\begin{align*}
\P \left( \|U\|_p > \frac{\eta}{6} \right)
&\leq
\frac{6}{\eta}
\sum_{i=1}^{n}
\E \left[
\big\|
X_i - \E \left[ X_i \mid \cH_n \right]
\big\|_p
+ \big\|
\E \left[ X_i \mid \cH_0 \right]
\big\|_p
\right]
\leq \frac{\zeta}{\eta},
\end{align*}
%
with $\zeta = 6 \sum_{i=1}^{n} c_i (\zeta_{i} + \zeta_{n-i+1})$.
Combining Proposition~\ref{pro:yurinskii_sa_simplified} with this
martingale error bound yields the following result for mixingales.
%
\begin{corollary}[Strong approximation for vector-valued mixingales]%
\label{cor:yurinskii_sa_mixingale}

Assume the setup and notation of Theorem~\ref{thm:yurinskii_sa_dependent},
and suppose
the mixingale conditions \eqref{eq:yurinskii_mixingale_1} and
\eqref{eq:yurinskii_mixingale_2} hold. For each $\eta > 0$ and
$p \in [1,\infty]$ there
is a random vector $T \mid \cH_0 \sim \cN(0, \Sigma)$ with
%
\begin{align*}
\P\big(\|S-T\|_p > \eta\big)
&\leq
24 \left(
\frac{\beta_{p,2} \phi_p(d)^2}{\eta^3}
\right)^{1/3}
+ 17 \left(
\frac{\E \left[ \|\Omega\|_2 \right] \phi_p(d)^2}{\eta^2}
\right)^{1/3}
+ \frac{\zeta}{\eta}.
\end{align*}
%
If further $\pi_3 = 0$ then
%
\begin{align*}
\P\big(\|S-T\|_p > \eta\big)
&\leq
24 \left(
\frac{\beta_{p,3} \phi_p(d)^3}{\eta^4}
\right)^{1/4}
+ 17 \left(
\frac{\E \left[ \|\Omega\|_2 \right] \phi_p(d)^2}{\eta^2}
\right)^{1/3}
+ \frac{\zeta}{\eta}.
\end{align*}
%
\end{corollary}

The closest antecedent to Corollary~\ref{cor:yurinskii_sa_mixingale} is found in
\citet[Theorem~4]{li2020uniform}, who also considered Yurinskii's coupling for
mixingales. Our result improves on this work in the following manner: it
removes any requirements on the minimum eigenvalue of the quadratic variation
of the mixingale sequence; it allows for general $\ell^p$-norms with
$p\in[1,\infty]$; it establishes a coupling to a multivariate Gaussian
mixture distribution in general; and it permits third-order couplings
(when $\pi_3=0$). These improvements have important practical implications as
demonstrated in Sections \ref{sec:yurinskii_factor} and
\ref{sec:yurinskii_nonparametric},
where significantly better coupling approximation
errors are demonstrated for a variety of statistical applications. On the
technical side, our result is rigorously established using a conditional
version of Strassen's theorem \citep{chen2020jackknife}, a carefully crafted
regularization argument, and a third-order Lindeberg method
\citep[see][and references therein, for more discussion on the
standard second-order Lindeberg method]{chatterjee2006generalization}.
Furthermore, as explained in
Remark~\ref{rem:yurinskii_coupling_bounds_probability}, we
clarify a technical issue in \citet{li2020uniform} surrounding the derivation
of valid probability bounds for $\|S-T\|_p$.

Corollary~\ref{cor:yurinskii_sa_mixingale} focused on mixingales for
simplicity, but, as
previously discussed, any method for constructing a martingale approximation
$\tilde X_i$ and bounding the resulting error $U$ could be used instead in
Proposition~\ref{pro:yurinskii_sa_simplified} to derive a similar result.

\subsection{Martingales}
\label{sec:yurinskii_martingales}

For our second refinement, suppose that
$X_i$ form martingale differences with respect to $\cH_i$.
In this case, $\E[X_i \mid \cH_n] = X_i$ and $\E[X_i \mid \cH_0] = 0$,
so $U = 0$, and the martingale approximation error term vanishes.
Applying Proposition~\ref{pro:yurinskii_sa_simplified} in this setting
directly yields the following result.
%
\begin{corollary}[Strong approximation for vector-valued martingales]%
\label{cor:yurinskii_sa_martingale}

With the setup and notation of Theorem~\ref{thm:yurinskii_sa_dependent},
suppose that
$X_i$ is $\cH_i$-measurable satisfying $\E[X_i \mid \cH_{i-1}] = 0$ for
$1 \leq i \leq n$. Then, for each $\eta > 0$ and $p \in [1,\infty]$, there is
a random vector $T \mid \cH_0 \sim \cN(0, \Sigma)$ with
%
\begin{align}
\label{eq:yurinskii_sa_martingale_order_2}
\P\big(\|S-T\|_p > \eta\big)
&\leq
24 \left(
\frac{\beta_{p,2} \phi_p(d)^2}{\eta^3}
\right)^{1/3}
+ 17 \left(
\frac{\E \left[ \|\Omega\|_2 \right] \phi_p(d)^2}{\eta^2}
\right)^{1/3}.
\end{align}
%
If further $\pi_3 = 0$ then
%
\begin{align}
\label{eq:yurinskii_sa_martingale_order_3}
\P\big(\|S-T\|_p > \eta\big)
&\leq
24 \left(
\frac{\beta_{p,3} \phi_p(d)^3}{\eta^4}
\right)^{1/4}
+ 17 \left(
\frac{\E \left[ \|\Omega\|_2 \right] \phi_p(d)^2}{\eta^2}
\right)^{1/3}.
\end{align}
%
\end{corollary}

The closest antecedents to Corollary~\ref{cor:yurinskii_sa_martingale} are
\citet{belloni2018high} and \citet{li2020uniform}, who also implicitly or
explicitly considered Yurinskii's coupling for martingales. More specifically,
\citet[Theorem~1]{li2020uniform} established an explicit
$\ell^2$-norm Yurinskii coupling
for martingales under a strong assumption on the minimum eigenvalue of the
martingale quadratic variation, while \citet[Theorem~2.1]{belloni2018high}
established a central limit theorem for vector-valued martingale sequences
employing the standard second-order Lindeberg method, implying that their proof
could be adapted to deduce a Yurinskii coupling for martingales with the help
of a conditional version of Strassen's theorem \citep{chen2020jackknife} and
some additional nontrivial technical work.

Corollary~\ref{cor:yurinskii_sa_martingale} improves over this prior work as
follows.
With respect to \citet{li2020uniform}, our result establishes an $\ell^p$-norm
Gaussian mixture Yurinskii coupling for martingales without any requirements on
the minimum eigenvalue of the martingale quadratic variation, and permits a
third-order coupling if $\pi_3=0$. The first probability bound
\eqref{eq:yurinskii_sa_martingale_order_2} in
Corollary~\ref{cor:yurinskii_sa_martingale} gives the
same rate of strong approximation as that in Theorem~1 of \citet{li2020uniform}
when $p=2$, with non-random $\Sigma$, and when the eigenvalues of a normalized
version of $\Sigma$ are bounded away from zero. In
Section~\ref{sec:yurinskii_kde} we
demonstrate the crucial importance of removing this eigenvalue lower bound
restriction in applications involving nonparametric kernel estimators, while in
Section~\ref{sec:yurinskii_series} we demonstrate how the availability of a
third-order
coupling \eqref{eq:yurinskii_sa_martingale_order_3} can give improved
approximation rates
in applications involving nonparametric series estimators with conditionally
symmetrically distributed residual errors. Finally, our technical work improves
on \citet{li2020uniform} in two respects:
%
\begin{inlineroman}
\item
we employ a conditional version
of Strassen's theorem (see Lemma~\ref{lem:yurinskii_app_strassen}
in the appendix)
to appropriately handle the conditioning arguments; and
\item
we deduce valid
probability bounds for $\|S-T\|_p$, as the following
Remark~\ref{rem:yurinskii_coupling_bounds_probability} makes clear.
\end{inlineroman}

\begin{remark}[Yurinskii's coupling and bounds in probability]
\label{rem:yurinskii_coupling_bounds_probability}
Given a sequence of random vectors $S_n$, Yurinskii's method provides a
coupling in the following form: for each $n$ and any $\eta > 0$, there exists
a random vector $T_n$ with $\P\big(\|S_n - T_n\| > \eta\big) < r_n(\eta)$,
where $r_n(\eta)$ is the approximation error. Crucially, each coupling
variable $T_n$ is a function of the desired approximation level $\eta$ and,
as such, deducing bounds in probability on $\|S_n - T_n\|$ requires some
extra care. One option is to select a sequence $R_n \to \infty$ and note that
$\P\big(\|S_n - T_n\| > r_n^{-1}(1 / R_n)\big) < 1 / R_n \to 0$ and hence
$\|S_n - T_n\| \lesssim_\P r_n^{-1}(1 / R_n)$. In this case, $T_n$ depends on
the choice of $R_n$, which can in turn typically be chosen to diverge slowly
enough to cause no issues in applications.
\end{remark}

Technicalities akin to those outlined in
Remark~\ref{rem:yurinskii_coupling_bounds_probability} have been both addressed
and
neglected alike in the prior literature. \citet[Chapter 10.4, Example
16]{pollard2002user} apparently misses this subtlety, providing an
inaccurate bound in probability based on the Yurinskii coupling.
\citet{li2020uniform} seem to make the same mistake in the proof of their
Lemma~A2, which invalidates the conclusion of their Theorem~1. In contrast,
\citet{belloni2015some} and \citet{belloni2019conditional} directly provide
bounds in $o_\P$ instead of $O_\P$, circumventing these issues in a manner
similar to our approach involving a diverging sequence $R_n$.

To see how this phenomenon applies to our main results, observe that the
second-order martingale coupling given as
\eqref{eq:yurinskii_sa_martingale_order_2} in
Corollary~\ref{cor:yurinskii_sa_martingale} implies that for any
$R_n \to \infty$,
%
\begin{align*}
\|S - T\|_p
\lesssim_\P
\beta_{p,2}^{1/3}
\phi_p(d)^{2/3} R_n
+ \E[\|\Omega\|_2]^{1/2}
\phi_p(d) R_n.
\end{align*}
%
This bound is comparable to that obtained by \citet[Theorem~1]{li2020uniform}
with $p=2$, albeit with their formulation missing the $R_n$ correction terms.
In Section~\ref{sec:yurinskii_series} we discuss further their (amended)
result, in the
setting of nonparametric series estimation. Our approach using
$p = \infty$ obtains superior distributional approximation rates, alongside
exhibiting various other improvements such as the aforementioned third-order
coupling.

Turning to the comparison with \citet{belloni2018high}, our
Corollary~\ref{cor:yurinskii_sa_martingale} again offers the same improvements,
with the
only exception being that the authors did account for the implications of a
possibly vanishing minimum eigenvalue. However, their results exclusively
concern high-dimensional central limit theorems for vector-valued martingales,
and therefore while their findings
could in principle enable the derivation of a result similar to our
Corollary~\ref{cor:yurinskii_sa_martingale}, this would require additional
technical work
on their behalf in multiple ways
(see Appendix~\ref{app:yurinskii}):
%
\begin{inlineroman}
\item a correct application of a conditional
version of Strassen's theorem
(Lemma~\ref{lem:yurinskii_app_strassen});
\item the development of a third-order Borel set smoothing technique and
associated $\ell^p$-norm moment control
(Lemmas \ref{lem:yurinskii_app_smooth_approximation},
\ref{lem:yurinskii_app_gaussian_useful},
and \ref{lem:yurinskii_app_gaussian_pnorm});
\item a careful truncation scheme to account for
$\Omega\npreceq0$; and
\item a valid third-order Lindeberg argument
(Lemma \ref{lem:yurinskii_app_sa_martingale}),
among others.
\end{inlineroman}

\subsection{Independence}

As a final refinement, suppose that $X_i$ are independent and
zero-mean conditionally on $\cH_0$,
and take $\cH_i$ to be the filtration
generated by $X_1, \ldots, X_i$ and $\cH_0$ for $1 \leq i \leq n$.
Then, taking $\Sigma = \sum_{i=1}^n V_i$
gives $\Omega = 0$, and hence Corollary~\ref{cor:yurinskii_sa_martingale}
immediately yields the following result.
%
\begin{corollary}[Strong approximation for sums of independent vectors]%
\label{cor:yurinskii_sa_indep}

Take the setup of Theorem~\ref{thm:yurinskii_sa_dependent},
and let $X_i$ be independent given $\cH_0$,
with $\E[X_i \mid \cH_0] = 0$.
Then, for each $\eta > 0$ and $p \in [1,\infty]$,
with $\Sigma = \sum_{i=1}^n V_i$,
there is $T \mid \cH_0 \sim \cN(0, \Sigma)$ with
%
\begin{align}
\label{eq:yurinskii_sa_indep_order_2}
\P\big(\|S-T\|_p > \eta\big)
&\leq 24 \left( \frac{\beta_{p,2} \phi_p(d)^2}{\eta^3} \right)^{1/3}.
\end{align}
%
If further $\pi_3 = 0$ then
%
\begin{align*}
\P\big(\|S-T\|_p > \eta\big)
&\leq 24 \left( \frac{\beta_{p,3} \phi_p(d)^3}{\eta^4} \right)^{1/4}.
\end{align*}
%
\end{corollary}

Taking $\cH_0$ to be trivial,
\eqref{eq:yurinskii_sa_indep_order_2} provides an $\ell^p$-norm approximation
analogous to that presented in \citet{belloni2019conditional}.
By further
restricting to $p=2$, we recover the original Yurinskii coupling as presented
in \citet[Theorem~1]{lecam1988} and \citet[Theorem~10]{pollard2002user}. Thus,
in the independent data setting, our result improves on prior work as follows:
\begin{inlineroman}
\item
it establishes a coupling to a multivariate Gaussian mixture distribution;
and
\item
it permits a third-order coupling if $\pi_3=0$.
\end{inlineroman}

\subsection{Stylized example: factor modeling}
\label{sec:yurinskii_factor}

In this section, we present a simple statistical example of how our
improvements over prior coupling results can have important theoretical and
practical implications. Consider the stylized factor model
%
\begin{align*}
X_i = L f_i + \varepsilon_i, \qquad 1 \leq i \leq n,
\end{align*}
%
with random variables $L$ taking values in $\R^{d \times m}$, $f_i$ in $\R^m$,
and $\varepsilon_i$ in $\R^d$. We interpret $f_i$ as a latent factor variable
and $L$ as a random factor loading, with idiosyncratic disturbances
$\varepsilon_i$. See \citet{fan2020statistical}, and references therein, for a
textbook review of factor analysis in statistics and econometrics.

We employ the above factor model to give a first illustration of the
applicability of our main result Theorem~\ref{thm:yurinskii_sa_dependent}, the
user-friendly Proposition~\ref{pro:yurinskii_sa_simplified}, and their
specialized
Corollaries~\ref{cor:yurinskii_sa_mixingale}--\ref{cor:yurinskii_sa_indep}. We
consider three different sets of conditions to demonstrate the applicability of
each of our corollaries for mixingales, martingales, and independent data,
respectively. We assume throughout that
$(\varepsilon_1, \ldots, \varepsilon_n)$ is zero-mean and finite variance, and
that $(\varepsilon_1, \ldots, \varepsilon_n)$ is independent
of $L$ and $(f_1, \ldots, f_n)$. Let $\cH_i$ be the $\sigma$-algebra generated
by $L$, $(f_1, \ldots, f_i)$, and $(\varepsilon_1, \ldots, \varepsilon_i)$, with
$\cH_0$ the $\sigma$-algebra generated by $L$ alone.

\begin{itemize}
\item \emph{Independent data}.
Suppose that the factors $(f_1, \ldots,
f_n)$ are independent conditional on $L$ and satisfy
$\E [ f_i \mid L ] = 0$.
Then, since $X_i$ are independent conditional on $\cH_0$ and with
$\E [ X_i \mid \cH_0 ] = \E [ L f_i + \varepsilon_i \mid L ] = 0$,
we can apply Corollary~\ref{cor:yurinskii_sa_indep} to $\sum_{i=1}^n X_i$.
In general, we will obtain a coupling variable which has the Gaussian
mixture distribution $T \mid \cH_0 \sim \cN(0, \Sigma)$ where
$\Sigma= \sum_{i=1}^n (L\Var[f_i \mid L]L^\T +\Var[\varepsilon_i])$.
In the special case where $L$ is non-random
and $\cH_0$ is trivial, the coupling is Gaussian. Further,
if $f_i\mid L$ and $\varepsilon_i$ are symmetric about zero
and bounded, then $\pi_3=0$, and the coupling is improved.

\item \emph{Martingales}.
Suppose instead that we assume only a martingale
condition on the latent factor variables so that
$\E \left[ f_i \mid L, f_1, \ldots, f_{i-1} \right] = 0$.
Then $\E [ X_i \mid \cH_{i-1} ]
= L\, \E \left[ f_i \mid \cH_{i-1} \right] = 0$
and Corollary~\ref{cor:yurinskii_sa_martingale} is applicable to
$\sum_{i=1}^n X_i$.
The preceding comments on Gaussian mixture distributions
and third-order couplings continue to apply.

\item \emph{Mixingales}.
Finally, assume that the factors follow the
auto-regressive model $f_i = A f_{i-1} + u_i$ where
$A \in \R^{m \times m}$ is non-random and $(u_1, \ldots, u_n)$ are
zero-mean, independent, and independent of
$(\varepsilon_1, \ldots, \varepsilon_n)$.
Then $\E \left[ f_i \mid f_0 \right] = A^i f_0$, so taking
$p \in [1, \infty]$ we see that
$\E \big[ \| \E [ f_i \mid f_0 ] \|_p \big]
= \E \big[ \| A^i f_0 \|_p \big] \leq \|A\|_p^i\,\E [ \|f_0\|_p ]$,
and that clearly $f_i - \E [ f_i \mid \cH_n ] = 0$.
Thus, whenever $\|A\|_p < 1$, the geometric sum formula implies that
we can apply the mixingale result from
Corollary~\ref{cor:yurinskii_sa_mixingale} to
$\sum_{i=1}^n X_i$. The conclusions on Gaussian mixture distributions
and third-order couplings parallel the previous cases.
%
\end{itemize}

This simple application to factor modeling gives a preliminary illustration of
the power of our main results, encompassing settings which could not be handled
by employing Yurinskii couplings available in the existing literature. Even
with independent data, we offer new Yurinskii couplings to Gaussian mixture
distributions (due to the presence of the common random factor loading $L$),
which could be further improved whenever the factors and residuals possess
symmetric (conditional) distributions. Furthermore, our results do not impose
any restrictions on the minimum eigenvalue of $\Sigma$, thereby allowing for
more general factor structures. These improvements are maintained in the
martingale, mixingale, and weakly dependent stationary data settings.

\section{Strong approximation for martingale empirical processes}%
\label{sec:yurinskii_emp_proc}

In this section, we demonstrate how our main results can be applied to some more
substantive problems in statistics. Having until this point studied only
finite-dimensional (albeit potentially high-dimensional) random vectors, we now
turn our attention to infinite-dimensional stochastic processes. Specifically,
we consider empirical processes of the form
$S(f) = \sum_{i=1}^{n} f(X_i)$ for $f \in \cF$
a problem-specific class of real-valued
functions, where each $f(X_i)$ forms a martingale difference sequence with
respect to an appropriate filtration. We construct (conditionally) Gaussian
processes $T(f)$ for which an upper bound on the uniform coupling error
$\sup_{f \in \cF} |S(f) - T(f)|$ is precisely quantified. We control the
complexity of $\cF$ using metric entropy under Orlicz norms.

The novel strong approximation results which we present concern the entire
martingale empirical process $(S(f):f \in \cF)$, as opposed to just the scalar
supremum of the empirical process, $\sup_{f \in \cF} |S(f)|$. This distinction
has been carefully noted by \citet{chernozhukov2014gaussian}, who studied
Gaussian approximation of empirical process suprema in the independent data
setting and wrote (p.\ $1565$): ``A related but different problem is that of
approximating \textit{whole} empirical processes by a sequence of Gaussian
processes in the sup-norm. This problem is more difficult than
[approximating the supremum of the empirical process].''
Indeed, the results we establish in
this section are for a strong approximation for the entire empirical process by
a sequence of Gaussian mixture processes in the supremum norm, when the data
has a martingale difference structure
(cf.\ Corollary \ref{cor:yurinskii_sa_martingale}).
Our results can be further generalized to approximate martingale
empirical processes (cf.\ Corollary \ref{cor:yurinskii_sa_mixingale}), but we
do not
consider this extension to reduce notation and the technical burden.

\subsection{Motivating example: kernel density estimation}
\label{sec:yurinskii_kde}

We begin with a brief study of a canonical example of an empirical process
which is non-Donsker (thus precluding the use of uniform central limit
theorems) due to the presence of a function class whose complexity increases
with the sample size: the kernel density estimator with i.i.d.\ scalar data.
We give an overview of our general strategy for
strong approximation of stochastic processes
via discretization, and show explicitly in
Lemma~\ref{lem:yurinskii_kde_eigenvalue}
how it is crucial
that we do not impose lower bounds on the eigenvalues of the discretized
covariance matrix. Detailed calculations for this section are
relegated to Appendix~\ref{app:yurinskii} for conciseness.

Let $X_1, \ldots, X_n$ be i.i.d.\ $\Unif[0,1]$, take
$K(x) = \frac{1}{\sqrt{2 \pi}} e^{-x^2/2}$ the Gaussian kernel and let
$h \in (0,1]$ be a bandwidth. Then, for $a \in (0,1/4]$ and
$x \in \cX = [a, 1-a]$ to avoid boundary issues, the kernel density estimator
of the true density function $g(x) = 1$ is
%
\begin{align*}
\hat g(x)
&=
\frac{1}{n}
\sum_{i=1}^{n}
K_h( X_i - x),
\qquad K_h(u) = \frac{1}{h} K\left( \frac{u}{h} \right).
\end{align*}
%
Consider establishing a strong approximation for the stochastic process
$(\hat g(x)-\E [ \hat g(x) ] : x\in\cX)$
which is, upon rescaling, non-Donsker whenever
the bandwidth decreases to zero in large samples.
To match notation with the upcoming
general result for empirical processes, set
$f_x(u) = \frac{1}{n} (K_h( u - x) - \E[K_h( X_i - x)])$
so $S(x) \vcentcolon= S(f_x) = \hat g(x)-\E [ \hat g(x) ]$.
The next step is standard: a
mesh separates the local oscillations of the processes from
the finite-dimensional coupling.
For $\delta \in (0,1/2)$, set
$N = \left\lfloor 1 + \frac{1 - 2a}{\delta} \right\rfloor$
and $\cX_\delta = (a + (j-1)\delta : 1 \leq j \leq N)$.
Letting $T(x)$ be the approximating stochastic
process to be constructed, consider the decomposition
%
\begin{align*}
\sup_{x \in \cX}
\big|S(x) - T(x)\big|
&\leq
\sup_{|x-x'| \leq \delta}
\big|S(x) - S(x') \big|
+ \max_{x \in \cX_\delta}
|S(x) - T(x)|
+ \sup_{|x-x'| \leq \delta}
\big|T(x) - T(x')\big|.
\end{align*}
%
Writing $S(\cX_\delta)$ for
$\big(S(x) : x \in \cX_\delta\big)\in \mathbb{R}^N$,
noting that this is a sum of i.i.d.\ random vectors, we apply
Corollary~\ref{cor:yurinskii_sa_indep} as
$\max_{x \in \cX_\delta} |S(x) - T(x)|
= \| S(\cX_\delta) - T(\cX_\delta) \|_\infty$.
We obtain that for each $\eta > 0$ there is a Gaussian vector
$T(\cX_\delta)$ with the same covariance matrix as $S(\cX_\delta)$ satisfying
%
\begin{align*}
\P\left(
\|S(\cX_\delta) - T(\cX_\delta)\|_\infty > \eta
\right)
&\leq
31 \left(
\frac{N \log 2 N}{\eta^3 n^2 h^2}
\right)^{1/3}
\end{align*}
%
assuming that $1/h \geq \log 2 N$.
By the Vorob'ev--Berkes--Philipp theorem
\citep[Theorem~1.1.10]{dudley1999uniform},
$T(\cX_\delta)$ extends to a Gaussian process $T(x)$
defined for all $x \in \cX$ and with the same covariance structure
as $S(x)$.

Next, chaining with the Bernstein--Orlicz and sub-Gaussian norms
\citep[Section~2.2]{van1996weak} shows that if
$\log(N/h) \lesssim \log n$ and $n h \gtrsim \log n$,
%
\begin{align*}
\sup_{|x-x'| \leq \delta}
\big\|S(x) - S(x') \big\|_\infty
&\lesssim_\P
\delta
\sqrt{\frac{\log n}{n h^3}} \ \quad\text{and}\quad
\sup_{|x-x'| \leq \delta}
\big\|T(x) - T(x')\big\|_\infty
\lesssim_\P
\delta
\sqrt{\frac{\log n}{n h^3}}.
\end{align*}
%
Finally, for any $R_n\to\infty$
(see Remark~\ref{rem:yurinskii_coupling_bounds_probability}),
the resulting bound on the coupling error is
%
\begin{align*}
\sup_{x \in \cX}
\big| S(x) - T(x) \big|
&\lesssim_\P
\left( \frac{N \log 2N}{n^2 h^2} \right)^{1/3} R_n
+ \delta \sqrt{\frac{\log n}{n h^3}},
\end{align*}
%
where the mesh size $\delta$ can then be approximately
optimized to obtain the tightest possible strong approximation.

The discretization strategy outlined above is at the core of the proof strategy
for our upcoming Proposition~\ref{pro:yurinskii_emp_proc}. Since we will
consider
martingale empirical processes, our proof will rely on
Corollary~\ref{cor:yurinskii_sa_martingale}, which, unlike the martingale
Yurinskii
coupling established by \citet{li2020uniform}, does not require a lower bound
on the minimum eigenvalue of $\Sigma$. Using the simple kernel density example
just discussed, we now demonstrate precisely the crucial importance of removing
such eigenvalue conditions. The following
Lemma~\ref{lem:yurinskii_kde_eigenvalue} shows
that the discretized covariance matrix $\Sigma = n h\Var[S(\cX_\delta)]$ has
exponentially small eigenvalues, which in turn will negatively affect the
strong approximation bound if the \citet{li2020uniform} coupling were to be
used instead of the results in this dissertation.

\begin{lemma}[Minimum eigenvalue of a
kernel density estimator covariance matrix]%
\label{lem:yurinskii_kde_eigenvalue}
%
The minimum eigenvalue of
$\Sigma=n h\Var[S(\cX_\delta)] \in \R^{N \times N}$
satisfies the upper bound
%
\begin{align*}
\lambda_{\min}(\Sigma)
&\leq
2 e^{-h^2/\delta^2}
+ \frac{h}{\pi a \delta}
e^{-a^2 / h^2}.
\end{align*}
\end{lemma}
%
Figure~\ref{fig:yurinskii_min_eig} shows how the upper bound in Lemma
\ref{lem:yurinskii_kde_eigenvalue} captures the behavior of the simulated
minimum
eigenvalue of $\Sigma$. In particular, the smallest eigenvalue decays
exponentially fast in the discretization level $\delta$ and the bandwidth $h$.
As seen in the calculations above, the coupling rate depends on $\delta / h$,
while the bias will generally depend on $h$, implying that both $\delta$ and
$h$ must converge to zero to ensure valid statistical inference. In general,
this will lead to $\Sigma$ possessing extremely small eigenvalues, rendering
strong approximation approaches such as that of \citet{li2020uniform}
ineffective in such scenarios.
%
\begin{figure}[t]
\centering
\begin{subfigure}{0.49\textwidth}
\centering
%\includegraphics[scale=0.64]{graphics/sim_2.pdf}
\caption{$h = 0.03$}
\end{subfigure}
\begin{subfigure}{0.49\textwidth}
\centering
%\includegraphics[scale=0.64]{graphics/sim_1.pdf}
\caption{$h = 0.01$}
\end{subfigure}
\caption[Minimum eigenvalue of the kernel density covariance matrix]{
Upper bounds on the minimum eigenvalue of the discretized covariance
matrix in kernel density estimation,
with $n=100$ and $a = 0.2$.
Simulated: the kernel density estimator is simulated,
resampling the data $100$ times
to estimate its covariance.
Computing matrix: the minimum eigenvalue of the limiting covariance
matrix $\Sigma$ is computed explicitly.
Upper bound: the bound derived in
Lemma~\ref{lem:yurinskii_kde_eigenvalue}
is shown.
}
\label{fig:yurinskii_min_eig}
\end{figure}

The discussion in this section focuses on the strong approximation of the
centered process $\hat g(x)-\E [ \hat g(x) ]$. In practice, the goal is often
rather to approximate the feasible process $\hat g(x)- g(x)$. The difference
between these is captured by the smoothing bias $\E [ \hat g(x) ] - g(x)$,
which is straightforward to control in this case with
$\sup_{x \in \cX} \big| \E [ \hat g(x) ] - g(x) \big|
\lesssim \frac{h}{a} e^{-a^2 / (2 h^2)}$.
See Section \ref{sec:yurinskii_nonparametric} for further
comments.

\subsection{General result for martingale empirical processes}

We now give our general result on a strong approximation for
martingale empirical processes, obtained by applying
the first result \eqref{eq:yurinskii_sa_martingale_order_2} in
Corollary~\ref{cor:yurinskii_sa_martingale} with $p=\infty$
to a discretization of the empirical process,
as in Section~\ref{sec:yurinskii_kde}.
We then control the increments in the stochastic processes
using chaining with Orlicz norms,
but note that other tools are available,
including generalized entropy with bracketing \citep{geer2000empirical}
and sequential symmetrization \citep{rakhlin2015sequential}.

A class of functions is said to be \emph{pointwise measurable}
if it contains a countable subclass which is dense under
the pointwise convergence topology.
For a finite class $\cF$, write
$\cF(x) = \big(f(x) : f \in \cF\big)$.
Define the set of Orlicz functions
%
\begin{align*}
\Psi
&=
\left\{
\psi: [0, \infty) \to [0, \infty)
\text{ convex increasing, }
\psi(0) = 0,\
\limsup_{x,y \to \infty} \tfrac{\psi(x) \psi(y)}{\psi(C x y)} < \infty
\text{ for } C > 0
\right\}
\end{align*}
%
and, for real-valued $Y$, the Orlicz norm
$\vvvert Y \vvvert_\psi
= \inf
\left\{ C > 0:
\E \left[ \psi(|Y|/C) \leq 1 \right]
\right\}$
as in \citet[Section~2.2]{van1996weak}.

\begin{proposition}[Strong approximation for martingale empirical processes]%
\label{pro:yurinskii_emp_proc}

Let $X_i$ be random variables for $1 \leq i \leq n$ taking values in a
measurable space $\cX$, and $\cF$ be a pointwise measurable class of
functions from $\cX$ to $\R$. Let $\cH_0, \ldots, \cH_n$ be a filtration such
that each $X_i$ is $\cH_i$-measurable, with $\cH_0$ the trivial
$\sigma$-algebra, and suppose that $\E[f(X_i) \mid \cH_{i-1}] = 0$ for all
$f \in \cF$. Define $S(f) = \sum_{i=1}^n f(X_i)$ for $f\in\cF$ and let
$\Sigma: \cF \times \cF \to \R$ be an almost surely positive semi-definite
$\cH_0$-measurable random function. Suppose that for a non-random
metric $d$ on $\cF$, constant $L$, and $\psi \in \Psi$,
%
\begin{align}%
\label{eq:yurinskii_emp_proc_var}
\Sigma(f,f) - 2\Sigma(f,f') + \Sigma(f',f')
+ \bigvvvert S(f) - S(f') \bigvvvert_\psi^2
&\leq L^2 d(f,f')^2 \quad \text{a.s.}
\end{align}
%
Then for each $\eta > 0$ there is a process $T(f)$
which, conditional on $\cH_0$, is zero-mean and Gaussian,
satisfying $\E\big[ T(f) T(f') \mid \cH_0 \big] = \Sigma(f,f')$
for all $f, f' \in \cF$, and for all $t > 0$ has
%
\begin{align*}
&\P\left(
\sup_{f \in \cF}
\big| S(f) - T(f) \big|
\geq C_\psi(t + \eta)
\right)
\leq
C_\psi
\inf_{\delta > 0}
\inf_{\cF_\delta}
\Bigg\{
\frac{\beta_\delta^{1/3} (\log 2 |\cF_\delta|)^{1/3}}{\eta } \\
&\qquad\quad+
\left(\frac{\sqrt{\log 2 |\cF_\delta|}
\sqrt{\E\left[\|\Omega_\delta\|_2\right]}}{\eta }\right)^{2/3}
+ \psi\left(\frac{t}{L J_\psi(\delta)}\right)^{-1}
+ \exp\left(\frac{-t^2}{L^2 J_2(\delta)^2}\right)
\Bigg\}
\end{align*}
%
where $\cF_\delta$ is any finite $\delta$-cover of $(\cF,d)$
and $C_\psi$ is a constant depending only on $\psi$, with
%
\begin{align*}
\beta_\delta
&= \sum_{i=1}^n
\E\left[ \|\cF_\delta(X_i)\|^2_2\|\cF_\delta(X_i)\|_\infty
+ \|V_i(\cF_\delta)^{1/2}Z_i\|^2_2
\|V_i(\cF_\delta)^{1/2}Z_i\|_\infty \right], \\
V_i(\cF_\delta)
&=
\E\big[\cF_\delta(X_i) \cF_\delta(X_i)^\T \mid \cH_{i-1} \big],
\hspace*{27.7mm}
\Omega_\delta
=
\sum_{i=1}^n V_i(\cF_\delta) - \Sigma(\cF_\delta), \\
J_\psi(\delta)
&=
\int_0^\delta \psi^{-1}\big( N_\varepsilon \big)
\diff{\varepsilon}
+ \delta \psi^{-1} \big( N_\delta^2 \big),
\hspace*{19mm}
J_2(\delta)
= \int_0^\delta \sqrt{\log N_\varepsilon}
\diff{\varepsilon},
\end{align*}
%
where $N_\delta = N(\delta, \cF, d)$
is the $\delta$-covering number of $(\cF, d)$
and $Z_i$ are i.i.d.\ $\cN\big(0, I_{|\cF_\delta|}\big)$
independent of $\cH_n$.
If $\cF_\delta$ is a minimal $\delta$-cover
of $(\cF, d)$, then $|\cF_\delta| = N_\delta$.
\end{proposition}

Proposition~\ref{pro:yurinskii_emp_proc}
is given in a rather general form to accommodate a range of different
settings and applications.
In particular, consider the following well-known Orlicz functions.
%
\begin{description}

\item[Polynomial:]
$\psi(x) = x^a$ for $a \geq 2$
has $\vvvert X \vvvert_2 \leq \vvvert X \vvvert_\psi$ and
$\sqrt{\log x} \leq \sqrt{a} \psi^{-1}(x)$.

\item[Exponential:]
$\psi(x) = \exp(x^a) - 1$ for $a \in [1,2]$
has $\vvvert X \vvvert_2 \leq 2\vvvert X \vvvert_\psi$ and
$\sqrt{\log x} \leq \psi^{-1}(x)$.

\item[Bernstein:]
$\psi(x) = \exp
\Big(
\Big(\frac{\sqrt{1+2ax}-1}{a}\Big)^{2}
\Big)-1$
for $a > 0$ has
$\vvvert X \vvvert_2 \leq (1+a)\vvvert X \vvvert_\psi$ \\ and
$\sqrt{\log x}~\leq~\psi^{-1}(x)$.

\end{description}
%
For these Orlicz functions and when $\Sigma(f, f') = \Cov[S(f), S(f')]$ is
non-random, the terms involving $\Sigma$ in \eqref{eq:yurinskii_emp_proc_var}
can be
controlled by the Orlicz $\psi$-norm term; similarly, $J_2$ is bounded by
$J_\psi$. Further, $C_\psi$ can be replaced by a universal constant $C$ which
does not depend on the parameter $a$. See Section~2.2 in \citet{van1996weak}
for details. If the conditional third moments of $f(X_i)$ given $\cH_{i-1}$ are
all zero (if $f$ and $X_i$ are appropriately symmetric, for example), then the
second inequality in Corollary~\ref{cor:yurinskii_sa_martingale} can be applied
to obtain
a tighter coupling inequality; the details of this are omitted for brevity, and
the proof would proceed in exactly the same manner.

In general, however, Proposition~\ref{pro:yurinskii_emp_proc} allows for a
random
covariance function, yielding a coupling to a stochastic process that is
Gaussian only conditionally. Such a process can equivalently be viewed as a
mixture of Gaussian processes, writing $T=\Sigma^{1/2} Z$ with an operator
square root and where $Z$ is a Gaussian white noise on $\cF$ independent of
$\cH_0$. This extension is in contrast with much of the existing strong
approximation and empirical process literature, which tends to focus on
couplings and weak convergence results with marginally Gaussian processes
\citep{settati2009gaussian,chernozhukov2016empirical}.

A similar approach was taken by \citet{berthet2006revisiting}, who used a
Gaussian coupling due to \citet{zaitsev1987estimates,zaitsev1987gaussian} along
with a discretization method to obtain strong approximations for empirical
processes with independent data. They handled fluctuations in the stochastic
processes with uniform $L^2$ covering numbers and bracketing numbers where we
opt instead for chaining with Orlicz norms. Our version using the martingale
Yurinskii coupling can improve upon theirs in approximation rate even for
independent data in certain circumstances. Suppose the setup of
Proposition~1 in \citet{berthet2006revisiting}; that is, $X_1, \ldots, X_n$ are
i.i.d.\ and $\sup_{\cF} \|f\|_\infty \leq M$, with the VC-type assumption
$\sup_\Q N(\varepsilon, \cF, d_\Q) \leq c_0 \varepsilon^{-\nu_0}$ where
$d_\Q(f,f')^2 = \E_\Q\big[(f-f')^2\big]$ for a measure $\Q$ on $\cX$ and
$M, c_0, \nu_0$ are constants. Using uniform $L^2$ covering numbers
rather than Orlicz chaining in our Proposition~4 gives the following.
Firstly, as $X_i$ are i.i.d., take $\Sigma(f, f') = \Cov[S(f), S(f')]$ so
$\Omega_\delta = 0$. Let $\cF_\delta$ be a minimal $\delta$-cover of
$(\cF, d_\P)$ with cardinality $N_\delta \lesssim \delta^{-\nu_0}$ where
$\delta \to 0$. It is easy to show that
$\beta_\delta \lesssim n \delta^{-\nu_0} \sqrt{\log(1/\delta)}$.
Theorem~2.2.8 and Theorem~2.14.1 in \citet{van1996weak} then give
%
\begin{align*}
\E\left[
\sup_{d_\P(f,f') \leq \delta}
\Big(
|S(f) - S(f')|
+ |T(f) - T(f')|
\Big)
\right]
&\lesssim
\sup_\Q
\int_0^\delta
\sqrt{n \log N(\varepsilon, \cF, d_\Q)}
\diff{\varepsilon} \\
&\lesssim
\delta \sqrt{n\log(1/\delta)},
\end{align*}
%
where we used the VC-type property to bound the entropy integral.
So by our Proposition~\ref{pro:yurinskii_emp_proc},
for any sequence $R_n \to \infty$
(see Remark~\ref{rem:yurinskii_coupling_bounds_probability}),
%
\begin{align*}
\sup_{f \in \cF}
\big| S(f) - T(f) \big|
&\lesssim_\P
n^{1/3} \delta^{-\nu_0/3}
\sqrt{\log(1/\delta)} R_n
+ \delta \sqrt{n\log(1/\delta)}
\lesssim_\P
n^{\frac{2+\nu_0}{6+2\nu_0}}
\sqrt{\log n} R_n,
\end{align*}
%
where we minimized over $\delta$ in the last step.
\citet[Proposition~1]{berthet2006revisiting} achieved
%
\begin{align*}
\sup_{f \in \cF}
\big| S(f) - T(f) \big|
&\lesssim_\P
n^{\frac{5\nu_0}{4+10\nu_0}}
(\log n)^{\frac{4+5\nu_0}{4+10\nu_0}},
\end{align*}
%
showing that our approach achieves a better approximation rate whenever
$\nu_0 > 4/3$. In particular, our method is superior in richer function classes
with larger VC-type dimension. For example, if $\cF$ is smoothly parameterized
by $\theta \in \Theta \subseteq \R^d$ where $\Theta$ contains an open set, then
$\nu_0 > 4/3$ corresponds to $d \geq 2$ and our rate is better as soon as the
parameter space is more than one-dimensional. The difference in approximation
rate is due to Zaitsev's coupling having better dependence on the sample size
but worse dependence on the dimension. In particular, Zaitsev's coupling is
stated only in $\ell^2$-norm and hence
\citet[Equation~5.3]{berthet2006revisiting} are compelled to use the inequality
$\|\cdot\|_\infty \leq \|\cdot\|_2$ in the coupling step, a bound which is
loose when the dimension of the vectors (here on the order of
$\delta^{-\nu_0}$) is even moderately large. We use the fact that our version
of Yurinskii's coupling applies directly to the supremum norm, giving sharper
dependence on the dimension.

In Section~\ref{sec:yurinskii_local_poly} we apply
Proposition~\ref{pro:yurinskii_emp_proc} to
obtain strong approximations for local polynomial estimators in the
nonparametric regression setting. In contrast with the series estimators of the
upcoming Section~\ref{sec:yurinskii_series}, local polynomial estimators are
not linearly
separable and hence cannot be analyzed directly using the finite-dimensional
Corollary~\ref{cor:yurinskii_sa_martingale}.

\section{Applications to nonparametric regression}
\label{sec:yurinskii_nonparametric}

We illustrate the applicability of our previous strong approximation results
with two substantial and classical examples in nonparametric regression
estimation. Firstly, we present an analysis of partitioning-based series
estimators, where we can apply Corollary~\ref{cor:yurinskii_sa_martingale}
directly due to an intrinsic linear separability property. Secondly, we
consider local polynomial estimators, this time using
Proposition~\ref{pro:yurinskii_emp_proc} due to a non-linearly separable
martingale empirical process.

\subsection{Partitioning-based series estimators}
\label{sec:yurinskii_series}

Partitioning-based least squares methods are essential tools for estimation and
inference in nonparametric regression, encompassing splines, piecewise
polynomials, compactly supported wavelets and decision trees as special cases.
See \citet{cattaneo2020large} for further details and references throughout
this section. We illustrate the usefulness of
Corollary~\ref{cor:yurinskii_sa_martingale}
by deriving a Gaussian strong approximation for partitioning series estimators
based on multivariate martingale data. Proposition~\ref{pro:yurinskii_series}
shows how
we achieve the best known rate of strong approximation for independent data by
imposing an additional mild $\alpha$-mixing condition to control the time
series dependence of the regressors.

Consider the nonparametric regression setup with martingale difference
residuals defined by $Y_i = \mu(W_i) + \varepsilon_i$ for $ 1 \leq i \leq n$
where the regressors $W_i$ have compact connected support $\cW \subseteq \R^m$,
$\cH_i$ is the $\sigma$-algebra generated by
$(W_1, \ldots, W_{i+1}, \varepsilon_1, \ldots, \varepsilon_i)$,
$\E[\varepsilon_i \mid \cH_{i-1}] = 0$ and $\mu: \cW \to \R$ is the estimand.
Let $p(w)$ be a $k$-dimensional vector of bounded basis functions on $\cW$
which are locally supported on a quasi-uniform partition
\citep[Assumption~2]{cattaneo2020large}. Under minimal regularity conditions,
the least-squares partitioning-based series estimator is
$\hat\mu(w) = p(w)^{\T} \hat H^{-1} \sum_{i=1}^n p(W_i) Y_i$
with $\hat H = \sum_{i=1}^n p(W_i) p(W_i)^\T$.
The approximation power of the estimator $\hat\mu(w)$ derives from letting
$k\to\infty$ as $n\to\infty$. The assumptions made on $p(w)$ are mild enough to
accommodate splines, wavelets, piecewise polynomials, and certain types of
decision trees. For such a tree, $p(w)$ is comprised of indicator functions
over $k$ axis-aligned rectangles forming a partition of $\cW$ (a Haar basis),
provided that the partitions are constructed using independent data
(e.g., with sample splitting).

Our goal is to approximate the law of the stochastic process
$(\hat\mu(w)-\mu(w):w\in\cW)$, which upon rescaling is typically not
asymptotically tight as $k \to \infty$ and thus does not converge weakly.
Nevertheless, exploiting the intrinsic linearity of the estimator $\hat\mu(w)$,
we can apply Corollary~\ref{cor:yurinskii_sa_martingale} directly to construct
a Gaussian
strong approximation. Specifically, we write
%
\begin{equation*}
\hat\mu(w) - \mu(w)
= p(w)^\T H^{-1} S
+ p(w)^\T \big(\hat H^{-1} - H^{-1}\big) S
+ \Bias(w),
\end{equation*}
%
where $H= \sum_{i=1}^n \E\left[p(W_i) p(W_i)^\T\right]$
is the expected outer product matrix, $S = \sum_{i=1}^n p(W_i) \varepsilon_i$
is the score vector, and
$\Bias(w) = p(w)^{\T} \hat H^{-1}\sum_{i=1}^n p(W_i) \mu(W_i) - \mu(w)$.
Imposing some mild time series restrictions and assuming stationarity,
it is not difficult to show
(see Section~\ref{sec:yurinskii_app_proofs})
that $\|\hat H - H\|_1 \lesssim_\P \sqrt{n k}$ and
$\sup_{w\in\cW} |\Bias(w)| \lesssim_\P k^{-\gamma}$
for some $\gamma>0$, depending on the specific structure of the basis
functions, the dimension $m$ of the regressors, and the smoothness of the
regression function $\mu$. It remains to study the $k$-dimensional
mean-zero martingale $S$ by applying
Corollary~\ref{cor:yurinskii_sa_martingale} with
$X_i=p(W_i) \varepsilon_i$. Controlling the convergence of the quadratic
variation term $\E[\|\Omega\|_2]$ requires some time series dependence
assumptions; we impose an $\alpha$-mixing condition on $(W_1, \ldots, W_n)$ for
illustration \citep{bradley2005basic}.

\begin{proposition}[Strong approximation for partitioning series estimators]%
\label{pro:yurinskii_series}
%
Consider the nonparametric regression setup described above
and further assume the following:
%
\begin{enumerate}[label=(\roman*)]

\item
$(W_i, \varepsilon_i)_{1 \leq i \leq n}$
is strictly stationary.

\item
$W_1, \ldots, W_n$ is $\alpha$-mixing with mixing coefficients
satisfying $\sum_{j=1}^\infty \alpha(j) < \infty$.

\item
$W_i$ has a Lebesgue density on $\cW$
which is bounded above and away from zero.

\item
$\E\big[|\varepsilon_i|^3 \big] < \infty$
and
$\E\big[\varepsilon_i^2 \mid \cH_{i-1}\big]=\sigma^2(W_i)$
is bounded away from zero.

\item
$p(w)$ is a basis with $k$ features satisfying
Assumptions~2 and~3 in \citet{cattaneo2020large}.

\end{enumerate}
%
Then, for any sequence $R_n \to \infty$,
there is a zero-mean Gaussian process
$G(w)$ indexed on $\cW$
with $\Var[G(w)] \asymp\frac{k}{n}$
satisfying
$\Cov[G(w), G(w')]
= \Cov[p(w)^\T H^{-1} S,\, p(w')^\T H^{-1} S]$
and
%
\begin{align*}
\sup_{w \in \cW}
\left| \hat\mu(w) - \mu(w) - G(w) \right|
&\lesssim_\P
\sqrt{\frac{k}{n}}
\left( \frac{k^3 (\log k)^3}{n} \right)^{1/6} R_n
+ \sup_{w \in \cW} |\Bias(w)|
\end{align*}
%
assuming the number of basis functions satisfies $k^3 / n \to 0$.
If further $\E \left[ \varepsilon_i^3 \mid \cH_{i-1} \right] = 0$ then
%
\begin{align*}
\sup_{w \in \cW}
\left| \hat\mu(w) - \mu(w) - G(w) \right|
&\lesssim_\P
\sqrt{\frac{k}{n}}
\left( \frac{k^3 (\log k)^2}{n} \right)^{1/4} R_n
+ \sup_{w \in \cW} |\Bias(w)|.
\end{align*}
%
\end{proposition}

The core concept in the proof of Proposition~\ref{pro:yurinskii_series} is to
apply
Corollary~\ref{cor:yurinskii_sa_martingale} with
$S = \sum_{i=1}^n p(W_i) \varepsilon_i$
and $p=\infty$ to construct $T \sim \cN\big(0, \Var[S]\big)$ such that
$\|S - T \|_\infty$ is small, and then setting $G(w) = p(w)^\T H^{-1} T$. So
long as the bias can be appropriately controlled, this result allows for
uniform inference procedures such as uniform confidence bands or shape
specification testing. The condition $k^3 / n \to 0$ is the same (up to logs)
as that imposed by \citet{cattaneo2020large} for i.i.d. data, which gives the
best known strong approximation rate for this problem. Thus,
Proposition~\ref{pro:yurinskii_series} gives the same best approximation rate
without
requiring any extra restrictions for $\alpha$-mixing time series data.

Our results improve substantially on \citet[Theorem~1]{li2020uniform}: using
the notation of our Corollary~\ref{cor:yurinskii_sa_martingale}, and with any
sequence
$R_n \to \infty$, a valid (see
Remark~\ref{rem:yurinskii_coupling_bounds_probability})
version of their martingale Yurinskii coupling is
%
\begin{align*}
\|S-T\|_2
\lesssim_\P
d^{1/2} r^{1/2}_n
+ (B_n d)^{1/3} R_n,
\end{align*}
%
where $B_n = \sum_{i=1}^n \E[\|X_i\|_2^3]$ and $r_n$ is a term controlling the
convergence of the quadratic variation, playing a similar role to our
term $\E[\|\Omega\|_2]$. Under the assumptions of our
Proposition~\ref{pro:yurinskii_series}, applying this
result with $S = \sum_{i=1}^n p(W_i) \varepsilon_i$ yields a rate no better
than $\|S-T\|_2 \lesssim_\P (n k)^{1/3} R_n$. As such, they attain a rate of
strong approximation no faster than
%
\begin{align*}
\sup_{w \in \cW}
\left| \hat\mu(w) - \mu(w) - G(w) \right|
&\lesssim_\P
\sqrt{\frac{k}{n}}
\left( \frac{k^5}{n} \right)^{1/6} R_n
+ \sup_{w \in \cW} |\Bias(w)|.
\end{align*}
%
Hence, for this approach to yield a valid strong approximation, the number of
basis functions must satisfy $k^5/n \to 0$, a more restrictive assumption than
our $k^3 / n \to 0$ (up to logs). This difference is due to
\citet{li2020uniform} using the $\ell^2$-norm version of Yurinskii's coupling
rather than the recently established $\ell^\infty$ version. Further,
our approach allows for an improved rate of distributional approximation
whenever the residuals have zero conditional third moment.

To illustrate the statistical applicability of
Proposition~\ref{pro:yurinskii_series}, consider constructing a feasible uniform
confidence band for the regression function $\mu$, using standardization and
Studentization for statistical power improvements. We assume throughout that
the bias is negligible. Proposition~\ref{pro:yurinskii_series} and
anti-concentration for
Gaussian suprema \citep[Corollary~2.1]{chernozhukov2014anti} yield
a distributional approximation for the supremum statistic whenever
$k^3(\log n)^6 / n \to 0$, giving
%
\begin{align*}
\sup_{t \in \R}
\left|
\P\left(
\sup_{w \in \cW}
\left|
\frac{\hat\mu(w)-\mu(w)}{\sqrt{\rho(w,w)}}
\right| \leq t
\right)
-
\P\left(
\sup_{w \in \cW}
\left|
\frac{G(w)}{\sqrt{\rho(w,w)}}
\right| \leq t
\right)
\right|
&\to 0,
\end{align*}
%
where $\rho(w,w') = \E[G(w)G(w')]$. Further, by a Gaussian--Gaussian
comparison result \citep[Lemma~3.1]{chernozhukov2013gaussian} and
anti-concentration, we show (see the proof of
Proposition~\ref{pro:yurinskii_series}) that with $\bW = (W_1, \ldots, W_n)$ and
$\bY = (Y_1, \ldots, Y_n)$,
%
\begin{align*}
\sup_{t \in \R}
\left|
\P\left(
\sup_{w \in \cW}
\left|
\frac{\hat\mu(w)-\mu(w)}{\sqrt{\hat\rho(w,w)}}
\right| \leq t
\right)
- \P\left(
\sup_{w \in \cW}
\left|
\frac{\hat G(w)}{\sqrt{\hat\rho(w,w)}}
\right| \leq t \biggm| \bW, \bY
\right)
\right|
&\to_\P 0,
\end{align*}
%
where $\hat G(w)$ is a zero-mean Gaussian process
conditional on $\bW$ and $\bY$ with conditional covariance function
$\hat\rho(w,w')
=\E\big[\hat G(w) \hat G(w') \mid \bW, \bY \big]
= p(w)^\T \hat H^{-1} \hat V \hat H^{-1}p(w')$
for some estimator $\hat V$ satisfying
$\frac{k (\log n)^2}{n}
\big\|\hat V-\Var[S]\big\|_2 \to_\P 0$.
For example, one could use the plug-in estimator
$\hat V=\sum_{i=1}^n p(W_i) p(W_i)^\T \hat{\sigma}^2(W_i)$
where $\hat{\sigma}^2(w)$ satisfies
$(\log n)^2 \sup_{w \in \cW}
|\hat{\sigma}^2(w)-\sigma^2(w)| \to_\P 0$.
This leads to the following feasible and asymptotically valid
$100(1-\tau)\%$
uniform confidence band for partitioning-based series estimators
based on martingale data.

\begin{proposition}[Feasible uniform confidence bands for partitioning
series estimators]%
\label{pro:yurinskii_series_feasible}
%
Assume the setup of the preceding section. Then
%
\begin{align*}
\P\Big(
\mu(w) \in
\Big[
\hat\mu(w) \pm \hat q(\tau)
\sqrt{\hat\rho(w,w)}
\Big]
\ \text{for all }
w \in \cW \Big)
\to 1-\tau,
\end{align*}
%
where
%
\begin{align*}
\hat{q}(\tau)
&=
\inf
\left\{
t \in \R:
\P\left(
\sup_{w \in \cW}
\left|
\frac{\hat G(w)}{\sqrt{\hat\rho(w,w)}}
\right|
\leq t
\Bigm| \bW, \bY
\right)
\geq \tau
\right\}
\end{align*}
%
is the conditional quantile of the supremum of the Studentized Gaussian
process. This can be estimated by resampling the conditional law of
$\hat G(w) \mid \bW, \bY$ with a discretization of $w \in \cW$.
\end{proposition}

\subsection{Local polynomial estimators}
\label{sec:yurinskii_local_poly}

As a second example application we consider nonparametric regression estimation
with martingale data employing local polynomial methods
\citep{fan1996local}. In contrast with the partitioning-based series
methods of Section~\ref{sec:yurinskii_series}, local polynomials induce
stochastic
processes which are not linearly separable, allowing us to showcase the
empirical process result given in Proposition \ref{pro:yurinskii_emp_proc}.

As before, suppose that
$Y_i = \mu(W_i) + \varepsilon_i$
for $ 1 \leq i \leq n$
where $W_i$ has compact connected support $\cW \subseteq \R^m$,
$\cH_i$ is the $\sigma$-algebra generated by
$(W_1, \ldots, W_{i+1}, \varepsilon_1, \ldots, \varepsilon_i)$,
$\E[\varepsilon_i \mid \cH_{i-1}] = 0$,
and $\mu: \cW \to \R$ is the estimand. Let $K$ be a kernel function on $\R^m$
and $K_h(w) = h^{-m} K(w/h)$ for some bandwidth $h > 0$.
Take $\gamma \geq 0$ a fixed polynomial order and let
$k = (m+\gamma)!/(m!\gamma!)$ be the number of monomials up to order $\gamma$.
Using multi-index notation,
let $p(w)$ be the $k$-dimensional vector
collecting the monomials $w^{\kappa}/\kappa!$
for $0 \leq |\kappa| \leq \gamma$,
and set $p_h(w) = p(w/h)$.
The local polynomial regression estimator of $\mu(w)$ is,
with $e_1 = (1, 0, \ldots, 0)^\T \in \R^k$ the first standard unit vector,
%
\begin{align*}
\hat{\mu}(w)
&=
e_1^\T\hat{\beta}(w)
&\text{where} &
&\hat{\beta}(w)
&=
\argmin_{\beta \in \R^{k}}
\sum_{i=1}^n
\left(Y_i - p_h(W_i-w)^\T \beta \right)^2
K_h(W_i-w).
\end{align*}

Our goal is again to approximate the distribution of the entire stochastic
process, $(\hat{\mu}(w)-\mu(w):w\in\cW)$, which upon rescaling is non-Donsker
if $h \to 0$, and decomposes as follows:
%
\begin{align*}
\hat{\mu}(w)-\mu(w)
&= e_1^\T H(w)^{-1} S(w)
+ e_1^\T \big(\hat H(w)^{-1} - H(w)^{-1}\big) S(w)
+ \Bias(w)
\end{align*}
%
where
$\hat H(w) = \sum_{i=1}^n K_h(W_i-w) p_h(W_i-w) p_h(W_i-w)^\T$,
$H(w) = \E \big[ \hat H(w) \big]$,
$S(w)= \sum_{i=1}^n K_h(W_i-w) p_h(W_i-w) \varepsilon_i$,
and
$\Bias(w) = e_1^\T \hat H(w)^{-1}
\sum_{i=1}^n K_h(W_i-w) p_h(W_i-w) \mu(W_i) - \mu(w)$.
A key distinctive feature of local polynomial regression is that both
$\hat H(w)$ and $S(w)$ are functions of the evaluation point $w\in\cW$;
contrast this with the partitioning-based series estimator discussed in
Section~\ref{sec:yurinskii_series}, for which neither $\hat H$ nor $S$ depend
on $w$.
Therefore we use Proposition \ref{pro:yurinskii_emp_proc} to obtain a Gaussian
strong
approximation for the martingale empirical process directly.

Under mild regularity conditions, including stationarity for simplicity
and an $\alpha$-mixing assumption on the time-dependence of the data, we show
$\sup_{w\in\cW} \|\hat H(w)-H(w)\|_2
\lesssim_\P \sqrt{n h^{-2m}\log n}$.
Further,
$\sup_{w\in\cW} |\Bias(w)|
\lesssim_\P h^\gamma$
provided that the regression function is sufficiently smooth.
It remains to analyze the martingale empirical process given by
$\big(e_1^\T H(w)^{-1} S(w) : w\in\cW\big)$
via Proposition \ref{pro:yurinskii_emp_proc} by setting
%
\begin{align*}
\cF = \left\{
(W_i, \varepsilon_i) \mapsto
e_1^\T H(w)^{-1}
K_h(W_i-w) p_h(W_i-w) \varepsilon_i
: w \in \cW
\right\}.
\end{align*}
%
With this approach, we obtain the following result.

\begin{proposition}[Strong approximation for local polynomial estimators]%
\label{pro:yurinskii_local_poly}

Under the nonparametric regression setup described above,
assume further that
%
\begin{enumerate}[label=(\roman*)]

\item
$(W_i, \varepsilon_i)_{1 \leq i \leq n}$
is strictly stationary.

\item
$(W_i, \varepsilon_i)_{1 \leq i \leq n}$
is $\alpha$-mixing with mixing coefficients
$\alpha(j) \leq e^{-2 j / C_\alpha}$
for some $C_\alpha > 0$.

\item
$W_i$ has a Lebesgue density on $\cW$
which is bounded above and away from zero.

\item
$\E\big[e^{|\varepsilon_i|/C_\varepsilon}\big] < \infty$
for $C_\varepsilon > 0$ and
$\E\left[\varepsilon^2_i \mid \cH_{i-1}\right]=\sigma^2(W_i)$
is bounded away from zero.

\item
$K$ is a non-negative Lipschitz
compactly supported kernel with
$\int K(w) \diff{w} = 1$.

\end{enumerate}
%
Then for any $R_n \to \infty$,
there is a zero-mean Gaussian process
$T(w)$ on $\cW$
with $\Var[T(w)] \asymp\frac{1}{n h^m}$
satisfying
$\Cov[T(w), T(w')]
= \Cov[e_1^\T H(w)^{-1} S(w),\, e_1^\T H(w')^{-1} S(w')]$
and
%
\begin{align*}
\sup_{w \in \cW}
\left|\hat \mu(w) - \mu(w) - T(w) \right|
&\lesssim_\P
\frac{R_n}{\sqrt{n h^m}}
\left(
\frac{(\log n)^{m+4}}{n h^{3m}}
\right)^{\frac{1}{2m+6}}
+ \sup_{w \in \cW} |\Bias(w)|,
\end{align*}
%
provided that the bandwidth sequence satisfies
$n h^{3m} \to \infty$.
%
\end{proposition}

If the residuals further satisfy
$\E \left[ \varepsilon_i^3 \mid \cH_{i-1} \right] = 0$, then
a third-order Yurinskii coupling delivers an improved rate of strong
approximation for Proposition~\ref{pro:yurinskii_local_poly}; this is omitted
here for
brevity. For completeness, the proof of
Proposition~\ref{pro:yurinskii_local_poly}
verifies that if the regression function $\mu(w)$ is $\gamma$ times
continuously differentiable on $\cW$ then
$\sup_w |\Bias(w)| \lesssim_\P h^\gamma$. Further, the assumption that $p(w)$
is a vector of monomials is unnecessary in general; any collection of bounded
linearly independent functions which exhibit appropriate approximation power
will suffice \citep{eggermont2009maximum}. As such, we can encompass local
splines and wavelets, as well as polynomials, and also choose whether or not to
include interactions between the regressor variables. The bandwidth restriction
of $n h^{3m} \to \infty$ is analogous to that imposed in
Proposition~\ref{pro:yurinskii_series} for partitioning-based series
estimators, and as
far as we know, has not been improved upon for non-i.i.d.\ data.

Applying an anti-concentration result for Gaussian process suprema, such as
Corollary~2.1 in \citet{chernozhukov2014anti}, allows one to write a
Kolmogorov--Smirnov bound comparing the law of
$\sup_{w \in \cW}|\hat\mu(w) - \mu(w)|$ to that of $\sup_{w \in \cW}|T(w)|$.
With an appropriate covariance estimator, we can further replace $T(w)$ by a
feasible version $\hat T(w)$ or its Studentized counterpart, enabling
procedures for uniform inference analogous to the confidence bands constructed
in Section~\ref{sec:yurinskii_series}. We omit the details of this to conserve
space but
note that our assumptions on $W_i$ and $\varepsilon_i$ ensure that
Studentization is possible even when the discretized covariance matrix has
small eigenvalues (Section~\ref{sec:yurinskii_kde}), as we normalize only by
the diagonal
entries. \citet[Remark~3.1]{chernozhukov2014gaussian} achieve better rates for
approximating the supremum of the $t$-process based on i.i.d.\ data in
Kolmogorov--Smirnov distance by bypassing the step where we first approximate
the entire stochastic process (see Section~\ref{sec:yurinskii_emp_proc} for a
discussion).
Nonetheless, our approach targeting the entire process allows for a
potential future
treatment of other functionals as well as the supremum.

We finally remark that in this setting of kernel-based local empirical
processes, it is essential that our initial strong approximation result
(Corollary~\ref{cor:yurinskii_sa_martingale}) does not impose a lower bound on
the
eigenvalues of the variance matrix $\Sigma$. This effect was demonstrated by
Lemma \ref{lem:yurinskii_kde_eigenvalue},
Figure~\ref{fig:yurinskii_min_eig}, and their surrounding discussion in
Section~\ref{sec:yurinskii_kde}. As such, the result of \citet{li2020uniform} is
unsuited for this application, even in its simplest formulation,
due to the strong minimum eigenvalue assumption.

\section{Conclusion}
\label{sec:yurinskii_conclusion}

In this chapter we introduced as our main result a new version of Yurinskii's
coupling which strictly generalizes all previously known forms of the result.
Our formulation gave a Gaussian mixture coupling for approximate martingale
vectors in $\ell^p$-norm where $1 \leq p \leq \infty$, with no restrictions on
the minimum eigenvalues of the associated covariance matrices. We further
showed how to obtain an improved approximation whenever third moments of the
data are negligible. We demonstrated the applicability of this main result by
first deriving a user-friendly version, and then specializing it to mixingales,
martingales, and independent data, illustrating the benefits with a collection
of simple factor models. We then considered the problem of constructing uniform
strong approximations for martingale empirical processes, demonstrating how our
new Yurinskii coupling can be employed in a stochastic process setting. As
substantive illustrative applications of our theory to some
well-established problems in statistical methodology, we showed how to use our
coupling results for both vector-valued and empirical process-valued
martingales in developing uniform inference procedures for partitioning-based
series estimators and local polynomial models in nonparametric regression. At
each stage we addressed issues of feasibility, compared our work with the
existing literature, and provided implementable statistical inference
procedures. The work in this chapter is based on \citet{cattaneo2022yurinskii}.

\appendix


\chapter{Supplement to Inference with Mondrian Random Forests}
\label{app:mondrian}

In this section we present the full proofs of all our results,
and also state some useful technical preliminary and
intermediate lemmas, along with some further properties
of the Mondrian process not required for our primary analysis.
See Section~\ref{sec:mondrian_overview_proofs} in the main text
for an overview of the main proof strategies and a discussion of
the challenges involved.
We use the following simplified notation for convenience,
whenever it is appropriate.
We write $\I_{i b}(x) = \I \left\{ X_i \in T_b(x) \right\}$
and $N_b(x) = \sum_{i=1}^{n} \I_{i b}(x)$,
as well as $\I_b(x) = \I \left\{ N_b(x) \geq 1 \right\}$.

\section{Preliminary lemmas}

We begin by bounding the maximum size of any cell
in a Mondrian forest containing $x$.
This result is used regularly throughout many of our other proofs,
and captures the ``localizing'' behavior of the Mondrian random
forest estimator, showing that Mondrian cells have side lengths
at most on the order of $1/\lambda$.

\begin{lemma}[Upper bound on the largest cell in a Mondrian forest]%
\label{lem:mondrian_app_largest_cell}
%
Let $T_1, \ldots, T_b \sim \cM\big([0,1]^d, \lambda\big)$
and take $x \in (0,1)^d$. Then for all $t > 0$
%
\begin{align*}
\P \left(
\max_{1 \leq b \leq B}
\max_{1 \leq j \leq d}
|T_b(x)_j|
\geq \frac{t}{\lambda}
\right)
&\leq
2dB e^{-t/2}.
\end{align*}

\end{lemma}

\begin{proof}[Lemma~\ref{lem:mondrian_app_largest_cell}]
%
We use the distribution of the Mondrian cell shape
\citep[Proposition~1]{mourtada2020minimax}. We have
$|T_b(x)_j| = \left( \frac{E_{bj1}}{\lambda} \wedge x_j \right)
+ \left( \frac{E_{bj2}}{\lambda} \wedge (1-x_j) \right)$
where $E_{bj1}$ and $E_{bj2}$
are i.i.d.\ $\Exp(1)$ variables for
$1 \leq b \leq B$ and $1 \leq j \leq d$.
Thus $|T_b(x)_j| \leq \frac{E_{bj1} + E_{bj2}}{\lambda}$
so by a union bound
%
\begin{align*}
\P \left(
\max_{1 \leq b \leq B}
\max_{1 \leq j \leq d}
|T_b(x)_j|
\geq \frac{t}{\lambda}
\right)
&\leq
\P \left(
\max_{1 \leq b \leq B}
\max_{1 \leq j \leq d}
(E_{bj1} \vee E_{bj2})
\geq \frac{t}{2}
\right) \\
&\leq
2dB\,
\P \left(
E_{bj1}
\geq \frac{t}{2}
\right)
\leq
2dB e^{-t/2}.
\end{align*}
%
\end{proof}

Next is another localization result,
showing that the union
of the cells $T_b(x)$ containing $x$ does not contain ``too many''
samples $X_i$.
Thus the Mondrian random forest estimator fitted at $x$
only depends on $n/\lambda^d$ (the effective sample size)
data points up to logarithmic terms.

\begin{lemma}[Upper bound on the number of active data points]%
\label{lem:mondrian_app_active_data}
Suppose Assumptions~\ref{ass:mondrian_data} and \ref{ass:mondrian_estimator}
hold,
and define
$N_{\cup}(x) =
\sum_{i=1}^{n} \I \left\{ X_i \in \bigcup_{b=1}^{B} T_b(x) \right\}$.
Then for $t > 0$ and sufficiently large $n$,
with $\|f\|_\infty = \sup_{x \in [0,1]^d} f(x)$,
%
\begin{align*}
\P \left( N_{\cup}(x) > t^{d+1}
\frac{n}{\lambda^d}
\|f\|_\infty
\right)
&\leq
4 d B e^{-t/4}.
\end{align*}
\end{lemma}

\begin{proof}[Lemma~\ref{lem:mondrian_app_active_data}]

Note
$N_\cup(x) \sim
\Bin\left(n, \int_{\bigcup_{b=1}^{B} T_b(x)} f(s) \diff s \right)
\leq \Bin\left(n, 2^d \max_{1 \leq b \leq B} \max_{1 \leq j \leq d}
|T_b(x)_j|^d \|f\|_\infty \right)$
conditionally on $\bT$.
If $N \sim \Bin(n,p)$ then, by Bernstein's inequality,
$\P\left( N \geq (1 + t) n p\right)
\leq \exp\left(-\frac{t^2 n^2 p^2 / 2}{n p(1-p) + t n p / 3}\right)
\leq \exp\left(-\frac{3t^2 n p}{6 + 2t}\right)$.
Thus for $t \geq 2$,
%
\begin{align*}
\P \left( N_{\cup}(x) > (1+t) n \frac{2^d t^d}{\lambda^d}
\|f\|_\infty
\Bigm| \max_{1 \leq b \leq B} \max_{1 \leq j \leq d}
|T_j(x)| \leq \frac{t}{\lambda}
\right)
&\leq
\exp\left(- \frac{2^d t^{d} n}{\lambda^d}\right).
\end{align*}
%
By Lemma~\ref{lem:mondrian_app_largest_cell},
$\P \left( \max_{1 \leq b \leq B} \max_{1 \leq j \leq d}
|T_j(x)| > \frac{t}{\lambda} \right)
\leq 2 d B e^{-t/2}$.
Hence
%
\begin{align*}
&\P \left( N_{\cup}(x) > 2^{d+1} t^{d+1} \frac{n}{\lambda^d}
\|f\|_\infty
\right) \\
&\quad\leq
\P \left( N_{\cup}(x) > 2 t n \frac{2^d t^d}{\lambda^d}
\|f\|_\infty
\Bigm| \max_{1 \leq b \leq B} \max_{1 \leq j \leq d}
|T_j(x)| \leq \frac{t}{\lambda}
\right)
+ \P \left( \max_{1 \leq b \leq B} \max_{1 \leq j \leq d}
|T_j(x)| > \frac{t}{\lambda}
\right) \\
&\quad\leq
\exp\left(- \frac{2^d t^{d} n}{\lambda^d}\right)
+ 2 d B e^{-t/2}.
\end{align*}
%
Replacing $t$ by $t/2$ gives that for sufficiently large $n$ such that
$n / \lambda^d \geq 1$,
%
\begin{align*}
\P \left( N_{\cup}(x) > t^{d+1}
\frac{n}{\lambda^d}
\|f\|_\infty
\right)
&\leq
4 d B e^{-t/4}.
\end{align*}
%
\end{proof}

Next we give a series of results culminating in a
generalized moment bound for the denominator appearing
in the Mondrian random forest estimator.
We begin by providing a moment bound for the truncated inverse binomial
distribution, which will be useful for controlling
$\frac{\I_b(x)}{N_b(x)} \leq 1 \wedge \frac{1}{N_b(x)}$
because conditional on $T_b$ we have
$N_b(x) \sim \Bin \left( n, \int_{T_b(x)} f(s) \diff s \right)$.
Our constants could be significantly suboptimal but they are sufficient
for our applications.

\begin{lemma}[An inverse moment bound for the binomial distribution]%
\label{lem:mondrian_app_binomial_bound}
For $n \geq 1$ and $p \in [0,1]$,
let $N \sim \Bin(n, p)$ and $a_1, \ldots, a_k \geq 0$.
Then
%
\begin{align*}
\E\left[
\prod_{j=1}^k
\left(
1 \wedge
\frac{1}{N + a_j}
\right)
\right]
&\leq
(9k)^k
\prod_{j=1}^k
\left(
1 \wedge
\frac{1}{n p + a_j}
\right).
\end{align*}
\end{lemma}

\begin{proof}[Lemma~\ref{lem:mondrian_app_binomial_bound}]
By Bernstein's inequality,
$\P\left( N \leq n p - t \right)
\leq \exp\left(-\frac{t^2/2}{n p(1-p) + t/3}\right)
\leq \exp\left(-\frac{3t^2}{6n p + 2t}\right)$.
Therefore we have
$\P\left( N \leq n p/4 \right)
\leq \exp\left(-\frac{27 n^2 p^2 / 16}{6n p + 3 n p / 2}\right)
= e^{-9 n p / 40}$.
Partitioning by this event gives
%
\begin{align*}
\E\left[
\prod_{j=1}^k
\left(
1 \wedge
\frac{1}{N + a_j}
\right)
\right]
&\leq
e^{-9 n p / 40}
\prod_{j=1}^k
\frac{1}{1 \vee a_j}
+ \prod_{j=1}^k
\frac{1}{1 \vee (\frac{n p}{4} + a_j)} \\
&\leq
\prod_{j=1}^k
\frac{1}{\frac{9 n p}{40k} + (1 \vee a_j)}
+ \prod_{j=1}^k
\frac{1}{1 \vee (\frac{n p}{4} + a_j)} \\
&\leq
\prod_{j=1}^k
\frac{1}{1 \vee \left(\frac{9 n p}{40k} + a_j\right)}
+ \prod_{j=1}^k
\frac{1}{1 \vee (\frac{n p}{4} + a_j)} \\
&\leq
2 \prod_{j=1}^k
\frac{1}{1 \vee \left(\frac{9 n p}{40k} + a_j\right)}
\leq
2 \prod_{j=1}^k
\frac{40k/9}{1 \vee \left(n p + a_j\right)} \\
&\leq
(9k)^k
\prod_{j=1}^k
\left(
1 \wedge
\frac{1}{n p + a_j}
\right).
\end{align*}
\end{proof}

Our next result is probably the most technically involved,
allowing one to bound moments of
(products of) $\frac{\I_b(x)}{N_b(x)}$ by the corresponding moments of
(products of) $\frac{1}{n |T_b(x)|}$, again based on the heuristic
that $N_b(x)$ is conditionally binomial so concentrates around
its conditional expectation
$n \int_{T_b(x)} f(x) \diff s \asymp n |T_b(x)|$.
By independence of the trees,
the latter expected products then factorize
since the dependence on the data $X_i$ has been eliminated.
The proof is complicated, and relies on the following induction procedure.
First we consider the common refinement consisting of the
subcells $\cR$ generated by all possible intersections
of $T_b(x)$ over the selected trees
(say $T_{b}(x), T_{b'}(x), T_{b''}(x)$
though there could be arbitrarily many).
Note that $N_b(x)$ is the sum of the number of
samples $X_i$ in each such subcell in $\cR$.
We then apply Lemma~\ref{lem:mondrian_app_binomial_bound} repeatedly
to each subcell in $\cR$ in turn, replacing
the number of samples $X_i$ in that subcell with its volume
multiplied by $n$, and controlling the error incurred at each step.
We record the subcells which have been ``checked'' in this manner
using the class $\cD \subseteq \cR$ and proceed by finite induction,
beginning with $\cD = \emptyset$ and ending at $\cD = \cR$.

\begin{lemma}[Generalized moment bound for
Mondrian random forest denominators]%
\label{lem:mondrian_app_moment_denominator}

Suppose Assumptions~\ref{ass:mondrian_data}
and \ref{ass:mondrian_estimator} hold.
Let $T_b \sim \cM\big([0,1]^d, \lambda\big)$
be independent and $k_b \geq 1$ for $1 \leq b \leq B_0$.
Then with $k = \sum_{b=1}^{B_0} k_b$,
for sufficiently large $n$,
%
\begin{align*}
\E\left[
\prod_{b=1}^{B_0}
\frac{\I_b(x)}{N_b(x)^{k_b}}
\right]
&\leq
\left( \frac{36k}{\inf_{x \in [0,1]^d} f(x)} \right)^{2^{B_0} k}
\prod_{b=1}^{B_0}
\E \left[
1 \wedge
\frac{1}{(n |T_b(x)|)^{k_b}}
\right].
\end{align*}
\end{lemma}

\begin{proof}[Lemma~\ref{lem:mondrian_app_moment_denominator}]

Define the common refinement of
$\left\{ T_b(x) : 1 \leq b \leq {B_0} \right\}$ as
the class of sets
%
\begin{align*}
\cR
&= \left\{ \bigcap_{b=1}^{B_0} D_b :
D_b \in
\big\{ T_b(x), T_b(x)^{\comp} \big\}
\right\}
\bigsetminus
\left\{
\emptyset,\,
\bigcap_{b=1}^{B_0}
T_b(x)^\comp
\right\}
\end{align*}
%
and let $\cD \subset \cR$.
We will proceed by induction on the elements of $\cD$,
which represents the subcells we have checked,
starting from $\cD = \emptyset$ and finishing at $\cD = \cR$.
For $D \in \cR$ let
$\cA(D) = \left\{ 1 \leq b \leq {B_0} : D \subseteq T_b(x) \right\}$
be the indices of the trees which are active on subcell $D$,
and for $1 \leq b \leq {B_0}$ let
$\cA(b) = \left\{ D \in \cR : D \subseteq T_b(x) \right\}$
be the subcells which are contained in $T_b(x)$,
so that $b \in \cA(D) \iff D \in \cA(b)$.
For a subcell $D \in \cR$,
write $N_b(D) = \sum_{i=1}^{n} \I \left\{ X_i \in D \right\}$
so that $N_b(x) = \sum_{D \in \cA(b)} N_b(D)$.
Note that for any $D \in \cR \setminus \cD$,
%
\begin{align*}
&\E \left[
\prod_{b=1}^{B_0}
\frac{1}{
1 \vee \left(
\sum_{D' \in \cA(b) \setminus \cD}
N_b(D')
+ n \sum_{D' \in \cA(b) \cap \cD}
|D'|
\right)^{k_b}
}
\right] \\
&\quad=
\E \left[
\prod_{b \notin \cA(D)}
\frac{1}{
1 \vee \left(
\sum_{D' \in \cA(b) \setminus \cD}
N_b(D')
+ n \sum_{D' \in \cA(b) \cap \cD}
|D'|
\right)^{k_b}
} \right. \\
&\left.
\qquad
\times\,\E\left[
\prod_{b \in \cA(D)}
\frac{1}{
1 \vee \left(
\sum_{D' \in \cA(b) \setminus \cD}
N_b(D')
+ n \sum_{D' \in \cA(b) \cap \cD}
|D'|
\right)^{k_b}
} \right.\right. \\
&\left.\left.
\quad\qquad\qquad\biggm|
\bT,
N_b(D') : D' \in \cR
\setminus
(\cD \cup \{D\})
\right]
\right].
\end{align*}
%
Now the inner conditional expectation is over $N_b(D)$ only.
Since $f$ is bounded away from zero,
%
\begin{align*}
N_b(D)
&\sim \Bin\left(
n - \sum_{D' \in \cR \setminus (\cD \cup \{D\})} N_b(D'), \
\frac{\int_{D} f(s) \diff s}
{1 - \int_{\bigcup \left( \cR \setminus \cD \right) \setminus D}
f(s) \diff s}
\right) \\
&\geq \Bin\left(
n - \sum_{D' \in \cR \setminus (\cD \cup \{D\})} N_b(D'), \
|D| \inf_{x \in [0,1]^d} f(x)
\right)
\end{align*}
%
conditional on $\bT$ and
$N_b(D') : D' \in \cR \setminus (\cD \cup \{D\})$.
For sufficiently large $t$ by Lemma~\ref{lem:mondrian_app_active_data}
%
\begin{align*}
\P \left(
\sum_{D' \in \cR \setminus (\cD \cup \{D\})} N_b(D')
> t^{d+1} \frac{n}{\lambda^d} \|f\|_\infty \right)
&\leq
\P \left( N_{\cup}(x) > t^{d+1}
\frac{n}{\lambda^d}
\|f\|_\infty
\right)
\leq
4 d B_0 e^{-t/4}.
\end{align*}
%
Thus
$N_b(D) \geq \Bin(n/2, |D| \inf_x f(x))$
conditional on
$\left\{ \bT, N_b(D') : D' \in \cR \setminus (\cD \cup \{D\}) \right\}$
with probability at least
$1 - 4 d B_0 e^{\frac{-\sqrt \lambda}{8 \|f\|_\infty}}$.
So by Lemma~\ref{lem:mondrian_app_binomial_bound},
%
\begin{align*}
&\E \Bigg[
\prod_{b \in \cA(D)} \!
\frac{1}{
1 \vee \left(
\sum_{D' \in \cA(b) \setminus \cD}
N_b(D')
+ n \sum_{D' \in \cA(b) \cap \cD}
|D'|
\right)^{k_b}
}
\biggm|
\!
\bT,
N_b(D')\! : D' \in \cR \setminus \! (\cD \cup \{D\})
\Bigg] \\
&\quad\leq
\E \! \left[
\prod_{b \in \cA(D)}
\frac{(9k)^{k_b}}{
1 \vee \left(
\sum_{D' \in \cA(b) \setminus (\cD \cup \{D\})}
N_b(D')
+ n |D| \inf_x f(x) / 2
+ n \sum_{D' \in \cA(b) \cap \cD}
|D'|
\right)^{k_b}}
\right] \\
&\qquad+
4 d B_0 e^{\frac{-\sqrt \lambda}{8 \|f\|_\infty}} \\
&\quad\leq
\left( \frac{18k}{\inf_x f(x)} \right)^k
\! \E \! \left[
\prod_{b \in \cA(D)}
\frac{1}{
1 \vee \left(
\sum_{D' \in \cA(b) \setminus (\cD \cup \{D\})}
N_b(D')
+ n \sum_{D' \in \cA(b) \cap (\cD \cup \{D\})}
|D'|
\right)^{k_b}}
\right] \\
&\qquad+
4 d B_0 e^{\frac{-\sqrt \lambda}{8 \|f\|_\infty}}.
\end{align*}
%
Therefore plugging this back into the marginal expectation yields
%
\begin{align*}
&\E\left[
\prod_{b=1}^{B_0}
\frac{1}{
1 \vee \left(
\sum_{D' \in \cA(b) \setminus \cD}
N_b(D')
+ n \sum_{D' \in \cA(b) \cap \cD}
|D'|
\right)^{k_b}
}
\right] \\
&\quad\leq
\left( \frac{18k}{\inf_x f(x)} \right)^k
\E \left[
\prod_{b=1}^{B_0}
\frac{1}{
1 \vee \left(
\sum_{D' \in \cA(b) \setminus (\cD \cup \{D\})}
N_b(D')
+ n \sum_{D' \in \cA(b) \cap (\cD \cup \{D\})}
|D'|
\right)^{k_b}}
\right] \\
&\qquad+
4 d B_0 e^{\frac{-\sqrt \lambda}{8 \|f\|_\infty}}.
\end{align*}
%
Now we apply induction,
starting with $\cD = \emptyset$ and
adding $D \in \cR \setminus \cD$ to $\cD$ until
$\cD = \cR$.
This takes at most $|\cR| \leq 2^{B_0}$ steps and yields
%
\begin{align*}
\E\left[
\prod_{b=1}^{B_0}
\frac{\I_b(x)}{N_b(x)^{k_b}}
\right]
&\leq
\E\left[
\prod_{b=1}^{B_0}
\frac{1}{1 \vee N_b(x)^{k_b}}
\right]
=
\E\left[
\prod_{b=1}^{B_0}
\frac{1}{1 \vee \left( \sum_{D \in \cA(b)} N_b(D) \right)^{k_b}}
\right]
\leq \cdots \\
&\leq
\left( \frac{18k}{\inf_x f(x)} \right)^{2^{B_0} k}
\left(
\prod_{b=1}^{B_0}
\,\E \left[
\frac{1}{1 \vee (n |T_b(x)|)^{k_b}}
\right]
+ 4 d B_0 2^{B_0} e^{\frac{-\sqrt \lambda}{8 \|f\|_\infty}}
\right),
\end{align*}
%
where the expectation factorizes due to independence of $T_b(x)$.
The last step is to remove the trailing exponential term.
To do this, note that by Jensen's inequality,
%
\begin{align*}
\prod_{b=1}^{B_0}
\,\E \left[
\frac{1}{1 \vee (n |T_b(x)|)^{k_b}}
\right]
&\geq
\prod_{b=1}^{B_0}
\frac{1}
{\E \left[ 1 \vee (n |T_b(x)|)^{k_b} \right]}
\geq
\prod_{b=1}^{B_0}
\frac{1}{n^{k_b}}
= n^{-k}
\geq
4 d B_0 2^{B_0} e^{\frac{-\sqrt \lambda}{8 \|f\|_\infty}}
\end{align*}
%
for sufficiently large $n$
because $B_0$, $d$, and $k$ are fixed while
$\log \lambda \gtrsim \log n$.
\end{proof}

Now that moments of (products of) $\frac{\I_b(x)}{N_b(x)}$
have been bounded by moments of
(products of) $\frac{1}{n |T_b(x)|}$, we establish further
explicit bounds for these in the next result.
Note that the problem has been reduced to determining
properties of Mondrian cells, so once again we return to the
exact cell shape distribution given by \citet{mourtada2020minimax},
and evaluate the appropriate expectations by integration.
Note that the truncation by taking the minimum with one inside the expectation
is essential here, as otherwise second moment of the inverse Mondrian cell
volume is not even finite. As such, there is a ``penalty'' of $\log n$
when bounding truncated second moments,
and the upper bound for the $k$th moment is significantly
larger than the naive assumption of $(\lambda^d / n)^k$
whenever $k \geq 3$.
This ``small cell'' phenomenon in which the inverse volumes of Mondrian cells
have heavy tails is a recurring challenge.

\begin{lemma}[Inverse moments of the volume of a Mondrian cell]%
\label{lem:mondrian_app_moment_cell}

Suppose Assumption~\ref{ass:mondrian_estimator} holds
and let $T \sim \cM\big([0,1]^d, \lambda\big)$.
Then for sufficiently large $n$,
%
\begin{align*}
\E\left[
1 \wedge
\frac{1}{(n |T(x)|)^k}
\right]
&\leq
\left(
\frac{\lambda^d}{n}
\right)^{\I \left\{ k = 1 \right\}}
\left(
\frac{3 \lambda^{2d} \log n}{n^2}
\right)^{\I \left\{ k \geq 2 \right\}}
\prod_{j=1}^{d} \frac{1}{x_j (1-x_j)}.
\end{align*}
%
\end{lemma}

\begin{proof}[Lemma~\ref{lem:mondrian_app_moment_cell}]

By \citet[Proposition~1]{mourtada2020minimax},
$|T(x)| = \prod_{j=1}^{d}
\left(
\left(\frac{1}{\lambda} E_{j1} \right) \wedge x_j
+ \left( \frac{1}{\lambda} E_{j2} \right) \wedge (1-x_j)
\right)$
where $E_{j1}$ and $E_{j2}$
are mutually independent $\Exp(1)$ random variables.
Thus for $0<t<1$,
using the fact that $E_{j1} + E_{j2} \sim \Gam(2, 1)$,
%
\begin{align*}
\E \left[
\frac{1}{1 \vee (n |T(x)|)^k}
\right]
&\leq
\frac{1}{n^k}
\E \left[
\frac{\I\{\min_j (E_{j1} + E_{j2}) \geq t\}}{|T(x)|^k}
\right]
+ \P \left(\min_{1 \leq j \leq d} (E_{j1} + E_{j2}) < t\right) \\
&\leq
\frac{1}{n^k}
\prod_{j=1}^d
\E \left[
\frac{\I\{E_{j1} + E_{j2} \geq t\}}
{\left(\frac{1}{\lambda} E_{j1} \wedge x_j
+ \frac{1}{\lambda} E_{j2} \wedge (1-x_j)\right)^k}
\right]
+ d\, \P \left(E_{j1} < t\right) \\
&\leq
\frac{\lambda^{d k}}{n^k}
\prod_{j=1}^d
\frac{1}{x_j(1-x_j)}
\E \left[
\frac{\I\{E_{j1} + E_{j2} \geq t\}}
{(E_{j1} + E_{j2})^k \wedge 1}
\right]
+ d (1 - e^{-t}) \\
&\leq
\frac{\lambda^{d k}}{n^k}
\prod_{j=1}^d
\frac{1}{x_j(1-x_j)}
\int_{t}^{1}
\frac{e^{-s}}{s^{k-1}}
\diff s
+ d t \\
&\leq
d t
+ \frac{\lambda^{d k}}{n^k}
\prod_{j=1}^d
\frac{1}{x_j(1-x_j)}
\times
\begin{cases}
1-t & \text{if } k = 1, \\
-\log t & \text{if } k = 2.
\end{cases}
\end{align*}
%
If $k>2$ we use
$\frac{1}{1 \vee (n |T(x)|)^k} \leq \frac{1}{1 \vee (n |T(x)|)^{k-1}}$
to reduce $k$. Now if $k = 1$ we let $t \to 0$, giving
%
\begin{align*}
\E \left[
\frac{1}{1 \vee (n |T(x)|)}
\right]
&\leq
\frac{\lambda^d}{n}
\prod_{j=1}^d
\frac{1}{x_j(1-x_j)},
\end{align*}
%
and if $k = 2$ then we set $t = 1/n^2$ so that for
sufficiently large $n$,
%
\begin{align*}
\E \left[
\frac{1}{1 \vee (n |T(x)|)^2}
\right]
&\leq
\frac{d}{n^2}
+ \frac{2 \lambda^{2d} \log n}{n^2}
\prod_{j=1}^d
\frac{1}{x_j(1-x_j)}
\leq
\frac{3 \lambda^{2d} \log n}{n^2}
\prod_{j=1}^d
\frac{1}{x_j(1-x_j)}.
\end{align*}
%
Lower bounds which match up to constants for the first moment and up to
logarithmic terms for the second moment are obtained as
$\E \left[ 1 \wedge \frac{1}{(n|T(x)|)^2} \right]
\geq \E \left[ 1 \wedge \frac{1}{n|T(x)|} \right]^2$
by Jensen, and
%
\begin{align*}
\E \left[ 1 \wedge \frac{1}{n|T(x)|} \right]
&\geq \frac{1}{1 + n \E \left[ |T(x)| \right]}
\geq \frac{1}{1 + 2^d n / \lambda^d}
\gtrsim \frac{\lambda^d}{n}.
\end{align*}
\end{proof}

The endeavor to bound moments of (products of) $\frac{\I_b(x)}{N_b(x)}$ is
concluded with the next result, combining the previous two lemmas to give a
bound without expectations on the right.

\begin{lemma}[Simplified generalized moment bound for
Mondrian forest denominators]%
\label{lem:mondrian_app_simple_moment_denominator}
%
Suppose Assumptions~\ref{ass:mondrian_data}
and \ref{ass:mondrian_estimator} hold.
Let $T_b \sim \cM\big([0,1]^d, \lambda\big)$
and $k_b \geq 1$ for $1 \leq b \leq B_0$.
Then with $k = \sum_{b=1}^{B_0} k_b$,
%
\begin{align*}
&\E\left[
\prod_{b=1}^{B_0}
\frac{\I_b(x)}{N_b(x)^{k_b}}
\right] \\
&\quad\leq
\left( \frac{36k}{\inf_{x \in [0,1]^d} f(x)} \right)^{2^{B_0} k}
\left(
\prod_{j=1}^{d} \frac{1}{x_j (1-x_j)}
\right)^{B_0}
\prod_{b=1}^{B_0}
\left(
\frac{\lambda^d}{n}
\right)^{\I \left\{ k_b = 1 \right\}}
\left(
\frac{\lambda^{2d} \log n}{n^2}
\right)^{\I \left\{ k_b \geq 2 \right\}}
\end{align*}
%
for sufficiently large $n$.
%
\end{lemma}

\begin{proof}[Lemma~\ref{lem:mondrian_app_simple_moment_denominator}]
This follows directly from
Lemmas~\ref{lem:mondrian_app_moment_denominator} and
\ref{lem:mondrian_app_moment_cell}.
\end{proof}

Our final preliminary lemma is concerned with further properties of
the inverse truncated binomial distribution, again with the aim
of analyzing $\frac{\I_b(x)}{N_b(x)}$.
This time, instead of merely upper bounding the moments,
we aim to give convergence results for those moments,
again in terms of moments of $\frac{1}{n |T_b(x)|}$.
This time we only need to handle the first
and second moment, so this result does not strictly generalize
Lemma~\ref{lem:mondrian_app_binomial_bound} except in simple cases.
The proof is by Taylor's theorem and the Cauchy--Schwarz inequality,
using explicit expressions for moments of the binomial distribution
and bounds from Lemma~\ref{lem:mondrian_app_binomial_bound}.

\begin{lemma}[Expectation inequalities for the binomial distribution]%
\label{lem:mondrian_app_binomial_expectation}
Let $N \sim \Bin(n, p)$ and take $a, b \geq 1$. Then
%
\begin{align*}
0
&\leq
\E \left[
\frac{1}{N+a}
\right]
- \frac{1}{n p+a}
\leq
\frac{2^{19}}{(n p+a)^2}, \\
0
&\leq
\E \left[
\frac{1}{(N+a)(N+b)}
\right]
- \frac{1}{(n p+a)(n p+b)}
\leq
\frac{2^{27}}{(n p +a)(n p +b)}
\left(
\frac{1}{n p + a}
+ \frac{1}{n p + b}
\right).
\end{align*}

\end{lemma}

\begin{proof}[Lemma~\ref{lem:mondrian_app_binomial_expectation}]

For the first result,
Taylor's theorem with Lagrange remainder
for $N \mapsto \frac{1}{N+a}$ around $n p$ gives
%
\begin{align*}
\E \left[
\frac{1}{N+a}
\right]
&=
\E \left[
\frac{1}{n p+a}
- \frac{N - n p}{(n p+a)^2}
+ \frac{(N - n p)^2}{(\xi+a)^3}
\right]
\end{align*}
%
for some $\xi$ between $n p$ and $N$. The second term in the expectation
is zero-mean, showing the non-negativity part, and the
Cauchy--Schwarz inequality for the remaining term gives
%
\begin{align*}
\E \left[
\frac{1}{N+a}
\right]
- \frac{1}{n p+a}
&\leq
\E \left[
\frac{(N - n p)^2}{(n p+a)^3}
+ \frac{(N - n p)^2}{(N+a)^3}
\right] \\
&\leq
\frac{\E\big[(N - n p)^2\big]}{(n p+a)^3}
+ \sqrt{
\E\big[(N - n p)^4\big]
\E \left[
\frac{1}{(N+a)^6}
\right]}.
\end{align*}
%
Now we use $\E\big[(N - n p)^4\big] \leq n p(1+3n p)$
and apply Lemma~\ref{lem:mondrian_app_binomial_bound} to see that
%
\begin{align*}
\E \left[
\frac{1}{N+a}
\right]
- \frac{1}{n p+a}
&\leq
\frac{n p}{(n p+a)^3}
+ \sqrt{\frac{54^6 n p(1+3 n p)}{(n p + a)^6}}
\leq
\frac{2^{19}}{(n p+a)^2}.
\end{align*}
%
For the second result,
Taylor's theorem applied to $N \mapsto \frac{1}{(N+a)(N+b)}$
around $n p$ gives
%
\begin{align*}
\E \left[
\frac{1}{(N+a)(N+b)}
\right]
&=
\E \left[
\frac{1}{(n p+a)(n p + b)}
- \frac{(N - n p)(2 n p + a + b)}{(n p + a)^2 (n p + b)^2}
\right] \\
&\quad+
\E \left[
\frac{(N - n p)^2}{(\xi+a)(\xi+b)}
\left(
\frac{1}{(\xi + a)^2}
+ \frac{1}{(\xi + a)(\xi + b)}
+ \frac{1}{(\xi + b)^2}
\right)
\right]
\end{align*}
%
for some $\xi$ between $n p$ and $N$. The second term on the right is
zero-mean, showing non-negativity, and applying the Cauchy--Schwarz
inequality to the remaining term gives
%
\begin{align*}
&\E \left[
\frac{1}{(N+a)(N+b)}
\right]
- \frac{1}{n p+a} \\
&\quad\leq
\E \left[
\frac{2 (N - n p)^2}{(N+a)(N+b)}
\left(
\frac{1}{(N + a)^2}
+ \frac{1}{(N + b)^2}
\right)
\right] \\
&\qquad+
\E \left[
\frac{2 (N - n p)^2}{(n p +a)(n p +b)}
\left(
\frac{1}{(n p + a)^2}
+ \frac{1}{(n p + b)^2}
\right)
\right] \\
&\quad\leq
\sqrt{
4 \E \left[ (N - n p)^4 \right]
\E \left[
\frac{1}{(N + a)^6 (N+b)^2}
+ \frac{1}{(N + b)^6 (N+a)^2}
\right]} \\
&\qquad+
\frac{2 \E\big[(N - n p)^2\big]}{(n p +a)(n p +b)}
\left(
\frac{1}{(n p + a)^2}
+ \frac{1}{(n p + b)^2}
\right).
\end{align*}
%
Now we use
$\E\big[(N - n p)^4\big] \leq n p(1+3n p)$
and apply Lemma~\ref{lem:mondrian_app_binomial_bound} to see that
%
\begin{align*}
\E \left[
\frac{1}{(N+a)(N+b)}
\right]
- \frac{1}{n p+a}
&\leq
\sqrt{
\frac{4n p (1 + 3n p) \cdot 72^8}{(n p + a)^2 (n p + b)^2}
\left(
\frac{1}{(n p + a)^4}
+ \frac{1}{(n p + b)^4}
\right)} \\
&\quad+
\frac{2 n p}{(n p +a)(n p +b)}
\left(
\frac{1}{(n p + a)^2}
+ \frac{1}{(n p + b)^2}
\right) \\
&\leq
\frac{2^{27}}{(n p + a) (n p + b)}
\left(
\frac{1}{n p + a}
+ \frac{1}{n p + b}
\right).
\end{align*}
%
\end{proof}

\section{Proofs of main results}
\label{sec:mondrian_app_proofs}

\subsection{Mondrian random forests}

We give rigorous proofs of the central limit theorem,
bias characterization, and variance estimation
results for the Mondrian random forest estimator without debiasing.
See Section~\ref{sec:mondrian_overview_proofs} in the main text
for details on our approaches to these proofs.

\begin{proof}[Theorem~\ref{thm:mondrian_clt}]
From the debiased version
(Theorem~\ref{thm:mondrian_clt_debiased}) with $J=0$, $a_0 = 1$, and
$\omega_0 = 1$.
\end{proof}

\begin{proof}[Theorem~\ref{thm:mondrian_bias}]

\proofparagraph{removing the dependence on the trees}

By measurability and with $\mu(X_i) = \E[Y_i \mid X_i]$ almost surely,
%
\begin{align*}
\E \left[ \hat \mu(x) \mid \bX, \bT \right]
- \mu(x)
&=
\frac{1}{B}
\sum_{b=1}^B
\sum_{i=1}^n \big( \mu(X_i) - \mu(x) \big)
\frac{\I_{i b}(x)}{N_b(x)}.
\end{align*}
%
Conditional on $\bX$,
the terms in the outer sum depend only on $T_b$ so are i.i.d.
As $\mu$ is Lipschitz,
%
\begin{align*}
&\Var \big[
\E \left[ \hat \mu(x) \mid \bX, \bT \right]
- \mu(x)
\mid \bX
\big]
\leq
\frac{1}{B}
\E \left[
\left(
\sum_{i=1}^n \big( \mu(X_i) - \mu(x) \big)
\frac{\I_{i b}(x)}{N_b(x)}
\right)^2
\Bigm| \bX
\right] \\
&\quad\lesssim
\frac{1}{B}
\E \left[
\max_{1 \leq i \leq n}
\big\| X_i - x \big\|_2^2
\left(
\sum_{i=1}^n
\frac{\I_{i b}(x)}{N_b(x)}
\right)^2
\Bigm| \bX
\right]
\lesssim
\frac{1}{B}
\sum_{j=1}^{d}
\E \left[
|T(x)_j|^2
\right]
\lesssim
\frac{1}{\lambda^2 B},
\end{align*}
%
using the law of $T(x)_j$ from \citet[Proposition~1]{mourtada2020minimax}.
By Chebyshev's inequality,
%
\begin{align*}
\big|
\E \left[ \hat \mu(x) \mid \bX, \bT \right]
- \E \left[ \hat \mu(x) \mid \bX \right]
\big|
&\lesssim_\P
\frac{1}{\lambda \sqrt B}.
\end{align*}

\proofparagraph{showing the conditional bias converges in probability}

Now $\E \left[ \hat\mu(x) \mid \bX \right]$
is a non-linear function of the i.i.d.\ random variables $X_i$,
so we use the Efron--Stein inequality
\citep{efron1981jackknife} to bound its variance.
Let $\tilde X_{i j} = X_i$ if $i \neq j$ and be an
independent copy of $X_j$, denoted $\tilde X_j$, if $i = j$.
Write $\tilde \bX_j = (\tilde X_{1j}, \ldots, \tilde X_{n j})$
and similarly
$\tilde \I_{i j b}(x) = \I \big\{ \tilde X_{i j} \in T_b(x) \big\}$
and $N_{j b}(x) = \sum_{i=1}^{n} \tilde \I_{i j b}(x)$.
%
\begin{align}
\nonumber
&\Var \left[
\sum_{i=1}^{n}
\big( \mu(X_i) - \mu(x) \big)
\E \left[
\frac{\I_{i b}(x)}{N_b(x)}
\Bigm| \bX
\right]
\right] \\
\nonumber
&\quad\leq
\frac{1}{2}
\sum_{j=1}^{n}
\E \! \left[
\! \left(
\sum_{i=1}^{n}
\big( \mu(X_i) - \mu(x) \big)
\E \! \left[
\frac{\I_{i b}(x)}{N_b(x)}
\Bigm| \bX
\right]
- \sum_{i=1}^{n}
\left( \mu(\tilde X_{i j}) - \mu(x) \right)
\E \! \left[
\frac{\tilde \I_{i j b}(x)}{\tilde N_{j b}(x)}
\Bigm| \tilde \bX_j
\right]
\right)^{\! \! 2}
\right] \\
\nonumber
&\quad\leq
\frac{1}{2}
\sum_{j=1}^{n}
\E \left[
\left(
\sum_{i=1}^{n}
\left(
\big( \mu(X_i) - \mu(x) \big)
\frac{\I_{i b}(x)}{N_b(x)}
- \left( \mu(\tilde X_{i j}) - \mu(x) \right)
\frac{\tilde \I_{i j b}(x)}{\tilde N_{j b}(x)}
\right)
\right)^2
\right] \\
\nonumber
&\quad\leq
\sum_{j=1}^{n}
\E \left[
\left(
\sum_{i \neq j}
\big( \mu(X_i) - \mu(x) \big)
\left(
\frac{\I_{i b}(x)}{N_b(x)} - \frac{\I_{i b}(x)}{\tilde N_{j b}(x)}
\right)
\right)^{\!\!2} \,
\right] \\
\label{eq:mondrian_app_bias_efron_stein}
&\qquad+
2 \sum_{j=1}^{n}
\E \left[
\left( \mu(X_j) - \mu(x) \right)^2
\frac{\I_{j b}(x)}{N_b(x)^2}
\right].
\end{align}
%
For the first term in \eqref{eq:mondrian_app_bias_efron_stein} to be non-zero,
we must have $|N_b(x) - \tilde N_{j b}(x)| = 1$.
Writing $N_{-j b}(x) = \sum_{i \neq j} \I_{i b}(x)$,
assume by symmetry that
$\tilde N_{j b}(x) = N_{-j b}(x)$ and $N_b(x) = N_{-j b}(x) + 1$,
and $\I_{j b}(x) = 1$.
As $f$ is bounded and $\mu$ is Lipschitz,
writing $\I_{-j b}(x) = \I \left\{ N_{-j b}(x) \geq 1 \right\}$,
%
\begin{align*}
&\sum_{j=1}^{n}
\E \left[
\left(
\sum_{i \neq j}
\left( \mu(X_i) - \mu(x) \right)
\left(
\frac{\I_{i b}(x)}{N_b(x)} - \frac{\I_{i b}(x)}{\tilde N_{j b}(x)}
\right)
\right)^{\! 2} \,
\right] \\
&\quad\lesssim
\sum_{j=1}^{n}
\E \left[
\max_{1 \leq l \leq d}
|T_b(x)_l|^2
\left(
\frac{\sum_{i \neq j}\I_{i b}(x) \I_{j b}(x)}
{N_{-j b}(x)(N_{-j b}(x) + 1)}
\right)^2
\right]
\lesssim
\E \left[
\max_{1 \leq l \leq d}
|T_b(x)_l|^2
\frac{\I_{b}(x)}{N_{b}(x)}
\right].
\end{align*}
%
For $t > 0$, partition by
$\left\{ \max_{1 \leq l \leq d} |T_b(x)_l| \geq t/\lambda \right\}$
and apply Lemma~\ref{lem:mondrian_app_largest_cell} and
Lemma~\ref{lem:mondrian_app_simple_moment_denominator}:
%
\begin{align*}
\E \left[
\max_{1 \leq l \leq d}
|T_b(x)_l|^2
\frac{\I_{b}(x)}{N_{b}(x)}
\right]
&\leq
\P \left(
\max_{1 \leq l \leq d} |T_b(x)_l| \geq t/\lambda
\right)
+ (t / \lambda)^2\,
\E \left[
\frac{\I_{b}(x)}{N_{b}(x)}
\right] \\
&\lesssim
e^{-t/2}
+ \left( \frac{t}{\lambda} \right)^2
\frac{\lambda^d}{n}
\lesssim
\frac{1}{n^2}
+ \frac{(\log n)^2}{\lambda^2}
\frac{\lambda^d}{n}
\lesssim
\frac{(\log n)^2}{\lambda^2}
\frac{\lambda^{d}}{n},
\end{align*}
%
where we set $t = 4 \log n$.
For the second term in \eqref{eq:mondrian_app_bias_efron_stein} we have
%
\begin{align*}
\sum_{j=1}^{n}
\E \left[
\left( \mu(X_j) - \mu(x) \right)^2
\frac{\I_{j b}(x)}{N_b(x)^2}
\right]
&\lesssim
\E \left[
\max_{1 \leq l \leq d}
|T_b(x)_l|^{2}
\frac{\I_{b}(x)}{N_b(x)}
\right]
\lesssim
\frac{(\log n)^2}{\lambda^2}
\frac{\lambda^{d}}{n}
\end{align*}
%
in the same manner.
Hence
%
\begin{align*}
\Var \left[
\sum_{i=1}^{n}
\left( \mu(X_i) - \mu(x) \right)
\E \left[
\frac{\I_{i b}(x)}{N_b(x)}
\Bigm| \bX
\right]
\right]
&\lesssim
\frac{(\log n)^2}{\lambda^2}
\frac{\lambda^{d}}{n},
\end{align*}
%
and so by Chebyshev's inequality,
%
\begin{align*}
\big|
\E \left[ \hat \mu(x) \mid \bX, \bT \right]
- \E \left[ \hat \mu(x) \right]
\big|
&\lesssim_\P
\frac{1}{\lambda \sqrt B}
+ \frac{\log n}{\lambda}
\sqrt{ \frac{\lambda^{d}}{n} }.
\end{align*}

\proofparagraph{computing the limiting bias}

It remains to compute the limit of
$\E \left[ \hat \mu(x) \right] - \mu(x)$.
Let $\bX_{-i} = (X_1, \ldots, X_{i-1}, X_{i+1}, \ldots, X_n)$
and $N_{-i b}(x) = \sum_{j=1}^n \I\{j \neq i\} \I\{X_j \in T_b(x)\}$.
Then
%
\begin{align*}
&\E \left[ \hat \mu(x) \right]
- \mu(x)
=
\E \left[
\sum_{i=1}^{n}
\left( \mu(X_i) - \mu(x) \right)
\frac{\I_{i b}(x)}{N_b(x)}
\right] \\
&\quad=
\sum_{i=1}^{n}
\E \left[
\E \left[
\frac{\left( \mu(X_i) - \mu(x) \right)\I_{i b}(x)}
{N_{-i b}(x) + 1}
\bigm| \bT, \bX_{-i}
\right]
\right]
= n \,
\E \left[
\frac{\int_{T_b(x)} \left( \mu(s) - \mu(x) \right) f(s) \diff s}
{N_{-i b}(x) + 1}
\right].
\end{align*}
%
By Lemma~\ref{lem:mondrian_app_binomial_expectation}, as
$N_{-i b}(x) \sim \Bin\left(n-1,
\int_{T_b(x)} f(s) \diff s \right)$
given $\bT$ and $f$ is bounded below,
%
\begin{align*}
\left|
\E \! \left[
\frac{1}{N_{-i b}(x) + 1}
\Bigm| \bT
\right]
- \frac{1}{(n-1) \! \int_{T_b(x)} \! f(s) \diff s + 1}
\right|
&\lesssim
\frac{1}{n^2 \! \left( \int_{T_b(x)} f(s) \diff s \right)^2}
\wedge 1
\lesssim
\frac{1}{n^2 |T_b(x)|^2}
\wedge 1,
\end{align*}
%
and also
%
\begin{align*}
\left|
\frac{1}{(n-1) \int_{T_b(x)} f(s) \diff s + 1}
- \frac{1}{n \int_{T_b(x)} f(s) \diff s}
\right|
&\lesssim
\frac{1}{n^2 \left( \int_{T_b(x)} f(s) \diff s\right)^2}
\wedge 1
\lesssim
\frac{1}{n^2 |T_b(x)|^2}
\wedge 1.
\end{align*}
%
So by Lemmas~\ref{lem:mondrian_app_largest_cell}
and \ref{lem:mondrian_app_moment_cell},
since $f$ is Lipschitz and bounded, using Cauchy--Schwarz,
%
\begin{align*}
&\left|
\E \left[ \hat \mu(x) \right]
- \mu(x)
- \E \left[
\frac{\int_{T_b(x)} \left( \mu(s) - \mu(x) \right) f(s) \diff s}
{\int_{T_b(x)} f(s) \diff s}
\right]
\right|
\lesssim
\E \left[
\frac{n \int_{T_b(x)} \left| \mu(s) - \mu(x) \right| f(s) \diff s}
{n^2 |T_b(x)|^2 \vee 1}
\right] \\
&\qquad\lesssim
\E \left[
\frac{\max_{1 \leq l \leq d} |T_b(x)_l| }
{n |T_b(x)| \vee 1}
\right] \\
&\qquad\lesssim
\frac{2 \log n}{\lambda} \,
\E \left[
\frac{1}{n |T_b(x)| \vee 1}
\right]
+ \P \left( \max_{1 \leq l \leq d} |T_b(x)_l| >
\frac{2 \log n}{\lambda} \right)^{1/2}
\E \left[
\frac{1}
{n^2 |T_b(x)|^2 \vee 1}
\right]^{1/2} \\
&\qquad\lesssim
\frac{\log n}{\lambda} \,
\frac{\lambda^d}{n}
+ \frac{d}{n}
\frac{\lambda^d \sqrt{\log n}}{n}
\lesssim
\frac{\log n}{\lambda} \,
\frac{\lambda^d}{n}.
\end{align*}
%
Next set
$A = \frac{1}{f(x) |T_b(x)|} \int_{T_b(x)} (f(s) - f(x)) \diff s
\geq \inf_{s \in [0,1]^d} \frac{f(s)}{f(x)} - 1$.
Use the Maclaurin series of $\frac{1}{1+x}$
up to order $\flbeta$ to see
$\frac{1}{1 + A} = \sum_{k=0}^{\flbeta} (-1)^k A^k
+ O \left( A^{\flbeta + 1} \right)$.
Hence
%
\begin{align*}
&\E \left[
\frac{\int_{T_b(x)} \left( \mu(s) - \mu(x) \right) f(s) \diff s}
{\int_{T_b(x)} f(s) \diff s}
\right]
=
\E \left[
\frac{\int_{T_b(x)} \left( \mu(s) - \mu(x) \right) f(s) \diff s}
{f(x) |{T_b(x)}|}
\frac{1}{1 + A}
\right] \\
&\quad=
\E \left[
\frac{\int_{T_b(x)} \left( \mu(s) - \mu(x) \right) f(s) \diff s}
{f(x) |{T_b(x)}|}
\left(
\sum_{k=0}^{\flbeta}
(-1)^k
A^k
+ O \left( |A|^{\flbeta + 1} \right)
\right)
\right].
\end{align*}
%
Note that since $f$ and $\mu$ are Lipschitz,
and by integrating the tail probability given in
Lemma~\ref{lem:mondrian_app_largest_cell}, the Maclaurin remainder term is
bounded by
%
\begin{align*}
&\E \left[
\frac{\int_{T_b(x)} \left| \mu(s) - \mu(x) \right| f(s) \diff s}
{f(x) |{T_b(x)}|}
|A|^{\flbeta + 1}
\right] \\
&\qquad=
\E \left[
\frac{\int_{T_b(x)} \left| \mu(s) - \mu(x) \right| f(s) \diff s}
{f(x) |{T_b(x)}|}
\left(
\frac{1}{f(x) |{T_b(x)}|} \int_{T_b(x)} (f(s) - f(x)) \diff s
\right)^{\flbeta + 1}
\right] \\
&\qquad\lesssim
\E \left[
\max_{1 \leq l \leq d}
|T_b(x)_l|^{\flbeta+2}
\right]
=
\int_{0}^{\infty}
\P \left(
\max_{1 \leq l \leq d}
|T_b(x)_l|
\geq t^{\frac{1}{\flbeta+2}}
\right)
\diff t
\leq
\int_{0}^{\infty}
2 d e^{- \lambda t^{\frac{1}{\flbeta+2}} / 2}
\diff t \\
&\qquad=
\frac{2^{\flbeta + 3} d (\flbeta + 2)! }
{\lambda^{\flbeta + 2}}
\lesssim
\frac{1}{\lambda^{\beta}},
\end{align*}
%
since $\int_0^\infty e^{-a x^{1/k}} \diff x
= a^{-k} k!$.
To summarize the progress so far, we have
%
\begin{align*}
&\left|
\E \left[
\hat \mu(x)
\right]
- \mu(x)
- \sum_{k=0}^{\flbeta}
(-1)^k \,
\E \left[
\frac{\int_{T_b(x)} \left( \mu(s) - \mu(x) \right) f(s) \diff s}
{f(x)^{k+1} |T_b(x)|^{k+1}}
\left(
\int_{T_b(x)} (f(s) - f(x)) \diff s
\right)^k
\right]
\right| \\
&\qquad\lesssim
\frac{\log n}{\lambda}
\frac{\lambda^d}{n}
+ \frac{1}{\lambda^\beta}.
\end{align*}
%
We evaluate the expectation.
By Taylor's theorem, with $\nu$ a multi-index,
as $f \in \cH^\beta$,
%
\begin{align*}
\left(
\int_{T_b(x)} (f(s) - f(x)) \diff s
\right)^k
&=
\left(
\sum_{|\nu| = 1}^\flbeta
\frac{\partial^\nu f(x)}{\nu !}
\! \int_{T_b(x)}
\!\! (s - x)^\nu
\diff s
\right)^k
+ O \! \left(
|T_b(x)| \max_{1 \leq l \leq d} |T_b(x)_l|^\beta
\right).
\end{align*}
%
Next, by the multinomial theorem
with a multi-index $u$ indexed by $\nu$ with $|\nu| \geq 1$,
%
\begin{align*}
\left(
\sum_{|\nu| = 1}^\flbeta
\frac{\partial^\nu f(x)}{\nu !}
\int_{T_b(x)}
(s - x)^\nu
\diff s
\right)^k
&=
\sum_{|u| = k}
\binom{k}{u}
\left(
\frac{\partial^\nu f(x)}{\nu !}
\int_{T_b(x)} (s-x)^\nu \diff s
\right)^u
\end{align*}
%
where $\binom{k}{u}$ is a multinomial coefficient.
By Taylor's theorem with $f, \mu \in \cH^\beta$,
%
\begin{align*}
&\int_{T_b(x)} \left( \mu(s) - \mu(x) \right) f(s) \diff s \\
&\quad=
\sum_{|\nu'|=1}^{\flbeta}
\sum_{|\nu''|=0}^{\flbeta}
\frac{\partial^{\nu'} \mu(x)}{\nu' !}
\frac{\partial^{\nu''} f(x)}{\nu'' !}
\int_{T_b(x)} (s-x)^{\nu' + \nu''} \diff s
+ O \left( |T_b(x)| \max_{1 \leq l \leq d} |T_b(x)_l|^\beta \right).
\end{align*}
%
Now by integrating the tail probabilities in
Lemma~\ref{lem:mondrian_app_largest_cell},
$ \E \left[ \max_{1 \leq l \leq d} |T_b(x)_l|^\beta \right]
\lesssim \frac{1}{\lambda^\beta}$.
Therefore, by Lemma~\ref{lem:mondrian_app_moment_cell},
writing $T_b(x)^\nu$ for $\int_{T_b(x)} (s-x)^\nu \diff s$,
%
\begin{align*}
&\sum_{k=0}^{\flbeta}
(-1)^k \,
\E \left[
\frac{\int_{T_b(x)} \left( \mu(s) - \mu(x) \right) f(s) \diff s}
{f(x)^{k+1} |T_b(x)|^{k+1}}
\left(
\int_{T_b(x)} (f(s) - f(x)) \diff s
\right)^k
\right] \\
&\,=
\! \sum_{k=0}^{\flbeta}
(-1)^k \,
\E \!
\left[
\! \frac{
\sum_{|\nu'|=1}^{\flbeta}
\! \sum_{|\nu''|=0}^{\flbeta}
\! \frac{\partial^{\nu'} \mu(x)}{\nu' !}
\frac{\partial^{\nu''} f(x)}{\nu'' !}
T_b(x)^{\nu' + \nu''\!\!\!}
}{f(x)^{k+1} |T_b(x)|^{k+1}}
\!\! \sum_{|u| = k}
\! \binom{k}{u}
\!\!
\left(
\frac{\partial^\nu f(x)}{\nu !}
T_b(x)^\nu
\right)^{\!\! u}
\right]
\! + O \! \left(
\frac{1}{\lambda^\beta}
\right) \\
&\,=
\sum_{|\nu'|=1}^{\flbeta}
\sum_{|\nu''|=0}^{\flbeta}
\sum_{|u|=0}^{\flbeta}
\frac{\partial^{\nu'} \mu(x)}{\nu' !}
\frac{\partial^{\nu''} f(x)}{\nu'' !}
\left( \frac{\partial^\nu f(x)}{\nu !} \right)^u
\binom{|u|}{u}
\frac{(-1)^{|u|}}{f(x)^{|u|+1}}
\E \left[
\frac{ T_b(x)^{\nu' + \nu''} (T_b(x)^\nu)^u}{|T_b(x)|^{|u|+1}}
\right] \\
&\quad+
O \left(
\frac{1}{\lambda^\beta}
\right) .
\end{align*}
%
We show this is a polynomial in $1/\lambda$.
For $1 \leq j \leq d$, define
$E_{1j*} \sim \Exp(1) \wedge (\lambda x_j)$
and $E_{2j*} \sim \Exp(1) \wedge (\lambda (1-x_j))$
independent so
$T_b(x) = \prod_{j=1}^{d} [x_j - E_{1j*} / \lambda, x_j + E_{2j*} / \lambda]$.
Then
%
\begin{align*}
T_b(x)^\nu
&=
\int_{T_b(x)} (s-x)^\nu \diff s
= \prod_{j=1}^d
\int_{x_j - E_{1j*}/\lambda}^{x_j+E_{2j*}/\lambda}
(s - x_j)^{\nu_j} \diff s
= \prod_{j=1}^d
\int_{-E_{1j*}}^{E_{2j*}} (s / \lambda)^{\nu_j} 1/\lambda \diff s \\
&=
\lambda^{-d - |\nu|}
\prod_{j=1}^d
\int_{-E_{1j*}}^{E_{2j*}} s^{\nu_j} \diff s
= \lambda^{-d - |\nu|}
\prod_{j=1}^d
\frac{E_{2j*}^{\nu_j + 1} + (-1)^{\nu_j} E_{1j*}^{\nu_j + 1}}
{\nu_j + 1}.
\end{align*}
%
So by independence over $j$,
%
\begin{align}
\label{eq:mondrian_app_bias_calc}
&\E \left[
\frac{ T_b(x)^{\nu' + \nu''} (T_b(x)^\nu)^u}{|T_b(x)|^{|u|+1}}
\right] \\
\nonumber
&\quad=
\lambda^{- |\nu'| - |\nu''| - |\nu| \cdot u}
\prod_{j=1}^d
\E \left[
\frac{E_{2j*}^{\nu'_j + \nu''_j + 1}
+ (-1)^{\nu'_j + \nu''_j} E_{1j*}^{\nu'_j + \nu''_j + 1}}
{(\nu'_j + \nu''_j + 1) (E_{2j*} + E_{1j*})}
\frac{\left(E_{2j*}^{\nu_j + 1}
+ (-1)^{\nu_j} E_{1j*}^{\nu_j + 1}\right)^u}
{(\nu_j + 1)^u (E_{2j*} + E_{1j*})^{|u|}}
\right].
\end{align}
%
The final step is to replace $E_{1j*}$
by $E_{1j} \sim \Exp(1)$ and similarly for $E_{2j*}$.
For some $C > 0$,
%
\begin{align*}
\P \! \left(
\bigcup_{j=1}^{d}
\left(
\left\{
E_{1j*} \neq E_{1j}
\right\}
\cup
\left\{
E_{2j*} \neq E_{2j}
\right\}
\right)
\! \right)
&\leq
2d\,
\P \! \left(
\Exp(1) \geq \lambda \min_{1 \leq j \leq d}
(x_j \wedge (1-x_j))
\! \right)
\leq
2d e^{-C \lambda}.
\end{align*}
%
Further, the quantity inside the expectation in
\eqref{eq:mondrian_app_bias_calc}
is bounded almost surely by one and so
the error incurred by replacing
$E_{1j*}$ and $E_{2j*}$ by $E_{1j}$ and $E_{2j}$
in \eqref{eq:mondrian_app_bias_calc}
is at most $2 d e^{-C \lambda} \lesssim \lambda^{-\beta}$.
Thus the limiting bias is
%
\begin{align}
\nonumber
&\E \left[ \hat \mu(x) \right]
- \mu(x) \\
\nonumber
&\quad=
\sum_{|\nu'|=1}^{\flbeta}
\sum_{|\nu''|=0}^{\flbeta}
\sum_{|u|=0}^{\flbeta}
\frac{\partial^{\nu'} \mu(x)}{\nu' !}
\frac{\partial^{\nu''} f(x)}{\nu'' !}
\left( \frac{\partial^\nu f(x)}{\nu !} \right)^u
\binom{|u|}{u}
\frac{(-1)^{|u|}}{f(x)^{|u|+1}}
\, \lambda^{- |\nu'| - |\nu''| - |\nu| \cdot u} \\
\nonumber
&\qquad\quad\times
\prod_{j=1}^d
\E \left[
\frac{E_{2j}^{\nu'_j + \nu''_j + 1}
+ (-1)^{\nu'_j + \nu''_j} E_{1j}^{\nu'_j + \nu''_j + 1}}
{(\nu'_j + \nu''_j + 1) (E_{2j} + E_{1j})}
\frac{\left(E_{2j}^{\nu_j + 1}
+ (-1)^{\nu_j} E_{1j}^{\nu_j + 1}\right)^u}
{(\nu_j + 1)^u (E_{2j} + E_{1j})^{|u|}}
\right] \\
\label{eq:mondrian_app_bias}
&\qquad+
O \left( \frac{\log n}{\lambda} \frac{\lambda^d}{n} \right)
+ O \left( \frac{1}{\lambda^\beta} \right),
\end{align}
%
recalling that $u$ is a multi-index which is indexed by the multi-index $\nu$.
This is a polynomial in $\lambda$ of degree at most $\flbeta$,
since higher-order terms can be absorbed into $O(1 / \lambda^\beta)$,
which has finite coefficients depending only on
the derivatives up to order $\flbeta$ of $f$ and $\mu$ at $x$.
Now we show that the odd-degree terms in this polynomial are all zero.
Note that a term is of odd degree if and only if
$|\nu'| + |\nu''| + |\nu| \cdot u$ is odd.
This implies that there exists $1 \leq j \leq d$ such that
exactly one of either
$\nu'_j + \nu''_j$ is odd or
$\sum_{|\nu|=1}^{\flbeta} \nu_j u_\nu$ is odd.

If $\nu'_j + \nu''_j$ is odd, then
$\sum_{|\nu|=1}^{\flbeta} \nu_j u_\nu$ is even, so
$|\{\nu : \nu_j u_\nu \text{ is odd}\}|$ is even.
Consider the effect of swapping $E_{1j}$ and $E_{2j}$,
an operation which preserves their joint law, in each of
%
\begin{align}
\label{eq:mondrian_app_bias_odd_1}
\frac{E_{2j}^{\nu'_j + \nu''_j + 1}
- (-E_{1j})^{\nu'_j + \nu''_j + 1}}
{E_{2j} + E_{1j}}
\end{align}
%
and
%
\begin{align}
\label{eq:mondrian_app_bias_odd_2}
&\frac{\left(E_{2j}^{\nu_j + 1}
- (-E_{1j})^{\nu_j + 1}\right)^u}
{(E_{2j} + E_{1j})^{|u|}}
= \!\!\!
\prod_{\substack{|\nu| = 1 \\
\nu_j u_\nu \text{ even}}}^\beta
\!\!\!
\frac{\left(E_{2j}^{\nu_j + 1}
- (-E_{1j})^{\nu_j + 1}\right)^{u_\nu}}
{(E_{2j} + E_{1j})^{u_\nu}}
\!\!\!
\prod_{\substack{|\nu| = 1 \\
\nu_j u_\nu \text{ odd}}}^\beta
\!\!\!
\frac{\left(E_{2j}^{\nu_j + 1}
- (-E_{1j})^{\nu_j + 1}\right)^{u_\nu}}
{(E_{2j} + E_{1j})^{u_\nu}}.
\end{align}
%
Clearly, $\nu'_j + \nu''_j$ being odd inverts the
sign of \eqref{eq:mondrian_app_bias_odd_1}.
For \eqref{eq:mondrian_app_bias_odd_2},
each term in the first product has either
$\nu_j$ even or $u_\nu$ even, so its sign is preserved.
Every term in the second product of \eqref{eq:mondrian_app_bias_odd_2}
has its sign inverted due to both $\nu_j$ and $u_\nu$ being odd,
but there are an even number of terms,
preserving the overall sign.
Therefore the expected product
of \eqref{eq:mondrian_app_bias_odd_1} and \eqref{eq:mondrian_app_bias_odd_2}
is zero by symmetry.

If however $\nu'_j + \nu''_j$ is even, then
$\sum_{|\nu|=1}^{\flbeta} \nu_j u_\nu$ is odd so
$|\{\nu : \nu_j u_\nu \text{ is odd}\}|$ is odd.
Clearly, the sign of \eqref{eq:mondrian_app_bias_odd_1} is preserved.
Again the sign of the first product in \eqref{eq:mondrian_app_bias_odd_2}
is preserved, and the sign of every term in \eqref{eq:mondrian_app_bias_odd_2}
is inverted. However there are now an odd number of terms in the
second product, so its overall sign is inverted.
Therefore the expected product
of \eqref{eq:mondrian_app_bias_odd_1} and \eqref{eq:mondrian_app_bias_odd_2}
is again zero.

\proofparagraph{calculating the second-order bias}

Next we calculate some special cases, beginning with
the form of the leading second-order bias,
where the exponent in $\lambda$ is
$|\nu'| + |\nu''| + u \cdot |\nu| = 2$,
proceeding by cases on the values of $|\nu'|$, $|\nu''|$, and $|u|$.
Firstly, if $|\nu'| = 2$ then $|\nu''| = |u| = 0$.
Note that if any $\nu'_j = 1$ then the expectation in
\eqref{eq:mondrian_app_bias} is zero.
Hence we can assume $\nu'_j \in \{0, 2\}$, yielding
%
\begin{align*}
\frac{1}{2 \lambda^2}
\! \sum_{j=1}^d
\frac{\partial^2 \mu(x)}{\partial x_j^2}
\frac{1}{3}
\E \! \left[
\frac{E_{2j}^{3} + E_{1j}^{3}} {E_{2j} + E_{1j}}
\right]
&\!=
\frac{1}{2 \lambda^2}
\! \sum_{j=1}^d
\frac{\partial^2 \mu(x)}{\partial x_j^2}
\frac{1}{3}
\E \! \left[
E_{1j}^{2}
+ E_{2j}^{2}
- E_{1j} E_{2j}
\right]
= \frac{1}{2 \lambda^2}
\! \sum_{j=1}^d
\frac{\partial^2 \mu(x)}{\partial x_j^2},
\end{align*}
%
where we used that $E_{1j}$ and $E_{2j}$ are independent $\Exp(1)$.
Next we consider $|\nu'| = 1$ and $|\nu''| = 1$, so $|u| = 0$.
Note that if $\nu'_j = \nu''_{j'} = 1$ with $j \neq j'$ then the
expectation in \eqref{eq:mondrian_app_bias} is zero.
So we need only consider $\nu'_j = \nu''_j = 1$, giving
%
\begin{align*}
\frac{1}{\lambda^2}
\frac{1}{f(x)}
\sum_{j=1}^{d}
\frac{\partial \mu(x)}{\partial x_j}
\frac{\partial f(x)}{\partial x_j}
\frac{1}{3}
\E \left[
\frac{E_{2j}^{3} + E_{1j}^{3}}
{E_{2j} + E_{1j}}
\right]
&=
\frac{1}{\lambda^2}
\frac{1}{f(x)}
\sum_{j=1}^{d}
\frac{\partial \mu(x)}{\partial x_j}
\frac{\partial f(x)}{\partial x_j}.
\end{align*}
%
Finally, we have the case where $|\nu'| = 1$, $|\nu''| = 0$
and $|u|=1$. Then $u_\nu = 1$ for some $|\nu| = 1$ and zero otherwise.
Note that if $\nu'_j = \nu_{j'} = 1$ with $j \neq j'$ then the
expectation is zero. So we need only consider $\nu'_j = \nu_j = 1$, giving
%
\begin{align*}
&- \frac{1}{\lambda^2}
\frac{1}{f(x)}
\sum_{j=1}^{d}
\frac{\partial \mu(x)}{\partial x_j}
\frac{\partial f(x)}{\partial x_j}
\frac{1}{4}
\E \left[
\frac{(E_{2j}^2 - E_{1j}^2)^2}
{(E_{2j} + E_{1j})^2}
\right] \\
&\quad=
- \frac{1}{4 \lambda^2}
\frac{1}{f(x)}
\sum_{j=1}^{d}
\frac{\partial \mu(x)}{\partial x_j}
\frac{\partial f(x)}{\partial x_j}
\E \left[
E_{1j}^2
+ E_{2j}^2
- 2 E_{1j} E_{2j}
\right]
=
- \frac{1}{2 \lambda^2}
\frac{1}{f(x)}
\sum_{j=1}^{d}
\frac{\partial \mu(x)}{\partial x_j}
\frac{\partial f(x)}{\partial x_j}.
\end{align*}
%
Hence the second-order bias term is
%
\begin{align*}
\frac{1}{2 \lambda^2}
\sum_{j=1}^d
\frac{\partial^2 \mu(x)}{\partial x_j^2}
+ \frac{1}{2 \lambda^2}
\frac{1}{f(x)}
\sum_{j=1}^{d}
\frac{\partial \mu(x)}{\partial x_j}
\frac{\partial f(x)}{\partial x_j}.
\end{align*}

\proofparagraph{calculating the bias if the data is uniformly distributed}

If $X_i \sim \Unif\big([0,1]^d\big)$ then $f(x) = 1$ and
the bias expansion from \eqref{eq:mondrian_app_bias} becomes
%
\begin{align*}
\sum_{|\nu'|=1}^{\flbeta}
\lambda^{- |\nu'|}
\frac{\partial^{\nu'} \mu(x)}{\nu' !}
\prod_{j=1}^d
\E \left[
\frac{E_{2j}^{\nu'_j + 1}
+ (-1)^{\nu'_j} E_{1j}^{\nu'_j + 1}}
{(\nu'_j + 1) (E_{2j} + E_{1j})}
\right].
\end{align*}
%
This is zero if any $\nu_j'$ is odd,
so we group these terms based on the exponent of $\lambda$ to see
%
\begin{align*}
\frac{B_r(x)}{\lambda^{2r}}
&=
\frac{1}{\lambda^{2r}}
\sum_{|\nu|=r}
\frac{\partial^{2 \nu} \mu(x)}{(2 \nu) !}
\prod_{j=1}^d
\frac{1}{2\nu_j + 1}
\E \left[
\frac{E_{2j}^{2\nu_j + 1} + E_{1j}^{2\nu_j + 1}}
{E_{2j} + E_{1j}}
\right].
\end{align*}
%
Since $\int_0^\infty \frac{e^{-t}}{a+t} \diff t = e^a \Gamma(0,a)$
and $\int_0^\infty s^a \Gamma(0, a) \diff s = \frac{a!}{a+1}$,
with $\Gamma(0, a) = \int_a^\infty \frac{e^{-t}}{t} \diff t$
the upper incomplete gamma function,
the expectation is easily calculated as
%
\begin{align*}
\E \left[
\frac{E_{2j}^{2\nu_j + 1} + E_{1j}^{2\nu_j + 1}}
{E_{2j} + E_{1j}}
\right]
&=
2
\int_{0}^{\infty}
s^{2\nu_j + 1}
e^{-s}
\int_{0}^{\infty}
\frac{e^{-t}}
{s + t}
\diff t
\diff s \\
&=
2 \int_{0}^{\infty}
s^{2\nu_j + 1}
\Gamma(0, s)
\diff s
=
\frac{(2 \nu_j + 1)!}{\nu_j + 1},
\end{align*}
%
so finally
%
\begin{align*}
\frac{B_r(x)}{\lambda^{2r}}
&=
\frac{1}{\lambda^{2r}}
\sum_{|\nu|=r}
\frac{\partial^{2 \nu} \mu(x)}{(2 \nu) !}
\prod_{j=1}^d
\frac{1}{2\nu_j + 1}
\frac{(2 \nu_j + 1)!}{\nu_j + 1}
=
\frac{1}{\lambda^{2r}}
\sum_{|\nu|=r}
\partial^{2 \nu} \mu(x)
\prod_{j=1}^d
\frac{1}{\nu_j + 1}.
\end{align*}
%
\end{proof}

\begin{proof}[Theorem~\ref{thm:mondrian_variance_estimation}]
This follows from the debiased version in
Theorem~\ref{thm:mondrian_variance_estimation_debiased}
with $J=0$, $a_0 = 1$, and $\omega_0 = 1$.
\end{proof}

\begin{proof}[Theorem~\ref{thm:mondrian_confidence}]
%
By Theorem~\ref{thm:mondrian_bias}
and Theorem~\ref{thm:mondrian_variance_estimation},
%
\begin{align*}
\sqrt{\frac{n}{\lambda^d}}
\frac{\hat \mu(x) - \mu(x)}{\hat \Sigma(x)^{1/2}}
&=
\sqrt{\frac{n}{\lambda^d}}
\frac{\hat \mu(x) - \E \left[ \hat \mu(x) \mid \bX, \bT \right]}
{\hat \Sigma(x)^{1/2}}
+ \sqrt{\frac{n}{\lambda^d}}
\frac{\E \left[ \hat \mu(x) \mid \bX, \bT \right] - \mu(x)}
{\hat \Sigma(x)^{1/2}} \\
&=
\sqrt{\frac{n}{\lambda^d}}
\frac{\hat \mu(x) - \E \left[ \hat \mu(x) \mid \bX, \bT \right]}
{\hat \Sigma(x)^{1/2}}
+ \sqrt{\frac{n}{\lambda^d}} \,
O_\P \left(
\frac{1}{\lambda^{\beta \wedge 2}}
+ \frac{1}{\lambda \sqrt B}
+ \frac{\log n}{\lambda} \sqrt{\frac{\lambda^d}{n}}
\right).
\end{align*}
%
The first term now converges weakly to $\cN(0,1)$ by
Slutsky's theorem, Theorem~\ref{thm:mondrian_clt},
and Theorem~\ref{thm:mondrian_variance_estimation},
while the second term is $o_\P(1)$ by assumption.
Validity of the confidence interval follows immediately.
%
\end{proof}

\subsection{Debiased Mondrian random forests}

We give rigorous proofs of the central limit theorem,
bias characterization, variance estimation,
confidence interval validity, and minimax optimality
results for the debiased Mondrian random forest estimator.

\begin{proof}[Theorem~\ref{thm:mondrian_clt_debiased}]

We use the martingale central limit theorem given by
\citet[Theorem~3.2]{hall1980martingale}.
For each $1 \leq i \leq n$ define
$\cH_{n i}$ to be the filtration
generated by $\bT$, $\bX$, and
$(\varepsilon_j : 1 \leq j \leq i)$,
noting that $\cH_{n i} \subseteq \cH_{(n+1)i}$
because $B$ increases weakly as $n$ increases.
Let $\I_{i b r}(x) = \I\{X_i \in T_{b r}(x)\}$
where $T_{b r}(x)$ is the cell containing $x$ in tree $b$
used to construct $\hat \mu_r(x)$,
and similarly let $N_{b r}(x) = \sum_{i=1}^n \I_{i b r}(x)$
and $\I_{b r}(x) = \I\{N_{b r}(x) \geq 1\}$.
Define the $\cH_{n i}$-measurable and square integrable
variables
%
\begin{align*}
S_i(x)
&=
\sqrt{\frac{n}{\lambda^d}}
\sum_{r=0}^{J}
\omega_r
\frac{1}{B} \sum_{b=1}^B
\frac{\I_{i b r}(x) \varepsilon_i} {N_{b r}(x)},
\end{align*}
%
which satisfy the martingale
difference property
$\E [ S_i(x) \mid \cH_{n i} ] = 0$.
Further,
%
\begin{align*}
\sqrt{\frac{n}{\lambda^d}}
\big(
\hat\mu_\rd(x)
- \E\left[
\hat\mu_\rd(x) \mid \bX, \bT
\right]
\big)
= \sum_{i=1}^n S_i(x).
\end{align*}
%
By \citet[Theorem~3.2]{hall1980martingale}
it suffices to check that
%
\begin{inlineroman}
\item $\max_i |S_i(x)| \to 0$ in probability,%
\label{it:mondrian_app_hall_prob}
\item $\E\left[\max_i S_i(x)^2\right] \lesssim 1$, and%
\label{it:mondrian_app_hall_exp}
\item $\sum_i S_i(x)^2 \to \Sigma_\rd(x)$ in probability.
\label{it:mondrian_app_hall_var}
\end{inlineroman}

\proofparagraph{checking condition \ref{it:mondrian_app_hall_prob}}
%
Since $J$ is fixed and
$\E[|\varepsilon_i|^3 \mid X_i]$ is bounded,
by Jensen's inequality and
Lemma~\ref{lem:mondrian_app_simple_moment_denominator},
%
\begin{align*}
\E\left[\max_{1 \leq i \leq n} |S_i(x)| \right]
&=
\E\left[\max_{1 \leq i \leq n}
\left|
\sqrt{\frac{n}{\lambda^d}}
\sum_{r=0}^{J}
\omega_r
\frac{1}{B} \sum_{b=1}^B
\frac{\I_{i b r}(x) \varepsilon_i} {N_{b r}(x)}
\right|
\right] \\
&\leq
\sqrt{\frac{n}{\lambda^d}}
\sum_{r=0}^{J}
|\omega_r|
\frac{1}{B}
\E\left[\max_{1 \leq i \leq n}
\left|
\sum_{b=1}^B
\frac{\I_{i b r}(x) \varepsilon_i} {N_{b r}(x)}
\right|
\right] \\
&\leq
\sqrt{\frac{n}{\lambda^d}}
\sum_{r=0}^{J}
|\omega_r|
\frac{1}{B}
\E\left[
\sum_{i=1}^{n}
\left(
\sum_{b=1}^B
\frac{\I_{i b r}(x) |\varepsilon_i|} {N_{b r}(x)}
\right)^3
\right]^{1/3} \\
&=
\sqrt{\frac{n}{\lambda^d}}
\sum_{r=0}^{J}
|\omega_r|
\frac{1}{B}
\E\left[
\sum_{i=1}^{n}
|\varepsilon_i|^3
\sum_{b=1}^B
\sum_{b'=1}^B
\sum_{b''=1}^B
\frac{\I_{i b r}(x) } {N_{b r}(x)}
\frac{\I_{i b' r}(x) } {N_{b' r}(x)}
\frac{\I_{i b'' r}(x) } {N_{b'' r}(x)}
\right]^{1/3} \\
&\lesssim
\sqrt{\frac{n}{\lambda^d}}
\sum_{r=0}^{J}
|\omega_r|
\frac{1}{B^{2/3}}
\E\left[
\sum_{b=1}^B
\sum_{b'=1}^B
\frac{\I_{b r}(x)} {N_{b r}(x)}
\frac{\I_{b' r}(x)} {N_{b' r}(x)}
\right]^{1/3} \\
&\lesssim
\sqrt{\frac{n}{\lambda^d}}
\sum_{r=0}^{J}
|\omega_r|
\frac{1}{B^{2/3}}
\left(
B^2 \frac{a_r^{2d} \lambda^{2d}}{n^2}
+ B \frac{a_r^{2d} \lambda^{2d} \log n}{n^2}
\right)^{1/3} \\
&\lesssim
\left( \frac{\lambda^d}{n} \right)^{1/6}
+ \left( \frac{\lambda^d}{n} \right)^{1/6}
\left( \frac{\log n}{B} \right)^{1/3}
\to 0.
\end{align*}

\proofparagraph{checking condition \ref{it:mondrian_app_hall_exp}}
%
Since $\E[\varepsilon_i^2 \mid X_i]$ is bounded
and by Lemma~\ref{lem:mondrian_app_simple_moment_denominator},
%
\begin{align*}
\E\left[\max_{1 \leq i \leq n} S_i(x)^2 \right]
&=
\E\left[
\max_{1 \leq i \leq n}
\left(
\sqrt{\frac{n}{\lambda^d}}
\sum_{r=0}^{J}
\omega_r
\frac{1}{B} \sum_{b=1}^B
\frac{\I_{i b r}(x) \varepsilon_i} {N_{b r}(x)}
\right)^2
\right] \\
&\leq
\frac{n}{\lambda^d}
\frac{1}{B^2}
(J+1)^2
\max_{0 \leq r \leq J}
\omega_r^2
\,\E\left[
\sum_{i=1}^{n}
\sum_{b=1}^B
\sum_{b'=1}^B
\frac{\I_{i b r}(x) \I_{i b' r}(x) \varepsilon_i^2}
{N_{b r}(x) N_{b' r}(x)}
\right] \\
&\lesssim
\frac{n}{\lambda^d}
\max_{0 \leq r \leq J}
\E\left[
\frac{\I_{b r}(x)}{N_{b r}(x)}
\right]
\lesssim
\frac{n}{\lambda^d}
\max_{0 \leq r \leq J}
\frac{a_r^d \lambda^d}{n}
\lesssim 1.
\end{align*}

\proofparagraph{checking condition \ref{it:mondrian_app_hall_var}}

Next, we have
%
\begin{align}
\label{eq:mondrian_app_clt_condition_sum}
\sum_{i=1}^n
S_i(x)^2
&=
\sum_{i=1}^n
\left(
\sqrt{\frac{n}{\lambda^d}}
\sum_{r=0}^{J}
\omega_r
\frac{1}{B} \sum_{b=1}^B
\frac{\I_{i b r}(x) \varepsilon_i} {N_{b r}(x)}
\right)^2 \\
&=
\nonumber
\frac{n}{\lambda^d}
\frac{1}{B^2}
\sum_{i=1}^n
\sum_{r=0}^{J}
\sum_{r'=0}^{J}
\omega_r
\omega_{r'}
\sum_{b=1}^B
\sum_{b'=1}^B
\frac{\I_{i b r}(x) \I_{i b' r'}(x) \varepsilon_i^2}
{N_{b r}(x) N_{b' r'}(x)} \\
\nonumber
&=
\frac{n}{\lambda^d}
\frac{1}{B^2}
\sum_{i=1}^n
\sum_{r=0}^{J}
\sum_{r'=0}^{J}
\omega_r
\omega_{r'}
\sum_{b=1}^B
\left(
\frac{\I_{i b r}(x) \I_{i b r'}(x) \varepsilon_i^2}
{N_{b r}(x) N_{b r'}(x)}
+ \sum_{b' \neq b}
\frac{\I_{i b r}(x) \I_{i b' r'}(x) \varepsilon_i^2}
{N_{b r}(x) N_{b' r'}(x)}
\right).
\end{align}
%
By boundedness of $\E[\varepsilon_i^2 \mid X_i]$
and Lemma~\ref{lem:mondrian_app_simple_moment_denominator},
the first term in \eqref{eq:mondrian_app_clt_condition_sum}
vanishes as
%
\begin{align*}
\frac{n}{\lambda^d}
\frac{1}{B^2}
\sum_{i=1}^n
\sum_{r=0}^{J}
\sum_{r'=0}^{J}
\omega_r
\omega_{r'}
\sum_{b=1}^B
\E \left[
\frac{\I_{i b r}(x) \I_{i b r'}(x) \varepsilon_i^2}
{N_{b r}(x) N_{b r'}(x)}
\right]
&\lesssim
\frac{n}{\lambda^d}
\frac{1}{B^2}
\max_{0 \leq r \leq J}
\sum_{b=1}^B
\E \left[
\frac{\I_{b r}(x)}{N_{b r}(x)}
\right]
\lesssim
\frac{1}{B}
\to 0.
\end{align*}
%
For the second term in \eqref{eq:mondrian_app_clt_condition_sum},
the law of total variance gives
%
\begin{align}
\nonumber
&\Var \left[
\frac{n}{\lambda^d}
\frac{1}{B^2}
\sum_{i=1}^n
\sum_{r=0}^{J}
\sum_{r'=0}^{J}
\omega_r
\omega_{r'}
\sum_{b=1}^B
\sum_{b' \neq b}
\frac{\I_{i b r}(x) \I_{i b' r'}(x) \varepsilon_i^2}
{N_{b r}(x) N_{b' r'}(x)}
\right] \\
\nonumber
&\quad\leq
(J+1)^4
\max_{0 \leq r, r' \leq J}
\omega_r
\omega_{r'}
\Var \left[
\frac{n}{\lambda^d}
\frac{1}{B^2}
\sum_{i=1}^n
\sum_{b=1}^B
\sum_{b' \neq b}
\frac{\I_{i b r}(x) \I_{i b' r'}(x) \varepsilon_i^2}
{N_{b r}(x) N_{b' r'}(x)}
\right] \\
\nonumber
&\quad\lesssim
\max_{0 \leq r, r' \leq J}
\E \left[
\Var \left[
\frac{n}{\lambda^d}
\frac{1}{B^2}
\sum_{i=1}^n
\sum_{b=1}^B
\sum_{b' \neq b}
\frac{\I_{i b r}(x) \I_{i b' r'}(x) \varepsilon_i^2}
{N_{b r}(x) N_{b' r'}(x)}
\Bigm| \bX, \bY
\right]
\right] \\
\label{eq:mondrian_app_total_variance}
&\qquad+
\max_{0 \leq r, r' \leq J}
\Var \left[
\E \left[
\frac{n}{\lambda^d}
\frac{1}{B^2}
\sum_{i=1}^n
\sum_{b=1}^B
\sum_{b' \neq b}
\frac{\I_{i b r}(x) \I_{i b' r'}(x) \varepsilon_i^2}
{N_{b r}(x) N_{b' r'}(x)}
\Bigm| \bX, \bY
\right]
\right]
\end{align}
%
For the first term in \eqref{eq:mondrian_app_total_variance},
%
\begin{align*}
&\E \left[
\Var \left[
\frac{n}{\lambda^d}
\frac{1}{B^2}
\sum_{i=1}^n
\sum_{b=1}^B
\sum_{b' \neq b}
\frac{\I_{i b r}(x) \I_{i b' r'}(x) \varepsilon_i^2}
{N_{b r}(x) N_{b' r'}(x)}
\Bigm| \bX, \bY
\right]
\right] \\
&\quad=
\frac{n^2}{\lambda^{2d}}
\frac{1}{B^4}
\sum_{i=1}^n
\sum_{j=1}^n
\sum_{b=1}^B
\sum_{b' \neq b}
\sum_{\tilde b=1}^B
\sum_{\tilde b' \neq \tilde b}
\E \Bigg[
\varepsilon_i^2
\varepsilon_j^2
\left(
\frac{\I_{i b r}(x) \I_{i b' r'}(x) }
{N_{b r}(x) N_{b' r'}(x)}
- \E
\left[
\frac{\I_{i b r}(x) \I_{i b' r'}(x) }
{N_{b r}(x) N_{b' r'}(x)}
\Bigm| \bX
\right]
\right) \\
&\qquad\quad
\times
\left(
\frac{\I_{j \tilde b r}(x) \I_{j \tilde b' r'}(x) }
{N_{\tilde b r}(x) N_{ \tilde b' r'}(x)}
- \E
\left[
\frac{\I_{j \tilde b r}(x) \I_{j \tilde b' r'}(x) }
{N_{\tilde b r}(x) N_{\tilde b' r'}(x)}
\Bigm| \bX
\right]
\right)
\Bigg].
\end{align*}
%
Since $T_{b r}$ is independent of $T_{b' r'}$ given
$\bX, \bY$, the summands are zero
whenever $\big|\{b, b', \tilde b, \tilde b'\}\big| = 4$.
Since $\E[ \varepsilon_i^2 \mid X_i]$ is bounded
and by the Cauchy--Schwarz inequality
and Lemma~\ref{lem:mondrian_app_simple_moment_denominator},
%
\begin{align*}
&\E \left[
\Var \left[
\frac{n}{\lambda^d}
\frac{1}{B^2}
\sum_{i=1}^n
\sum_{b=1}^B
\sum_{b' \neq b}
\frac{\I_{i b r}(x) \I_{i b' r'}(x) \varepsilon_i^2}
{N_{b r}(x) N_{b' r'}(x)}
\Bigm| \bX, \bY
\right]
\right] \\
&\quad\lesssim
\frac{n^2}{\lambda^{2d}}
\frac{1}{B^3}
\sum_{b=1}^B
\sum_{b' \neq b}
\E \left[
\left(
\sum_{i=1}^n
\frac{\I_{i b r}(x) \I_{i b' r'}(x) }
{N_{b r}(x) N_{b' r'}(x)}
\right)^2
\right]
\lesssim
\frac{n^2}{\lambda^{2d}}
\frac{1}{B}
\E \left[
\frac{\I_{b r}(x)}{N_{b r}(x)}
\frac{\I_{b' r'}(x)}{N_{b' r'}(x)}
\right]
\lesssim
\frac{1}{B}
\to 0.
\end{align*}
%
For the second term in \eqref{eq:mondrian_app_total_variance},
the random variable inside the variance is a nonlinear
function of the i.i.d.\ variables $(X_i, \varepsilon_i)$,
so we apply the Efron--Stein inequality
\citep{efron1981jackknife}.
Let $(\tilde X_{i j}, \tilde Y_{i j}) = (X_i, Y_i)$
if $i \neq j$ and be an
independent copy of $(X_j, Y_j)$,
denoted $(\tilde X_j, \tilde Y_j)$, if $i = j$,
and define $\tilde \varepsilon_{i j} = \tilde Y_{i j} - \mu(\tilde X_{i j})$.
Write
$\tilde \I_{i j b r}(x) = \I \big\{ \tilde X_{i j} \in T_{b r}(x) \big\}$
and
$\tilde \I_{j b r}(x) = \I \big\{ \tilde X_{j} \in T_{b r}(x) \big\}$,
and also
$\tilde N_{j b r}(x) = \sum_{i=1}^{n} \tilde \I_{i j b r}(x)$.
We use the leave-one-out notation
$N_{-j b r}(x) = \sum_{i \neq j} \I_{i b r}(x)$
and also write
$N_{-j b r \cap b' r'} = \sum_{i \neq j} \I_{i b r}(x) \I_{i b' r'}(x)$.
Since $\E[ \varepsilon_i^4 \mid X_i]$ is bounded,
%
\begin{align*}
&\Var \left[
\E \left[
\frac{n}{\lambda^d}
\frac{1}{B^2}
\sum_{i=1}^n
\sum_{b=1}^B
\sum_{b' \neq b}
\frac{\I_{i b r}(x) \I_{i b' r'}(x) \varepsilon_i^2}
{N_{b r}(x) N_{b' r'}(x)}
\Bigm| \bX, \bY
\right]
\right] \\
&\quad\leq
\Var \left[
\E \left[
\frac{n}{\lambda^d}
\sum_{i=1}^n
\frac{\I_{i b r}(x) \I_{i b' r'}(x) \varepsilon_i^2}
{N_{b r}(x) N_{b' r'}(x)}
\Bigm| \bX, \bY
\right]
\right] \\
&\quad\leq
\frac{1}{2}
\frac{n^2}{\lambda^{2d}}
\sum_{j=1}^{n}
\E \left[
\left(
\sum_{i=1}^n
\left(
\frac{\I_{i b r}(x) \I_{i b' r}(x) \varepsilon_i^2}
{N_{b r}(x) N_{b' r'}(x)}
- \frac{\tilde \I_{i j b r}(x) \tilde \I_{i j b' r'}(x)
\tilde \varepsilon_{i j}^2}
{\tilde N_{j b r}(x) \tilde N_{j b' r'}(x)}
\right)
\right)^2
\right] \\
&\quad\leq
\frac{n^2}{\lambda^{2d}}
\sum_{j=1}^{n}
\E \left[
\left(
\left|
\frac{1}
{N_{b }(x) N_{b' r'}(x)}
- \frac{1}
{\tilde N_{j b r}(x) \tilde N_{j b' r'}(x)}
\right|
\sum_{i \neq j}
\I_{i b r}(x) \I_{i b' r'}(x) \varepsilon_i^2
\right)^2
\right] \\
&\qquad+
\frac{n^2}{\lambda^{2d}}
\sum_{j=1}^{n}
\E \left[
\left(
\left(
\frac{\I_{j b r}(x) \I_{j b' r'}(x) \varepsilon_j^2}
{N_{b r}(x) N_{b' r'}(x)}
- \frac{\tilde \I_{j b r}(x) \tilde \I_{j b' r'}(x)
\tilde \varepsilon_j^2}
{\tilde N_{j b r}(x) \tilde N_{j b' r'}(x)}
\right)
\right)^2
\right] \\
&\quad\lesssim
\frac{n^2}{\lambda^{2d}}
\sum_{j=1}^{n}
\E \left[
N_{-j b r \cap b' r}(x)^2
\left|
\frac{1}
{N_{b r}(x) N_{b' r'}(x)}
- \frac{1}
{\tilde N_{j b r}(x) \tilde N_{j b' r'}(x)}
\right|^2
+ \frac{\I_{j b r}(x) \I_{j b' r'}(x)}
{N_{b r}(x)^2 N_{b' r'}(x)^2}
\right].
\end{align*}
%
For the first term in the above display, note that
%
\begin{align*}
&\left|
\frac{1}{N_{b r}(x) N_{b' r'}(x)}
- \frac{1} {\tilde N_{j b r}(x) \tilde N_{j b' r'}(x)}
\right| \\
&\quad\leq
\frac{1}{N_{b r}(x)}
\left|
\frac{1} {N_{b' r'}(x)} - \frac{1} {\tilde N_{j b' r'}(x)}
\right|
+ \frac{1}{\tilde N_{j b' r'}(x)}
\left|
\frac{1} {N_{b r}(x)} - \frac{1} {\tilde N_{j b r}(x)}
\right| \\
&\quad\leq
\frac{1}{N_{-j b r}(x)}
\frac{1} {N_{-j b' r'}(x)^2}
+ \frac{1}{N_{-j b' r'}(x)}
\frac{1} {N_{-j b r}(x)^2}
\end{align*}
%
since $|N_{b r}(x) - \tilde N_{j b r}(x)| \leq 1$
and $|N_{b' r'}(x) - \tilde N_{j b' r'}(x)| \leq 1$.
Further, these terms are non-zero only on the events
$\{ X_j \in T_{b r}(x) \} \cup \{ \tilde X_j \in T_{b r}(x) \}$
and $\{ X_j \in T_{b' r'}(x) \} \cup \{ \tilde X_j \in T_{b' r'}(x) \}$
respectively, so
%
\begin{align*}
&\Var \left[
\E \left[
\frac{n}{\lambda^d}
\frac{1}{B^2}
\sum_{i=1}^n
\sum_{b=1}^B
\sum_{b' \neq b}
\frac{\I_{i b r}(x) \I_{i b' r'}(x) \varepsilon_i^2}
{N_{b r}(x) N_{b' r'}(x)}
\Bigm| \bX, \bY
\right]
\right] \\
&\, \lesssim
\frac{n^2}{\lambda^{2d}}
\sum_{j=1}^{n}
\E \left[
\frac{\I_{j b' r'}(x) + \tilde \I_{j b' r'}(x)}{N_{-j b r}(x)^2}
\frac{N_{-j b r \cap b' r}(x)^2} {N_{-j b' r'}(x)^4}
\right. \\
&\left.
\qquad+
\frac{\I_{j b r}(x) + \tilde \I_{j b r}(x)}{N_{-j b' r'}(x)^2}
\frac{N_{-j b r \cap b' r}(x)^2} {N_{-j b r}(x)^4}
+
\frac{\I_{j b r}(x) \I_{j b' r'}(x)}
{N_{b r}(x)^2 N_{b' r'}(x)^2}
\right] \\
&\, \lesssim
\frac{n^2}{\lambda^{2d}}
\sum_{j=1}^{n}
\E \left[
\frac{\I_{j b r}(x) \I_{b r}(x) \I_{b' r'}(x)}
{N_{b r}(x)^2 N_{b' r'}(x)^2}
\right]
\lesssim
\frac{n^2}{\lambda^{2d}}
\E \left[
\frac{\I_{b r}(x) \I_{b' r'}(x)}
{N_{b r}(x) N_{b' r'}(x)^2}
\right] \\
&\lesssim
\frac{n^2}{\lambda^{2d}}
\frac{\lambda^d}{n}
\frac{\lambda^{2d} \log n}{n^2}
\lesssim
\frac{\lambda^d \log n}{n}
\to 0,
\end{align*}
%
where we used Lemma~\ref{lem:mondrian_app_simple_moment_denominator}.
So
$\sum_{i=1}^{n} S_i(x)^2 - n \,\E \left[ S_i(x)^2 \right]
= O_\P \left( \frac{1}{\sqrt B} + \sqrt{\frac{\lambda^d \log n}{n}} \right)
= o_\P(1)$.

\proofparagraph{calculating the limiting variance}
%
Thus by \citet[Theorem~3.2]{hall1980martingale}
we conclude that
%
\begin{align*}
\sqrt{\frac{n}{\lambda^d}}
\big(
\hat\mu_\rd(x)
- \E\left[
\hat\mu_\rd(x) \mid \bX, \bT
\right]
\big)
&\rightsquigarrow
\cN\big(0, \Sigma_\rd(x)\big)
\end{align*}
%
as $n \to \infty$, assuming that the limit
%
\begin{align*}
\Sigma_\rd(x)
&=
\lim_{n \to \infty}
\sum_{r=0}^{J}
\sum_{r'=0}^{J}
\omega_r
\omega_{r'}
\frac{n^2}{\lambda^d}
\E \left[
\frac{\I_{i b r}(x) \I_{i b' r'}(x) \varepsilon_i^2}
{N_{b r}(x) N_{b' r'}(x)}
\right]
\end{align*}
%
exists. Now we verify this and calculate the limit.
Since $J$ is fixed, it suffices to find
%
\begin{align*}
\lim_{n \to \infty}
\frac{n^2}{\lambda^d}
\E \left[
\frac{\I_{i b r}(x) \I_{i b' r'}(x) \varepsilon_i^2}
{N_{b r}(x) N_{b' r'}(x)}
\right]
\end{align*}
%
for each $0 \leq r, r' \leq J$.
Firstly, note that
%
\begin{align*}
\frac{n^2}{\lambda^d}
\E \left[
\frac{\I_{i b r}(x) \I_{i b' r'}(x) \varepsilon_i^2}
{N_{b r}(x) N_{b' r'}(x)}
\right]
&=
\frac{n^2}{\lambda^d}
\E \left[
\frac{\I_{i b r}(x) \I_{i b' r'}(x) \sigma^2(X_i)}
{N_{b r}(x) N_{b' r'}(x)}
\right] \\
&=
\frac{n^2}{\lambda^d}
\sigma^2(x)
\E \left[
\frac{\I_{i b r}(x) \I_{i b' r'}(x)}
{N_{b r}(x) N_{b' r'}(x)}
\right] \\
&\quad+
\frac{n^2}{\lambda^d}
\E \left[
\frac{\I_{i b r}(x) \I_{i b' r'}(x)
\big(\sigma^2(X_i) - \sigma^2(x) \big)}
{N_{b r}(x) N_{b' r'}(x)}
\right].
\end{align*}
%
Since $\sigma^2$ is Lipschitz and
$\P \left(\max_{1 \leq l \leq d}
|T_b(x)_l| \geq t/\lambda \right) \leq 2d e^{-t/2}$
by Lemma~\ref{lem:mondrian_app_largest_cell},
%
\begin{align*}
\frac{n^2}{\lambda^d}
\E \left[
\frac{\I_{i b r}(x) \I_{i b' r'}(x)
\big|\sigma^2(X_i) - \sigma^2(x) \big|}
{N_{b r}(x) N_{b' r'}(x)}
\right]
&\leq
2de^{-t/2}
\frac{n^2}{\lambda^d}
+ \frac{n^2}{\lambda^d}
\frac{t}{\lambda}
\E \left[
\frac{\I_{i b r}(x) \I_{i b' r'}(x)}
{N_{b r}(x) N_{b' r'}(x)}
\right] \\
&\lesssim
\frac{n^2}{\lambda^d}
\frac{\log n}{\lambda}
\frac{\lambda^d}{n^2}
\lesssim
\frac{\log n}{\lambda},
\end{align*}
%
by Lemma~\ref{lem:mondrian_app_simple_moment_denominator},
where we set $t = 4 \log n$.
Therefore
%
\begin{align*}
\frac{n^2}{\lambda^d}
\E \left[
\frac{\I_{i b r}(x) \I_{i b' r'}(x) \varepsilon_i^2}
{N_{b r}(x) N_{b' r'}(x)}
\right]
&=
\sigma^2(x)
\frac{n^2}{\lambda^d}
\E \left[
\frac{\I_{i b r}(x) \I_{i b' r'}(x)}
{N_{b r}(x) N_{b' r'}(x)}
\right]
+ O \left( \frac{\log n}{\lambda} \right).
\end{align*}
%
Next, by conditioning on
$T_{b r}$, $T_{b' r'}$, $N_{-i b r}(x)$, and $N_{-i b' r'}(x)$,
%
\begin{align*}
&\E \left[
\frac{\I_{i b r}(x) \I_{i b' r'}(x)}
{N_{b r}(x) N_{b' r'}(x)}
\right]
= \E \left[
\frac{\int_{T_{b r}(x) \cap T_{b' r'}(x)} f(\xi) \diff \xi}
{(N_{-i b r}(x)+1) (N_{-i b' r'}(x)+1)}
\right] \\
&\quad= f(x) \,
\E \left[
\frac{|T_{b r}(x) \cap T_{b' r'}(x)|}
{(N_{-i b r}(x)+1) (N_{-i b' r'}(x)+1)}
\right]
+
\E \left[
\frac{\int_{T_{b r}(x) \cap T_{b' r'}(x)}
(f(\xi) - f(x)) \diff \xi}
{(N_{-i b r}(x)+1) (N_{-i b' r'}(x)+1)}
\right] \\
&\quad=
f(x) \,
\E \left[
\frac{|T_{b r}(x) \cap T_{b' r'}(x)|}
{(N_{-i b r}(x)+1) (N_{-i b' r'}(x)+1)}
\right]
+ O \left(
\frac{\lambda^d}{n^2}
\frac{(\log n)^{d+1}}{\lambda}
\right)
\end{align*}
%
arguing using Lemma~\ref{lem:mondrian_app_largest_cell},
the Lipschitz property of $f(x)$,
and Lemma~\ref{lem:mondrian_app_simple_moment_denominator}. So
%
\begin{align*}
\frac{n^2}{\lambda^d}
\E \! \left[
\frac{\I_{i b r}(x) \I_{i b' r'}(x) \varepsilon_i^2}
{N_{b r}(x) N_{b' r'}(x)}
\right]
&=
\sigma^2(x)
f(x)
\frac{n^2}{\lambda^d}
\E \! \left[
\frac{|T_{b r}(x) \cap T_{b' r'}(x)|}
{(N_{-i b r}(x)+1) (N_{-i b' r'}(x)+1)}
\right]
\! + O \! \left(
\frac{(\log n)^{d+1}}{\lambda}
\right).
\end{align*}
%
Now we apply the binomial result in
Lemma~\ref{lem:mondrian_app_binomial_expectation}
to approximate the expectation. With
$N_{-i b' r' \setminus b r}(x) =
\sum_{j \neq i} \I\{X_j \in T_{b' r'}(x) \setminus T_{b r}(x)\}$,
%
\begin{align*}
&\E \left[
\frac{|T_{b r}(x) \cap T_{b' r'}(x)|}
{(N_{-i b r}(x)+1) (N_{-i b' r'}(x)+1)}
\right]
= \E \left[
\frac{|T_{b r}(x) \cap T_{b' r'}(x)|}
{N_{-i b r}(x)+1}
\right. \\
&\qquad\left.
\times \,
\E \left[
\frac{1}
{N_{-i b' r' \cap b r}(x)+N_{-i b' r' \setminus b r}(x)+1}
\Bigm| \bT, N_{-i b' r' \cap b r}(x), N_{-i b r \setminus b' r'}(x)
\right]
\right].
\end{align*}
%
Now conditional on
$\bT$, $N_{-i b' r' \cap b r}(x)$, and $N_{-i b r \setminus b' r'}(x)$,
%
\begin{align*}
N_{-i b' r' \setminus b r}(x)
&\sim \Bin\left(
n - 1 - N_{-i b r}(x), \
\frac{\int_{T_{b' r'}(x) \setminus T_{b r}(x)} f(\xi) \diff \xi}
{1 - \int_{T_{b r}(x)}
f(\xi) \diff \xi}
\right).
\end{align*}
%
We bound these parameters above and below.
Firstly, by Lemma~\ref{lem:mondrian_app_active_data} with $B=1$,
%
\begin{align*}
\P \left( N_{-i b r}(x) >
t^{d+1}
\frac{n}{\lambda^d}
\right)
&\leq
4 d e^{- t / (4 \|f\|_\infty(1 + 1/a_r))}
\leq
e^{- t / C}
\end{align*}
%
for some $C > 0$ and sufficiently large $t$.
Next, if $f$ is $L$-Lipschitz in $\ell^2$,
by Lemma~\ref{lem:mondrian_app_largest_cell},
%
\begin{align*}
&\P \left(
\left|
\frac{\int_{T_{b' r'}(x) \setminus T_{b r}(x)} f(\xi) \diff \xi}
{1 - \int_{T_{b r}(x)} f(\xi)
\diff \xi}
- f(x) |T_{b' r'}(x) \setminus T_{b r}(x)|
\right|
> t \, \frac{|T_{b' r'}(x) \setminus T_{b r}(x)|}{\lambda}
\right) \\
&\quad\leq
\P \left(
\int_{T_{b' r'}(x) \setminus T_{b r}(x)}
\left| f(\xi) - f(x) \right|
\diff \xi
> t \, \frac{|T_{b' r'}(x) \setminus T_{b r}(x)|}{2 \lambda}
\right) \\
&\qquad+
\P \left(
\frac{\int_{T_{b' r'}(x) \setminus T_{b r}(x)} f(\xi) \diff \xi
\cdot \int_{T_{b r}(x)} f(\xi) \diff \xi}
{1 - \int_{T_{b r}(x)} f(\xi) \diff \xi}
> t \, \frac{|T_{b' r'}(x) \setminus T_{b r}(x)|}{2\lambda}
\right) \\
&\quad\leq
\P \left(
L d\,
|T_{b' r'}(x) \setminus T_{b r}(x)|
\max_{1 \leq j \leq d} |T_{b' r'}(x)_j|
> t \, \frac{|T_{b' r'}(x) \setminus T_{b r}(x)|}{2\lambda}
\right) \\
&\qquad+
\P \left(
\|f\|_\infty
\,|T_{b' r'}(x) \setminus T_{b r}(x)|
\frac{\|f\|_\infty |T_{b r}(x)|}
{1 - \|f\|_\infty |T_{b r}(x)|}
> t \, \frac{|T_{b' r'}(x) \setminus T_{b r}(x)|}{2\lambda}
\right) \\
&\quad\leq
\P \left(
\max_{1 \leq j \leq d} |T_{b' r'}(x)_j|
> \frac{t}{2\lambda L d}
\right)
+\P \left(
|T_{b r}(x)|
> \frac{t}{4\lambda \|f\|_\infty^2}
\right) \\
&\quad\leq
2 d e^{-t a_r /(4L d)}
+ 2 d e^{-t a_r / (8 \|f\|_\infty^2)}
\leq e^{-t/C},
\end{align*}
%
for large $t$,
increasing $C$ as necessary.
Thus with probability at least $1 - e^{-t/C}$,
increasing $C$,
%
\begin{align*}
N_{-i b' r' \setminus b r}(x)
&\leq \Bin\left(
n, \,
|T_{b' r'}(x) \setminus T_{b r}(x)|
\left( f(x) + \frac{t}{\lambda} \right)
\right) \\
N_{-i b' r' \setminus b r}(x)
&\geq
\Bin\left(
n
\left( 1 - \frac{t^{d+1}}{\lambda^d}
- \frac{1}{n} \right), \,
|T_{b' r'}(x) \setminus T_{b r}(x)|
\left( f(x) - \frac{t}{\lambda} \right)
\right).
\end{align*}
%
So by Lemma~\ref{lem:mondrian_app_binomial_expectation} conditionally on
$\bT$, $N_{-i b' r' \cap b r}(x)$, and $N_{-i b r \setminus b' r'}(x)$,
we have with probability at least $1 - e^{-t/C}$ that
%
\begin{align*}
&\left|
\E \left[
\frac{1}
{N_{-i b' r' \cap b r}(x)+N_{-i b' r' \setminus b r}(x)+1}
\Bigm| \bT, N_{-i b' r' \cap b r}(x), N_{-i b r \setminus b' r'}(x)
\right]
\right.
\\
&\left.
\qquad-
\frac{1}
{N_{-i b' r' \cap b r}(x) + n f(x) |T_{b' r'}(x) \setminus T_{b r}(x)|+1}
\right| \\
&\quad\lesssim
\frac{1 + \frac{n t}{\lambda} |T_{b' r'}(x) \setminus T_{b r}(x)|}
{\left(N_{-i b' r' \cap b r}(x)
+ n |T_{b' r'}(x) \setminus T_{b r}(x)|+1\right)^2}.
\end{align*}
%
Therefore, by the same approach as the proof of
Lemma~\ref{lem:mondrian_app_moment_denominator},
taking $t = 3 C \log n$,
%
\begin{align*}
&
\left|
\E \left[
\frac{|T_{b r}(x) \cap T_{b' r'}(x)|}
{(N_{-i b r}(x)+1) (N_{-i b' r'}(x)+1)}
\right.\right. \\
&\left.\left.
\qquad -
\frac{|T_{b r}(x) \cap T_{b' r'}(x)|}
{(N_{-i b r}(x)+1)
(N_{-i b' r' \cap b r}(x)+n f(x)
|T_{b' r'}(x) \setminus T_{b r}(x)|+1)}
\right]
\right| \\
&\quad\lesssim
\E \left[
\frac{|T_{b r}(x) \cap T_{b' r'}(x)|}{N_{-i b r}(x)+1}
\frac{1 + \frac{n t}{\lambda} |T_{b' r'}(x) \setminus T_{b r}(x)|}
{\left(N_{-i b' r' \cap b r}(x)
+ n |T_{b' r'}(x) \setminus T_{b r}(x)|+1\right)^2}
\right]
+
e^{-t/C} \\
&\quad\lesssim
\E \left[
\frac{|T_{b r}(x) \cap T_{b' r'}(x)|}
{n |T_{b r}(x)|+1}
\frac{1 + \frac{n t}{\lambda} |T_{b' r'}(x) \setminus T_{b r}(x)|}
{(n |T_{b' r'}(x)| + 1)^2}
\right]
+ e^{-t/C} \\
&\quad\lesssim
\E \left[
\frac{1}{n}
\frac{1}
{(n |T_{b' r'}(x)| + 1)^2}
+ \frac{1}{n}
\frac{t / \lambda}
{n |T_{b' r'}(x)| + 1}
\right]
+ e^{-t/C} \\
&\quad\lesssim
\frac{\lambda^{2d} \log n}{n^3}
+ \frac{\log n}{n \lambda}
\frac{\lambda^d}{n}
\lesssim
\frac{\lambda^d}{n^2}
\left(
\frac{\lambda^{d} \log n}{n}
+ \frac{\log n}{\lambda}
\right).
\end{align*}
%
Now apply the same argument to the other
term in the expectation, to see that
%
\begin{align*}
&\left|
\E \left[
\frac{1}
{N_{-i b r \cap b' r'}(x)+N_{-i b r \setminus b' r'}(x)+1}
\Bigm| \bT, N_{-i b r \cap b' r'}(x), N_{-i b' r' \setminus b r}(x)
\right]
\right. \\
&\left.
\qquad-
\frac{1}
{N_{-i b r \cap b' r'}(x) + n f(x) |T_{b r}(x) \setminus T_{b' r'}(x)|+1}
\right| \\
&\quad\lesssim
\frac{1 + \frac{n t}{\lambda} |T_{b r}(x) \setminus T_{b' r'}(x)|}
{\left(N_{-i b r \cap b' r'}(x)
+ n |T_{b r}(x) \setminus T_{b' r'}(x)|+1\right)^2}.
\end{align*}
%
with probability at least $1 - e^{-t/C}$,
and so likewise again with $t = 3 C \log n$,
%
\begin{align*}
&\frac{n^2}{\lambda^d}
\left|
\E \left[
\frac{|T_{b r}(x) \cap T_{b' r'}(x)|}{N_{-i b r}(x)+1}
\frac{1}
{N_{-i b' r' \cap b r}(x)+n f(x) |T_{b' r'}(x) \setminus T_{b r}(x)|+1}
\right]
\right.
\\
&\left.
\quad-
\E \left[
\frac{|T_{b r}(x) \cap T_{b' r'}(x)|}
{N_{-i b r \cap b' r'}(x) + n f(x) |T_{b r}(x) \setminus T_{b' r'}(x)|+1}
\right.\right. \\
&\qquad\qquad\left.\left.
\times
\frac{1}
{N_{-i b' r' \cap b r}(x)+n f(x) |T_{b' r'}(x) \setminus T_{b r}(x)|+1}
\right]
\right| \\
&\lesssim
\frac{n^2}{\lambda^d} \,
\E \left[
\frac{1 + \frac{n t}{\lambda} |T_{b r}(x) \setminus T_{b' r'}(x)|}
{\left(N_{-i b r \cap b' r'}(x)
+ n |T_{b r}(x) \setminus T_{b' r'}(x)|+1\right)^2}
\right. \\
&\qquad\qquad\left.
\times
\frac{|T_{b r}(x) \cap T_{b' r'}(x)|}
{N_{-i b' r' \cap b r}(x)+n f(x) |T_{b' r'}(x) \setminus T_{b r}(x)|+1}
\right]
+ \frac{n^2}{\lambda^d}
e^{-t/C} \\
&\lesssim
\frac{\lambda^d \log n}{n}
+ \frac{\log n}{\lambda}.
\end{align*}
%
Thus far we have proven that
%
\begin{align*}
&\frac{n^2}{\lambda^d}
\E \left[
\frac{\I_{i b r}(x) \I_{i b' r'}(x) \varepsilon_i^2}
{N_{b r}(x) N_{b' r'}(x)}
\right]
= \sigma^2(x)
f(x)
\frac{n^2}{\lambda^d} \\
&\quad\times
\E \left[
\frac{|T_{b r}(x) \cap T_{b' r'}(x)|}
{N_{-i b r \cap b' r'}(x) + n f(x) |T_{b r}(x) \setminus T_{b' r'}(x)|+1}
\right. \\
&\left.
\qquad\qquad
\times
\frac{1}
{N_{-i b' r' \cap b r}(x)+n f(x) |T_{b' r'}(x) \setminus T_{b r}(x)|+1}
\right] \\
&\quad+
O \left(
\frac{(\log n)^{d+1}}{\lambda}
+ \frac{\lambda^d \log n}{n}
\right).
\end{align*}
%
We remove the $N_{-i b r \cap b' r'}(x)$ terms.
With probability at least $1 - e^{-t/C}$, conditional on $\bT$,
%
\begin{align*}
N_{-i b r \cap b' r'}(x)
&\leq \Bin\left(
n, \,
|T_{b r}(x) \cap T_{b' r'}(x)|
\left( f(x) + \frac{t}{\lambda} \right)
\right), \\
N_{-i b r \cap b' r'}(x)
&\geq
\Bin\left(
n
\left( 1 - \frac{t^{d+1}}{\lambda^d}
- \frac{1}{n} \right), \,
|T_{b r}(x) \cap T_{b' r'}(x)|
\left( f(x) - \frac{t}{\lambda} \right)
\right).
\end{align*}
%
Therefore, by Lemma~\ref{lem:mondrian_app_binomial_expectation}
applied conditionally on $\bT$,
with probability at least $1 - e^{-t/C}$,
%
\begin{align*}
&
\left|
\E \! \left[
\frac{1}
{N_{-i b r \cap b' r'}(x)
+ n f(x) |T_{b r}(x) \!\setminus\! T_{b' r'}(x)|+1}
\frac{1}
{N_{-i b' r' \cap b r}(x)
+ n f(x) |T_{b' r'}(x) \!\setminus\! T_{b r}(x)|+1}
\! \Bigm| \! \bT
\right]
\right.
\\
&\left.
\qquad-
\frac{1}
{n f(x) |T_{b r}(x)|+1}
\frac{1}
{n f(x) |T_{b' r'}(x)|+1}
\right| \\
&\quad\lesssim
\frac{1 + \frac{n t}{\lambda} |T_{b r}(x) \cap T_{b' r'}(x)|}
{(n |T_{b r}(x)| + 1)(n |T_{b' r'}(x)| + 1)}
\left(
\frac{1}{n |T_{b r}(x)| + 1}
+ \frac{1}{n |T_{b' r'}(x)| + 1}
\right).
\end{align*}
%
Now by Lemma~\ref{lem:mondrian_app_moment_cell},
with $t = 3 C \log n$,
%
\begin{align*}
&\frac{n^2}{\lambda^d}
\left|
\E \! \left[
\frac{|T_{b r}(x) \cap T_{b' r'}(x)|}
{N_{-i b r \cap b' r'}(x)
+ n f(x) |T_{b r}(x) \!\setminus\! T_{b' r'}(x)|+1}
\frac{1}
{N_{-i b' r' \cap b r}(x)
+ n f(x) |T_{b' r'}(x) \!\setminus\! T_{b r}(x)|+1}
\right]
\right. \\
&\left.
\qquad-
\E \left[
\frac{|T_{b r}(x) \cap T_{b' r'}(x)|}
{n f(x) |T_{b r}(x)|+1}
\frac{1}
{n f(x) |T_{b' r'}(x)|+1}
\right]
\right| \\
&\quad\lesssim
\frac{n^2}{\lambda^d}
\E \left[
|T_{b r}(x) \cap T_{b' r'}(x)|
\frac{1 + \frac{n t}{\lambda} |T_{b r}(x) \cap T_{b' r'}(x)|}
{(n |T_{b r}(x)| + 1)(n |T_{b' r'}(x)| + 1)}
\frac{1}{n |T_{b r}(x)| + 1}
+ \frac{1}{n |T_{b' r'}(x)| + 1}
\right] \\
&\qquad+
\frac{n^2}{\lambda^d}
e^{-t/C} \\
&\quad\lesssim
\frac{n^2}{\lambda^d}
\frac{1}{n^3}
\E \left[
\frac{1 + \frac{n t}{\lambda} |T_{b r}(x) \cap T_{b' r'}(x)|}
{|T_{b r}(x)| |T_{b' r'}(x)|}
\right]
+ \frac{n^2}{\lambda^d}
e^{-t/C} \\
&\quad\lesssim
\frac{1}{n \lambda^d}
\E \left[
\frac{1}{|T_{b r}(x)| |T_{b' r'}(x)|}
\right]
+ \frac{t}{\lambda^{d+1}}
\E \left[
\frac{1}{|T_{b r}(x)|}
\right]
+ \frac{n^2}{\lambda^d}
e^{-t/C} \\
&\quad\lesssim
\frac{\lambda^d}{n}
+ \frac{\log n}{\lambda}.
\end{align*}
%
This allows us to deduce that
%
\begin{align*}
\frac{n^2}{\lambda^d}
\E \left[
\frac{\I_{i b r}(x) \I_{i b' r'}(x) \varepsilon_i^2}
{N_{b r}(x) N_{b' r'}(x)}
\right]
&=
\sigma^2(x)
f(x)
\frac{n^2}{\lambda^d}
\E \left[
\frac{|T_{b r}(x) \cap T_{b' r'}(x)|}
{(n f(x) |T_{b r}(x)|+1)(n f(x) |T_{b' r'}(x)|+1)}
\right] \\
&\quad+
O \left(
\frac{(\log n)^{d+1}}{\lambda}
+ \frac{\lambda^d \log n}{n}
\right).
\end{align*}
%
Now that we have reduced the limiting variance to an expression
only involving the sizes of Mondrian cells,
we can exploit their exact distribution to compute this expectation.
Recall from \citet[Proposition~1]{mourtada2020minimax}
that we can write
%
\begin{align*}
|T_{b r}(x)|
&= \prod_{j=1}^{d}
\left(
\frac{E_{1j}}{a_r \lambda} \wedge x_j
+ \frac{E_{2j}}{a_r \lambda} \wedge (1 - x_j)
\right), \\
|T_{b' r'}(x)|
&=
\prod_{j=1}^{d}
\left(
\frac{E_{3j}}{a_{r'} \lambda} \wedge x_j
+ \frac{E_{4j}}{a_{r'} \lambda} \wedge (1 - x_j)
\right), \\
|T_{b r }(x)\cap T_{b' r'}(x)|
&= \prod_{j=1}^{d}
\left(
\frac{E_{1j}}{a_r \lambda} \wedge
\frac{E_{3j}}{a_{r'} \lambda}
\wedge x_j
+ \frac{E_{2j}}{a_r \lambda} \wedge
\frac{E_{4j}}{a_{r'} \lambda}
\wedge (1 - x_j)
\right)
\end{align*}
%
where $E_{1j}$, $E_{2j}$, $E_{3j}$, and $E_{4j}$
are independent and $\Exp(1)$.
Define their non-truncated versions
%
\begin{align*}
|\tilde T_{b r}(x)|
&=
a_r^{-d}
\lambda^{-d}
\prod_{j=1}^{d}
\left( E_{1j} + E_{2j} \right), \\
|\tilde T_{b' r'}(x)|
&=
a_{r'}^{-d}
\lambda^{-d}
\prod_{j=1}^{d}
\left( E_{3j} + E_{4j} \right), \\
|\tilde T_{b r}(x) \cap \tilde T_{b' r'}(x)|
&=
\lambda^{-d}
\prod_{j=1}^{d}
\left(
\frac{E_{1j}}{a_r}
\wedge
\frac{E_{3j}}{a_{r'}}
+ \frac{E_{2j}}{a_r}
\wedge
\frac{E_{4j}}{a_{r'}}
\right),
\end{align*}
%
and note that
%
\begin{align*}
&\P \left(
\big( \tilde T_{b r}(x), \tilde T_{b' r'}(x),
\tilde T_{b r}(x) \cap T_{b' r'}(x) \big)
\neq
\big( T_{b r}(x), T_{b' r'}(x), T_{b r}(x) \cap T_{b' r'}(x) \big)
\right) \\
&\,\leq
\sum_{j=1}^{d}
\big(
\P(E_{1j} \geq a_r \lambda x_j)
+ \P(E_{3j} \geq a_{r'} \lambda x_j)
+ \P(E_{2j} \geq a_r \lambda (1 - x_j))
+ \P(E_{4j} \geq a_{r'} \lambda (1 - x_j))
\big) \\
&\,\leq e^{-C \lambda}
\end{align*}
%
for some $C > 0$ and sufficiently large $\lambda$.
So by Cauchy--Schwarz and Lemma~\ref{lem:mondrian_app_moment_cell},
%
\begin{align*}
&
\frac{n^2}{\lambda^d}
\left|
\E \left[
\frac{|T_{b r}(x) \cap T_{b' r'}(x)|}
{n f(x) |T_{b r}(x)|+1}
\frac{1}
{n f(x) |T_{b' r'}(x)|+1}
\right]
- \E \left[
\frac{|\tilde T_{b r}(x) \cap T_{b' r'}(x)|}
{n f(x) |\tilde T_{b r}(x)|+1}
\frac{1}
{n f(x) |\tilde T_{b' r'}(x)|+1}
\right]
\right| \\
&\quad\lesssim
\frac{n^2}{\lambda^d}
e^{-C \lambda}
\lesssim
e^{-C \lambda / 2}
\end{align*}
%
as $\log \lambda \gtrsim \log n$.
Therefore
%
\begin{align*}
\frac{n^2}{\lambda^d}
\E \left[
\frac{\I_{i b r}(x) \I_{i b' r'}(x) \varepsilon_i^2}
{N_{b r}(x) N_{b' r'}(x)}
\right]
&=
\sigma^2(x)
f(x)
\frac{n^2}{\lambda^d}
\E \left[
\frac{|\tilde T_{b r}(x) \cap \tilde T_{b' r'}(x)|}
{(n f(x) |\tilde T_{b r}(x)|+1)(n f(x) |\tilde T_{b' r'}(x)|+1)}
\right] \\
&\quad+
O \left(
\frac{(\log n)^{d+1}}{\lambda}
+ \frac{\lambda^d \log n}{n}
\right).
\end{align*}
%
We remove the superfluous units in the denominators.
Firstly, by independence of the trees,
%
\begin{align*}
& \frac{n^2}{\lambda^d}
\left|
\E \left[
\frac{|\tilde T_{b r}(x) \cap \tilde T_{b' r'}(x)|}
{(n f(x) |\tilde T_{b r}(x)|+1)(n f(x) |\tilde T_{b' r'}(x)|+1)}
\right]
- \E \left[
\frac{|\tilde T_{b r}(x) \cap \tilde T_{b' r'}(x)|}
{(n f(x) |\tilde T_{b r}(x)|+1)(n f(x) |\tilde T_{b' r'}(x)|)}
\right]
\right| \\
&\quad\lesssim
\frac{n^2}{\lambda^d}
\E \left[
\frac{|\tilde T_{b r}(x) \cap \tilde T_{b' r'}(x)|}
{n |\tilde T_{b r}(x)|}
\frac{1}
{n^2 |\tilde T_{b' r'}(x)|^2}
\right]
\lesssim
\frac{1}{n \lambda^d}
\E \left[
\frac{1}{|T_{b r}(x)|}
\right]
\E \left[
\frac{1}{|T_{b' r'}(x)|}
\right]
\lesssim
\frac{\lambda^d}{n}.
\end{align*}
%
Secondly, we have in exactly the same manner that
%
\begin{align*}
\frac{n^2}{\lambda^d}
\left|
\E \left[
\frac{|\tilde T_{b r}(x) \cap T_{b' r'}(x)|}
{(n f(x) |\tilde T_{b r}(x)|+1)(n f(x) |\tilde T_{b' r'}(x)|)}
\right]
- \E \left[
\frac{|\tilde T_{b r}(x) \cap T_{b' r'}(x)|}
{n^2 f(x)^2 |\tilde T_{b r}(x)| |\tilde T_{b' r'}(x)|}
\right]
\right|
&\lesssim
\frac{\lambda^d}{n}.
\end{align*}
%
Therefore
%
\begin{align*}
\frac{n^2}{\lambda^d}
\E \left[
\frac{\I_{i b r}(x) \I_{i b' r'}(x) \varepsilon_i^2}
{N_{b r}(x) N_{b' r'}(x)}
\right]
&=
\frac{\sigma^2(x)}{f(x)}
\frac{1}{\lambda^d}
\E \left[
\frac{|\tilde T_{b r}(x) \cap \tilde T_{b' r'}(x)|}
{|\tilde T_{b r}(x)| |\tilde T_{b' r'}(x)|}
\right]
+ O \left(
\frac{(\log n)^{d+1}}{\lambda}
+ \frac{\lambda^d \log n}{n}
\right).
\end{align*}
%
It remains to compute this integral.
By independence over $1 \leq j \leq d$,
%
\begin{align*}
&\E \left[
\frac{|\tilde T_{b r}(x) \cap \tilde T_{b' r'}(x)|}
{|\tilde T_{b r}(x)| |\tilde T_{b' r'}(x)|}
\right] \\
&\quad=
a_r^d a_{r'}^d \lambda^d
\prod_{j=1}^d
\E \left[
\frac{ (E_{1j} / a_r) \wedge (E_{3j} / a_{r'})
+ (E_{2j} a_r) \wedge (E_{4j} / a_{r'}) }
{ \left( E_{1j} + E_{2j} \right) \left( E_{3j} + E_{4j} \right)}
\right] \\
&\quad=
2^d a_r^d a_{r'}^d \lambda^d
\prod_{j=1}^d
\E \left[
\frac{ (E_{1j} / a_r) \wedge (E_{3j} / a_{r'})}
{ \left( E_{1j} + E_{2j} \right) \left( E_{3j} + E_{4j} \right) }
\right] \\
&\quad=
2^d a_r^d a_{r'}^d \lambda^d
\prod_{j=1}^d
\int_{0}^{\infty}
\int_{0}^{\infty}
\int_{0}^{\infty}
\int_{0}^{\infty}
\frac{ (t_1 / a_r) \wedge (t_3 / a_{r'}) }
{ \left( t_1 + t_2 \right) \left( t_3 + t_4 \right) }
e^{-t_1 - t_2 - t_3 - t_4}
\diff t_1
\diff t_2
\diff t_3
\diff t_4 \\
&\quad=
2^d a_r^d a_{r'}^d \lambda^d
\prod_{j=1}^d
\int_{0}^{\infty}
\int_{0}^{\infty}
((t_1 / a_r) \wedge (t_3 / a_{r'}))
e^{-t_1 - t_3} \\
&\qquad\times
\left(
\int_{0}^{\infty}
\frac{e^{-t_2}}{t_1 + t_2}
\diff t_2
\right)
\left(
\int_{0}^{\infty}
\frac{e^{-t_4}}{t_3 + t_4}
\diff t_4
\right)
\diff t_1
\diff t_3 \\
&\quad=
2^d a_r^d a_{r'}^d \lambda^d
\prod_{j=1}^d
\int_{0}^{\infty}
\int_{0}^{\infty}
((t / a_r) \wedge (s / a_{r'}))
\Gamma(0, t)
\Gamma(0, s)
\diff t
\diff s,
\end{align*}
%
as $\int_0^\infty \frac{e^{-t}}{a + t} \diff t = e^a \Gamma(0, a)$
with $\Gamma(0, a) = \int_a^\infty \frac{e^{-t}}{t} \diff t$. Now
%
\begin{align*}
&2
\int_{0}^{\infty}
\int_{0}^{\infty}
((t / a_r) \wedge (s / a_{r'}))
\Gamma(0, t)
\Gamma(0, s)
\diff t
\diff s \\
&\quad=
\int_0^\infty
\Gamma(0, t)
\left(
\frac{1}{a_{r'}}
\int_0^{a_{r'} t / a_r}
2 s \Gamma(0, s)
\diff{s}
+
\frac{t}{a_r}
\int_{a_{r'} t / a_r}^\infty
2 \Gamma(0, s)
\diff{s}
\right)
\diff{t} \\
&\quad=
\int_0^\infty
\Gamma(0, t)
\left(
\frac{t}{a_r}
e^{- \frac{a_{r'}}{a_r}t}
- \frac{1}{a_{r'}} e^{- \frac{a_{r'}}{a_r}t}
+ \frac{1}{a_{r'}}
- \frac{a_{r'}}{a_r^2} t^2
\Gamma\left(0, \frac{a_{r'}}{a_r} t\right)
\right)
\diff{t} \\
&\quad=
\frac{1}{a_r}
\int_0^\infty
t e^{- \frac{a_{r'}}{a_r} t}
\Gamma(0, t)
\diff{t}
- \frac{1}{a_{r'}}
\int_0^\infty
e^{- \frac{a_{r'}}{a_r} t}
\Gamma(0, t)
\diff{t} \\
&\qquad+
\frac{1}{a_{r'}}
\int_0^\infty
\Gamma(0, t)
\diff{t}
-
\frac{a_{r'}}{a_r^2}
\int_0^\infty
t^2 \Gamma\left(0, \frac{a_{r'}}{a_r} t\right)
\Gamma(0, t)
\diff{t},
\end{align*}
%
since
$\int_0^a 2 t \Gamma(0, t) \diff t = a^2 \Gamma(0, a) - a e^{-a} -e^{-a} + 1$
and
$\int_a^\infty \Gamma(0, t) \diff t = e^{-a} - a \Gamma(0, a)$.
Next, we use
%
$ \int_{0}^{\infty} \Gamma(0, t) \diff t = 1$,
$\int_{0}^{\infty} e^{-at} \Gamma(0, t) \diff t
= \frac{\log(1+a)}{a}$,
$\int_{0}^{\infty} t e^{-at} \Gamma(0, t) \diff t
= \frac{\log(1+a)}{a^2} - \frac{1}{a(a+1)}$,
and
$\int_{0}^{\infty} t^2 \Gamma(0, t) \Gamma(0, at) \diff t
= - \frac{2a^2 + a + 2}{3a^2 (a+1)} + \frac{2(a^3 + 1) \log(a+1)}{3a^3}
- \frac{2 \log a}{3}$
to see
%
\begin{align*}
&2
\int_{0}^{\infty}
\int_{0}^{\infty}
((t / a_r) \wedge (s / a_{r'}))
\Gamma(0, t)
\Gamma(0, s)
\diff t
\diff s \\
&\quad=
\frac{a_r \log(1+a_{r'} / a_r)}{a_{r'}^2}
- \frac{a_r / a_{r'}}{a_r + a_{r'}}
- \frac{a_r \log(1 + a_{r'} / a_r)}{a_{r'}^2}
+ \frac{1}{a_{r'}} \\
&\qquad+
\frac{2 a_{r'}^2 + a_r a_{r'} + 2 a_r^2}
{3 a_r a_{r'} (a_r + a_{r'})}
- \frac{2(a_{r'}^3 + a_r^3) \log(a_{r'} / a_r+1)}{3 a_r^2 a_{r'}^2}
+ \frac{2 a_{r'} \log (a_{r'} / a_r)}{3 a_r^2} \\
&\quad=
\frac{2}{3 a_r} + \frac{2}{3 a_{r'}}
- \frac{2(a_r^3 + a_{r'}^3 ) \log(a_{r'} / a_{r}+1)}
{3 a_r^2 a_{r'}^2}
+ \frac{2 a_{r'} \log (a_{r'} / a_{r})}{3 a_r^2} \\
&\quad=
\frac{2}{3 a_r}
+ \frac{2}{3 a_{r'}}
- \frac{2 a_{r'} \log(a_{r} / a_{r'} + 1)}{3 a_r^2}
- \frac{2 a_r \log(a_{r'} / a_{r} + 1)}{3 a_{r'}^2} \\
&\quad=
\frac{2}{3 a_r}
\left(
1 - \frac{a_{r'}}{a_r}
\log\left(\frac{a_{r}}{a_{r'}} + 1\right)
\right)
+ \frac{2}{3 a_{r'}}
\left(
1 - \frac{a_r }{a_{r'}}
\log\left(\frac{a_{r'}}{a_{r}} + 1\right)
\right).
\end{align*}
%
Finally, we conclude by giving the limiting variance.
%
\begin{align*}
&\sum_{r=0}^{J}
\sum_{r'=0}^{J}
\omega_r
\omega_{r'}
\frac{n^2}{\lambda^d}
\E \left[
\frac{\I_{i b r}(x) \I_{i b' r'}(x) \varepsilon_i^2}
{N_{b r}(x) N_{b' r'}(x)}
\right] \\
&\quad=
\frac{\sigma^2(x)}{f(x)}
\sum_{r=0}^{J}
\sum_{r'=0}^{J}
\omega_r
\omega_{r'}
\left(
\frac{2 a_{r'}}{3}
\left(
1 - \frac{a_{r'}}{a_r}
\log\left(\frac{a_r}{a_{r'}} + 1\right)
\right)
+ \frac{2 a_r}{3}
\left(
1 - \frac{a_r}{a_{r'}}
\log\left(\frac{a_{r'}}{a_r} + 1\right)
\right)
\right)^d \\
&\qquad+
O \left(
\frac{(\log n)^{d+1}}{\lambda}
+ \frac{\lambda^d \log n}{n}
\right).
\end{align*}
%
So the limit exists, and
with $\ell_{r r'} = \frac{2 a_r}{3} \left( 1 - \frac{a_{r}}{a_{r'}}
\log\left(\frac{a_{r'}}{a_{r}} + 1\right) \right)$,
the limiting variance is
%
\begin{align*}
\Sigma_\rd(x)
&=
\frac{\sigma^2(x)}{f(x)}
\sum_{r=0}^{J} \sum_{r'=0}^{J} \omega_r \omega_{r'}
\left( \ell_{r r'} + \ell_{r' r} \right)^d.
\end{align*}
%
\end{proof}

The new bias characterization with debiasing is an algebraic
consequence of the original bias characterization and the construction
of the debiased Mondrian random forest estimator.

\begin{proof}[Theorem~\ref{thm:mondrian_bias_debiased}]

By the definition of the debiased estimator and
Theorem~\ref{thm:mondrian_bias}, since $J$ and $a_r$ are fixed,
%
\begin{align*}
\E \big[ \hat \mu_\rd(x) \mid \bX, \bT \big]
&=
\sum_{l=0}^J
\omega_l
\E \big[
\hat \mu_l(x)
\Bigm| \bX, \bT
\big] \\
&=
\sum_{l=0}^J
\omega_l
\left(
\mu(x)
+ \sum_{r=1}^{\lfloor \flbeta / 2 \rfloor}
\frac{B_r(x)}{a_l^{2r} \lambda^{2r}}
\right)
+ O_\P \left(
\frac{1}{\lambda^\beta}
+ \frac{1}{\lambda \sqrt B}
+ \frac{\log n}{\lambda} \sqrt{\frac{\lambda^d}{n}}
\right).
\end{align*}
%
It remains to evaluate the first term.
Recalling that $A_{r s} = a_{r-1}^{2 - 2s}$
and $A \omega = e_0$, we have
%
\begin{align*}
&\sum_{l=0}^J
\omega_l
\left(
\mu(x)
+ \sum_{r=1}^{\lfloor \flbeta / 2 \rfloor}
\frac{B_r(x)}{a_l^{2r} \lambda^{2r}}
\right) \\
&\quad=
\mu(x)
\sum_{l=0}^J
\omega_l
+
\sum_{r=1}^{\lfloor \flbeta / 2 \rfloor}
\frac{B_r(x)}{\lambda^{2r}}
\sum_{l=0}^J
\frac{\omega_l}{a_l^{2r}} \\
&\quad=
\mu(x)
(A \omega)_1
+ \sum_{r=1}^{\lfloor \flbeta / 2 \rfloor \wedge J}
\frac{B_r(x)}{\lambda^{2r}}
(A \omega)_{r+1}
+ \sum_{r = (\lfloor \flbeta / 2 \rfloor \wedge J) + 1}
^{\lfloor \flbeta / 2 \rfloor}
\frac{B_r(x)}{\lambda^{2r}}
\sum_{l=0}^J
\frac{\omega_l}{a_l^{2r}} \\
&\quad=
\mu(x)
+ \I\{\lfloor \flbeta / 2 \rfloor \geq J + 1\}
\frac{B_{J+1}(x)}{\lambda^{2J + 2}}
\sum_{l=0}^J
\frac{\omega_l}{a_l^{2J + 2}}
+ O \left( \frac{1}{\lambda^{2J + 4}} \right) \\
&\quad=
\mu(x)
+ \I\{2J + 2 < \beta\}
\frac{\bar\omega B_{J+1}(x)}{\lambda^{2J + 2}}
+ O \left( \frac{1}{\lambda^{2J + 4}} \right).
\end{align*}
%
\end{proof}

\begin{proof}[Theorem~\ref{thm:mondrian_variance_estimation_debiased}]

\proofparagraph{consistency of $\hat\sigma^2(x)$}

Recall that
%
\begin{align}
\label{eq:mondrian_app_sigma2_hat_proof}
\hat\sigma^2(x)
&=
\frac{1}{B}
\sum_{b=1}^{B}
\frac{\sum_{i=1}^n Y_i^2 \, \I\{X_i \in T_b(x)\}}
{\sum_{i=1}^n \I\{X_i \in T_b(x)\}}
- \hat \mu(x)^2.
\end{align}
%
The first term in \eqref{eq:mondrian_app_sigma2_hat_proof}
is simply a Mondrian forest estimator of
$\E[Y_i^2 \mid X_i = x] = \sigma^2(x) + \mu(x)^2$,
which is bounded and Lipschitz,
where $\E[Y_i^4 \mid X_i]$ is bounded almost surely.
So its conditional bias is controlled
by Theorem~\ref{thm:mondrian_bias} and is at most
$O_\P \left( \frac{1}{\lambda} +
\frac{\log n}{\lambda} \sqrt{\lambda^d / n} \right)$.
Its variance is
at most $\frac{\lambda^d}{n}$ by Theorem~\ref{thm:mondrian_clt_debiased}.
Consistency of the second term in \eqref{eq:mondrian_app_sigma2_hat_proof}
follows directly from Theorems~\ref{thm:mondrian_bias} and
\ref{thm:mondrian_clt_debiased} with the same bias and variance bounds.
Therefore
%
\begin{align*}
\hat\sigma^2(x)
&=
\sigma^2(x)
+ O_\P \left(
\frac{1}{\lambda}
+ \sqrt{\frac{\lambda^d}{n}}
\right).
\end{align*}

\proofparagraph{consistency of the sum}
%
Note that
%
\begin{align*}
&\frac{n}{\lambda^d}
\sum_{i=1}^n
\left(
\sum_{r=0}^J
\omega_r
\frac{1}{B}
\sum_{b=1}^B
\frac{\I\{X_i \in T_{r b}(x)\}}
{\sum_{i=1}^n \I\{X_i \in T_{r b}(x)\}}
\right)^2 \\
&\quad=
\frac{n}{\lambda^d}
\frac{1}{B^2}
\sum_{i=1}^n
\sum_{r=0}^J
\sum_{r'=0}^J
\omega_r
\omega_{r'}
\sum_{b=1}^B
\sum_{b'=1}^B
\frac{\I_{i b r}(x) \I_{i b' r'}(x)}
{N_{b r}(x) N_{b' r'}(x)}.
\end{align*}
%
This is exactly the same as the quantity in
\eqref{eq:mondrian_app_clt_condition_sum}, if we were to take
$\varepsilon_i$ to be $\pm 1$ with equal probability.
Thus we immediately have convergence in probability
by the proof of Theorem~\ref{thm:mondrian_clt_debiased}:
%
\begin{align*}
\frac{n}{\lambda^d}
\sum_{i=1}^n
\left(
\sum_{r=0}^J
\omega_r
\frac{1}{B}
\sum_{b=1}^B
\frac{\I\{X_i \in T_{r b}(x)\}}
{\sum_{i=1}^n \I\{X_i \in T_{r b}(x)\}}
\right)^2
&=
\frac{n^2}{\lambda^d}
\sum_{r=0}^J
\sum_{r'=0}^J
\omega_r
\omega_{r'}
\E \left[
\frac{\I_{i b r}(x) \I_{i b' r'}(x)}
{N_{b r}(x) N_{b' r'}(x)}
\right] \\
&\quad+
O_\P \left(
\frac{1}{\sqrt B}
+ \sqrt{\frac{\lambda^d \log n}{n}}
\right).
\end{align*}

\proofparagraph{conclusion}

By the proof of Theorem~\ref{thm:mondrian_clt_debiased}
with $\varepsilon_i$ being $\pm 1$ with equal probability,
and by previous parts,
%
\begin{align*}
\hat\Sigma_\rd(x)
= \Sigma_\rd(x)
+ O_\P \left(
\frac{(\log n)^{d+1}}{\lambda}
+ \frac{1}{\sqrt B}
+ \sqrt{\frac{\lambda^d \log n}{n}}
\right).
\end{align*}

\end{proof}

\begin{proof}[Theorem~\ref{thm:mondrian_confidence_debiased}]
%
By Theorem~\ref{thm:mondrian_bias_debiased}
and Theorem~\ref{thm:mondrian_variance_estimation_debiased},
%
\begin{align*}
\sqrt{\frac{n}{\lambda^d}}
\frac{\hat \mu_\rd(x) - \mu(x)}{\hat \Sigma_\rd(x)^{1/2}}
&=
\sqrt{\frac{n}{\lambda^d}}
\frac{\hat \mu_\rd(x) - \E \left[ \hat \mu_\rd(x) \mid \bX, \bT \right]}
{\hat \Sigma_\rd(x)^{1/2}}
+ \sqrt{\frac{n}{\lambda^d}}
\frac{\E \left[ \hat \mu_\rd(x) \mid \bX, \bT \right] - \mu(x)}
{\hat \Sigma_\rd(x)^{1/2}} \\
&=
\sqrt{\frac{n}{\lambda^d}}
\frac{\hat \mu_\rd(x) - \E \left[ \hat \mu_\rd(x) \mid \bX, \bT \right]}
{\hat \Sigma_\rd(x)^{1/2}}
+ \sqrt{\frac{n}{\lambda^d}} \,
O_\P \left(
\frac{1}{\lambda^\beta}
+ \frac{1}{\lambda \sqrt B}
+ \frac{\log n}{\lambda} \sqrt{\frac{\lambda^d}{n}}
\right).
\end{align*}
%
The first term converges weakly to $\cN(0,1)$ by
Slutsky's theorem and Theorems~\ref{thm:mondrian_clt_debiased}
and \ref{thm:mondrian_variance_estimation_debiased},
while the second is $o_\P(1)$ by assumption.
Validity of the confidence interval follows.
%
\end{proof}

\begin{proof}[Theorem~\ref{thm:mondrian_minimax}]

Theorem~\ref{thm:mondrian_bias_debiased}
and the proof of Theorem~\ref{thm:mondrian_clt_debiased}
with $J = \lfloor \flbeta / 2 \rfloor$ gives
%
\begin{align*}
\E \left[
\big(
\hat \mu_\rd(x)
- \mu(x)
\big)^2
\right]
&=
\E \left[
\big(
\hat \mu_\rd(x)
- \E \left[ \hat \mu_\rd(x) \mid \bX, \bT \right]
\big)^2
\right]
+ \E \left[
\big(
\E \left[ \hat \mu_\rd(x) \mid \bX, \bT \right]
- \mu(x)
\big)^2
\right] \\
&\lesssim
\frac{\lambda^d}{n}
+ \frac{1}{\lambda^{2\beta}}
+ \frac{1}{\lambda^2 B}.
\end{align*}
%
We use here an $L^2$ version of Theorem~\ref{thm:mondrian_bias_debiased}
which is immediate from the proof of Theorem~\ref{thm:mondrian_bias},
since we leveraged Chebyshev's inequality. Now since
$\lambda \asymp n^{\frac{1}{d + 2 \beta}}$ and
$B \gtrsim n^{\frac{2 \beta - 2}{d + 2 \beta}}$,
%
\begin{align*}
\E \left[
\big(
\hat \mu_\rd(x)
- \mu(x)
\big)^2
\right]
&\lesssim
n^{-\frac{2\beta}{d + 2 \beta}}.
\end{align*}
\end{proof}

\section{Further properties of the Mondrian process}

In section, we state and prove a collection of lemmas concerning
various properties of the Mondrian process. While they are not used directly
in our analysis of Mondrian random forest estimators, we believe that
these results, along with the techniques displayed during their proofs,
may be of potential independent interest.

Our analysis of Mondrian random forest estimators in the main text
is for the most part
conducted pointwise, in the sense that we first fix $x \in [0,1]^d$
and then analyze $\hat\mu(x)$. This means that we interact with the Mondrian
process
only through $T(x)$; that is, the cell in $T$ which contains the point $x$.
As such, we rely only on local properties of $T$, and may consider just a
single Mondrian cell. The lemmas in this section take a more global approach
to analyzing the Mondrian process, and we make statements about the
entire process $T$, rather than individual cells $T(x)$.
Such results may be useful for a future investigation of the uniform
properties of Mondrian forest estimators, as well as
being interesting in their own right.

We begin with a tail bound for the number of cells appearing
in a Mondrian tree, offering a multiplicative
exponential inequality which
complements the exact expectation result given in
\citet[Proposition~2]{mourtada2020minimax}.
The resulting bound in probability is the same up to
logarithmic terms, and the sharp tail decay is useful
in combination with union bounds in our upcoming results.

\begin{lemma}[Tail bound for the number of cells in a Mondrian tree]
\label{lem:mondrian_app_cells_tail}

Let $D \subseteq \R^d$ be a rectangle and
$T \sim \cM(D, \lambda)$. Writing
$\# T$ for the number of cells in $T$,
%
\begin{align*}
\P\left(
\# T > 3 (1 + \lambda |D|_1)^d
(t + 1 + d \log(1 + \lambda |D|_1))
\right)
&\leq
e^{-t}.
\end{align*}

\end{lemma}

\begin{proof}[Lemma~\ref{lem:mondrian_app_cells_tail}]

We refer to this method as the ``subcell trick''
and attribute it to \citet{mourtada2017universal}.
For $\varepsilon > 0$, partition $D$ into
at most $(1 + 1/\varepsilon)^d$ cells $D' \in \cD_\varepsilon$
with side lengths at most $(|D_1| \varepsilon, \ldots, |D_d| \varepsilon)$.
Denote the restriction of a tree $T$ to a subcell $D'$ by $T \cap D'$.
Since a split in $T$ induces a split in at least one $T \cap D'$,
by a union bound
%
\begin{align*}
\P\left(\# T > t \right)
&\leq
\P\left(\sum_{D' \in \cD_\varepsilon}
\# (T \cap D') > t \right)
\leq
\sum_{D' \in \cD_\varepsilon}
\P\left(
\# (T \cap D') >
\frac{t}{\# \cD_\varepsilon}
\right).
\end{align*}
%
Now $\# (T \cap D')$ is dominated by a Yule process
with parameter $|D'|_1$ stopped at time $\lambda$
\citep[proof of Lemma~2]{mourtada2017universal},
so using that fact that if
$X \sim \Yule(a)$
then $\P(X_t > n) \leq (1-e^{-at})^{n-1}$,
%
\begin{align*}
\P\left(\# T > t \right)
&\leq
\# \cD_\varepsilon \,
(1 - e^{-\lambda |D|_1 \varepsilon})^{t / \# \cD_\varepsilon - 1}
\leq
(1 + 1/\varepsilon)^d
(1 - e^{-\lambda |D|_1 \varepsilon})^{t (1 + 1/\varepsilon)^{-d} - 1}.
\end{align*}
%
Set $\varepsilon = \frac{1}{\lambda |D|_1}$,
note $1-1/e \leq e^{-1/3}$
and replace $t$ by
$3 (1 + \lambda |D|_1)^d
(t + 1 + d \log(1 + \lambda |D|_1))$:
%
\begin{align*}
&\P\left(\# T > t \right)
\leq
(1 + \lambda |D|_1)^d
(1 - 1/e)^{t (1 + \lambda |D|_1)^{-d} - 1}
\leq
2 (1 + \lambda |D|_1)^d
e^{-t (1 + \lambda |D|_1)^{-d} / 3}, \\
&\P\left(\# T >
3
(1 + \lambda |D|_1)^d
(t + 1 + d \log(1 + \lambda |D|_1))
\right)
\leq
e^{-t}.
\end{align*}
%
\end{proof}

Next we provide a rigorous justification to the observation that the cells
in a Mondrian process should have the same shape distribution, though
of course they are not independent. To state and prove this result,
we need a way to identify a particular cell by endowing the
cells in a Mondrian tree with a natural order.

\begin{definition}[Canonical order of cells in a Mondrian tree]
Let $T \sim \cM(D, \lambda)$.
Each cell in a fixed realization of $T$ can be described by
a word from the alphabet $\{l, r\}$,
where $l$ indicates the cell to the left of a split
and $r$ indicates the cell to the right.
For example, if there are no splits we have one cell
described by the empty word.
After one split there are two cells, denoted
$l$ and $r$.
Now suppose that the cell $r$ splits again, giving two splits and three cells,
denoted $l$, $r l$, and $r r$.
Define the canonical ordering of the cells of $T$ by applying
the lexicographic order to their words, with $l < r$.
Note that it does not matter which coordinate each split occurs in:
in two dimensions, $l$ might refer to the ``left'' or ``bottom''
and $r$ to the ``right'' or ``top'' cell.
\end{definition}

\begin{lemma}[Cells in a Mondrian tree have identically distributed shapes]
\label{lem:mondrian_app_cells_identically_distributed}

Let $T \sim \cM(D, \lambda)$
with ordered cells $D'_1, \ldots, D'_{\# T}$.
For $\varepsilon_1, \ldots, \varepsilon_d \geq 0$
and $1 \leq i \leq k$,
%
\begin{align*}
\P\left(
|D'_{i1}| \leq \varepsilon_1,
\ldots, |D'_{id}| \leq \varepsilon_d,
\# T = k
\right)
&=
\P\left(
|D'_{11}| \leq \varepsilon_1,
\ldots, |D'_{1d}| \leq \varepsilon_d,
\# T = k
\right).
\end{align*}
%
Marginalizing over $\# T$
with $E_j$ i.i.d.\ $\Exp(1)$,
\citet[Proposition~1]{mourtada2020minimax} gives
%
\begin{align*}
\P\left(
|D'_{i1}| > \varepsilon_1,
\ldots, |D'_{id}| > \varepsilon_d
\right)
&=
\prod_{j=1}^d
\P\left(
\frac{E_j}{\lambda} \wedge |D_j|
> \varepsilon_j
\right)
= \prod_{j=1}^d
\I\{|D_j| > \varepsilon_j\}
e^{-\lambda \varepsilon_j}.
\end{align*}

\end{lemma}

We observe a version of the famous Poisson process inspection or waiting time
paradox in the sizes of Mondrian cells. The above
Lemma~\ref{lem:mondrian_app_cells_identically_distributed} shows that for a
large enough
lifetime $\lambda$, the volume of any cell $D$ has the same distribution as the
volume of a corner cell, and is asymptotically
$\E[|D|] \asymp \E \left[ \prod_{j=1}^{d} (E_j / \lambda) \right]
= 1/\lambda^d$.
This is consistent with \citet[Proposition~2]{mourtada2020minimax} who give
$\E[\# T] \asymp \lambda^d$.
However, if instead of selecting a cell directly,
we instead select a fixed interior point $x$
and query the cell $T(x)$ which contains it, we find that
$\E[|T(x)|] \asymp \E \left[
\prod_{j=1}^{d} ((E_{1j} + E_{2j}) / \lambda) \right]
= 2^d/\lambda^d$, where $E_{1j}, E_{2j}$ are i.i.d.\ $\Exp(1)$,
by \citet[Proposition~1]{mourtada2020minimax}.
Since $T(x)$ contains $x$ by construction, a size-biasing phenomenon occurs
and we see that $T(x)$ is on average larger than a typical Mondrian cell.

\begin{proof}[Lemma~\ref{lem:mondrian_app_cells_identically_distributed}]

Let $w$ be the word associated with the cell $D_i \in T$.
Note that $i=1$ if and only if $r \notin w$, as then $D_i$ is the left child
of every split.
So suppose $r \in w$.
Let $\tilde w$ be the word obtained by replacing all occurrences
of $r$ in $w$ with an $l$.
Each such replacement corresponds to a split in $T$.
Let $\tilde T$ be the same process as $T$ but with the following
modification: for each split where a replacement was made,
change the uniform random variable $S$
(from the definition of $T$, see Section~\ref{sec:mondrian_process}) to $1-S$.
Since $S$ is independent of everything else in the construction of $T$,
we observe that $\tilde T \sim \cM(D, \lambda)$ also.
Further, there is almost surely exactly one cell in $\tilde T$
which has the same shape as $D$, as the uniform distribution has no atoms.
Denote this cell by $\tilde D$ and note that
the replacements imply that its word in $\tilde T$
is $\tilde w$.
Thus $\tilde D = \tilde D_1$ in $\tilde T$ and so
$(|D_{i1}|, \ldots, |D_{i d}|, \# T)
= (|\tilde D_{11}|, \ldots, |\tilde D_{1d}|, \# \tilde T)$.
Equality of the distributions follows.
\end{proof}

As our next result we provide a tail bound for the size of the largest
Mondrian cell. The cells within a Mondrian tree are of course not independent,
and in fact there should intuitively be some negative correlation between their
sizes, due to the fact that they must all fit within the original cell $D$.

\begin{lemma}[Tail bound on largest Mondrian cell]
\label{lem:mondrian_app_largest_cell_tail}

Let $T \sim \cM(D, \lambda)$.
For any $\varepsilon > 0$,
%
\begin{align*}
\P\left(
\max_{D' \in T}
\max_{1 \leq j \leq d}
|D'_j| > \varepsilon
\right)
&\leq
5d (1 + \lambda |D|_1)^{d+1}
e^{-\lambda \varepsilon}.
\end{align*}
%
\end{lemma}

\begin{proof}[Lemma~\ref{lem:mondrian_app_largest_cell_tail}]

Let $D_i$ be the ordered cells of $T$ and take $k \geq 1$.
By union bounds and
Lemma~\ref{lem:mondrian_app_cells_identically_distributed},
%
\begin{align*}
\P\left(
\max_{D' \in T}
\max_{1 \leq j \leq d}
|D'_j| > \varepsilon
\right)
&\leq
\sum_{l=1}^k
\P\left(
\max_{1 \leq i \leq l}
\max_{1 \leq j \leq d}
|D_{i j}| > \varepsilon,
\# T = l
\right)
+ \P\left( \# T > k \right) \\
&\leq
\sum_{l=1}^k
\sum_{i=1}^l
\sum_{j=1}^d
\P\big(
|D_{i j}| > \varepsilon,
\# T = l
\big)
+ \P\left( \# T > k \right) \\
&\leq
\sum_{l=1}^k
l d \,
\P\big(
|D_{1j}| > \varepsilon,
\# T = l
\big)
+ \P\left( \# T > k \right) \\
&\leq
k d \,
\P\big(|D_{1 j}| > \varepsilon \big)
+ \P\left( \# T > k \right).
\end{align*}
%
For the first term we use the exact distribution of
$D_1$ from Lemma~\ref{lem:mondrian_app_cells_identically_distributed}
and for the second term we apply Lemma~\ref{lem:mondrian_app_cells_tail}.
%
\begin{align*}
\P\left(
\max_{D' \in T}
\max_{1 \leq j \leq d}
|D'_j| > \varepsilon
\right)
&\leq
k d \, \P\big(|D_{1 j}| > \varepsilon \big)
+ \P\left( \# T > k \right) \\
&\leq
k d \, e^{-\lambda \varepsilon}
+ 2 (1 + \lambda |D|_1)^d
e^{-k (1 + \lambda |D|_1)^{-d} / 3}.
\end{align*}
%
Finally, set
$k = \big\lceil 3 \lambda \varepsilon (1 + \lambda |D|_1)^d \big\rceil$
and note the bound is trivial unless $\varepsilon \leq |D|_1$.
%
\begin{align*}
\P\left(
\max_{D' \in T}
\max_{1 \leq j \leq d}
|D'_j| > \varepsilon
\right)
&\leq
\big( 3 \lambda \varepsilon (1 + \lambda |D|_1)^d + 1 \big)
d \, e^{-\lambda \varepsilon}
+ 2 (1 + \lambda |D|_1)^d
e^{-\lambda \varepsilon} \\
&\leq
3d (1 + \lambda |D|_1)^{d+1}
e^{-\lambda \varepsilon}
+ 2 (1 + \lambda |D|_1)^d
e^{-\lambda \varepsilon} \\
&\leq
5d (1 + \lambda |D|_1)^{d+1}
e^{-\lambda \varepsilon}.
\end{align*}
%
\end{proof}

For the remainder of this section, we turn our attention to the partitions
generated by Mondrian random forests. In particular, we study the refinement
generated by overlaying $B$ independent Mondrian processes with possibly
different lifetime parameters, and intersecting their resulting individual
partitions.

\begin{definition}[Partition refinement]%
%
Let $T_1, \ldots, T_B$ be partitions of a set.
Their common refinement is
%
\begin{align*}
\bigwedge_{b=1}^B T_b
= \left\{
\bigcap_{b=1}^B D_b:
D_b \in T_b
\right\}
\bigsetminus
\left\{ \emptyset \right\}.
\end{align*}
%
\end{definition}

We begin our analysis of Mondrian forest refinements with a pair of simple
inequalities for bounding the total number of refined cells
in Lemma~\ref{lem:mondrian_app_refinement_inequalities}. This result does not
depend
on the probabilistic structure of the Mondrian process, and holds for any
rectangular partitions.

\begin{lemma}[Inequalities for refinements of rectangular partitions]
\label{lem:mondrian_app_refinement_inequalities}

Let $T_1, \ldots, T_B$ be rectangular partitions of a $d$-dimensional
rectangle $D$. Then
%
\begin{align}
\label{eq:mondrian_app_refinement_1}
\# \bigwedge_{b=1}^B T_b
&\leq \prod_{b=1}^B \# T_b,
\end{align}
%
and for all $B \leq d$ there exist $T_b$ such that
\eqref{eq:mondrian_app_refinement_1} holds with equality.
If $\# T_{b j}$ denotes the number of splits
made by $T_b$ in dimension $j$, then
%
\begin{align}
\label{eq:mondrian_app_refinement_2}
\# \bigwedge_{b=1}^B T_b
&\leq \prod_{j=1}^d
\left( 1 + \sum_{b=1}^B \# T_{b j} \right),
\end{align}
%
and for all $B \geq d$ there exist $T_b$ such that
\eqref{eq:mondrian_app_refinement_2} holds with equality.

\end{lemma}

\begin{proof}[Lemma~\ref{lem:mondrian_app_refinement_inequalities}]

The first inequality \eqref{eq:mondrian_app_refinement_1}
follows because every cell in
$\bigwedge_b T_b$ is the intersection of cells
$D_b \in T_b$ for $1 \leq b \leq B$, and there at at most
$\prod_{b=1}^{B} \# T_b$ ways to choose these.
This bound is achievable when $B \leq d$ by setting
$T_b$ to be a tree with splits only in dimension $b$,
so that every such intersection of cells
gives a cell in the refinement.

For the second inequality \eqref{eq:mondrian_app_refinement_2},
we construct a new forest of trees.
In particular, for each $1 \leq j \leq d$ define
$A_j$ to be the set of locations in $D_j$ where a tree $T_b$
makes a split in dimension $j$ for some $b$.
Define $T'_j$ to be a tree which has splits
only in dimension $j$ and at the locations prescribed by $A_j$.
Clearly, since every split in $T'_j$
comes from a split in some $T_b$ in dimension $j$,
we have $\# T'_j \leq 1 + \sum_b \# T_{b j}$.
Applying the first inequality to this new forest yields
$\# \bigwedge_j T'_j \leq \prod_j \# T'_j
\leq \prod_j \big( 1 + \sum_b \# T_{b j} \big)$.
Finally, note that $\bigwedge_j T'_j$
is a refinement of $\bigwedge_b T_b$ and the result follows.
This bound is achievable when $B \geq d$ by letting
$T_b$ have splits only in dimension $b$ when $b \leq d$
and to be the trivial partition otherwise.
%
\end{proof}

The inequalities in Lemma~\ref{lem:mondrian_app_refinement_inequalities} provide
rather crude bounds for the number of cells in a Mondrian forest
refinement as they do not take into account the random structure.
Indeed, it should be clear that the ``worst case'' scenarios, involving
trees which contain splits only in a single direction, should be extremely
unlikely under the Mondrian law. In Lemma~\ref{lem:mondrian_app_refinement} we
confirm
this intuition and provide an exact value for the expected number of cells
in a Mondrian refinement by direct calculation. This result strictly generalizes
the single tree version provided as \citet[Proposition~2]{mourtada2020minimax}.

\begin{lemma}[Expected number of cells in a Mondrian forest refinement]
\label{lem:mondrian_app_refinement}

Let $D$ be a $d$-dimensional rectangle
and take $\lambda_b > 0$ for $1 \leq b \leq B$.
Let $T_b \sim \cM(D, \lambda_b)$ be independent.
Then the expected number of cells in their refinement is exactly
%
\begin{align*}
\E\left[\# \bigwedge_{b=1}^B T_b \right]
&= \prod_{j=1}^d \left(
1 + |D_j| \sum_{b=1}^B \lambda_b
\right).
\end{align*}
%
\end{lemma}

\begin{proof}[Lemma~\ref{lem:mondrian_app_refinement}]

By \citet[Proposition~2]{mourtada2020minimax}
we have the result for a single tree:
%
\begin{align}
\label{eq:mondrian_app_single_tree}
\E\left[\# T_b \right]
&= \prod_{j=1}^d \left(
1 + |D_j| \lambda_b
\right).
\end{align}
%
We proceed by induction on $B$.
By the tower law,
%
\begin{align*}
\E\left[\# \bigwedge_{b=1}^B T_b \right]
&=
\E\left[
\sum_{D' \in T_B}
\#
\bigwedge_{b=1}^{B-1} (T_b \cap D')
\right]
= \E\left[
\sum_{D' \in T_B}
\E\left[
\#
\bigwedge_{b=1}^{B-1} (T_b \cap D')
\biggm| T_B
\right]
\right].
\end{align*}
%
Now by the restriction property of Mondrian processes
\citep[Fact~2]{mourtada2020minimax},
observe that $T_b \cap D' \sim \cM(D', \lambda_b)$
conditional on $T_B$.
Then by the induction hypothesis,
%
\begin{align*}
\E\left[
\#
\bigwedge_{b=1}^{B-1} (T_b \cap D')
\biggm| T_B
\right]
&=
\prod_{j=1}^d \left(
1 + |D'_j| \sum_{b=1}^{B-1} \lambda_b
\right)
= \E\big[
\# T_{D'} \mid T_B
\big]
\end{align*}
%
where $T_{D'} \sim \cM\big(D', \sum_{b=1}^{B-1} \lambda_B\big)$
conditional on $T_B$,
by the result for a single tree \eqref{eq:mondrian_app_single_tree}.
The restriction property finally shows that there exist realizations
of $T_{D'}$ which ensure that
$\sum_{D' \in T_B} \# T_{D'}$ is equal in distribution
to $\# T$, where $T \sim \cM(D, \sum_{b=1}^B \lambda_b)$,
so by \eqref{eq:mondrian_app_single_tree},
%
\begin{align*}
\E\left[\# \bigwedge_{b=1}^B T_b \right]
&=
\E\left[
\sum_{D' \in T_B}
\E\big[
\# T_{D'} \mid T_B
\big]
\right]
=
\E\big[\# T \big]
= \prod_{j=1}^d \left(
1 + |D_j| \sum_{b=1}^B \lambda_b
\right).
\end{align*}
%
\end{proof}

While the exact expectation calculation in
Lemma~\ref{lem:mondrian_app_refinement} is neat,
sharper control on the tail
behavior of the number of cells in a Mondrian refinement is desired.
Lemma~\ref{lem:mondrian_app_refinement_tail} provides this, again
making use of the subcell trick to convert a crude bound based on
Lemma~\ref{lem:mondrian_app_refinement_inequalities} into a useful tail
inequality.
We assume for simplicity that all of the lifetimes are identical.

\begin{lemma}[Tail bound on the number of cells in a Mondrian forest refinement]
\label{lem:mondrian_app_refinement_tail}

Let $T_b \sim \cM(D, \lambda)$ be i.i.d.\ for $1 \leq b \leq B$. Then
%
\begin{align*}
\P\left(
\# \bigwedge_{b=1}^B T_b
> 3^d 2^{d^2} B^d (1+\lambda|D|_1)^d t^d
\right)
&\leq
2^{d+1} d B (1 + \lambda |D|_1)^d e^{-t}.
\end{align*}

\end{lemma}

\begin{proof}[Lemma~\ref{lem:mondrian_app_refinement_tail}]

We begin with a coarse estimate and refine it with the subcell trick.
By Lemma~\ref{lem:mondrian_app_refinement_inequalities}
\eqref{eq:mondrian_app_refinement_2},
for any $t > 0$, recalling that $\# T_{b j}$ is the number
of splits made by $T_b$ in dimension $j$,
%
\begin{align}
\nonumber
\P\left(
\# \bigwedge_{b=1}^B T_b
> t
\right)
&\leq
\P\left(
\prod_{j=1}^d
\left(
1 + \sum_{b=1}^B \# T_{b j}
\right)
> t
\right)
\leq
\sum_{j=1}^d
\P\left(
1 + \sum_{b=1}^B \# T_{b j}
> t^{1/d}
\right) \\
\label{eq:mondrian_app_refinement_tail_coarse}
&\leq
d\, \P\left(
\sum_{b=1}^B \# T_b
> t^{1/d}
\right)
\leq
d B\,
\P\left(
\# T_b > \frac{t^{1/d}}{B}
\right).
\end{align}
%
By the subcell trick, partition $D$ into
at most $(1 + 1/\varepsilon)^d$ cells
$D' \in \cD_\varepsilon$ with side lengths at most
$(|D_1| \varepsilon, \ldots, |D_d| \varepsilon)$.
As every cell in $\bigwedge_b T_b$ corresponds to
at least one cell in $\bigwedge_b (T_b \cap D')$,
%
\begin{align*}
\P\left(
\# \bigwedge_{b=1}^B T_b
> t
\right)
&\leq
\P\left(
\sum_{D' \in \cD_\varepsilon}
\# \bigwedge_{b=1}^B (T_b \cap D')
> t
\right)
\leq
\sum_{D' \in \cD_\varepsilon}
\P\left(
\# \bigwedge_{b=1}^B (T_b \cap D')
> \frac{t}{\# \cD_\varepsilon}
\right).
\end{align*}
%
Applying the coarse estimate \eqref{eq:mondrian_app_refinement_tail_coarse}
to $\# \bigwedge_b (T_b \cap D')$ gives
%
\begin{align*}
\P\left(
\# \bigwedge_{b=1}^B T_b
> t
\right)
&\leq
d B \sum_{D' \in \cD_\varepsilon}
\P\left(
\# (T_b \cap D')
> \frac{t^{1/d}}{B \# \cD_\varepsilon^{1/d}}
\right).
\end{align*}
%
Now apply Lemma~\ref{lem:mondrian_app_cells_tail}
and set $\varepsilon = \frac{1}{\lambda |D|_1}$ to obtain
%
\begin{align*}
\P\left(
\# \bigwedge_{b=1}^B T_b
> t
\right)
&\leq
d B \sum_{D' \in \cD_\varepsilon}
\P\left(
\# (T_b \cap D')
> \frac{t^{1/d}}{B \# \cD_\varepsilon^{1/d}}
\right) \\
&\leq
d B \sum_{D' \in \cD_\varepsilon}
2 (1 + \lambda |D'|_1)^d
e^{- t^{1/d} \# \cD_\varepsilon^{-1/d} B^{-1}
(1 + \lambda |D'|_1)^{-d} / 3} \\
&\leq
2 d B (1 + 1 / \varepsilon)^d
(1 + \lambda \varepsilon |D|_1)^d
e^{- t^{1/d} (1 + 1/\varepsilon)^{-1} B^{-1}
(1 + \lambda \varepsilon |D|_1)^{-d} / 3} \\
&\leq
2^{d+1} d B (1 + \lambda |D|_1)^d
e^{- t^{1/d} (1 + \lambda |D|_1)^{-1} B^{-1} 2^{-d} / 3}.
\end{align*}
%
Finally, replacing $t$ by $3^d 2^{d^2} B^d (1+\lambda|D|_1)^d t^d$ we have
%
\begin{align*}
\P\left(
\# \bigwedge_{b=1}^B T_b
> 3^d 2^{d^2} B^d (1+\lambda|D|_1)^d t^d
\right)
&\leq
2^{d+1} d B (1 + \lambda |D|_1)^d e^{-t}.
\end{align*}
%
\end{proof}

\chapter{Supplement to Dyadic Kernel Density Estimators}
\label{app:kernel}

This section contains complementary detailed expositions of some
of our main results, along with additional technical lemmas
which may be of independent interest. We also provide full proofs
for all of our theoretical contributions.

\section{Supplementary main results}

In this first section we provide more detailed versions of some of the
results presented in the main text, alongside some intermediate
lemmas which were skipped for conciseness.
We begin with some extra notation used throughout this appendix.

For real vectors,
$\|\cdot\|_p$ is the standard $\ell^p$-norm defined for $p \in [1, \infty]$.
For real square matrices,
$\|\cdot\|_p$ is the operator
norm induced by the corresponding vector norm.
In particular,
$\|\cdot\|_1$
is the maximum absolute column sum,
$\|\cdot\|_\infty$
is the maximum absolute row sum,
and $\|\cdot\|_2$
is the maximum singular value.
For real symmetric matrices,
$\|\cdot\|_2$
coincides with the maximum absolute eigenvalue.
We use $\|\cdot\|_{\max}$
to denote the largest absolute entry of a real matrix.
For real-valued functions,
$\|\cdot\|_\infty$
denotes the (essential) supremum norm.
For a bounded set $\cX \subseteq \R$ and $a \geq 0$
we use $[\cX \pm a]$ to denote the compact interval
$[\inf \cX - a, \ \sup \cX + a]$.
For measurable subsets of $\R^d$
we use $\Leb$ to denote the Lebesgue measure,
and for finite sets we use $|\cdot|$
for the cardinality.
Write $\sum_i$
for $\sum_{i=1}^n$
when clear from context.
Similarly, use $\sum_{i<j}$
for $\sum_{i=1}^{n-1} \sum_{j=i+1}^n$
and $\sum_{i<j<r}$
for $\sum_{i=1}^{n-2} \sum_{j=i+1}^{n-1} \sum_{r=j+1}^n$.

\subsection{Strong approximation}
\label{sec:kernel_app_strong_approx}

We give a detailed construction of the
strong approximation of the dyadic empirical process $\hat f_W$.
We begin by using the
K{\'o}mlos--Major--Tusn{\'a}dy (KMT) approximation
to obtain a strong approximation for $L_n$
in Lemma~\ref{lem:kernel_app_strong_approx_Ln}.
Since $E_n$ is an empirical process of i.n.i.d.\ variables,
the KMT approximation is not valid.
Instead we apply a conditional version of
Yurinskii's coupling to obtain a
conditional strong approximation for $E_n$
in Lemma~\ref{lem:kernel_app_conditional_strong_approx_En},
and then construct an unconditional
strong approximation for $E_n$
in Lemma~\ref{lem:kernel_app_unconditional_strong_approx_En}.
These approximations are combined to give a
strong approximation for $\hat f_W$
in Theorem~\ref{thm:kernel_app_strong_approx_fW}.
We do not need to approximate
the negligible $Q_n$.

This section is largely concerned with
distributional properties,
and, as such, will frequently involve
\emph{copies} of processes.
We say that $X'$ is a copy of a random variable $X$
if they have the same distribution,
though they may be defined on different probability spaces.
To ensure that all of the joint distributional properties of
such processes are preserved,
we also carry over a copy of the latent variables
$(\bA_n, \bV_n)$
to the new space.

Many of the technical details regarding
the copying and embedding of stochastic processes
are covered by the
Vorob'ev--Berkes--Philipp theorem,
which is stated and discussed in
Lemma~\ref{lem:kernel_app_vbp}.
In particular, this theorem can be used
for random vectors or for stochastic processes
indexed by a compact rectangle in $\R^d$
with a.s.\ continuous sample paths.

We present more detailed versions of
Lemmas~\ref{lem:kernel_strong_approx_Ln},
\ref{lem:kernel_conditional_strong_approx_En},
and \ref{lem:kernel_unconditional_strong_approx_En}
as Lemmas~\ref{lem:kernel_app_strong_approx_Ln},
\ref{lem:kernel_app_conditional_strong_approx_En},
and \ref{lem:kernel_app_unconditional_strong_approx_En}
respectively, taking care to describe how copies of
the stochastic processes are constructed,
and also providing smoothness properties for
the resulting sample paths.

\begin{lemma}[Strong approximation of $L_n$]
\label{lem:kernel_app_strong_approx_Ln}

Suppose that Assumptions
\ref{ass:kernel_data}
and
\ref{ass:kernel_bandwidth} hold.
For each $n \geq 2$
there exists
on some probability space
a copy of $\big(\bA_n, \bV_n, L_n\big)$,
denoted $\big(\bA_n', \bV_n', L_n'\big)$,
and a mean-zero Gaussian process
$Z^{L\prime}_n$
indexed on $\cW$ satisfying
%
\begin{align*}
\P\left(
\sup_{w \in \cW}
\big| \sqrt{n} L'_n(w) - Z_n^{L\prime}(w)\big|
> \Du
\frac{t + C_1 \log n}{\sqrt{n}}
\right)
&\leq C_2 e^{-C_3 t},
\end{align*}
%
for some positive constants
$C_1$, $C_2$, $C_3$,
and for all $t > 0$.
By integration of tail probabilities,
%
\begin{align*}
\E\left[
\sup_{w \in \cW}
\big| \sqrt{n} L_n'(w) - Z_n^{L\prime}(w)\big|
\right]
&\lesssim
\frac{\Du \log n}{\sqrt{n}}.
\end{align*}
%
Further,
$Z_n^{L\prime}$ has the same covariance structure as
$\sqrt{n} L_n'$ in the sense that for all $w, w' \in \cW$,
%
\begin{align*}
\E\left[
Z_n^{L\prime}(w)
Z_n^{L\prime}(w')
\right]
&=
n
\E\left[
L_n'(w)
L_n'(w')
\right].
\end{align*}
%
It also satisfies the following
trajectory regularity property
for any $\delta_n \in (0, 1/2]$:
%
\begin{align*}
\E\left[
\sup_{|w-w'| \leq \delta_n}
\big|
Z_n^{L\prime}(w)
- Z_n^{L\prime}(w')
\big|
\right]
&\lesssim
\Du
\delta_n \sqrt{\log 1/\delta_n},
\end{align*}
%
and has continuous trajectories.
The process $Z_n^{L\prime}$
is a function only of $\bA_n'$
and some random noise
which is independent of $(\bA_n', \bV_n')$.

\end{lemma}

\begin{lemma}[Conditional strong approximation of $E_n$]
\label{lem:kernel_app_conditional_strong_approx_En}

Suppose Assumptions
\ref{ass:kernel_data} and \ref{ass:kernel_bandwidth} hold.
For $n \geq 2$
and $t_n > 0$ with $\left|\log t_n\right| \lesssim \log n$,
there exists on some probability space
a copy of
$\big(\bA_n, \bV_n, E_n\big)$,
denoted
$\big(\bA_n', \bV_n', E_n'\big)$,
and a process
$\tilde Z^{E\prime}_n$
which is Gaussian conditional on $\bA_n'$
and mean-zero conditional on $\bA_n'$,
satisfying
%
\begin{align*}
\P\left(
\sup_{w \in \cW}
\big|
\sqrt{n^2h} E_n'(w) - \tilde Z_n^{E\prime}(w)
\big|
> t_n
\Bigm\vert \bA_n'
\right)
&\leq
C_1
t_n^{-2}
n^{-1/2}
h^{-3/4}
(\log n)^{3/4},
\end{align*}
$\bA_n'$-almost surely
for some constant $C_1 > 0$.
Setting $t_n = n^{-1/4} h^{-3/8} (\log n)^{3/8} R_n$
for any sequence $R_n \to \infty$
and taking an expectation gives
%
\begin{align*}
\sup_{w \in \cW}
\big|
\sqrt{n^2h} E_n'(w) - \tilde Z_n^{E\prime}(w)
\big|
&\lesssim_\P
n^{-1/4}
h^{-3/8} (\log n)^{3/8} R_n.
\end{align*}
%
Further,
$\tilde Z_n^{E\prime}$ has the same conditional covariance as
$\sqrt{n^2h} E_n'$ in that for all $w, w' \in \cW$,
%
\begin{align*}
\E\left[
\tilde Z_n^{E\prime}(w)
\tilde Z_n^{E\prime}(w')
\bigm\vert \bA_n'
\right]
&=
n^2h
\E\left[
E_n'(w)
E_n'(w')
\bigm\vert \bA_n'
\right].
\end{align*}
%
It also satisfies the following
trajectory regularity property
for any $\delta_n \in (0, 1/(2h)]$:
%
\begin{align*}
\E\left[
\sup_{|w-w'| \leq \delta_n}
\big|
\tilde Z_n^{E\prime}(w)
- \tilde Z_n^{E\prime}(w')
\big|
\right]
&\lesssim
\frac{\delta_n}{h}
\sqrt{\log \frac{1}{h\delta_n}},
\end{align*}
%
and has continuous trajectories.

\end{lemma}

\begin{lemma}[Unconditional strong approximation of $E_n$]
\label{lem:kernel_app_unconditional_strong_approx_En}

Suppose Assumptions
\ref{ass:kernel_data} and \ref{ass:kernel_bandwidth} hold.
Let $\big(\bA_n', \bV_n', \tilde Z_n^{E\prime}\big)$
be defined as in
Lemma~\ref{lem:kernel_app_conditional_strong_approx_En}.
For each $n \geq 2$
there exists
(on some probability space)
a copy of
$\big(\bA_n', \bV_n', \tilde Z_n^{E\prime}\big)$,
denoted
$\big(\bA_n'', \bV_n'', \tilde Z_n^{E\dprime}\big)$,
and a centered
Gaussian process
$Z^{E\dprime}_n$
satisfying
%
\begin{align*}
\E\left[
\sup_{w \in \cW}
\big|\tilde Z_n^{E\dprime}(w) - Z_n^{E\dprime}(w)\big|
\right]
&\lesssim
n^{-1/6} (\log n)^{2/3}.
\end{align*}
%
Further,
$Z_n^{E\dprime}$ has the same
(unconditional) covariance structure as
$\tilde Z_n^{E\dprime}$ and $\sqrt{n^2h} E_n$
in the sense that for all $w, w' \in \cW$,
%
\begin{align*}
\E\left[
Z_n^{E\dprime}(w)
Z_n^{E\dprime}(w')
\right]
&=
\E\left[
\tilde Z_n^{E\dprime}(w)
\tilde Z_n^{E\dprime}(w')
\right]
=
n^2h \,
\E\left[
E_n(w)
E_n(w')
\right].
\end{align*}
%
It also satisfies the following
trajectory regularity property
for any $\delta_n \in (0, 1/(2h)]$:
%
\begin{align*}
\E\left[
\sup_{|w-w'| \leq \delta_n}
\big|
Z_n^{E\dprime}(w)
- Z_n^{E\dprime}(w')
\big|
\right]
&\lesssim
\frac{\delta_n}{h}
\sqrt{\log \frac{1}{h\delta_n}}.
\end{align*}
%
Finally, $Z_n^{E\dprime}$ is independent of $\bA_n''$
and has continuous trajectories.

\end{lemma}

We combine these strong approximations to deduce a coupling for $\hat f_W$ in
Theorem~\ref{thm:kernel_app_strong_approx_fW}, taking care with independence
to ensure the approximating processes are jointly Gaussian.

\begin{theorem}[Strong approximation of $\hat f_W$]
\label{thm:kernel_app_strong_approx_fW}

Suppose that Assumptions \ref{ass:kernel_data} and \ref{ass:kernel_bandwidth}
hold. For each $n \geq 2$ and any sequence $R_n \to \infty$ there exists on
some probability space a centered Gaussian process $Z_n^{f\prime}$ and a copy
of $\hat f_W$, denoted $\hat f_W'$, satisfying
%
\begin{align*}
&\sup_{w \in \cW}
\Big|
\hat f_W'(w) - \E[\hat f_W'(w)]
- Z_n^{f\prime}(w)
\Big| \\
&\quad\lesssim_\P
n^{-1} \log n
+ n^{-5/4} h^{-7/8} (\log n)^{3/8} R_n
+ n^{-7/6} h^{-1/2} (\log n)^{2/3}.
\end{align*}
%
Further, $Z_n^{f\prime}$ has the same covariance
structure as
$\hat f_W'(w)$
in the sense that for all
$w, w' \in \cW$,
%
\begin{align*}
\E\big[Z_n^{f\prime}(w) Z_n^{f\prime}(w')\big]
&=
\Cov\Big[
\hat f_W'(w),
\hat f_W'(w')
\Big]
= \Sigma_n(w,w').
\end{align*}
%
It has continuous trajectories satisfying the following regularity property
for any $\delta_n \in (0, 1/2]$:
%
\begin{align*}
\E\left[
\sup_{|w-w'| \leq \delta_n}
\Big|
Z_n^{f\prime}(w)
- Z_n^{f\prime}(w')
\Big|
\right]
&\lesssim
\frac{\Du}{\sqrt n} \delta_n
\sqrt{\log \frac{1}{\delta_n}}
+ \frac{1}{\sqrt{n^2h}}
\frac{\delta_n}{h}
\sqrt{\log \frac{1}{h\delta_n}}.
\end{align*}
%
\end{theorem}

The main result Theorem~\ref{thm:kernel_strong_approx_Tn}
now follows easily using Theorem~\ref{thm:kernel_app_strong_approx_fW},
the bias bound from Theorem~\ref{thm:kernel_bias},
and properties of $\Sigma_n$ established in
Lemma~\ref{lem:kernel_variance_bounds}.

\subsection{Covariance estimation}
\label{sec:kernel_app_covariance_estimation}

In this section we carefully construct a consistent estimator for the
covariance function $\Sigma_n$. Firstly, we characterize $\Sigma_n$ in
Lemma~\ref{lem:kernel_app_covariance_structure}. In
Lemma~\ref{lem:kernel_app_covariance_estimation}
we define the estimator and demonstrate that it converges in probability in a
suitable sense. In Lemma~\ref{lem:kernel_app_alternative_covariance_estimator}
we give an
alternative representation which is more amenable to computation.

\begin{lemma}[Covariance structure]
\label{lem:kernel_app_covariance_structure}

Suppose Assumptions~\ref{ass:kernel_data} and~\ref{ass:kernel_bandwidth}
hold. Then $\Sigma_n$, as defined in Section~\ref{sec:kernel_degeneracy},
admits the following representations,
where $1 \leq i < j < r \leq n$.
%
\begin{align*}
\Sigma_n(w,w')
&=
\frac{2}{n(n-1)}
\,\Cov\!\big[
k_h(W_{i j},w),
k_h(W_{i j},w')
\big]
+
\frac{4(n-2)}{n(n-1)}
\,\Cov\!\big[
k_h(W_{i j},w),
k_h(W_{i r},w')
\big] \\
&=
\frac{2}{n(n-1)}
\,\Cov\!\big[
k_h(W_{i j},w),
k_h(W_{i j},w')
\big] \\
&\quad+
\frac{4(n-2)}{n(n-1)}
\,\Cov\!\big[
\E[k_h(W_{i j},w) \mid A_i],
\E[k_h(W_{i j},w') \mid A_i]
\big],
\end{align*}
%
\end{lemma}

\begin{lemma}[Covariance estimation]
\label{lem:kernel_app_covariance_estimation}

Grant Assumptions \ref{ass:kernel_data} and \ref{ass:kernel_bandwidth},
and suppose $n h \gtrsim \log n$ and $f_W(w) > 0$ on $\cW$. Define
%
\begin{align*}
S_{i j r}(w,w')
&=
\frac{1}{6}
\Big(
k_h(W_{i j},w)
k_h(W_{i r},w')
+ k_h(W_{i j},w)
k_h(W_{jr},w')
+ k_h(W_{i r},w)
k_h(W_{i j},w') \\
&\quad+
k_h(W_{i r},w)
k_h(W_{jr},w')
+ k_h(W_{jr},w)
k_h(W_{i j},w')
+ k_h(W_{jr},w)
k_h(W_{i r},w')
\Big), \\
\hat \Sigma_n(w,w')
&=
\frac{4}{n^2(n-1)^2}
\sum_{i<j}
k_h(W_{i j},w)
k_h(W_{i j},w')
+
\frac{24}{n^2(n-1)^2}
\sum_{i<j<r}
S_{i j r}(w,w') \\
&\quad-
\frac{4n-6}{n(n-1)}
\hat f_W(w)
\hat f_W(w').
\end{align*}
%
Then $\hat \Sigma_n$
is uniformly entrywise-consistent in the sense that
%
\begin{align*}
\sup_{w,w' \in \cW}
\left|
\frac{\hat \Sigma_n(w,w') - \Sigma_n(w,w')}
{\sqrt{\Sigma_n(w,w) + \Sigma_n(w',w')}}
\right|
&\lesssim_\P
\frac{\sqrt{\log n}}{n}.
\end{align*}

\end{lemma}

\begin{lemma}[Alternative covariance estimator representation]
\label{lem:kernel_app_alternative_covariance_estimator}

Suppose that Assumptions~\ref{ass:kernel_data}
and~\ref{ass:kernel_bandwidth} hold,
and let $\hat \Sigma_n$
be the covariance estimator defined
in Lemma~\ref{lem:kernel_app_covariance_estimation}.
Then the following alternative representation
for $\hat \Sigma_n$ holds,
which may be easier to compute
as it does not involve any triple summations
over the data.
Let $S_i(w) = \frac{1}{n-1}
\sum_{j = 1}^{i-1} k_h(W_{j i}, w)
+ \frac{1}{n-1} \sum_{j = i+1}^n k_h(W_{i j}, w)$
estimate $\E[k_h(W_{i j},w) \mid A_i]$.
%
\begin{align*}
\hat \Sigma_n(w,w')
&=
\frac{4}{n^2}
\sum_{i=1}^n
S_i(w) S_i(w')
- \frac{4}{n^2(n-1)^2}
\sum_{i<j}
k_h(W_{i j},w)
k_h(W_{i j},w') \\
&\quad-
\frac{4n-6}{n(n-1)}
\hat f_W(w)
\hat f_W(w').
\end{align*}
%
\end{lemma}

We show how to obtain a positive semi-definite estimator $\hat \Sigma_n^+$
which is uniformly entrywise-consistent for $\Sigma_n$. Define $\hat \Sigma_n$
as in Lemma~\ref{lem:kernel_app_covariance_estimation} and consider the
following
optimization problem over bivariate functions.
%
\begin{equation}
\label{eq:kernel_app_sdp}
%
\begin{aligned}
\minimize
\qquad
& \sup_{w,w' \in \cW}
\left|
\frac{M(w,w') - \hat\Sigma_n(w,w')}
{\sqrt{\hat \Sigma_n(w,w) + \hat \Sigma_n(w',w')}}
\right|
\quad \textup{ over } M: \cW \times \cW \to \R
\\
\subjectto
\qquad
& M \textup{ is symmetric and positive semi-definite}, \\
& \big|M(w,w') - M(w, w'')\big|
\leq \frac{4}{n h^3}
C_\rk C_\rL
|w'-w''|
\textup{ for all }
w, w', w'' \in \cW.
\end{aligned}
%
\end{equation}

\begin{lemma}[Consistency of $\hat \Sigma_n^+$]
\label{lem:kernel_app_sdp}

Suppose that Assumptions~\ref{ass:kernel_data}
and~\ref{ass:kernel_bandwidth} hold, and that
$n h \gtrsim \log n$ and $f_W(w) > 0$ on $\cW$.
Then the optimization problem \eqref{eq:kernel_app_sdp}
has an approximately optimal solution $\hat\Sigma_n^+$
which is uniformly entrywise-consistent
for $\Sigma_n$ in the sense that
%
\begin{align*}
\sup_{w,w' \in \cW}
\left|
\frac{\hat \Sigma_n^+(w,w') - \Sigma_n(w,w')}
{\sqrt{\Sigma_n(w,w) + \Sigma_n(w',w')}}
\right|
&\lesssim_\P
\frac{\sqrt{\log n}}{n}.
\end{align*}

\end{lemma}

The optimization problem \eqref{eq:kernel_app_sdp} is stated for functions
rather than
matrices so is infinite-dimensional. However, when restricting to finite-size
matrices, Lemma~\ref{lem:kernel_app_sdp} still holds and does not depend on the
size
of the matrices. Furthermore, the problem then becomes a semi-definite program
and so can be solved to arbitrary precision in polynomial time in the size of
the matrices \citep{laurent2005semidefinite}.

The Lipschitz-type constraint in the optimization problem
\eqref{eq:kernel_app_sdp}
ensures that $\hat \Sigma_n^+$ is sufficiently smooth and is a technicality
required by some of the later proofs. In practice this constraint is readily
verified.

\begin{lemma}[Positive semi-definite variance estimator bounds]
\label{lem:kernel_app_variance_estimator_bounds}

Suppose that Assumptions~\ref{ass:kernel_data}
and~\ref{ass:kernel_bandwidth} hold, and that
$n h \gtrsim \log n$ and $f_W(w) > 0$ on $\cW$.
Then $\hat \Sigma_n^+(w,w) \geq 0$
almost surely for all $w \in \cW$ and
%
\begin{align*}
\frac{\Dl^2}{n} + \frac{1}{n^2h}
&\lesssim_\P
\inf_{w \in \cW} \hat \Sigma_n^+(w,w)
\leq
\sup_{w \in \cW} \hat \Sigma_n^+(w,w)
\lesssim_\P
\frac{\Du^2}{n} + \frac{1}{n^2h}.
\end{align*}

\end{lemma}

\subsection{Feasible uniform confidence bands}

We use the strong approximation derived in
Section~\ref{sec:kernel_app_strong_approx} and the
positive semi-definite covariance estimator introduced in
Section~\ref{sec:kernel_app_covariance_estimation} to construct feasible
uniform
confidence bands. We drop the prime notation for copies of processes
in the interest of clarity.

\begin{lemma}[Proximity of the standardized and studentized $t$-statistics]
\label{lem:kernel_app_studentized_t_statistic}

Let Assumptions \ref{ass:kernel_data} and
\ref{ass:kernel_bandwidth} hold, and suppose that
$n h \gtrsim \log n$ and $f_W(w) > 0$ on $\cW$.
Define for $w \in \cW$
the Studentized $t$-statistic process
%
\begin{align*}
\hat T_n(w) = \frac{\hat f_W(w) - f_W(w)}
{\sqrt{\hat\Sigma_n^+(w,w)}}.
\end{align*}
%
Then
%
\begin{align*}
\sup_{w \in \cW}
\left| \hat T_n(w) - T_n(w) \right|
&\lesssim_\P
\sqrt{\frac{\log n}{n}}
\left(
\sqrt{\log n} + \frac{\sqrt n h^{p \wedge \beta}}
{\Dl + 1/\sqrt{n h}}
\right)
\frac{1}{\Dl + 1/\sqrt{n h}}.
\end{align*}

\end{lemma}

\begin{lemma}[Feasible Gaussian approximation
of the infeasible Gaussian process]
\label{lem:kernel_app_distributional_approx_feasible_gaussian}

Let Assumptions \ref{ass:kernel_data} and \ref{ass:kernel_bandwidth}
hold, and suppose that
$n h \gtrsim \log n$ and $f_W(w) > 0$ on $\cW$.
Define a process $\hat Z_n^T(w)$ which,
conditional on the data $\bW_n$,
is conditionally mean-zero and
conditionally Gaussian, and whose
conditional covariance structure is
%
\begin{align*}
\E\left[
\hat Z_n^T(w) \hat Z_n^T(w')
\bigm| \bW_n \right]
&=
\frac{\hat \Sigma_n^+(w,w')}
{\sqrt{\hat \Sigma_n^+(w,w) \hat \Sigma_n^+(w',w')}}
\end{align*}
%
Then the following conditional
Kolmogorov--Smirnov result holds.
%
\begin{align*}
\sup_{t \in \R}
\left|
\P\left(
\sup_{w \in \cW}
\left| Z_n^T(w) \right|
\leq t
\right)
- \P\left(
\sup_{w \in \cW}
\left| \hat Z_n^T(w) \right|
\leq t
\biggm\vert \bW_n
\right)
\right|
&\lesssim_\P
\frac{n^{-1/6}(\log n)^{5/6}}
{\Dl^{1/3} + (n h)^{-1/6}}.
\end{align*}

\end{lemma}

\begin{lemma}[Feasible Gaussian approximation of the studentized $t$-statistic]
\label{lem:kernel_app_feasible_gaussian_approx}

Let Assumptions \ref{ass:kernel_data}, \ref{ass:kernel_bandwidth}
and \ref{ass:kernel_rates} hold, and suppose that $f_W(w) > 0$ on $\cW$.
Then
%
\begin{align*}
\sup_{t \in \R}
\left|
\P\left(
\sup_{w \in \cW}
\left| \hat T_n(w) \right|
\leq t
\right)
- \P\left(
\sup_{w \in \cW}
\left| \hat Z_n^T(w) \right|
\leq t
\Bigm\vert \bW_n
\right)
\right|
&\ll_\P
1.
\end{align*}

\end{lemma}

These intermediate lemmas can be used to establish the valid and feasible
uniform confidence bands presented in Theorem~\ref{thm:kernel_ucb} in the main
text. See Section~\ref{sec:kernel_app_proofs} for details.

\subsection{Counterfactual dyadic density estimation}

In this section we give a detailed analysis of the counterfactual
estimator of Section~\ref{sec:kernel_counterfactual}.
We begin with an assumption describing the counterfactual setup.

\begin{assumption}[Counterfactual data generation]
\label{ass:kernel_app_counterfactual}

For each $r \in \{0,1\}$,
let $\bW_n^r$, $\bA_n^r$, and $\bV_n^r$ be as in
Assumption~\ref{ass:kernel_data}.
Let $X_i^r$ be finitely-supported variables,
setting $\bX_n^r = (X_1^r, \ldots, X_n^r)$.
Suppose that $(A_i^r, X_i^r)$ are
independent over $1 \leq i \leq n$
and that $\bX_n^r$ is independent of $\bV_n^r$.
Assume that $W_{i j}^r \mid X_i^r, X_j^r$
has a Lebesgue density
$f_{W \mid XX}^r(\,\cdot \mid x_1, x_2) \in \cH^\beta_{C_\rH}(\cW)$
and that $X_i^r$ has positive
probability mass function
$p_X^r(x)$ on a common support $\cX$.
Suppose that
$(\bA_n^0, \bV_n^0, \bX_n^0)$
and $(\bA_n^1, \bV_n^1, \bX_n^1)$
are independent.

\end{assumption}

The counterfactual density of $W_{i j}$ in population $1$ had $X_i, X_j$
followed population $0$ is
%
\begin{align*}
f_W^{1 \triangleright 0}(w)
&=
\E\left[
f_{W \mid XX}^1\big(w \mid X_1^0, X_2^0\big)
\right]
= \sum_{x_1 \in \cX}
\sum_{x_2 \in \cX}
f_{W \mid XX}^{1}(w \mid x_1, x_2)
\psi(x_1)
\psi(x_2)
p_X^{1}(x_1)
p_X^{1}(x_2),
\end{align*}
%
with $\psi(x) = p_X^0(x)/p_X^1(x)$ for $x \in \cX$.
Define the counterfactual dyadic kernel density estimator
%
\begin{align*}
\hat f_W^{1 \triangleright 0}(w)
&=
\frac{2}{n(n-1)}
\sum_{i=1}^{n-1}
\sum_{j=i+1}^n
\hat \psi(X_i^1)
\hat \psi(X_j^1)
k_h(W_{i j}^1, w),
\end{align*}
%
where
$\hat\psi(x) = \I\{\hat p_X^{1}(x) > 0\}\hat p_X^{0}(x) / \hat p_X^{1}(x)$
and $\hat p_X^{r}(x) = \frac{1}{n}\sum_{i = 1}^n \I\{X_i^r = x\}$.
Since $p_X^r(x) > 0$,
%
\begin{align*}
\hat\psi(x) - \psi(x)
&=
\frac{\hat p_X^{0}(x) - p_X^0(x)}{p_X^1(x)}
- \frac{p_X^0(x)}{p_X^1(x)}
\frac{\hat p_X^{1}(x) - p_X^1(x)}{p_X^1(x)} \\
&\quad+
\frac{\hat p_X^{1}(x) - p_X^1(x)}{p_X^1(x)}
\frac{\hat p_X^{1}(x) p_X^0(x) - \hat p_X^{0}(x)p_X^1(x)}
{\hat p_X^{1}(x) p_X^1(x)} \\
&=
\frac{1}{n}
\sum_{r=1}^n \kappa(X_r^0, X_r^1, x)
+ O_\P\left(\frac{1}{n}\right)
\end{align*}
%
is an asymptotic linear representation where
%
\begin{align*}
\kappa(X_i^0, X_i^1, x)
&=
\frac{\I\{X_i^0 = x\} - p_X^0(x)}{p_X^1(x)}
- \frac{p_X^0(x)}{p_X^1(x)}
\frac{\I\{X_i^1 = x\} - p_X^1(x)}{p_X^1(x)}
\end{align*}
%
satisfies
$\E[\kappa(X_i^0, X_i^1, x)] = 0$.
We now establish uniform consistency and feasible strong
approximation results for the counterfactual density estimator.

\begin{lemma}[Bias of $\hat f_W^{1 \triangleright 0}$]
\label{lem:kernel_app_counterfactual_bias}

Suppose that Assumptions~\ref{ass:kernel_data},
\ref{ass:kernel_bandwidth}, and \ref{ass:kernel_app_counterfactual} hold.
Then
%
\begin{align*}
\sup_{w \in \cW}
\big|
\E\big[\hat f_W^{1 \triangleright 0}(w)\big]
- f_W^{1 \triangleright 0}(w)
\big|
\lesssim
h^{p \wedge \beta} + \frac{1}{n}.
\end{align*}

\end{lemma}

\begin{lemma}[Hoeffding-type decomposition for
$\hat f_W^{1 \triangleright 0}$]
\label{lem:kernel_app_counterfactual_hoeffding}

Suppose that Assumptions~\ref{ass:kernel_data},
\ref{ass:kernel_bandwidth}, and
\ref{ass:kernel_app_counterfactual} hold.
With $k_{i j} = k_h(W_{i j}^1, w)$,
$\kappa_{r i} = \kappa(X_r^0, X_r^1, X_i^1)$
and $\psi_i = \psi(X_i^1)$, define the projections
%
\begin{align*}
u
&=
\E\left[
k_{i j}
\psi_i
\psi_j
\right], \\
u_i
&=
\frac{2}{3} \psi_i
\E\left[
k_{i j}
\psi_j
\mid A_i^1 \right]
+
\frac{2}{3} \E\left[
k_{jr}
\psi_j \kappa_{i r}
\mid X_i^0, X_i^1 \right]
- \frac{2}{3} u, \\
u_{i j}
&=
\frac{1}{3}
\psi_i
\psi_j
\E\left[
k_{i j}
\mid A_i^1, A_j^1 \right]
+
\frac{1}{3}
\psi_i
\E\left[
k_{i r} \psi_r
\mid A_i^1 \right]
+
\frac{1}{3}
\psi_i
\E\left[
k_{i r} \kappa_{jr}
\mid A_i^1, X_j^0, X_j^1 \right] \\
&\quad+
\frac{1}{3}
\kappa_{j i}
\E\left[
k_{i r} \psi_r
\mid A_i^1 \right]
+ \frac{1}{3}
\psi_j
\E\left[
k_{jr} \psi_r
\mid A_j^1 \right]
+
\frac{1}{3}
\psi_j
\E\left[
k_{jr} \kappa_{i r}
\mid X_i^0, X_i^1, A_j^1 \right] \\
&\quad+
\frac{1}{3}
\kappa_{i j}
\E\left[
k_{jr} \psi_r
\mid A_j^1 \right]
- u_i - u_j + u, \\
u_{i j r}
&=
\frac{1}{3}
\psi_i \psi_j
\E\left[
k_{i j}
\mid A_i^1, A_j^1 \right]
+
\frac{1}{3}
\psi_i \kappa_{r j}
\E\left[
k_{i j}
\mid A_i^1, A_j^1 \right]
+
\frac{1}{3}
\psi_j \kappa_{r i}
\E\left[
k_{i j}
\mid A_i^1, A_j^1 \right] \\
&\quad+
\frac{1}{3}
\psi_i \psi_r
\E\left[
k_{i r}
\mid A_i^1, A_r^1 \right]
+ \frac{1}{3}
\psi_i \kappa_{jr}
\E\left[
k_{i r}
\mid A_i^1, A_r^1 \right]
+
\frac{1}{3}
\psi_r \kappa_{j i}
\E\left[
k_{i r}
\mid A_i^1, A_r^1 \right] \\
&\quad+
\frac{1}{3}
\psi_j \psi_r
\E\left[
k_{jr}
\mid A_j^1, A_r^1 \right]
+ \frac{1}{3}
\psi_j \kappa_{i r}
\E\left[
k_{jr}
\mid A_j^1, A_r^1 \right]
+
\frac{1}{3}
\psi_r \kappa_{i j}
\E\left[
k_{jr}
\mid A_j^1, A_r^1 \right] \\
&\quad-
u_{i j} - u_{i r} - u_{jr}
+ u_i + u_j + u_r
- u, \\
v_{i j r}
&=
\frac{1}{3}
k_{i j} \big(\psi_i \psi_j +\psi_i \kappa_{r j} +\psi_j \kappa_{r i} \big)
+ \frac{1}{3}
k_{i r} \big(\psi_i \psi_r +\psi_i \kappa_{jr} +\psi_r \kappa_{j i} \big) \\
&\quad+
\frac{1}{3}
k_{jr} \big(\psi_j \psi_r +\psi_j \kappa_{i r} +\psi_r \kappa_{i j} \big).
\end{align*}
%
With $l_i^{1 \triangleright 0}(w) = u_i$
and $e_{i j r}^{1 \triangleright 0}(w) = v_{i j r} - u_{i j r}$,
set
%
\begin{align*}
L_n^{1 \triangleright 0}(w)
&=
\frac{3}{n} \sum_{i=1}^n
l_i^{1 \triangleright 0}(w)
&\text{and} &
&E_n^{1 \triangleright 0}(w)
&=
\frac{6}{n(n-1)(n-2)}
\sum_{i=1}^{n-2}
\sum_{j=i+1}^{n-1}
\sum_{r=i+1}^n
e_{i j r}^{1 \triangleright 0}(w).
\end{align*}
%
Then the following Hoeffding-type decomposition holds,
where $O_\P(1/n)$ is uniform in $w \in \cW$.
%
\begin{align*}
\hat f_W^{1 \triangleright 0}(w)
= \E\big[\hat f_W^{1 \triangleright 0}(w)\big]
+ L_n^{1 \triangleright 0}(w)
+ E_n^{1 \triangleright 0}(w)
+ O_\P\left( \frac{1}{n} \right).
\end{align*}
%
Further,
the stochastic processes
$L_n^{1 \triangleright 0}$
and $E_n^{1 \triangleright 0}$
are mean-zero and orthogonal
in $L^2(\P)$.
Define the upper and lower degeneracy constants as
%
\begin{align*}
\Du^{1 \triangleright 0}
&=
\limsup_{n \to \infty}
\sup_{w \in \cW}
\Var\big[
l_i^{1 \triangleright 0}(w)
\big]^{1/2}
&\text{and}&
&
\Dl^{1 \triangleright 0}
&=
\liminf_{n \to \infty}
\inf_{w \in \cW}
\Var\big[
l_i^{1 \triangleright 0}(w)
\big]^{1/2}.
\end{align*}

\end{lemma}

\begin{lemma}[Uniform consistency of $\hat f_W^{1 \triangleright 0}$]
\label{lem:kernel_app_counterfactual_uniform_consistency}

Suppose that Assumptions~\ref{ass:kernel_data},
\ref{ass:kernel_bandwidth}, and \ref{ass:kernel_app_counterfactual} hold.
Then
%
\begin{align*}
\E\left[
\sup_{w \in \cW}
\big|\hat f_W^{1 \triangleright 0}(w)
- f_W^{1 \triangleright 0}(w)
\right]
&\lesssim
h^{p \wedge \beta}
+ \frac{\Du^{1 \triangleright 0}}{\sqrt n}
+ \sqrt{\frac{\log n}{n^2h}}.
\end{align*}

\end{lemma}

\begin{lemma}[Strong approximation of $\hat f_W^{1 \triangleright 0}$]
\label{lem:kernel_app_counterfactual_sa}

On an appropriately enlarged probability space
and for any sequence $R_n \to \infty$,
there exists a mean-zero Gaussian process
$Z_n^{f, 1 \triangleright 0}$
with the same covariance structure as
$\hat f_W^{1 \triangleright 0}(w)$ satisfying
%
\begin{align*}
&\sup_{w \in \cW}
\left|
\hat f_W^{1 \triangleright 0}(w)
- \E\big[\hat f_W^{1 \triangleright 0}(w)\big]
- Z_n^{f, 1 \triangleright 0}(w)
\right| \\
&\quad\lesssim_\P
n^{-1} \log n
+ n^{-5/4} h^{-7/8} (\log n)^{3/8} R_n
+ n^{-7/6} h^{-1/2} (\log n)^{2/3}.
\end{align*}

\end{lemma}

\begin{lemma}[Counterfactual covariance structure]
\label{lem:kernel_app_counterfactual_covariance_structure}

Writing $k_{i j}'$ for $k_h(W_{i j}^1, w')$ etc.,
the counterfactual covariance function is
%
\begin{align*}
&\Sigma_n^{1 \triangleright 0}(w,w')
= \Cov\left[
\hat f_W^{1 \triangleright 0}(w),
\hat f_W^{1 \triangleright 0}(w')
\right] \\
&\quad=
\frac{4}{n}
\E\left[
\Big(
\psi_i
\E\big[
k_{i j} \psi_j
\mid A_i^1
\big]
+ \E\left[
k_{r j} \psi_r
\kappa_{i j}
\mid X_i^0, X_i^1
\right]
\Big)
\right. \\
&\left.
\qquad\qquad\quad
\times
\Big(
\psi_i
\E\big[
k_{i j}' \psi_j
\mid A_i^1
\big]
+ \E\left[
k_{r j}' \psi_r \kappa_{i j}
\mid X_i^0, X_i^1
\right]
\Big)
\right] \\
&\qquad+
\frac{2}{n^2}
\E\left[
k_{i j} k_{i j}'
\psi_i^2 \psi_j^2
\right]
- \frac{4}{n}
\E\left[
k_{i j} \psi_i \psi_j
\right]
\E\left[
k_{i j}' \psi_i \psi_j
\right]
+ O\left( \frac{1}{n^{3/2}} + \frac{1}{\sqrt{n^4h}} \right).
\end{align*}

\end{lemma}

\begin{lemma}[Gaussian approximation
of the standardized counterfactual $t$-statistic]
\label{lem:kernel_app_counterfactual_infeasible_t_statistic}

Let Assumptions \ref{ass:kernel_data},
\ref{ass:kernel_bandwidth}, and
\ref{ass:kernel_app_counterfactual}
hold, and suppose
$f_W^{1 \triangleright 0}(w) > 0$ on $\cW$.
Define
%
\begin{align*}
T_n^{1 \triangleright 0}(w)
&= \frac{\hat f_W^{1 \triangleright 0}(w)
- f_W^{1 \triangleright 0}(w)}
{\sqrt{\Sigma_n^{1 \triangleright 0}(w,w)}}
\quad\text{and}\quad
Z_n^{T, 1 \triangleright 0}(w)
= \frac{Z_n^{f, 1 \triangleright 0}(w)}
{\sqrt{\Sigma_n^{1 \triangleright 0}(w,w)}}.
\end{align*}
%
Then with $R_n \to \infty$ as in Lemma~\ref{lem:kernel_app_counterfactual_sa},
%
\begin{align*}
&\sup_{w \in \cW}
\left|
T_n^{1 \triangleright 0}(w) - Z_n^{T, 1 \triangleright 0}(w)
\right| \\
&\quad\lesssim_\P
\frac{
n^{-1/2} \log n
+ n^{-3/4} h^{-7/8} (\log n)^{3/8} R_n
+ n^{-2/3} h^{-1/2} (\log n)^{2/3}
+ n^{1/2} h^{p \wedge \beta}}
{\Dl^{1 \triangleright 0} + 1/\sqrt{n h}}.
\end{align*}

\end{lemma}

\begin{theorem}[Infeasible counterfactual uniform confidence bands]
\label{thm:kernel_app_counterfactual_infeasible_ucb}

Let Assumptions \ref{ass:kernel_data}, \ref{ass:kernel_bandwidth},
\ref{ass:kernel_rates}, and \ref{ass:kernel_app_counterfactual}
hold and suppose that $f_W^{1 \triangleright 0}(w) > 0$ on $\cW$.
Let $\alpha \in (0,1)$ be a confidence level
and define $q^{1 \triangleright 0}_{1-\alpha}$ as the quantile
satisfying
%
\begin{align*}
\P\left(
\sup_{w \in \cW}
\left| Z_n^{T,1 \triangleright 0}(w) \right|
\leq q^{1 \triangleright 0}_{1-\alpha}
\right)
&=
1 - \alpha.
\end{align*}
%
Then
%
\begin{align*}
\P\left(
f_W^{1 \triangleright 0}(w)
\in
\left[
\hat f_W^{1 \triangleright 0}(w)
\pm
q^{1 \triangleright 0}_{1-\alpha}
\sqrt{\Sigma_n^{1 \triangleright 0}(w,w)}
\, \right]
\, \textup{for all }
w \in \cW
\right)
\to 1 - \alpha.
\end{align*}
\end{theorem}
%
We propose an estimator for the counterfactual covariance function
$\Sigma_n^{1 \triangleright 0}$. First let
%
\begin{align*}
\hat\kappa(X_i^0, X_i^1, x)
&=
\frac{\I\{X_i^0 = x\} - \hat p_X^0(x)}{\hat p_X^1(x)}
- \frac{\hat p_X^0(x)}{\hat p_X^1(x)}
\frac{\I\{X_i^1 = x\} - \hat p_X^1(x)}{\hat p_X^1(x)},
\end{align*}
%
and define the leave-out conditional expectation estimators
%
\begin{align*}
S_i^{1 \triangleright 0}(w)
&=
\hat\E\left[
k_h(W_{i j}^1,w) \psi(X_j^1) \mid A_i^1
\right] \\
&=
\frac{1}{n-1}
\left(
\sum_{j=1}^{i-1}
k_h(W_{j i}^1,w) \hat\psi(X_j^1)
+ \sum_{j=i+1}^n
k_h(W_{i j}^1,w) \hat\psi(X_j^1)
\right), \\
\tilde S_i^{1 \triangleright 0}(w)
&=
\hat\E\left[
k_h(W_{r j}^1,w) \psi(X_r^1)
\kappa(X_i^0, X_i^1, X_j^1) \mid X_i^0, X_i^1
\right] \\
&=
\frac{1}{n-1}
\sum_{j=1}^n
\I\{j \neq i\}
\hat\kappa(X_i^0, X_i^1, X_j^1)
S_j^{1 \triangleright 0}(w).
\end{align*}
%
Then set
%
\begin{align*}
\hat\Sigma_n^{1 \triangleright 0}(w,w')
&=
\frac{4}{n^2}
\sum_{i=1}^n
\left(
\hat\psi(X_i^1)
S_i^{1 \triangleright 0}(w)
+ \tilde S_i^{1 \triangleright 0}(w)
\right)
\left(
\hat\psi(X_i^1)
S_i^{1 \triangleright 0}(w')
+ \tilde S_i^{1 \triangleright 0}(w')
\right) \\
&\quad-
\frac{4}{n^3(n-1)}
\sum_{i<j}
k_h(W_{i j}^1, w)
k_h(W_{i j}^1, w')
\hat\psi(X_i^1)^2
\hat\psi(X_j^1)^2
- \frac{4}{n}
\hat f_W^{1 \triangleright 0}(w)
\hat f_W^{1 \triangleright 0}(w').
\end{align*}
%
We use a positive semi-definite approximation to
$\hat\Sigma_n^{1 \triangleright 0}$, denoted by
$\hat\Sigma_n^{+, 1 \triangleright 0}$,
and omit the proof of consistency for brevity.
To construct feasible uniform confidence bands,
define a process $\hat Z_n^{T, 1 \triangleright 0}(w)$ which,
conditional on the data $\bW_n^1$, $\bX_n^0$, and $\bX_n^1$
is conditionally mean-zero and conditionally Gaussian, and whose
conditional covariance structure is
%
\begin{align*}
\E\left[
\hat Z_n^{T, 1 \triangleright 0}(w)
\hat Z_n^{T, 1 \triangleright 0}(w')
\bigm| \bW_n^1, \bX_n^0, \bX_n^1 \right]
&=
\frac{\hat \Sigma_n^{+, 1 \triangleright 0}(w,w')}
{\sqrt{\hat \Sigma_n^{+, 1 \triangleright 0}(w,w)
\hat \Sigma_n^{+, 1 \triangleright 0}(w',w')}}.
\end{align*}
%
Let $\alpha \in (0,1)$ be a confidence level and define
$\hat q_{1-\alpha}^{1 \triangleright 0}$
as the conditional quantile satisfying
%
\begin{align*}
\P\left(
\sup_{w \in \cW}
\left| \hat Z_n^{T, 1 \triangleright 0}(w) \right|
\leq \hat q_{1-\alpha}^{1 \triangleright 0}
\Bigm\vert \bW_n^1, \bX_n^0, \bX_n^1
\right)
&=
1 - \alpha.
\end{align*}
%
Then assuming that the covariance estimator is appropriately consistent,
we have that
%
\begin{align*}
\P\left(
f_W^{1 \triangleright 0}(w)
\in
\left[
\hat f_W^{1 \triangleright 0}(w)
\pm
\hat q^{1 \triangleright 0}_{1-\alpha}
\sqrt{\hat\Sigma_n^{+, 1 \triangleright 0}(w,w)}
\,\right]
\,\textup{for all }
w \in \cW
\right)
\to 1 - \alpha.
\end{align*}

\section{Technical lemmas}
\label{sec:kernel_app_technical}

We present some lemmas which provide the technical foundations for our main
results. These lemmas are stated in as much generality as is reasonable,
and may be of independent interest.

\subsection{Maximal inequalities for i.n.i.d.\ empirical processes}

Firstly, we provide a maximal inequality
for empirical processes of
independent but not necessarily identically distributed
(i.n.i.d.)
random variables,
indexed by a class of functions.
This result is an extension
of Theorem~5.2 from \citet{chernozhukov2014gaussian},
which only covers i.i.d.\ random variables,
and is proven in the same manner.
Such a result is useful in the study of dyadic data
because when conditioning on latent variables,
we may encounter
random variables which are conditionally independent
but which do not necessarily follow the same
conditional distribution.

\begin{lemma}[A maximal inequality for i.n.i.d.\ empirical processes]
\label{lem:kernel_app_maximal_entropy}

Let $X_1, \dots, X_n$
be independent but not necessarily identically distributed
(i.n.i.d.)
random variables taking values in a
measurable space $(S,\cS)$.
Denote the joint distribution of
$X_1,\ldots,X_n$ by $\P$
and the marginal distribution of
$X_i$ by $\P_i$,
and let $\bar\P = n^{-1} \sum_i \P_i$.
Let $\cF$ be a class of Borel measurable functions
from $S$ to $\R$
which is pointwise measurable
(i.e.\ it contains a countable subclass which
is dense under pointwise convergence).
Let $F$ be a strictly positive
measurable envelope function for $\cF$
(i.e.\ $|f(s)| \leq |F(s)|$ for all $f \in \cF$
and $s \in S$).
For a distribution $\Q$ and some $q \geq 1$,
define the $(\Q,q)$-norm of $f \in \cF$ as
$\|f\|_{\Q,q}^q = \E_{X \sim \Q}[f(X)^q]$
and suppose
$\|F\|_{\bar\P,2} < \infty$.
For $f \in \cF$
define the empirical process
%
\begin{align*}
G_n(f)
&=
\frac{1}{\sqrt n}
\sum_{i=1}^n
\big(
f(X_i) - \E[f(X_i)]
\big).
\end{align*}
%
Let $\sigma > 0$ satisfy
$\sup_{f \in \cF}
\|f\|_{\bar\P,2}
\leq
\sigma
\leq
\|F\|_{\bar\P,2}$
and
$M = \max_{1 \leq i \leq n} F(X_i)$.
Then with
$\delta = \sigma / \|F\|_{\bar\P,2} \in (0,1]$,
%
\begin{align*}
\E \left[
\sup_{f \in \cF}
\big| G_n(f) \big|
\right]
&\lesssim
\|F\|_{\bar\P,2}
\, J\big(\delta, \cF, F \big)
+
\frac{\|M\|_{\P,2} \, J(\delta, \cF, F)^2}{\delta^2 \sqrt{n}},
\end{align*}
%
where $\lesssim$ is up to a universal constant,
and $J(\delta, \cF, F)$ is the covering integral
%
\begin{align*}
J\big(\delta, \cF, F\big)
&=
\int_0^\delta
\sqrt{1 +
\sup_\Q \log N(\cF, \rho_\Q, \varepsilon \|F\|_{\Q,2})}
\diff{\varepsilon},
\end{align*}
%
with the supremum taken over finite discrete probability
measures $\Q$ on $(S, \cS)$.

\end{lemma}

\begin{lemma}[A VC class maximal inequality for i.n.i.d.\ empirical processes]
\label{lem:kernel_app_maximal_vc_inid}

Assume the same setup as in
Lemma~\ref{lem:kernel_app_maximal_entropy},
and suppose that $\cF$ forms a VC-type class
in that
%
\begin{align*}
\sup_\Q N(\cF, \rho_\Q, \varepsilon \|F\|_{\Q,2})
&\leq
(C_1/\varepsilon)^{C_2}
\end{align*}
%
for all $\varepsilon \in (0,1]$,
for some constants
$C_1 \geq e$
(where $e$ is the standard exponential constant)
and $C_2 \geq 1$.
Then for $\delta \in (0,1]$
we have the covering integral bound
%
$J\big(\delta, \cF, F\big) \leq 3 \delta \sqrt{C_2 \log (C_1/\delta)}$,
%
and so by Lemma~\ref{lem:kernel_app_maximal_entropy},
up to a universal constant,
%
\begin{align*}
\E \left[
\sup_{f \in \cF}
\big| G_n(f) \big|
\right]
&\lesssim
\sigma
\sqrt{C_2 \log (C_1/\delta)}
+
\frac{\|M\|_{\P,2} C_2 \log(C_1/\delta)}{\sqrt{n}} \\
&\lesssim
\sigma
\sqrt{C_2 \log \big(C_1 \|F\|_{\bar\P,2}/\sigma\big)}
+
\frac{\|M\|_{\P,2} C_2 \log \big(C_1 \|F\|_{\bar\P,2}/\sigma\big)}
{\sqrt{n}}.
\end{align*}
%
\end{lemma}

\subsection{Strong approximation results}

Next we provide two strong approximation results.
The first is a corollary of the KMT approximation
\citep{komlos1975approximation}
which applies to bounded-variation functions
of i.i.d.\ variables.
The second is an extension of the Yurinskii coupling
\citep{belloni2019conditional}
which applies to Lipschitz functions
of i.n.i.d.\ variables.

\begin{lemma}[A KMT approximation corollary]
\label{lem:kernel_app_kmt_corollary}

For $n \geq 1$
let $X_1, \ldots, X_n$
be i.i.d.\ real-valued random variables and
$g_n: \R \times \R \to \R$
be a function satisfying
the total variation bound
$\sup_{x \in \R} \|g_n(\cdot, x)\|_\TV < \infty$.
Then on some probability space
there exist independent copies of
$X_1, \ldots, X_n$,
denoted
$X_1', \ldots, X_n'$,
and a mean-zero Gaussian process $Z_n(x)$
such that if we define
the empirical process
%
\begin{align*}
G_n(x)
= \frac{1}{\sqrt n} \sum_{i=1}^n
\Big(g_n(X_i',x) - \E\big[g_n(X_i',x)\big]\Big),
\end{align*}
%
then
for some universal positive constants
$C_1$, $C_2$, and $C_3$,
%
\begin{align*}
\P\left(
\sup_{x \in \R}
\big|G_n(x) - Z_n(x)\big|
> \sup_{x \in \R} \|g_n(\cdot, x)\|_\TV
\, \frac{t + C_1 \log n}{\sqrt n}
\right)
\leq C_2 e^{-C_3 t}.
\end{align*}
%
Further, $Z_n$
has the same covariance structure as $G_n$
in the sense that for all $x,\, x' \in \R$,
%
\begin{align*}
\E\big[Z_n(x) Z_n(x')\big]
= \E\big[G_n(x) G_n(x')\big].
\end{align*}
%
By independently sampling from the law of
$Z_n$ conditional on $X_1', \ldots, X_n'$,
we can assume that
$Z_n$ is a function only of $X_1', \ldots, X_n'$
and some independent random noise.

\end{lemma}

\begin{lemma}[Yurinskii coupling for Lipschitz i.n.i.d.\ empirical processes]
\label{lem:kernel_app_yurinskii_corollary}

For $n \geq 2$ let $X_1, \dots, X_n$
be independent but not necessarily identically distributed
(i.n.i.d.) random variables
taking values in a measurable space $(S, \cS)$
and let $\cX_n \subseteq \R$
be a compact interval
with $\left|\log \Leb(\cX_n)\right| \leq C_1 \log n$
where $C_1 > 0$ is a constant.
Let $g_n$ be measurable on $S \times \cX_n$ satisfying
$\sup_{\xi \in S} \sup_{x \in \cX_n} |g_n(\xi, x)| \leq M_n$
and
$\sup_{x \in \cX_n} \max_{1 \leq i \leq n} \Var[g_n(X_i, x)]
\leq \sigma_n^2$,
with $\left|\log M_n\right| \leq C_1 \log n$
and $\left|\log \sigma_n^2\right| \leq C_1 \log n$.
Suppose that $g_n$ satisfies the following uniform
Lipschitz condition:
%
\begin{align*}
\sup_{\xi \in S}
\sup_{x,x' \in \cX_n}
\left|
\frac{g_n(\xi, x) - g_n(\xi, x')}
{x-x'}
\right|
\leq
l_{n,\infty},
\end{align*}
%
and also the following $L^2$
Lipschitz condition:
%
\begin{align*}
\sup_{x,x' \in \cX_n}
\E\left[
\frac{1}{n}
\sum_{i=1}^n
\left|
\frac{g_n(X_i, x) - g_n(X_i, x')}
{x-x'}
\right|^2
\right]^{1/2}
\leq
l_{n,2},
\end{align*}
%
where $0 < l_{n,2} \leq l_{n,\infty}$,
$\left|\log l_{n,2}\right| \leq C_1 \log n$, and
$\left|\log l_{n,\infty}\right| \leq C_1 \log n$.
Then for any $t_n > 0$ with
$\left|\log t_n\right| \leq C_1 \log n$,
there is a probability space carrying
independent copies of $X_1, \ldots, X_n$ denoted $X_1', \ldots, X_n'$
and a mean-zero Gaussian process $Z_n(x)$
such that if we define the empirical process
%
$G_n(x) = \frac{1}{\sqrt n} \sum_{i=1}^n
\big( g_n(X'_i,x) - \E[g_n(X'_i,x)] \big)$,
%
then
%
\begin{align*}
&\P\left(
\sup_{x \in \cX_n}
\big|
G_n(x) - Z_n(x)
\big|
> t_n
\right) \\
&\quad\leq
\frac{
C_2
\sigma_n
\sqrt{\Leb(\cX_n)}
\sqrt{\log n}
\sqrt{M_n + \sigma_n\sqrt{\log n}}
}{n^{1/4} t_n^2}
\sqrt{
l_{n,2}
\sqrt{\log n}
+ \frac{l_{n,\infty}}{\sqrt n}
\log n}
\end{align*}
%
where $C_2 > 0$ is a constant depending only on $C_1$.
Further, $Z_n$
has the same covariance structure as $G_n$
in the sense that for all $x, x' \in \cX_n$,
%
\begin{align*}
\E\big[Z_n(x) Z_n(x')\big]
= \E\big[G_n(x) G_n(x')\big].
\end{align*}

\end{lemma}

\subsection{The Vorob'ev--Berkes--Philipp theorem}

We present a generalization of the Vorob'ev--Berkes--Philipp theorem
\citep{dudley1999uniform}
which allows one to ``glue'' multiple random variables
or stochastic processes onto the same probability space,
while preserving some pairwise distributions.
We begin with some definitions.

\begin{definition}[Tree]
A \emph{tree} is a finite undirected graph which is connected and contains no
cycles or self-loops.
\end{definition}

\begin{definition}[Polish Borel probability space]
A \emph{Polish Borel probability space}
is a triple $(\cX, \cF, \P)$,
where $\cX$ is a Polish space
(a topological space metrizable by a complete separable metric),
$\cF$ is the Borel $\sigma$-algebra induced on $\cX$ by its topology,
and $\P$ is a probability measure on $(\cX, \cF)$.
Important examples of Polish spaces include $\R^d$ and
the Skorokhod space $\cD[0,1]^d$ for $d \geq 1$.
In particular,
one can consider vectors of real-valued random variables
or stochastic processes indexed by
compact subsets of $\R^d$ which have
almost surely continuous trajectories.
\end{definition}

\begin{definition}[Projection of a law]
Let $(\cX_1, \cF_1)$ and $(\cX_2, \cF_2)$
be measurable spaces, and
let $\P_{12}$ be a law on the
product space
$(\cX_1 \times \cX_2, \cF_1 \otimes \cF_2)$.
The \emph{projection} of $\P_{12}$
onto $\cX_1$ is the law
$\P_1$ defined on $(\cX_1, \cF_1)$
by $\P_1 = \P_{12} \circ \pi_1^{-1}$
where $\pi_1(x_1, x_2) = x_1$
is the first-coordinate projection.
\end{definition}

\begin{lemma}[Vorob'ev--Berkes--Philipp theorem, tree form]
\label{lem:kernel_app_vbp}

Let $\cT$ be a tree with vertex set $\cV = \{1, \ldots, n\}$
and edge set $\cE$.
Suppose that attached to each vertex $i$ is a
Polish Borel probability space
$(\cX_i, \cF_i, \P_i)$.
Suppose that attached to each edge $(i,j) \in \cE$
(where $i<j$ without loss of generality)
is a law $\P_{i j}$ on
$(\cX_i \times \cX_j, \cF_i \otimes \cF_j)$.
Assume that these laws are pairwise-consistent in the sense that
the projection of $\P_{i j}$ onto
$\cX_i$ (resp.\ $\cX_j$) is $\P_i$ (resp.\ $\P_j$)
for each $(i,j) \in \cE$.
Then there exists a law $\P$ on
%
\begin{align*}
\left(
\prod_{i=1}^n \cX_i, \
\bigotimes_{i=1}^n \cF_i
\right)
\end{align*}
%
such that the projection of $\P$
onto $\cX_i \times \cX_j$
is $\P_{i j}$ for each $(i,j) \in \cE$,
and therefore also the projection of $\P$
onto $\cX_i$ is $\P_i$ for each $i \in \cV$.

\end{lemma}

\begin{remark}
The requirement that $\cT$ must contain no cycles
is necessary in general.
To see this, consider the Polish Borel probability spaces
given by
$\cX_1 = \cX_2 = \cX_3 = \{0,1\}$,
their respective Borel $\sigma$-algebras,
and the pairwise-consistent probability measures:
%
\begin{align*}
1/2
&=
\P_1(0) = \P_2(0) = \P_3(0) \\
1/2
&=
\P_{12}(0,1) = \P_{12}(1,0) =
\P_{13}(0,1) = \P_{13}(1,0) =
\P_{23}(0,1) = \P_{23}(1,0).
\end{align*}
%
Each measure $\P_i$ places equal mass on 0 and 1, while $\P_{i j}$
asserts that each pair of realizations is a.s.\ not equal.
The graph of these laws forms a triangle,
which is not a tree.
Suppose that $(X_1,X_2,X_3)$ has distribution given by $\P$,
where $X_i \sim \P_i$ and $(X_i,X_j) \sim \P_{i j}$
for each $i,j$.
But then by definition of $\P_{i j}$ we have
$X_1 = 1-X_2 = X_3 = 1-X_1$ a.s.,
which is a contradiction.

\end{remark}

\begin{remark}

Two important applications of
Lemma~\ref{lem:kernel_app_vbp} include
the embedding of a random vector into a stochastic process
and the coupling of stochastic processes
onto the same probability space:
%
\begin{enumerate}[label=(\roman*)]

\item
Let $X_1$ and $X_2$
be stochastic processes with
trajectories in $\cD[0,1]$.
For $x_1, \ldots, x_n \in [0,1]$
let $\tilde X_1 = (X_1(x_1), \ldots, X_1(x_n))$
be a random vector
and suppose that $\tilde X_1'$
is a copy of $\tilde X_1$.
Then there is a law $\P$ on
$\cD[0,1] \times \R^n \times \cD[0,1]$
such that restriction of $\P$ to
$\cD[0,1] \times \R^n$
is the law of $(X_1, \tilde X_1)$,
while the restriction of $\P$ to
$\R^n \times \cD[0,1]$
is the law of $(\tilde X_1',X_2)$.
In other words,
we can embed the vector $\tilde X_1'$
into a stochastic process $X_1$
while maintaining the joint distribution of
$\tilde X_1'$ and $X_2$.

\item
Let $X_1, X_1', \ldots, X_n, X_n'$
be stochastic processes with
trajectories in $\cD[0,1]$,
where $X_i'$ is a copy of $X_i$
for each $1 \leq i \leq n-1$.
Suppose that
$\P\big(\|X_{i+1} - X_i'\| > t)\leq r_i$
for each $1 \leq i \leq n-1$,
where $\|\cdot\|$ is a norm on $\cD[0,1]$.
Then there exist copies of
$X_1, \ldots, X_n$
denoted
$X_1'', \ldots, X_n''$
satisfying
$\P\big(\|X_{i+1}'' - X_i''\| > t)\leq r_i$
for each $1 \leq i \leq n$.
That is, all of the inequalities
can be satisfied simultaneously
on the same probability space.

\end{enumerate}

\end{remark}

\section{Proofs}
\label{sec:kernel_app_proofs}

We present full proofs of all the results stated in
Chapter~\ref{ch:kernel} and Appendix~\ref{app:kernel}.

\subsection{Preliminary lemmas}

In this section we list some results
in probability and U-statistic theory
which are used in proofs of our main results.
Other auxiliary lemmas will be introduced when
they are needed.

\begin{lemma}[Bernstein's inequality for independent random variables]
\label{lem:kernel_app_bernstein}

Let $X_1, \ldots, X_n$ be independent real-valued
random variables with
$\E[X_i] = 0$, $|X_i| \leq M$, and
$\E[X_i^2] \leq \sigma^2$,
where $M$ and $\sigma$ are non-random.
Then for all $t>0$,
%
\begin{align*}
\P \left(
\left| \frac{1}{n} \sum_{i=1}^n X_i \right| \geq t
\right)
\leq 2 \exp \left( -
\frac{t^2 n}
{2 \sigma^2 + \frac{2}{3} M t}
\right).
\end{align*}

\end{lemma}

\begin{proof}[Lemma~\ref{lem:kernel_app_bernstein}]

See for example
Lemma~2.2.9 in~\citet{van1996weak}.
\end{proof}

\begin{lemma}[The matrix Bernstein inequality]
\label{lem:kernel_app_matrix_bernstein}

For $1 \leq i \leq n$
let $X_i$ be independent symmetric $d \times d$
real random matrices
with expected values $\mu_i = \E[X_i]$.
Suppose that
$\|X_i - \mu_i\|_2 \leq M$
almost surely for all $1 \leq i \leq n$
where $M$ is non-random, and define
$\sigma^2 = \big\| \sum_i \E[(X_i - \mu_i)^2] \big\|_2$.
Then there exists a universal constant $C > 0$
such that
for any $t > 0$ and $q \geq 1$,
%
\begin{align*}
\P\left(
\left\|
\sum_{i=1}^n
\left(
X_i - \mu_i
\right)
\right\|_2
\geq
2 \sigma \sqrt{t}
+ \frac{4}{3} M t
\right)
&\leq
2 d e^{-t}, \\
\E\left[
\left\|
\sum_{i=1}^n
\left(
X_i - \mu_i
\right)
\right\|_2^q
\right]^{1/q}
&\leq
C \sigma \sqrt{q + \log 2d}
+ C M (q + \log 2d).
\end{align*}
%
Another simplified version of this is as follows:
suppose that
$\|X_i\|_2 \leq M$ almost surely,
so that
$\|X_i - \mu_i\|_2 \leq 2M$.
Then since
$\sigma^2 \leq n M^2$,
we have
%
\begin{align*}
\P\left(
\left\|
\sum_{i=1}^n
\left(
X_i - \mu_i
\right)
\right\|_2
\geq
4M \big(t + \sqrt{n t}\big)
\right)
&\leq
2 d e^{-t}, \\
\E\left[
\left\|
\sum_{i=1}^n
\left(
X_i - \mu_i
\right)
\right\|_2^q
\right]^{1/q}
&\leq
C M
\big(q + \log 2d + \sqrt{n(q + \log 2d)}\big).
\end{align*}

\end{lemma}

\begin{proof}[Lemma~\ref{lem:kernel_app_matrix_bernstein}]

See Lemma~3.2 in \citet{minsker2019moment}.
\end{proof}

\begin{lemma}[A maximal inequality for Gaussian vectors]
\label{lem:kernel_app_gaussian_vector_maximal}

Take $n \geq 2$.
Let $X_i \sim \cN(0, \sigma_i^2)$
for $1 \leq i \leq n$
with $\sigma_i^2 \leq \sigma^2$.
Then
%
\begin{align}
\label{eq:kernel_app_gaussian_vector_maximal}
\E\left[
\max_{1 \leq i \leq n}
X_i
\right]
&\leq
\sigma \sqrt{2 \log n}, \\
\label{eq:kernel_app_gaussian_vector_maximal_abs}
\E\left[
\max_{1 \leq i \leq n}
|X_i|
\right]
&\leq
2 \sigma \sqrt{\log n}.
\end{align}
%
If $\Sigma_1$ and $\Sigma_2$ are constant
positive semi-definite $n \times n$ matrices
and $N \sim \cN(0,I_n)$,
then
%
\begin{align}
\label{eq:kernel_app_gaussian_difference_psd}
\E\Big[
\big\|
\Sigma_1^{1/2} N
- \Sigma_2^{1/2} N
\big\|_\infty
\Big]
&\leq
2 \sqrt{\log n} \,
\big\|
\Sigma_1 - \Sigma_2
\big\|_2^{1/2}.
\end{align}
%
If further $\Sigma_1$ is
positive definite,
then
%
\begin{align}
\label{eq:kernel_app_gaussian_difference_pd}
\E\Big[
\big\|
\Sigma_1^{1/2} N
- \Sigma_2^{1/2} N
\big\|_\infty
\Big]
&\leq
\sqrt{\log n} \,
\lambda_{\min}(\Sigma_1)^{-1/2} \,
\big\|
\Sigma_1 - \Sigma_2
\big\|_2.
\end{align}

\end{lemma}

\begin{proof}[Lemma~\ref{lem:kernel_app_gaussian_vector_maximal}]

For $t > 0$,
Jensen's inequality on the concave logarithm function
gives
%
\begin{align*}
\E\left[
\max_{1 \leq i \leq n}
X_i
\right]
&=
\frac{1}{t}
\E\left[
\log
\exp
\max_{1 \leq i \leq n}
t X_i
\right]
\leq
\frac{1}{t}
\log
\E\left[
\exp
\max_{1 \leq i \leq n}
t X_i
\right]
\leq
\frac{1}{t}
\log
\sum_{i=1}^n
\E\left[
\exp
t X_i
\right] \\
&=
\frac{1}{t}
\log
\sum_{i=1}^n
\exp
\left(
\frac{t^2 \sigma_i^2}{2}
\right)
\leq
\frac{1}{t}
\log n
+ \frac{t \sigma^2}{2},
\end{align*}
%
by the Gaussian moment generating function.
Minimizing with $t = \sqrt{2 \log n} / \sigma$
yields \eqref{eq:kernel_app_gaussian_vector_maximal}.
For \eqref{eq:kernel_app_gaussian_vector_maximal_abs},
we use the symmetry of the Gaussian distribution:
%
\begin{align*}
\E\left[
\max_{1 \leq i \leq n}
|X_i|
\right]
&=
\E\left[
\max_{1 \leq i \leq n}
\{X_i, -X_i\}
\right]
\leq
\sigma \sqrt{2 \log 2n}
\leq
2 \sigma \sqrt{\log n}.
\end{align*}
%
For \eqref{eq:kernel_app_gaussian_difference_psd}
and \eqref{eq:kernel_app_gaussian_difference_pd},
note that
$\Sigma_1^{1/2} N - \Sigma_2^{1/2} N$
is Gaussian with covariance matrix
$\big(\Sigma_1^{1/2} - \Sigma_2^{1/2}\big)^2$.
The variances of of its components are the diagonal
elements of this matrix, namely
%
\begin{align*}
\sigma_i^2
&=
\Var\big[
\big(\Sigma_1^{1/2} N - \Sigma_2^{1/2} N\big)_i
\big]
=
\Big(\big(
\Sigma_1^{1/2} - \Sigma_2^{1/2}
\big)^2\Big)_{ii}.
\end{align*}
%
Note that if $e_i$ is the
$i$th standard unit basis vector,
then for any real symmetric matrix $A$,
we have
$e_i^\T A^2 e_i = (A^2)_{ii}$,
so in particular
$(A^2)_{ii} \leq \|A\|_2^2$.
Therefore
%
\begin{align*}
\sigma_i^2
&\leq
\big\|
\Sigma_1^{1/2} - \Sigma_2^{1/2}
\big\|_2^2
=\vcentcolon
\sigma^2.
\end{align*}
%
Applying
\eqref{eq:kernel_app_gaussian_vector_maximal_abs}
then gives
%
\begin{align*}
\E\Big[
\big\|
\Sigma_1^{1/2} N
- \Sigma_2^{1/2} N
\big\|_\infty
\Big]
&\leq
2 \sqrt{\log n} \,
\big\|
\Sigma_1^{1/2} - \Sigma_2^{1/2}
\big\|_2.
\end{align*}
%
By Theorem~X.1.1
in \citet{bhatia1997matrix},
we can deduce
%
\begin{align*}
\big\|
\Sigma_1^{1/2} - \Sigma_2^{1/2}
\big\|_2
&\leq
\big\|
\Sigma_1 - \Sigma_2
\big\|_2^{1/2},
\end{align*}
%
giving
\eqref{eq:kernel_app_gaussian_difference_psd}.
If $\Sigma_1$
is positive definite,
Theorem~X.3.8 in
\citet{bhatia1997matrix} gives
\eqref{eq:kernel_app_gaussian_difference_pd}:
%
\begin{align*}
\big\|
\Sigma_1^{1/2} - \Sigma_2^{1/2}
\big\|_2
&\leq
\frac{1}{2}
\lambda_{\min}(\Sigma_1)^{-1/2} \,
\big\|
\Sigma_1 - \Sigma_2
\big\|_2.
\end{align*}
%
\end{proof}

\begin{lemma}[Maximal inequalities for Gaussian processes]
\label{lem:kernel_app_gaussian_process_maximal}

Let $Z$ be a separable
mean-zero Gaussian process indexed
by $x \in \cX$.
Recall that $Z$ is separable for example if
$\cX$ is Polish and $Z$ has
continuous trajectories.
Define its covariance structure on $\cX \times \cX$
by $\Sigma(x, x') = \E[Z(x) Z(x')]$,
and the corresponding semimetric on $\cX$ by
%
\begin{align*}
\rho(x,x')
&=
\E\big[\big(Z(x) - Z(x')\big)^2\big]^{1/2}
= \big(\Sigma(x,x)
- 2 \Sigma(x,x')
+ \Sigma(x',x')\big)^{1/2}.
\end{align*}
%
Let $N(\varepsilon, \cX, \rho)$
denote the $\varepsilon$-covering number of $\cX$
with respect to the semimetric $\rho$.
Define $\sigma = \sup_x \Sigma(x,x)^{1/2}$.
Then there exists a universal constant $C > 0$
such that for any $\delta > 0$,
%
\begin{align*}
\E\left[
\sup_{x \in \cX}
|Z(x)|
\right]
&\leq
C \sigma
+ C \int_0^{2\sigma}
\sqrt{\log N(\varepsilon, \cX, \rho)}
\diff{\varepsilon}, \\
\E\left[
\sup_{\rho(x,x') \leq \delta}
|Z(x) - Z(x')|
\right]
&\leq
C \int_0^{\delta}
\sqrt{\log N(\varepsilon, \cX, \rho)}
\diff{\varepsilon}.
\end{align*}

\end{lemma}

\begin{proof}[Lemma~\ref{lem:kernel_app_gaussian_process_maximal}]

See Corollary~2.2.8 in \citet{van1996weak},
noting that for any $x,x' \in \cX$, we have
$\E[|Z(x)|] \lesssim \sigma$ and
$\rho(x,x') \leq 2\sigma$,
implying that
$\log N(\varepsilon, \cX, \rho) = 0$
for all
$\varepsilon > 2 \sigma$.
\end{proof}

\begin{lemma}[Anti-concentration for Gaussian process absolute suprema]
\label{lem:kernel_app_anticoncentration}

Let $Z$ be a separable mean-zero Gaussian process
indexed by a semimetric space $\cX$ with
$\E[Z(x)^2] = 1$
for all $x \in \cX$.
Then for any $\varepsilon > 0$,
%
\begin{align*}
\sup_{t \in \R}
\P\left(
\left|
\sup_{x \in \cX}
\big| Z(x) \big|
- t
\right|
\leq \varepsilon
\right)
&\leq
4 \varepsilon
\left(
1 + \E\left[
\sup_{x \in \cX}
\big| Z(x) \big|
\right]
\right).
\end{align*}

\end{lemma}

\begin{proof}[Lemma~\ref{lem:kernel_app_anticoncentration}]

See Corollary~2.1
in \citet{chernozhukov2014anti}.
\end{proof}

\begin{lemma}[No slowest rate of convergence in probability]
\label{lem:kernel_app_slow_convergence}

Let $X_n$ be a sequence of real-valued random
variables with
$X_n = o_\P(1)$.
Then there exists a deterministic sequence
$\varepsilon_n \to 0$
such that
$\P\big(|X_n| > \varepsilon_n\big) \leq \varepsilon_n$
for all $n \geq 1$.

\end{lemma}

\begin{proof}[Lemma~\ref{lem:kernel_app_slow_convergence}]

Define the following deterministic sequence
for $k \geq 1$.
%
\begin{align*}
\tau_k
&=
\sup
\big\{
n \geq 1:
\P\big(|X_n| > 1/k\big)
> 1/k
\big\}
\vee
(\tau_{k-1} +1)
\end{align*}
%
with $\tau_0 = 0$.
Since $X_n = o_\P(1)$,
each $\tau_k$ is finite
and so we can define
$\varepsilon_n = \frac{1}{k}$
where $\tau_k < n \leq \tau_{k+1}$.
Then, noting that $\varepsilon_n \to 0$,
we have
$\P\big(|X_n| > \varepsilon_n\big)
= \P\big(|X_n| > 1/k\big) \leq 1/k = \varepsilon_n$.
\end{proof}

\begin{lemma}[General second-order Hoeffding-type decomposition]
\label{lem:kernel_app_general_hoeffding}

Let $\cU$ be a vector space.
Let $u_{i j} \in \cU$ be defined for
$1 \leq i, j \leq n$
and
$i \neq j$.
Suppose that $u_{i j} = u_{j i}$
for all $i,j$.
Then for any $u_i \in \cU$
(for $1 \leq i \leq n$)
and any $u \in \cU$,
the following decomposition holds:
%
\begin{align*}
\sum_{i=1}^n
\sum_{\substack{j=1 \\ j \neq i}}^n
\big(u_{i j} - u\big)
&=
2(n-1)
\sum_{i=1}^n
\big(u_i - u\big)
+
\sum_{i=1}^n
\sum_{\substack{j=1 \\ j \neq i}}^n
\big(u_{i j} - u_i - u_j + u\big).
\end{align*}

\end{lemma}

\begin{proof}[Lemma~\ref{lem:kernel_app_general_hoeffding}]

We compute the left hand side minus the right hand side,
beginning by observing that all of the
$u_{i j}$ and $u$ terms clearly cancel.
%
\begin{align*}
&\sum_{i=1}^n
\sum_{j \neq i}^n
\big(u_{i j} - u\big)
- 2(n-1)
\sum_{i=1}^n
\big(u_i - u\big)
-
\sum_{i=1}^n
\sum_{j \neq i}
\big(u_{i j} - u_i - u_j + u\big) \\
&\qquad=
- 2(n-1)
\sum_{i=1}^n
u_i
-
\sum_{i=1}^n
\sum_{j \neq i}^n
\big(- u_i - u_j\big)
=
- 2(n-1)
\sum_{i=1}^n
u_i
+
\sum_{i=1}^n
\sum_{j \neq i}^n
u_i
+
\sum_{j=1}^n
\sum_{i \neq j}^n
u_j \\
&\qquad=
- 2(n-1)
\sum_{i=1}^n
u_i
+
(n-1)
\sum_{i=1}^n
u_i
+
(n-1)
\sum_{j=1}^n
u_j
= 0.
\end{align*}
\end{proof}

\begin{lemma}[A U-statistic concentration inequality]
\label{lem:kernel_app_ustat_concentration}

Let $(S,\cS)$ be a measurable space and
$X_1, \ldots, X_n$ be i.i.d.\ $S$-valued random variables.
Let $H: S^m \to \R$ be a function of $m$ variables
satisfying the symmetry property
$H(x_1, \ldots, x_m) = H(x_{\tau (1)}, \ldots, x_{\tau (m)})$
for any $m$-permutation $\tau$.
Suppose also that
$\E[H(X_1, \ldots, X_m)] = 0$.
Let
$M = \|H\|_\infty$
and
$\sigma^2 = \E\big[\E[H(X_1, \ldots, X_m) \mid X_1]^2\big]$.
Define the U-statistic
%
\begin{align*}
U_n
&=
\frac{m!(n-m)!}{n!}
\sum_{1 \leq i_1 < \cdots < i_m \leq n}
H(X_1, \ldots, X_n).
\end{align*}
%
Then for any $t > 0$,
with $C_1(m)$, $C_2(m)$
positive constants depending only on $m$,
%
\begin{align*}
\P\left(
|U_n| > t
\right)
&\leq
4 \exp \left(
- \frac{n t^2}{C_1(m) \sigma^2 + C_2(m) M t}
\right).
\end{align*}
%
\end{lemma}

\begin{proof}[Lemma~\ref{lem:kernel_app_ustat_concentration}]
See Theorem~2 in \citet{arcones1995bernstein}.
\end{proof}

\begin{lemma}[A second-order U-process maximal inequality]
\label{lem:kernel_app_uprocess_maximal}

Let $X_1 \ldots, X_n$
be i.i.d.\ random variables taking values
in a measurable space $(S, \cS)$
with distribution $\P$.
Let $\cF$ be a class of measurable functions from
$S \times S$ to $\R$ which is also pointwise measurable.
Define the degenerate second-order U-process
%
\begin{align*}
U_n(f)
=
\frac{2}{n(n-1)}
\sum_{i<j}
&\Big(
f(X_i, X_j)
- \E\big[f(X_i,X_j) \mid X_i\big]
-
\E\big[f(X_i,X_j) \mid X_j\big]
+ \E\big[f(X_i,X_j)\big]
\Big)
\end{align*}
%
for $f \in \cF$.
Suppose that each $f \in \cF$ is symmetric in the sense that
$f(s_1,s_2) = f(s_2,s_1)$
for all $s_1, s_2 \in S$.
Let $F$ be a measurable envelope function for $\cF$
satisfying $|f(s_1,s_2)| \leq F(s_1,s_2)$
for all $s_1,s_2 \in S$.
For a law $\Q$ on
$(S \times S, \, \cS \otimes \cS)$,
define the $(\Q,q)$-norm of $f \in \cF$ by
$\|f\|_{\Q,q}^q = \E_\Q[|f|^q]$.
Assume that $\cF$ is VC-type in the following manner.
%
\begin{align*}
\sup_\Q
N\big(
\cF, \|\cdot\|_{\Q,2}, \varepsilon \|F\|_{\Q,2}
\big)
&\leq
(C_1/\varepsilon)^{C_2}
\end{align*}
%
for some constants
$C_1 \geq e$
and
$C_2 \geq 1$,
and for all $\varepsilon \in (0,1]$,
where $\Q$ ranges over all finite discrete laws
on
$S \times S$.
Let $\sigma > 0$ be any deterministic value satisfying
$\sup_{f \in \cF} \|f\|_{\P,2} \leq \sigma \leq \|F\|_{\P,2}$,
and define the random variable $M = \max_{i,j} |F(X_i, X_j)|$.
Then there exists a universal constant $C_3 > 0$
satisfying
%
\begin{align*}
n
\E\left[
\sup_{f \in \cF}
\big| U_n(f) \big|
\right]
&\leq
C_3 \sigma
\Big(
C_2 \log\big(C_1 \|F\|_{\P,2} / \sigma \big)
\Big)
+ \frac{C_3 \|M\|_{\P,2}}{\sqrt{n}}
\Big(
C_2 \log\big(C_1 \|F\|_{\P,2} / \sigma \big)
\Big)^2.
\end{align*}

\end{lemma}

\begin{proof}[Lemma~\ref{lem:kernel_app_uprocess_maximal}]

Apply Corollary~5.3
from \citet{chen2020jackknife}
with the order of the U-statistic fixed at
$r=2$,
and with $k=2$.
\end{proof}

\begin{lemma}[A U-statistic matrix concentration inequality]
\label{lem:kernel_app_ustat_matrix_concentration}

Let $X_1, \ldots, X_n$ be i.i.d.\ random variables
taking values in a measurable space $(S, \cS)$.
Suppose
$H: S^2 \to \R^{d \times d}$
is a measurable matrix-valued function
of two variables
satisfying the following:
%
\begin{enumerate}[label=(\roman*)]

\item
$H(X_1, X_2)$ is an almost surely symmetric matrix.

\item
$\|H(X_1, X_2)\|_2 \leq M$ almost surely.

\item
$H$ is a symmetric function in its arguments in that
$H(X_1, X_2) = H(X_2, X_1)$.

\item
$H$ is degenerate in the sense that
$\E[H(X_1, x_2)] = 0$ for all $x_2 \in S$.

\end{enumerate}
%
Let $U_n = \sum_i \sum_{j \neq i} H(X_i, X_j)$
be a U-statistic,
and define the variance-type constant
%
\begin{align*}
\sigma^2
&=
\E\left[
\left\|
\E\left[
H(X_i, X_j)^2
\mid X_j
\right]
\right\|_2
\right].
\end{align*}
%
Then for a universal constant $C > 0$
and for all $t > 0$,
%
\begin{align*}
\P\left(
\|U_n\|_2
\geq
C \sigma n (t + \log d)
+ C M \sqrt{n} (t + \log d)^{3/2}
\right)
&\leq
C e^{-t}.
\end{align*}
%
By Jensen's inequality,
$\sigma^2 \leq \E[ \| H(X_i, X_j)^2 \|_2 ]
= \E[ \| H(X_i, X_j) \|_2^2 ] \leq M^2$, giving the simpler
%
\begin{align*}
\P\left(
\|U_n\|_2
\geq
2 C M n
(t + \log d)^{3/2}
\right)
&\leq
C e^{-t}.
\end{align*}
%
From this last inequality we deduce a moment bound
by integration of tail probabilities:
%
\begin{align*}
\E\left[
\|U_n\|_2
\right]
&\lesssim
M n (\log d)^{3/2}.
\end{align*}

\end{lemma}

\begin{proof}[Lemma~\ref{lem:kernel_app_ustat_matrix_concentration}]

We apply results from \citet{minsker2019moment}.

\proofparagraph{decoupling}

Let $\bar U_n = \sum_{i=1}^n \sum_{j=1}^n H(X_i^{(1)}, X_j^{(2)})$
be a decoupled matrix U-statistic,
where $X^{(1)}$ and $X^{(2)}$
are i.i.d.\ copies of the sequence $X_1, \ldots, X_n$.
By Lemma~5.2 in \citet{minsker2019moment},
since we are only stating this result for
degenerate U-statistics of order 2,
there exists a universal constant $D_2$
such that for any $t > 0$,
we have
%
\begin{align*}
\P\left(
\|U_n\|_2 \geq t
\right)
&\leq
D_2
\P\left(
\|\bar U_n\|_2 \geq t / D_2
\right).
\end{align*}

\proofparagraph{concentration of the decoupled U-statistic}

By Equation~11 in \citet{minsker2019moment},
we have the following concentration inequality
for decoupled degenerate U-statistics.
For some universal constant $C_1$
and for any $t > 0$,
%
\begin{align*}
\P\left(
\|\bar U_n\|_2
\geq
C_1 \sigma n (t + \log d)
+ C_1 M \sqrt{n} (t + \log d)^{3/2}
\right)
&\leq
e^{-t}.
\end{align*}

\proofparagraph{concentration of the original U-statistic}

Hence we have
%
\begin{align*}
&\P\left(
\|U_n\|_2
\geq
C_1 D_2 \sigma n (t + \log d)
+ C_1 D_2 M \sqrt{n} (t + \log d)^{3/2}
\right) \\
&\quad\leq
D_2 \P\left(
\|\bar U_n\|_2
\geq
C_1 \sigma n (t + \log d)
+ C_1 M \sqrt{n} (t + \log d)^{3/2}
\right)
\leq
D_2 e^{-t}.
\end{align*}
%
The main result follows by setting
$C = C_1 + C_1 D_2$.

\proofparagraph{moment bound}

We now obtain a moment bound for the simplified version.
We already have that
%
\begin{align*}
\P\left(
\|U_n\|_2
\geq
2 C M n
(t + \log d)^{3/2}
\right)
&\leq
C e^{-t}.
\end{align*}
%
This implies that for any $t \geq \log d$,
we have
%
\begin{align*}
\P\left(
\|U_n\|_2
\geq
8 C M n
t^{3/2}
\right)
&\leq
C e^{-t}.
\end{align*}
%
Defining
$s = 8 C M n t^{3/2}$
so $t = \left( \frac{s}{8C M n} \right)^{2/3}$
shows that for any $s \geq 8C M n(\log d)^{3/2}$,
%
\begin{align*}
\P\left(
\|U_n\|_2
\geq
s
\right)
&\leq
C e^{-\left( \frac{s}{8C M n} \right)^{2/3}}.
\end{align*}
%
Hence the moment bound is obtained:
%
\begin{align*}
\E\left[
\|U_n\|_2
\right]
&=
\int_0^\infty
\P\left(
\|U_n\|_2
\geq
s
\right)
\diff{s} \\
&=
\int_0^{8C M n(\log d)^{3/2}}
\P\left(
\|U_n\|_2
\geq
s
\right)
\diff{s}
+
\int_{8C M n(\log d)^{3/2}}^\infty
\P\left(
\|U_n\|_2
\geq
s
\right)
\diff{s} \\
&\leq
8C M n(\log d)^{3/2}
+
\int_0^\infty
C e^{-\left( \frac{s}{8C M n} \right)^{2/3}}
\diff{s} \\
&=
8C M n(\log d)^{3/2}
+
8C M n
\int_0^\infty
e^{s^{-2/3}}
\diff{s}
\lesssim
Mn(\log d)^{3/2}.
\end{align*}
\end{proof}

\subsection{Technical lemmas}

Before presenting the proof of
Lemma~\ref{lem:kernel_app_maximal_entropy},
we give some auxiliary lemmas;
namely a symmetrization inequality
(Lemma~\ref{lem:kernel_app_symmetrization}),
a Rademacher contraction principle
(Lemma~\ref{lem:kernel_app_contraction}),
and a Hoffman--J{\o}rgensen inequality
(Lemma~\ref{lem:kernel_app_hoffmann}).
Recall that the Rademacher distribution
places probability mass of $1/2$
on each of the points $-1$ and $1$.

\begin{lemma}[A symmetrization inequality for i.n.i.d.\ variables]
\label{lem:kernel_app_symmetrization}

Let $(S, \cS)$ be a measurable space and
$\cF$ a class of Borel-measurable functions
from $S$ to $\R$ which is pointwise measurable
(i.e.\ it contains a countable dense subset
under pointwise convergence).
Let $X_1, \ldots, X_n$
be independent
but not necessarily identically distributed
$S$-valued random variables.
Let $a_1, \ldots, a_n$ be arbitrary points in $S$
and $\phi$ a non-negative non-decreasing convex function
from $\R$ to $\R$.
Define $\varepsilon_1, \ldots, \varepsilon_n$
as independent Rademacher
random variables,
independent of $X_1, \ldots, X_n$.
Then
%
\begin{align*}
\E \left[
\phi \left(
\sup_{f \in \cF}
\left|
\sum_{i=1}^n
\Big(
f(X_i)
- \E[f(X_i)]
\Big)
\right|
\right)
\right]
&\leq
\E \left[
\phi \left(
2
\sup_{f \in \cF}
\left|
\sum_{i=1}^n
\varepsilon_i
\Big(
f(X_i)
- a_i
\Big)
\right|
\right)
\right].
\end{align*}
%
Note that in particular this holds with $a_i = 0$
and also holds with $\phi(t) = t \vee 0$.

\end{lemma}

\begin{proof}[Lemma~\ref{lem:kernel_app_symmetrization}]

See Lemma~2.3.6 in
\citet{van1996weak}.
%
\end{proof}

\begin{lemma}[A Rademacher contraction principle]
\label{lem:kernel_app_contraction}

Let $\varepsilon_1, \ldots, \varepsilon_n$
be independent Rademacher random variables
and $\cT$ be a bounded subset of $\R^n$.
Define
$M = \sup_{t \in \cT} \max_{1 \leq i \leq n} |t_i|$.
Then, noting that the supremum is measurable
because $\cT$ is a subset of a separable metric space
and is therefore itself separable,
%
\begin{align*}
\E
\left[
\sup_{t \in \cT}
\left|
\sum_{i=1}^n
\varepsilon_i
t_i^2
\right|
\right]
&\leq
4M \,
\E
\left[
\sup_{t \in \cT}
\left|
\sum_{i=1}^n
\varepsilon_i
t_i
\right|
\right].
\end{align*}
%
This gives the following corollary.
Let $X_1, \ldots, X_n$ be mutually independent
and also independent of $\varepsilon_1, \ldots, \varepsilon_n$.
Let $\cF$ be a pointwise measurable class of functions
from a measurable space $(S, \cS)$ to $\R$,
with measurable envelope $F$.
Define $M = \max_i F(X_i)$.
Then we obtain
%
\begin{align*}
\E
\left[
\sup_{f \in \cF}
\left|
\sum_{i=1}^n
\varepsilon_i
f(X_i)^2
\right|
\right]
&\leq
4
\E
\left[
M
\sup_{f \in \cF}
\left|
\sum_{i=1}^n
\varepsilon_i
f(X_i)
\right|
\right].
\end{align*}

\end{lemma}

\begin{proof}[Lemma~\ref{lem:kernel_app_contraction}]

Apply Theorem~4.12 from \citet{ledoux1991probability} with $F$ the identity
function and
%
\begin{align*}
\psi_i(s)
= \psi(s)
&=
\min
\left(
\frac{s^2}{2M},
\frac{M}{2}
\right).
\end{align*}
%
This is a weak contraction
(i.e.\ 1-Lipschitz)
because it is continuous,
differentiable on $(-M,M)$
with derivative bounded by
$|\psi'(s)| \leq |s|/M \leq 1$,
and constant outside $(-M,M)$.
Note that since $|t_i| \leq M$
by definition,
we have $\psi_i(t_i) = t_i^2 / (2M)$.
Hence
by Theorem~4.12
from \citet{ledoux1991probability},
%
\begin{align*}
\E
\left[
F
\left(
\frac{1}{2}
\sup_{t \in \cT}
\left|
\sum_{i=1}^n
\varepsilon_i
\psi_i(t_i)
\right|
\right)
\right]
&\leq
\E
\left[
F
\left(
\sup_{t \in \cT}
\left|
\sum_{i=1}^n
\varepsilon_i
t_i
\right|
\right)
\right], \\
\E
\left[
\frac{1}{2}
\sup_{t \in \cT}
\left|
\sum_{i=1}^n
\varepsilon_i
\frac{t_i^2}{2M}
\right|
\right]
&\leq
\E
\left[
\sup_{t \in \cT}
\left|
\sum_{i=1}^n
\varepsilon_i
t_i
\right|
\right], \\
\E
\left[
\sup_{t \in \cT}
\left|
\sum_{i=1}^n
\varepsilon_i
t_i^2
\right|
\right]
&\leq
4M \,
\E
\left[
\sup_{t \in \cT}
\left|
\sum_{i=1}^n
\varepsilon_i
t_i
\right|
\right].
\end{align*}
%
For the corollary, set
$\cT = \left\{\big(f(X_1), \ldots, f(X_n)\big) : f \in \cF\right\}$.
For a fixed realization
$X_1, \ldots, X_n$,
%
\begin{align*}
\E_\varepsilon
\left[
\sup_{f \in \cF}
\left|
\sum_{i=1}^n
\varepsilon_i
f(X_i)^2
\right|
\right]
&=
\E_\varepsilon
\left[
\sup_{t \in \cT}
\left|
\sum_{i=1}^n
\varepsilon_i
t_i^2
\right|
\right] \\
&\leq 4
\E_\varepsilon
\left[
M
\sup_{t \in \cT}
\left|
\sum_{i=1}^n
\varepsilon_i
t_i
\right|
\right]
= 4 \E_\varepsilon
\left[
M
\sup_{f \in \cF}
\left|
\sum_{i=1}^n
\varepsilon_i
f(X_i)
\right|
\right].
\end{align*}
%
Taking an expectation over $X_1, \ldots, X_n$
and applying Fubini's theorem yields the result.
\end{proof}

\begin{lemma}[A Hoffmann--J{\o}rgensen inequality]
\label{lem:kernel_app_hoffmann}

Let $(S, \cS)$ be a measurable space
and $X_1, \ldots, X_n$
be $S$-valued random variables.
Suppose that
$\cF$ is a pointwise measurable class of functions from $S$ to $\R$
with finite envelope $F$.
Let $\varepsilon_1, \ldots, \varepsilon_n$
be independent Rademacher variables
independent of $X_1, \ldots, X_n$.
For $q \in (1, \infty)$,
%
\begin{align*}
\E \left[
\sup_{f \in \cF}
\left|
\sum_{i=1}^n
\varepsilon_i
f(X_i)
\right|
^q
\right]
^{1/q}
&\leq
C_q
\left(
\E \left[
\sup_{f \in \cF}
\left|
\sum_{i=1}^n
\varepsilon_i
f(X_i)
\right|
\right]
+
\E \left[
\max_{1 \leq i \leq n}
\sup_{f \in \cF}
\big| f(X_i) \big|^q
\right]^{1/q}
\right),
\end{align*}
%
where $C_q$ is a positive constant depending only on $q$.

\end{lemma}

\begin{proof}[Lemma~\ref{lem:kernel_app_hoffmann}]

We use Talagrand's formulation of
a Hoffmann--J{\o}rgensen inequality.
Consider the
independent
$\ell^\infty(\cF)$-valued
random functionals $u_i$ defined by
$u_i(f) = \varepsilon_i f(X_i)$,
where $\ell^\infty(\cF)$
is the Banach space of bounded functions from
$\cF$ to $\R$,
equipped with the norm
$\|u\|_\cF = \sup_{f \in \cF} |u(f)|$.
Then Remark~3.4 in \citet{kwapien1991hypercontraction} gives
%
\begin{align*}
\E \left[
\sup_{f \in \cF}
\left|
\sum_{i=1}^n
u_i(f)
\right|
^q
\right]
^{1/q}
&\leq
C_q
\left(
\E \left[
\sup_{f \in \cF}
\left|
\sum_{i=1}^n
u_i(f)
\right|
\right]
+
\E \left[
\max_{1 \leq i \leq n}
\sup_{f \in \cF}
\left|
u_i(f)
\right|^q
\right]^{1/q}
\right) \\
\E \left[
\sup_{f \in \cF}
\left|
\sum_{i=1}^n
\varepsilon_i
f(X_i)
\right|
^q
\right]
^{1/q}
&\leq
C_q
\left(
\E \left[
\sup_{f \in \cF}
\left|
\sum_{i=1}^n
\varepsilon_i
f(X_i)
\right|
\right]
+
\E \left[
\max_{1 \leq i \leq n}
\sup_{f \in \cF}
\big| f(X_i) \big|^q
\right]^{1/q}
\right).
\end{align*}
\end{proof}

\begin{proof}[Lemma~\ref{lem:kernel_app_maximal_entropy}]

We follow the proof of Theorem~5.2
from \citet{chernozhukov2014gaussian},
using our i.n.i.d.\ versions of the symmetrization inequality
(Lemma~\ref{lem:kernel_app_symmetrization}),
Rademacher contraction principle
(Lemma~\ref{lem:kernel_app_contraction}),
and Hoffmann--J{\o}rgensen inequality
(Lemma~\ref{lem:kernel_app_hoffmann}).

Without loss of generality,
we may assume that $J(1, \cF, F) < \infty$
as otherwise there is nothing to prove,
and that $F > 0$ everywhere on $S$.
Let $\P_n = n^{-1} \sum_i \delta_{X_i}$
be the empirical distribution
of $X_i$,
and define the empirical variance bound
$\sigma_n^2 = \sup_\cF n^{-1} \sum_i f(X_i)^2$.
By the i.n.i.d.\ symmetrization inequality
(Lemma~\ref{lem:kernel_app_symmetrization}),
%
\begin{align*}
\E \left[
\sup_{f \in \cF}
\big| G_n(f) \big|
\right]
&=
\frac{1}{\sqrt n}
\E \left[
\sup_{f \in \cF}
\left|
\sum_{i=1}^n
\Big(
f(X_i)
- \E[f(X_i)]
\Big)
\right|
\right]
\leq
\frac{2}{\sqrt n}
\E \left[
\sup_{f \in \cF}
\left|
\sum_{i=1}^n
\varepsilon_i
f(X_i)
\right|
\right],
\end{align*}
%
where $\varepsilon_1, \ldots, \varepsilon_n$
are independent Rademacher random variables,
independent of $X_1, \ldots, X_n$.
Then the standard entropy integral inequality
from the proof of Theorem~5.2 in
the supplemental materials for
\citet{chernozhukov2014gaussian}
gives for a universal constant $C_1 > 0$,
%
\begin{align*}
\frac{1}{\sqrt n}
\E \left[
\sup_{f \in \cF}
\left|
\sum_{i=1}^n
\varepsilon_i
f(X_i)
\right|
\Bigm\vert
X_1, \ldots, X_n
\right]
&\leq
C_1 \|F\|_{\P_n,2}
\, J(\sigma_n / \|F\|_{\P_n,2}, \cF, F).
\end{align*}
%
Taking marginal expectations
and applying Jensen's inequality along with
a convexity result for the covering integral,
as in Lemma~A.2 in \citet{chernozhukov2014gaussian}, gives
%
\begin{align*}
Z
&\vcentcolon=
\frac{1}{\sqrt n}
\E \left[
\sup_{f \in \cF}
\left|
\sum_{i=1}^n
\varepsilon_i
f(X_i)
\right|
\right]
\leq
C_1 \|F\|_{\bar\P,2}
\, J(\E[\sigma_n^2]^{1/2} / \|F\|_{\bar\P,2}, \cF, F).
\end{align*}
%
Now use symmetrization
(Lemma~\ref{lem:kernel_app_symmetrization}),
the contraction principle
(Lemma~\ref{lem:kernel_app_contraction}),
the Cauchy--Schwarz inequality,
and the Hoffmann--J{\o}rgensen inequality
(Lemma~\ref{lem:kernel_app_hoffmann})
to deduce that
%
\begin{align*}
\E[\sigma_n^2]
&=
\E\left[
\sup_{f \in \cF}
\frac{1}{n}
\sum_{i=1}^n
f(X_i)^2
\right]
\leq
\sup_{f \in \cF}
\E_{\bar\P} \left[
f(X_i)^2
\right]
+ \frac{1}{n}
\E\left[
\sup_{f \in \cF}
\left|
\sum_{i=1}^n
f(X_i)^2
- \E \left[
f(X_i)^2
\right]
\right|
\right] \\
&\leq
\sigma^2
+ \frac{2}{n}
\E\left[
\sup_{f \in \cF}
\left|
\sum_{i=1}^n
\varepsilon_i
f(X_i)^2
\right|
\right]
\leq
\sigma^2
+ \frac{8}{n}
\E\left[
M
\sup_{f \in \cF}
\left|
\sum_{i=1}^n
\varepsilon_i
f(X_i)
\right|
\right] \\
&\leq
\sigma^2
+ \frac{8}{n}
\E\left[
M^2
\right]^{1/2}
\E\left[
\sup_{f \in \cF}
\left|
\sum_{i=1}^n
\varepsilon_i
f(X_i)
\right|^2
\right]^{1/2} \\
&\leq
\sigma^2
+ \frac{8}{n}
\|M\|_{\P,2} \,
C_2
\left(
\E \left[
\sup_{f \in \cF}
\left|
\sum_{i=1}^n
\varepsilon_i
f(X_i)
\right|
\right]
+
\E \left[
\max_{1 \leq i \leq n}
\sup_{f \in \cF}
\big| f(X_i) \big|^2
\right]^{1/2}
\right) \\
&=
\sigma^2
+ \frac{8C_2}{n}
\|M\|_{\P,2} \,
\left(
\sqrt{n} Z
+
\|M\|_{\P,2}
\right)
\lesssim
\sigma^2
+
\frac{\|M\|_{\P,2} Z}{\sqrt n}
+
\frac{\|M\|_{\P,2}^2}{n},
\end{align*}
%
where $\lesssim$ indicates a bound up to a universal constant.
Hence taking a square root we see that,
following the notation from the proof of Theorem~5.2
in the supplemental materials to
\citet{chernozhukov2014gaussian},
%
\begin{align*}
\sqrt{\E[\sigma_n^2]}
&\lesssim
\sigma
+
\|M\|_{\P,2}^{1/2} Z^{1/2} n^{-1/4}
+
\|M\|_{\P,2} n^{-1/2}
\lesssim
\|F\|_{\bar\P,2}
\left( \Delta \vee \sqrt{DZ} \right),
\end{align*}
%
where
$\Delta^2 = \|F\|_{\bar\P,2}^{-2}
\big(\sigma^2 \vee (\|M\|_{\P,2}^2 / n) \big) \geq \delta^2$
and
$D = \|M\|_{\P,2} n^{-1/2} \|F\|_{\bar\P,2}^{-2}$.
Thus returning to our bound on $Z$,
we now have
%
\begin{align*}
Z
&\lesssim
\|F\|_{\bar\P,2}
\, J(\Delta \vee \sqrt{DZ}, \cF, F).
\end{align*}
%
The final steps proceed as
in the proof of Theorem~5.2
from \citet{chernozhukov2014gaussian},
considering cases separately for
$\Delta \geq \sqrt{DZ}$
and
$\Delta < \sqrt{DZ}$,
and applying convexity properties of
the entropy integral $J$.
\end{proof}

\begin{proof}[Lemma~\ref{lem:kernel_app_maximal_vc_inid}]

We assume the VC-type condition
%
$\sup_\Q N(\cF, \rho_\Q, \varepsilon \|F\|_{\Q,2}) \leq
(C_1/\varepsilon)^{C_2}$
%
for all $\varepsilon \in (0,1]$,
with constants
$C_1 \geq e$ and $C_2 \geq 1$.
Hence for $\delta \in (0,1]$,
the entropy integral can be bounded as
%
\begin{align*}
J\big(\delta, \cF, F\big)
&=
\int_0^\delta
\sqrt{1 +
\sup_\Q \log N(\cF, \rho_\Q, \varepsilon \|F\|_{\Q,2})}
\diff{\varepsilon}
\leq
\int_0^\delta
\sqrt{1 +
C_2 \log (C_1/\varepsilon)}
\diff{\varepsilon} \\
&\leq
\int_0^\delta
\left(
1 +
\sqrt{C_2 \log (C_1/\varepsilon)}
\right)
\diff{\varepsilon}
=
\delta
+ \sqrt{C_2}
\int_0^\delta
\sqrt{\log (C_1/\varepsilon)}
\diff{\varepsilon} \\
&\leq
\delta
+ \sqrt{\frac{C_2}{\log (C_1/\delta)}}
\int_0^\delta
\log (C_1/\varepsilon)
\diff{\varepsilon}
=
\delta
+ \sqrt{\frac{C_2}{\log (C_1/\delta)}}
\big(
\delta
+ \delta \log (C_1/\delta)
\big) \\
&\leq
3 \delta
\sqrt{C_2 \log (C_1/\delta)}.
\end{align*}
%
The remaining equations now follow
by Lemma~\ref{lem:kernel_app_maximal_entropy}.
\end{proof}

Before proving Lemma~\ref{lem:kernel_app_kmt_corollary},
we give a bounded-variation characterization
(Lemma~\ref{lem:kernel_app_bv_characterization}).

\begin{lemma}[A characterization of bounded-variation functions]
\label{lem:kernel_app_bv_characterization}

Let $\cV_1$ be
the class of real-valued functions on $[0,1]$
which are 0 at 1 and have total variation bounded by 1.
Also define the class of
half-interval indicator functions $\cI = \{\I[0,t]: t \in [0,1]\}$.
For any topological vector space $\cX$,
define the symmetric convex hull of a subset $\cY \subseteq \cX$ as
%
\begin{align*}
\symconv \cY
&=
\left\{
\sum_{i=1}^n
\lambda_i
y_i :
\sum_{i=1}^n
\lambda_i
= 1, \
\lambda_i
\geq 0, \
y_i \in \cY \cup -\cY, \
n \in \N
\right\}.
\end{align*}
%
Denote its closure by $\overline\symconv \ \cY$.
Under the pointwise convergence topology,
$\cV_1 \subseteq \overline\symconv \ \cI$.

\end{lemma}

\begin{proof}[Lemma~\ref{lem:kernel_app_bv_characterization}]

Firstly, let $\cD \subseteq \cV_1$
be the class of real-valued functions
on $[0,1]$
which are
0 at 1,
have total variation exactly 1,
and are weakly monotone decreasing.
Therefore, for $g \in \cD$, we have
$\|g\|_\TV = g(0) = 1$.
Let $S = \{s_1, s_2, \dots\} \subseteq [0,1]$
be the countable set of discontinuity points of $g$.
We want to find a sequence of
convex combinations of elements of
$\cI$ which converges pointwise to $g$.
To do this, first define the sequence of meshes
%
\begin{align*}
A_n =
\{s_k : 1 \leq k \leq n\}
\cup
\{k/n : 0 \leq k \leq n\},
\end{align*}
%
which satisfies
$\bigcup_n A_n = S \cup ([0,1] \cap \Q)$.
Endow $A_n$ with the ordering
induced by the canonical order on $\R$,
giving $A_n = \{a_1, a_2, \ldots\}$,
and define the sequence of functions
%
\begin{align*}
g_n(x)
= \sum_{k = 1}^{|A_n|-1}
\I[0,a_k]
\big( g(a_k) - g(a_{k+1}) \big),
\end{align*}
%
where clearly
$\I[0, a_k] \in \cI$,
$g(a_k) - g(a_{k+1}) \geq 0$,
and
$\sum_{k = 1}^{|A_n|-1}
\big(
g(a_k) - g(a_{k+1})
\big)
= g(0) - g(1) = 1$.
Therefore $g_n$ is a convex combination of elements of $\cI$.
Further, note that for
$a_k \in A_n$,
%
\begin{align*}
g_n(a_k)
= \sum_{j = k}^{|A_n|-1}
\big( g(a_j) - g(a_{j+1}) \big)
= g(a_k) - g(a_{|A_n|})
= g(a_k) - g(1)
= g(a_k).
\end{align*}
%
Hence if $x \in S$, then eventually $x \in A_n$ so $g_n(x) \to g(x)$.
Alternatively, if $x \not\in S$, then $g$ is continuous at $x$.
But $g_n \to g$ on the dense set $\bigcup_n A_n$,
so also $g_n(x) \to g(x)$.
Hence $g_n \to g$
pointwise on $[0,1]$.

Now take $f \in \cV_1$.
By the Jordan decomposition for
total variation functions
\citep{royden1988real},
we can write
$f = f^+ - f^-$,
with
$f^+$ and $f^-$ weakly decreasing,
$f^+(1) = f^-(1) = 0$,
and
$\|f^+\|_\TV + \|f^-\|_\TV = \|f\|_\TV$.
Supposing that both
$\|f^+\|_\TV$ and $\|f^-\|_\TV$
are strictly positive, let
$g_n^+$ approximate
the unit-variation function
$f^+/\|f^+\|_\TV$
and
$g_n^-$ approximate $f^-/\|f^-\|_\TV$
as above.
Then since trivially
%
\begin{align*}
f =
\|f^+\|_\TV f^+ / \|f^+\|_\TV
- \|f^-\|_\TV f^- / \|f^-\|_\TV
+ \big(1 - \|f^+\|_\TV - \|f^-\|_\TV) \cdot 0,
\end{align*}
%
we have that
the convex combination
%
\begin{align*}
g_n^+ \|f^+\|_\TV
- g_n^- \|f^-\|_\TV
+ \big(1 - \|f^+\|_\TV - \|f^-\|_\TV) \cdot 0
\end{align*}
%
converges pointwise to $f$.
This also holds if either of the total variations
$\|f^\pm\|_\TV$
are zero,
since then the corresponding sequence $g_n^\pm$
need not be defined.
Now note that each of
$g_n^+$, $\,-g_n^-$, and $0$
are in $\symconv \cI$, so
$f \in \overline\symconv \ \cI$
under pointwise convergence.
\end{proof}

\begin{proof}[Lemma~\ref{lem:kernel_app_kmt_corollary}]

We follow the Gaussian approximation method given in
Section~2 of \citet{gine2004kernel}.
The KMT approximation theorem \citep{komlos1975approximation}
asserts the existence
of a probability space
carrying $n$ i.i.d.\ uniform random variables
$\xi_1, \ldots, \xi_n \sim \Unif[0,1]$
and a standard Brownian motion
$B_n(s): s \in [0,1]$
such that if
%
\begin{align*}
\alpha_n(s)
&\vcentcolon=
\frac{1}{\sqrt{n}}
\sum_{i=1}^n
\big(
\I\{\xi_i \leq s\} - s
\big),
&\beta_n(s)
&\vcentcolon=
B_n(s) - s B_n(1),
\end{align*}
%
then
for some universal positive constants
$C_1$, $C_2$, $C_3$,
and for all $t > 0$,
%
\begin{align*}
\P\left(
\sup_{s \in [0,1]}
\big| \alpha_n(s) - \beta_n(s) \big|
> \frac{t + C_1\log n}{\sqrt{n}}
\right)
\leq C_2 e^{-C_3 t}.
\end{align*}
%
We can
view $\alpha_n$ and $\beta_n$ as random functionals
defined on the class of
half-interval indicator functions
$\cI = \big\{\I[0,s]: s \in [0,1]\big\}$
in the following way.
%
\begin{align*}
\alpha_n(\I[0,s])
&= \frac{1}{\sqrt{n}}
\sum_{i=1}^n
\big( \I[0,s](\xi_i) - \E[\I[0,s](\xi_i)]), \\
\beta_n(\I[0,s])
&= \int_0^1 \I[0,s](u) \diff{B_n(u)}
- B_n(1) \int_0^1 \I[0,s](u) \diff{u},
\end{align*}
%
where the integrals are defined as It{\^o} and
Riemann--Stieltjes integrals in
the usual way for stochastic integration against semimartingales
\citep[Chapter~5]{legall2016brownian}.
Now we extend their definitions to the class
$\cV_1$
of functions on $[0,1]$
which are 0 at 1 and have total variation bounded by 1.
This is achieved by
noting that by Lemma~\ref{lem:kernel_app_bv_characterization},
we have
$\cV_1 \subseteq \overline\symconv \ \cI$
where $\overline{\symconv} \ \cI$ is the
smallest
symmetric convex class containing $\cI$
which is closed under pointwise convergence.
Thus by the dominated convergence theorem,
every function in $\cV_1$ is approximated in $L^2$ by finite convex
combinations of functions in $\pm\cI$,
and the extension to $g \in \cV_1$ follows
by linearity and $L^2$ convergence of (stochastic) integrals:
%
\begin{align*}
\alpha_n(g)
&=
\frac{1}{\sqrt{n}}
\sum_{i=1}^n
\big( g(\xi_i) - \E[g(\xi_i)]),
&\beta_n(g)
&= \int_0^1 g(s) \diff{B_n(s)}
- B_n(1) \int_0^1 g(s) \diff{s}.
\end{align*}
%
Now we show that the norm induced on
$(\alpha_n - \beta_n)$
by the function class $\cV_1$ is a.s.\ identical to the
supremum norm.
Writing the sums as integrals and using integration by parts
for finite-variation Lebesgue--Stieltjes and It\^o integrals,
and recalling that $g(1) = \alpha_n(0) = B_n(0) = 0$,
%
\begin{align*}
\sup_{g \in \cV_1}
\big|\alpha_n(g) - \beta_n(g)\big|
&=
\sup_{g \in \cV_1}
\left|
\int_0^1 g(s) \diff{\alpha_n(s)}
- \int_0^1 g(s) \diff{B_n(s)}
+ B_n(1) \int_0^1 g(s) \diff{s}
\right| \\
&=
\sup_{g \in \cV_1}
\left|
\int_0^1 \alpha_n(s) \diff{g(s)}
- \int_0^1 B_n(s) \diff{g(s)}
+ B_n(1) \int_0^1 s \diff{g(s)}
\right| \\
&=
\sup_{g \in \cV_1}
\left|
\int_0^1 \big(\alpha_n(s) - \beta_n(s)\big)
\diff{g(s)}
\right|
= \sup_{s \in [0,1]}
\big|
\alpha_n(s) - \beta_n(s)
\big|,
\end{align*}
%
where in the last line
the upper bound is because $\|g\|_\TV \leq 1$,
and the lower bound is by taking
$g_\varepsilon = \pm \I[0,s_\varepsilon]$ where
$|\alpha_n(s_\varepsilon) - \beta_n(s_\varepsilon)|
\geq \sup_s |\alpha_n(s) - \beta_n(s)| -
\varepsilon$.
Hence we obtain
%
\begin{align}
\label{eq:kernel_app_kmt_concentration}
\P\left(
\sup_{g \in \cV_1}
\big|\alpha_n(g) - \beta_n(g)\big|
> \frac{t + C_1\log n}{\sqrt{n}}
\right)
\leq C_2 e^{-C_3 t}.
\end{align}
%
Now define $V_n = \sup_{x \in \R} \|g_n(\cdot, x)\|_\TV$,
noting that if $V_n = 0$ then the result is trivially true
by setting $Z_n = 0$.
Let $F_X$ be the common c.d.f.\ of $X_i$,
and define the quantile function
$F_X^{-1}(s) = \inf \{u: F_X(u) \geq s\}$ for $s \in [0,1]$,
writing $\inf \emptyset = \infty$
and $\inf \R = -\infty$.
Consider the function class
%
\begin{align*}
\cG_n = \big\{
V_n^{-1} g_n\big(F_X^{-1}(\cdot), x\big)
- V_n^{-1} g_n\big(F_X^{-1}(1), x\big)
: x \in \R \big\},
\end{align*}
%
noting that $g_n(\cdot,x)$
is finite-variation so
$g_n(\pm \infty, x)$
can be interpreted as
the relevant limit.
By monotonicity of $F_X$ and the definition of $V_n$,
the members of $\cG_n$ have total variation of at most $1$
and are 0 at 1, implying that
$\cG_n \subseteq \cV_1$.
Noting that $\alpha_n$ and $\beta_n$ are random
linear operators which a.s.\ annihilate
constant functions,
define
%
\begin{align*}
Z_n(x)
&=
\beta_n \Big(g_n\big(F_X^{-1}(\cdot), x\big)\Big)
= V_n \beta_n \Big(
V_n^{-1} g_n\big(F_X^{-1}(\cdot), x\big)
- V_n^{-1} g_n\big(F_X^{-1}(1), x\big)
\Big),
\end{align*}
%
which is a mean-zero continuous Gaussian process.
Its covariance structure is
%
\begin{align*}
&\E[Z_n(x) Z_n(x')] \\
&=
\E\bigg[
\left(
\int_0^1 g_n\big(F_X^{-1}(s),x\big) \diff{B_n(s)}
- B_n(1) \int_0^1 g_n\big(F_X^{-1}(s),x\big) \diff{s}
\right) \\
&\quad\times
\left(
\int_0^1 g_n\big(F_X^{-1}(s),x'\big) \diff{B_n(s)}
- B_n(1) \int_0^1 g_n\big(F_X^{-1}(s),x'\big) \diff{s}
\right)
\bigg] \\
&=
\E\left[
\int_0^1 g_n\big(F_X^{-1}(s),x\big) \diff{B_n(s)}
\int_0^1 g_n\big(F_X^{-1}(s),x'\big) \diff{B_n(s)}
\right] \\
&\quad- \int_0^1 g_n\big(F_X^{-1}(s),x\big) \diff{s} \
\E\left[
B_n(1) \int_0^1 g_n\big(F_X^{-1}(s),x'\big) \diff{B_n(s)}
\right] \\
&\quad-
\int_0^1 g_n\big(F_X^{-1}(s),x'\big) \diff{s} \
\E\left[
B_n(1) \int_0^1 g_n\big(F_X^{-1}(s),x\big) \diff{B_n(s)}
\right] \\
&\quad+
\int_0^1 g_n\big(F_X^{-1}(s),x\big) \diff{s}
\int_0^1 g_n\big(F_X^{-1}(s),x'\big) \diff{s} \
\E\left[
B_n(1)^2
\right] \\
&=
\int_0^1 g_n\big(F_X^{-1}(s),x\big)
g_n\big(F_X^{-1}(s),x'\big) \diff{s}
- \int_0^1 g_n\big(F_X^{-1}(s),x\big) \diff{s}
\int_0^1 g_n\big(F_X^{-1}(s),x'\big) \diff{s} \\
&=
\E\Big[
g_n\big(F_X^{-1}(\xi_i), x\big)
g_n\big(F_X^{-1}(\xi_i), x'\big)
\Big]
- \E\Big[
g_n\big(F_X^{-1}(\xi_i), x\big)
\Big]
\E\Big[
g_n\big(F_X^{-1}(\xi_i), x'\big)
\Big] \\
&=
\E\Big[
g_n\big(X_i, x\big)
g_n\big(X_i, x'\big)
\Big]
- \E\Big[
g_n\big(X_i, x\big)
\Big]
\E\Big[
g_n\big(X_i, x'\big)
\Big]
=
\E\big[
G_n(x)
G_n(x')
\big]
\end{align*}
%
as desired, by the It\^o isometry for stochastic integrals,
writing $B_n(1) = \int_0^1 \diff{B_n(s)}$;
and noting that $F_X^{-1}(\xi_i)$
has the same distribution as $X_i$.
Finally, note that
%
\begin{align*}
G_n(x)
&=
\alpha_n \Big(g_n\big(F_X^{-1}(\cdot), x\big)\Big)
= V_n \alpha_n \Big(
V_n^{-1} g_n\big(F_X^{-1}(\cdot), x\big)
- V_n^{-1} g_n\big(F_X^{-1}(1), x\big)
\Big),
\end{align*}
%
and so by \eqref{eq:kernel_app_kmt_concentration}
%
\begin{align*}
\P\left(
\sup_{x \in \R}
\Big|G_n(x) - Z_n(x)\Big|
> V_n \frac{t + C_1 \log n}{\sqrt n}
\right)
&\leq
\P\left(
\sup_{g \in \cV_1}
\big|\alpha_n(g) - \beta_n(g)\big|
> \frac{t + C_1\log n}{\sqrt{n}}
\right) \\
&\leq C_2 e^{-C_3 t}.
\end{align*}
\end{proof}

\begin{proof}[Lemma~\ref{lem:kernel_app_yurinskii_corollary}]

Take $0 < \delta_n \leq \Leb(\cX_n)$ and let
$\cX_n^\delta = \big\{ x_1, \dots, x_{|\cX_n^\delta|}\big\}$
be a $\delta_n$-covering of $\cX_n$ with cardinality
$|\cX_n^\delta| \leq \Leb(\cX_n)/\delta_n$.
Suppose that $\left|\log \delta_n\right| \lesssim C_1 \log n$
up to a universal constant.
We first use the Yurinskii coupling to
construct a Gaussian process
$Z_n$
which is close to $G_n$
on this finite cover.
Then we bound the fluctuations in $G_n$
and in $Z_n$
using entropy methods.

\proofparagraph{Yurinskii coupling}

Define the i.n.i.d.\
and mean-zero variables
%
\begin{align*}
h_i(x)
&=
\frac{1}{\sqrt n}
\Big(
g_n(X_i', x)
- \E[g_n(X_i', x)]
\Big),
\end{align*}
%
where $X_1', \ldots, X_n'$
are independent copies of $X_1, \ldots, X_n$
on some new probability space,
so that we have
$G_n(x) = \sum_{i=1}^n h_i(x)$
in distribution.
Also define the length-$|\cX_n^\delta|$ random vector
%
\begin{align*}
h_i^\delta
&=
\big(
h_i(x): x \in \cX_n^\delta
\big).
\end{align*}
%
By an extension of
Yurinskii's coupling
to general norms
\citep[supplemental materials, Lemma~38]{belloni2019conditional},
there exists on the new probability space a
Gaussian length-$|\cX_n^\delta|$ vector $Z_n^\delta$
which is mean-zero
and with the same covariance structure as
$
\sum_{i=1}^n
h_i^\delta
$
satisfying
%
\begin{align*}
\P\left(
\bigg\|
\sum_{i=1}^n
h_i^\delta
- Z_n^\delta
\bigg\|_\infty
> 3 t_n
\right)
\leq
\min_{s > 0}
\left(
2 \P\big( \|N\|_\infty > s)
+ \frac{\beta s^2}{t_n^3}
\right),
\end{align*}
%
where
%
\begin{align*}
\beta
= \sum_{i=1}^n
\Big(
\E\big[\|h_i^\delta\|_2^2 \,
\|h_i^\delta\|_\infty
\big]
+ \E\big[\|z_i\|_2^2 \,
\|z_i\|_\infty
\big]
\Big),
\end{align*}
%
with $z_i \sim \cN(0, \Var[h_i^\delta])$
independent and $N \sim \cN(0, I_{|\cX_n^\delta|})$.
By the bounds on $g_n$,
%
\begin{align*}
\E\big[\|h_i^\delta\|_2^2 \,
\|h_i^\delta\|_\infty \,
\big]
\leq
\frac{M_n}{\sqrt n}
\E\big[\|h_i^\delta\|_2^2 \,
\big]
=
\frac{M_n}{\sqrt n}
\sum_{x \in \cX_n^\delta}
\E\big[h_i(x)^2 \,
\big]
\leq
\frac{M_n}{\sqrt n}
\frac{|\cX_n^\delta| \sigma_n^2}{n}
\leq
\frac{M_n \sigma_n^2 \Leb(\cX_n)}{n^{3/2}\delta_n}.
\end{align*}
%
By the fourth moment bound for Gaussian variables,
%
\begin{align*}
\E\big[
\|z_i\|_2^4 \,
\big]
&\leq
|\cX_n^\delta| \,
\E\big[
\|z_i\|_4^4
\big]
\leq
|\cX_n^\delta|^2 \,
\max_j
\E\big[
(z_i^{(j)})^4
\big]
\leq
3
|\cX_n^\delta|^2 \,
\max_j
\E\big[
(z_i^{(j)})^2
\big]^2 \\
&=
3
|\cX_n^\delta|^2 \,
\max_{x \in \cX_n^\delta}
\E\big[
h_i(x)^2
\big]^2
\leq
\frac{3\sigma_n^4 \Leb(\cX_n)^2}{n^2\delta_n^2} .
\end{align*}
%
Also by Jensen's inequality
and for $|\cX_n^\delta| \geq 2$,
assuming $C_1 > 1$ without loss of generality,
%
\begin{align*}
\E\big[
\|z_i\|_\infty^2
\big]
&\leq
\frac{4 \sigma_n^2}{n}
\log
\E\big[
e^{\|z_i\|_\infty^2 / (4\sigma_n^2/n)}
\big]
\leq
\frac{4 \sigma_n^2}{n}
\log
\E\left[
\sum_{j=1}^{|\cX_n^\delta|}
e^{(z_i^{(j)})^2 / (4\sigma_n^2/n)}
\right]
\leq
\frac{4\sigma_n^2}{n}
\log \big(2|\cX_n^\delta|\big) \\
&\leq
\frac{4\sigma_n^2}{n}
\left(
\log 2 + \log \Leb(\cX_n) - \log \delta_n
\right)
\leq
\frac{12 C_1 \sigma_n^2 \log n}{n},
\end{align*}
%
where we used the moment
generating function of a $\chi_1^2$ random variable.
Therefore we can apply the Cauchy--Schwarz inequality
to obtain
%
\begin{align*}
\E\big[\|z_i\|_2^2 \,
\|z_i\|_\infty
\big]
&\leq
\sqrt{
\E\big[\|z_i\|_2^4
\big]}
\sqrt{
\E\big[
\|z_i\|_\infty^2
\big]}
\leq
\sqrt{
\frac{3\sigma_n^4 \Leb(\cX_n)^2}{n^2\delta_n^2}}
\sqrt{ \frac{12 C_1 \sigma_n^2 \log n}{n} } \\
&\leq
\frac{6\sigma_n^3
\Leb(\cX_n)
\sqrt{C_1 \log n}}{n^{3/2} \delta_n}.
\end{align*}
%
Now summing over the $n$ samples gives
%
\begin{align*}
\beta
\leq
\frac{M_n \sigma_n^2 \Leb(\cX_n)}{\sqrt n \delta_n}
+ \frac{6\sigma_n^3 \Leb(\cX_n) \sqrt{C_1 \log n}}
{\sqrt n \delta_n}
=
\frac{\sigma_n^2 \Leb(\cX_n)}{\sqrt n \delta_n}
\Big(M_n + 6\sigma_n \sqrt{C_1 \log n}\Big).
\end{align*}
%
By a union bound
and Gaussian tail probabilities,
we have that
$\P\big( \|N\|_\infty > s)
\leq 2|\cX_n^\delta| e^{-s^2/2}$.
Thus we get the following Yurinskii coupling inequality
for all $s > 0$:
%
\begin{align*}
\P\left(
\bigg\|
\sum_{i=1}^n
h_i^\delta
- Z_n^\delta
\bigg\|_\infty
> t_n
\right)
&\leq
\frac{4 \Leb(\cX_n)}{\delta_n}
e^{-s^2/2}
+ \frac{\sigma_n^2 \Leb(\cX_n) s^2}{\sqrt n \delta_n t_n^3}
\Big(M_n + 6 \sigma_n \sqrt{C_1 \log n}\Big).
\end{align*}
%
Note that
$Z_n^\delta$
now extends
by the Vorob'ev--Berkes--Philipp theorem
(Lemma~\ref{lem:kernel_app_vbp})
to a mean-zero Gaussian
process
$Z_n$ on the compact interval $\cX_n$
with covariance structure
%
\begin{align*}
\E\big[
Z_n(x)
Z_n(x')
\big]
=
\E\big[
G_n(x)
G_n(x')
\big],
\end{align*}
%
satisfying for any $s' > 0$
%
\begin{align*}
&\P\left(
\sup_{x \in \cX_n^\delta}
\big|
G_n(x) - Z_n(x)
\big|
> t_n
\right)
\leq
\frac{4 \Leb(\cX_n)}{\delta_n}
e^{-s^2/2}
+ \frac{\sigma_n^2 \Leb(\cX_n) s^2}{\sqrt n \delta_n t_n^3}
\Big(M_n + 6 \sigma_n \sqrt{C_1 \log n}\Big).
\end{align*}

\proofparagraph{regularity of $G_n$}

Next we bound the fluctuations in
the empirical process $G_n$.
Consider the following classes of functions on $S$
and their associated (constant) envelope functions.
By continuity of $g_n$,
each class is pointwise measurable
(to see this, restrict the index sets to rationals).
%
\begin{align*}
\cG_n
&=
\big\{
g_n(\cdot, x):
x \in \cX_n
\big\},
&\Env(\cG_n)
&=
M_n, \\
\cG_n^\delta
&=
\big\{
g_n(\cdot, x)
- g_n(\cdot, x'):
x, x' \in \cX_n,
|x-x'| \leq \delta_n
\big\},
&\Env(\cG_n^\delta)
&=
l_{n,\infty} \delta_n.
\end{align*}
%
We first show these are VC-type.
By the uniform Lipschitz assumption,
%
\begin{align*}
\big\|
g_n(\cdot, x)
- g_n(\cdot, x')
\big\|_\infty
&\leq l_{n,\infty} |x-x'|
\end{align*}
%
for all $x,x' \in \cX_n$.
Therefore, with $\Q$ ranging over the
finitely-supported distributions
on $(S, \cS)$,
noting that any $\|\cdot\|_\infty$-cover
is a $\rho_\Q$-cover,
%
\begin{align*}
\sup_\Q
N\big(\cG_n, \rho_\Q, \varepsilon l_{n,\infty} \!\Leb(\cX_n)\big)
&\leq
N\big(\cG_n, \|\cdot\|_\infty,
\varepsilon l_{n,\infty} \!\Leb(\cX_n)\big)
\leq
N\big(\cX_n, |\cdot|, \varepsilon \!\Leb(\cX_n)\big)
\leq
1/\varepsilon.
\end{align*}
%
Replacing $\varepsilon$ by
$\varepsilon M_n/(l_{n,\infty} \Leb(\cX_n))$
gives
%
\begin{align*}
\sup_\Q
N\big(\cG_n, \rho_\Q, \varepsilon M_n \big)
&\leq
\frac{l_{n,\infty} \Leb(\cX_n)}{\varepsilon M_n},
\end{align*}
%
and so $\cG_n$
is a VC-type class.
To see that $\cG_n^\delta$
is also a VC-type class,
we construct a cover in the following way.
Let $\cF_n$ be an $\varepsilon$-cover
for $(\cG_n, \|\cdot\|_\infty)$.
By the triangle inequality,
$\cF_n - \cF_n$ is a $2\varepsilon$-cover
for $(\cG_n - \cG_n, \|\cdot\|_\infty)$
of cardinality at most $|\cF_n|^2$,
where the subtractions are set subtractions.
Since $\cG_n^\delta \subseteq \cG_n - \cG_n$,
we see that $\cF_n - \cF_n$ is a $2\varepsilon$-external cover
for $\cG_n^\delta$. Thus
%
\begin{align*}
\sup_\Q
N\big(\cG_n^\delta, \rho_\Q, \varepsilon l_{n,\infty} \Leb(\cX_n)\big)
&\leq
N\big(\cG_n^\delta, \|\cdot\|_\infty,
\varepsilon l_{n,\infty} \Leb(\cX_n)\big) \\
&\leq
N\big(\cG_n, \|\cdot\|_\infty,
\varepsilon l_{n,\infty} \Leb(\cX_n)\big)^2
\leq
1/\varepsilon^2.
\end{align*}
%
Replacing $\varepsilon$ by
$\varepsilon \delta_n/\Leb(\cX_n)$
gives
%
\begin{align*}
\sup_\Q
N\big(\cG_n^\delta, \rho_\Q, \varepsilon l_{n,\infty} \delta_n \big)
&\leq
\frac{\Leb(\cX_n)^2}{\varepsilon^2 \delta_n^2}
\leq
(C_{1,n}/\varepsilon)^{2}
\end{align*}
%
with $C_{1,n} = \Leb(\cX_n) / \delta_n$,
demonstrating that $\cG_n^\delta$
forms a VC-type class.
We now apply the maximal inequality
for i.n.i.d.\ data
given in
Lemma~\ref{lem:kernel_app_maximal_vc_inid}.
To do this,
note that
$\sup_{\cG_n^\delta} \|g\|_{\bar\P,2}
\leq l_{n,2} \delta_n$
by the $L^2$ Lipschitz condition, and recall
$\Env(\cG_n^\delta) = l_{n,\infty} \delta_n$.
Therefore Lemma~\ref{lem:kernel_app_maximal_vc_inid} with
$\|F\|_{\bar\P,2} = l_{n,\infty} \delta_n$,
$\|M\|_{\P,2} = l_{n,\infty} \delta_n$,
and $\sigma = l_{n,2} \delta_n$
gives,
up to universal constants
%
\begin{align*}
&\E\left[
\sup_{g \in \cG_n^\delta}
\left|
\frac{1}{\sqrt{n}}
\sum_{i=1}^n
\Big(
g(X_i)
- \E[g(X_i)]
\Big)
\right|
\right] \\
&\quad\lesssim
\sigma
\sqrt{2 \log \big(C_{1,n} \|F\|_{\bar\P,2}/\sigma\big)}
+
\frac{\|M\|_{\P,2} 2 \log \big(C_{1,n} \|F\|_{\bar\P,2}/\sigma\big)}
{\sqrt{n}} \\
&\quad\lesssim
l_{n,2} \delta_n
\sqrt{C_1 \log n}
+
\frac{l_{n,\infty} \delta_n}{\sqrt n}
C_1 \log n,
\end{align*}
%
and hence by Markov's inequality,
%
\begin{align*}
&\P\left(
\sup_{|x-x'| \leq \delta_n}
\big|
G_n(x) - G_n(x')
\big|
> t_n
\right) \\
&=
\P\left(
\sup_{|x-x'| \leq \delta_n}
\frac{1}{\sqrt{n}}
\left|
\sum_{i=1}^n
\Big(
g_n(X_i, x) - \E[g_n(X_i, x)]
- g_n(X_i, x') + \E[g_n(X_i, x')]
\Big)
\right|
> t_n
\right) \\
&=
\P\left(
\sup_{g \in \cG_n^\delta}
\left|
\frac{1}{\sqrt{n}}
\sum_{i=1}^n
\Big(
g(X_i) - \E[g(X_i)]
\Big)
\right|
> t_n
\right)
\leq
\frac{1}{t}
\E\left[
\sup_{g \in \cG_n^\delta}
\left|
\frac{1}{\sqrt{n}}
\sum_{i=1}^n
\Big(
g(X_i) - \E[g(X_i)]
\Big)
\right|
\right] \\
&\lesssim
\frac{l_{n,2} \delta_n}{t_n}
\sqrt{C_1 \log n}
+ \frac{l_{n,\infty} \delta_n}{t_n \sqrt n} C_1 \log n.
\end{align*}

\proofparagraph{regularity of $Z_n$}

Next we bound the fluctuations in the Gaussian process
$Z_n$.
Let $\rho$ be the following semimetric:
%
\begin{align*}
\rho(x, x')^2
&=
\E\big[\big( Z_n(x) - Z_n(x') \big)^2\big]
=
\E\big[\big( G_n(x) - G_n(x') \big)^2\big] \\
&=
\frac{1}{n}
\sum_{i=1}^n
\E\big[\big( h_i(x) - h_i(x') \big)^2\big]
\leq
l_{n,2}^2 \, |x - x'|^2.
\end{align*}
%
Hence
$\rho(x, x')
\leq
l_{n,2} \, |x - x'|$.
By
the Gaussian process maximal inequality from
Lemma~\ref{lem:kernel_app_gaussian_process_maximal},
we obtain that
%
\begin{align*}
&\E\bigg[
\sup_{|x - x'| \leq \delta_n}
\big|
Z_n(x) - Z_n(x')
\big|
\bigg]
\lesssim
\E\bigg[
\sup_{\rho(x,x') \leq l_{n,2} \delta_n}
\big|
Z_n(x) - Z_n(x')
\big|
\bigg] \\
&\quad\leq
\int_0^{l_{n,2} \delta_n}
\sqrt{\log N(\varepsilon, \cX_n, \rho)}
\diff{\varepsilon}
\leq
\int_0^{l_{n,2} \delta_n}
\sqrt{\log N(\varepsilon / l_{n,2}, \cX_n, |\cdot|)}
\diff{\varepsilon} \\
&\quad\leq
\int_0^{l_{n,2} \delta_n}
\sqrt{\log \left( 1 + \frac{\Leb(\cX_n) l_{n,2}}{\varepsilon} \right)}
\diff{\varepsilon}
\leq
\int_0^{l_{n,2} \delta_n}
\sqrt{\log \left( \frac{2\Leb(\cX_n) l_{n,2}}{\varepsilon} \right)}
\diff{\varepsilon} \\
&\quad\leq
\log \left(\frac{2\Leb(\cX_n)}{\delta_n} \right)^{-1/2}
\int_0^{l_{n,2} \delta_n}
\log \left( \frac{2\Leb(\cX_n) l_{n,2}}{\varepsilon} \right)
\diff{\varepsilon} \\
&\quad=
\log \left(\frac{2\Leb(\cX_n)}{\delta_n} \right)^{-1/2}
\left(
l_{n,2} \delta_n \log \left( 2 \Leb(\cX_n) l_{n,2} \right)
+ l_{n,2} \delta_n
+ l_{n,2} \delta_n \log \left( \frac{1}{l_{n,2} \delta_n} \right)
\right) \\
&\quad=
\log \left(\frac{2\Leb(\cX_n)}{\delta_n} \right)^{-1/2}
l_{n,2} \delta_n
\left(
1 +
\log \left( \frac{2\Leb(\cX_n)}{\delta_n} \right)
\right)
\lesssim
l_{n,2} \delta_n
\sqrt{\log \left( \frac{\Leb(\cX_n)}{\delta_n} \right)} \\
&\quad\lesssim
l_{n,2} \delta_n
\sqrt{C_1 \log n},
\end{align*}
%
where we used that $\delta_n \leq \Leb(\cX_n)$.
So by Markov's inequality,
%
\begin{align*}
\P\left(
\sup_{|x - x'| \leq \delta_n}
\big|
Z_n(x) - Z_n(x')
\big|
> t_n
\right)
&\lesssim
t_n^{-1}
l_{n,2} \delta_n
\sqrt{C_1 \log n}.
\end{align*}

\proofparagraph{conclusion}

By the results of the previous parts,
we have up to universal constants that
%
\begin{align*}
&\P\left(
\sup_{x \in \cX_n}
\big|
G_n(x) - Z_n(x)
\big|
> t_n
\right) \\
&\quad\leq
\P\left(
\sup_{x \in \cX_n^\delta}
\big|
G_n(x) - Z_n(x)
\big|
> t_n / 3
\right)
+ \P\left(
\sup_{|x-x'| \leq \delta_n}
\big|
G_n(x) - G_n(x')
\big|
> t_n / 3
\right) \\
&\qquad+
\P\left(
\sup_{|x - x'| \leq \delta_n}
\big|
Z_n(x) - Z_n(x')
\big|
> t_n / 3
\right) \\
&\quad\lesssim
\frac{4 \Leb(\cX_n)}{\delta_n}
e^{-s^2/2}
+ \frac{\sigma_n^2 \Leb(\cX_n) s^2}{\sqrt n \delta_n t_n^3}
\Big(M_n + 6 \sigma_n \sqrt{C_1 \log n}\Big) \\
&\qquad+
\frac{l_{n,2} \delta_n}{t_n}
\sqrt{C_1 \log n}
+ \frac{l_{n,\infty} \delta_n}{t_n \sqrt n} C_1 \log n.
\end{align*}
%
Choosing an approximately optimal mesh size of
%
\begin{align*}
\delta_n
&=
\sqrt{
\frac{\sigma_n^2 \Leb(\cX_n) \log n}{\sqrt n t_n^3}
\Big(M_n + \sigma_n \sqrt{\log n}\Big)
} \Bigg/
\sqrt{
t_n^{-1}
l_{n,2}
\sqrt{\log n}
\left(
1 + \frac{l_{n,\infty} \sqrt{\log n}}{l_{n,2} \sqrt{n}}
\right)
}
\end{align*}
%
gives $\log |\delta_n| \lesssim C_1 \log n$ for a universal constant,
so with $s$ a large enough multiple of $\sqrt{\log n}$,
%
\begin{align*}
&\P\left(
\sup_{x \in \cX_n}
\big|
G_n(x) - Z_n(x)
\big|
> t_n
\right) \\
&\quad\lesssim
\frac{4 \Leb(\cX_n)}{\delta_n}
e^{-s^2/2}
+ \frac{\sigma_n^2 \Leb(\cX_n) s^2}{\sqrt n \delta_n t_n^3}
\Big(M_n + 6 \sigma_n \sqrt{C_1 \log n}\Big) \\
&\qquad+
\frac{l_{n,2} \delta_n}{t_n}
\sqrt{C_1 \log n}
+ \frac{l_{n,\infty} \delta_n}{t_n \sqrt n} C_1 \log n \\
&\quad\lesssim
\delta_n
\frac{l_{n,2} \sqrt {\log n}}{t_n}
\left( 1 + \frac{l_{n,\infty} \sqrt{\log n}}{l_{n,2} \sqrt n} \right) \\
&\quad\lesssim
\frac{\sigma_n \sqrt{\Leb(\cX_n)} \sqrt{\log n}
\sqrt{M_n + \sigma_n \sqrt{\log n}}}
{n^{1/4} t_n^2}
\sqrt{l_{n,2} \sqrt {\log n}
+ \frac{l_{n,\infty}}{\sqrt n} \log n}.
\end{align*}
%
\end{proof}

\begin{proof}[Lemma~\ref{lem:kernel_app_vbp}]

The proof is by induction on the number of vertices in the tree.
Let $\cT$ have $n$ vertices,
and suppose that vertex $n$ is a leaf
connected to vertex $n-1$ by an edge,
relabeling the vertices if necessary.
By the induction hypothesis we assume that there is a
probability measure $\P^{(n-1)}$
on $\prod_{i=1}^{n-1} \cX_i$
whose projections onto $\cX_i$ are $\P_i$
and whose projections onto $\cX_i \times \cX_j$ are $\P_{i j}$,
for $i,j \leq n-1$.
Now apply the original
Vorob'ev--Berkes--Philipp theorem,
which can be found as Theorem~1.1.10 in
\citet{dudley1999uniform},
to the spaces
$\prod_{i=1}^{n-2} \cX_i$,\,
$\cX_{n-1}$, and
$\cX_n$;
and to the laws
$\P^{(n-1)}$
and
$\P_{n-1, n}$.
This gives a law $\P^{(n)}$
which agrees with $\P_i$
at every vertex by definition,
and agrees with
$\P_{i j}$ for all $i,j \leq n-1$.
It also agrees with $\P_{n-1,n}$,
and this is the only edge touching vertex $n$.
Hence $\P^{(n)}$ satisfies the desired properties.
\end{proof}

\subsection{Main results}
\label{sec:kernel_app_main}

We give supplementary details for our main results on consistency, minimax
optimality, strong approximation, covariance estimation, feasible inference and
counterfactual estimation.
We begin with a basic fact about Lipschitz functions.

\begin{lemma}[Lipschitz kernels are bounded]
\label{lem:kernel_app_lipschitz_kernels_bounded}

Let $\cX \subseteq \R$ be a connected set.
Let $f: \cX \to \R$ satisfy the Lipschitz condition
$|f(x) - f(x')| \leq C |x-x'|$ for some $C > 0$
and all $x, x' \in \cX$.
Suppose also that $f$ is a kernel in the sense that
$\int_\cX f(x) \diff{x} = 1$.
Then we have
%
\begin{align*}
\sup_{x \in \cX} |f(x)|
&\leq
C \Leb(\cX) + \frac{1}{\Leb(\cX)}.
\end{align*}
%
Now let $g: \cX \to [0,\infty)$ satisfy
$|g(x) - g(x')| \leq C |x-x'|$ for some $C > 0$
and all $x, x' \in \cX$.
Suppose $g$ is a sub-kernel with
$\int_\cX g(x) \diff{x} \leq 1$.
Then for any $M \in \big(0, \Leb(\cX)\big]$,
we have
%
\begin{align*}
\sup_{x \in \cX} f(x)
&\leq
C M + \frac{1}{M}.
\end{align*}

\end{lemma}

Applying Lemma~\ref{lem:kernel_app_lipschitz_kernels_bounded}
to the density and kernel functions defined in
Assumptions~\ref{ass:kernel_data} and~\ref{ass:kernel_bandwidth}
yields the following.
Firstly, since $k_h(\cdot, w)$ is $C_\rL / h^2$-Lipschitz
on $[w \pm h] \cap \cW$ and integrates to one,
we have by the first inequality in
Lemma~\ref{lem:kernel_app_lipschitz_kernels_bounded} that
%
\begin{align*}
|k_h(s,w)|
&\leq \frac{2 C_\rL + 1}{h} + \frac{1}{\Leb(\cW)}.
\end{align*}
%
Since each of
$f_{W \mid AA}(\cdot \mid a,a')$,
$f_{W \mid A}(\cdot \mid a)$, and
$f_W$ is non-negative, and $C_\rH$-Lipschitz on $\cW$
and integrates to at most one over $\cW$,
taking $M = \frac{1}{\sqrt{C_\rH}} \wedge \Leb(\cW)$
in the second inequality in
Lemma~\ref{lem:kernel_app_lipschitz_kernels_bounded}
gives
%
\begin{align*}
f_{W \mid AA}(w \mid a,a')
&\leq 2 \sqrt{C_\rH} + \frac{1}{\Leb(\cW)}, \\
f_{W \mid A}(w \mid a)
&\leq 2 \sqrt{C_\rH} + \frac{1}{\Leb(\cW)}, \\
f_W(w)
&\leq 2 \sqrt{C_\rH} + \frac{1}{\Leb(\cW)}.
\end{align*}

\begin{proof}[Lemma~\ref{lem:kernel_app_lipschitz_kernels_bounded}]

We begin with the first inequality.
Note that if $\Leb(\cX) = \infty$ there is nothing to prove.
Suppose for contradiction that
$|f(x)| > C \Leb(\cX) + \frac{1}{\Leb(\cX)}$
for some $x \in \cX$.
If $f(x) \geq 0$
then by the Lipschitz property, for any $y \in \cX$,
%
\begin{align*}
f(y)
\geq f(x) - C|y-x|
> C \Leb(\cX) + \frac{1}{\Leb(\cX)} - C\Leb(\cX)
= \frac{1}{\Leb(\cX)}.
\end{align*}
%
Similarly, if $f(x) \leq 0$ then
%
\begin{align*}
f(y)
\leq f(x) + C|y-x|
< - C \Leb(\cX) - \frac{1}{\Leb(\cX)} + C\Leb(\cX)
= -\frac{1}{\Leb(\cX)}.
\end{align*}
%
But then either
$\int_\cX f(x) \diff{x} > \int_\cX 1/\Leb(\cX) \diff{x} = 1$
or
$\int_\cX f(x) \diff{x} < \int_\cX -1/\Leb(\cX) \diff{x} = -1 < 1$,
giving a contradiction.

For the second inequality,
assume that $f$ is non-negative on $\cX$,
and take $M \in \big(0, \Leb(\cX)\big]$.
Suppose for contradiction that
$f(x) > C M + \frac{1}{M}$
for some $x \in \cX$.
Then by the Lipschitz property, $f(y) \geq 1/M$
for all $y$ such that $|y - x| \leq M$.
Since $\cX$ is connected, we have
$\Leb(\cX \cap [x \pm M]) \geq M$
and so we deduce that
$\int_\cX f(x) \diff{x} > M/M = 1$
which is a contradiction.
\end{proof}

\begin{proof}[Theorem~\ref{thm:kernel_bias}]

Begin by defining
%
\begin{align*}
P_p(s,w)
&=
\sum_{r = 0}^p
\frac{f_W^{(r)}(w)}{r!}
{(s-w)^r}
\end{align*}
%
for $s, w \in \cW$
as the degree-$p$ Taylor polynomial of $f_W$,
centered at $w$ and evaluated at $s$.
Note that
for $p \leq \flbeta-1$,
by Taylor's theorem with Lagrange remainder,
%
\begin{align*}
f_W(s) - P_p(s,w)
&=
\frac{f_W^{(p+1)}(w')}{(p+1)!}
(s-w)^{p+1}
\end{align*}
%
for some $w'$ between $w$ and $s$.
Also note that for any $p$,
%
\begin{align*}
\int_{\cW}
k_h(s,w)
\big(
P_p(s,w)
- P_{p-1}(s,w)
\big)
\diff{s}
&=
\int_{\cW}
k_h(s,w)
\frac{f_W^{(p)}(w)}{p!}
(s-w)^p
\diff{s}
= h^p b_p(w).
\end{align*}
%
Further, by the order of the kernel,
%
\begin{align*}
\E\big[\hat f_W(w)\big]
- f_W(w)
&=
\int_{\cW}
k_h(s,w)
f_W(s)
\diff{s}
- f_W(w)
=
\int_{\cW}
k_h(s,w)
\big(f_W(s) - f_W(w)\big)
\diff{s} \\
&=
\int_{\cW}
k_h(s,w)
\big(f_W(s) - P_{p-1}(s,w)\big)
\diff{s}.
\end{align*}

\proofparagraph{low-order kernel}
Suppose that $p \leq \flbeta - 1$. Then
%
\begin{align*}
&\sup_{w \in \cW}
\big|
\E[\hat f_W(w)]
- f_W(w)
- h^p b_p(w)
\big| \\
&\quad=
\sup_{w \in \cW}
\left|
\int_{\cW}
k_h(s,w)
\big(f_W(s) - P_{p-1}(s,w)\big)
\diff{s}
- h^p b_p(w)
\right| \\
&\quad=
\sup_{w \in \cW}
\left|
\int_{\cW}
k_h(s,w)
\big(
f_W(s) - P_{p}(s,w)
+ P_{p}(s,w) - P_{p-1}(s,w)
\big)
\diff{s}
- h^p b_p(w)
\right| \\
&\quad=
\sup_{w \in \cW}
\left|
\int_{\cW}
k_h(s,w)
\big(
f_W(s) - P_{p}(s,w)
\big)
\diff{s}
\right|
= \sup_{w \in \cW}
\left|
\int_{\cW}
k_h(s,w)
\frac{f_W^{(p+1)}(w')}{(p+1)!}
(s-w)^{p+1}
\diff{s}
\right| \\
&\quad\leq
\sup_{w \in \cW}
\left|
\int_{[w \pm h]}
\frac{C_\rk}{h}
\frac{C_\rH}{(p+1)!}
h^{p+1}
\diff{s}
\right|
\leq
\frac{2C_\rk C_\rH}{(p+1)!}
h^{p+1}.
\end{align*}

\proofparagraph{order of kernel matches smoothness}
Suppose that $p = \flbeta$.
Then
%
\begin{align*}
&\sup_{w \in \cW}
\big|
\E[\hat f_W(w)]
- f_W(w)
- h^p b_p(w)
\big| \\
&\quad=
\sup_{w \in \cW}
\left|
\int_{\cW}
k_h(s,w)
\big(f_W(s) - P_{\flbeta - 1}(s,w)\big)
\diff{s}
- h^p b_p(w)
\right| \\
&\quad=
\sup_{w \in \cW}
\left|
\int_{\cW}
k_h(s,w)
\big(
f_W(s) - P_{\flbeta}(s,w)
+ P_{\flbeta}(s,w) - P_{\flbeta - 1}(s,w)
\big)
\diff{s}
- h^{\flbeta} b_{\flbeta}(w)
\right| \\
&\quad=
\sup_{w \in \cW}
\left|
\int_{\cW}
k_h(s,w)
\big(
f_W(s) - P_{\flbeta}(s,w)
\big)
\diff{s}
\right| \\
&\quad=
\sup_{w \in \cW}
\left|
\int_{\cW}
k_h(s,w)
\frac{f_W^{(\flbeta)}(w') - f_W^{(\flbeta)}(w)}{\flbeta!}
(s-w)^{\flbeta}
\diff{s}
\right| \\
&\quad\leq
\sup_{w \in \cW}
\left|
\int_{[w \pm h]}
\frac{C_\rk}{h}
\frac{C_\rH h^{\beta - \flbeta}}{\flbeta !}
h^{\flbeta}
\diff{s}
\right|
\leq
\frac{2 C_\rk C_\rH}{\flbeta !}
h^\beta.
\end{align*}

\proofparagraph{high-order kernel}
Suppose that $p \geq \flbeta+1$.
Then as in the previous part
%
\begin{align*}
\sup_{w \in \cW}
\big|
\E[\hat f_W(w)]
- f_W(w)
\big|
&=
\sup_{w \in \cW}
\left|
\int_{[w \pm h] \cap \cW}
\!\!\!\! k_h(s,w)
\big(
f_W(s) - P_{\flbeta}(s,w)
\big)
\diff{s}
\right|
\leq
\frac{2 C_\rk C_\rH}{\flbeta !}
h^\beta.
\end{align*}
\end{proof}

\begin{proof}[Lemma~\ref{lem:kernel_hoeffding}]

\proofparagraph{Hoeffding-type decomposition}

\begin{align*}
\hat f_W(w)
- E_n(w)
- \E[\hat f_W(w)]
&=
\frac{2}{n(n-1)}
\sum_{i=1}^{n-1}
\sum_{j=i+1}^{n}
\Big(
\E[k_h(W_{i j},w) \mid A_i, A_j]
- \E[k_h(W_{i j},w)]
\Big) \\
&=
\frac{1}{n(n-1)}
\sum_{i=1}^{n-1}
\sum_{j \neq i}
\Big(
\E[k_h(W_{i j},w) \mid A_i, A_j]
- \E[k_h(W_{i j},w)]
\Big),
\end{align*}
%
and apply Lemma~\ref{lem:kernel_app_general_hoeffding} with
%
\begin{align*}
u_{i j}
&=
\frac{1}{n(n-1)}
\E\big[k_h(W_{i j},w) \mid A_i, A_j\big],
&u_i
&=
\frac{1}{n(n-1)}
\E\big[k_h(W_{i j},w) \mid A_i\big], \\
u
&=
\frac{1}{n(n-1)}
\E\big[k_h(W_{i j},w)\big],
\end{align*}
%
to see
%
\begin{align*}
\hat f_W(w)
- E_n(w)
- \E[\hat f_W(w)]
&=
\frac{2}{n}
\sum_{i=1}^n
\big(u_i - u\big)
+ \frac{1}{n(n-1)}
\sum_{i=1}^n
\sum_{j \neq i}
\big(
u_{i j} - u_i - u_j + u
\big) \\
&=
\frac{2}{n}
\sum_{i=1}^n
l_i(w)
+ \frac{2}{n(n-1)}
\sum_{i=1}^n
\sum_{j = i+1}^n
q_{i j}(w)
=
L_n + Q_n.
\end{align*}

\proofparagraph{expectation and covariance of $L_n$, $Q_n$, and $E_n$}

$L_n$, $Q_n$, and $E_n$
are clearly mean-zero.
For orthogonality,
note that their summands
have the following properties,
for any $1 \leq i < j \leq n$
and $1 \leq r < s \leq n$,
and for any $w, w' \in \cW$:
%
\begin{align*}
\E\big[
l_i(w)
q_{rs}(w')
\big]
&=
\E\big[
l_i(w)
\E\big[
q_{rs}(w') \mid A_i
\big]
\big]
= 0, \\
\E\big[
l_i(w)
e_{rs}(w')
\big]
&=
\begin{cases}
\E\big[
l_i(w)
\big]
\E\big[
e_{rs}(w')
\big],
\text{ if } i \notin \{r,s\}, \\
\E\big[
l_i(w)
\E\big[
e_{rs}(w') \mid A_r, A_s
\big]
\big],
\text{ if } i \in \{r,s\},
\end{cases} \\
&=
0, \\
\E\big[
q_{i j}(w)
e_{rs}(w')
\big]
&=
\begin{cases}
\E\big[
q_{i j}(w)
\big]
\E\big[
e_{rs}(w')
\big],
\text{ if } \{i,j\} \cap \{r,s\} = \emptyset, \\
\E\big[
\E\big[
q_{i j}(w) \mid A_i
\big]
\E\big[
e_{rs}(w') \mid A_i
\big]
\big],
\text{ if } \{i,j\} \cap \{r,s\} = \{i\}, \\
\E\big[
\E\big[
q_{i j}(w) \mid A_j
\big]
\E\big[
e_{rs}(w') \mid A_j
\big]
\big],
\text{ if } \{i,j\} \cap \{r,s\} = \{j\}, \\
\E\big[
q_{i j}(w)
\E\big[
e_{rs}(w') \mid A_r, A_s
\big]
\big],
\text{ if } \{i,j\} = \{r,s\},
\end{cases} \\
&=
0,
\end{align*}
%
by independence of $\bA_n$ and $\bV_n$
and as $\E[q_{rs}(w) \mid A_i] = 0$
and $\E[e_{i j}(w) \mid A_i, A_j] = 0$.
\end{proof}

\begin{proof}[Lemma~\ref{lem:kernel_trichotomy}]

\proofparagraph{total degeneracy}

Suppose
$\Dl = 0$, so
$\Var[f_{W \mid A}(w \mid A_i)] = 0$
for all $w \in \cW$.
Therefore, for all $w \in \cW$,
we have $f_{W \mid A}(w) = f_W(w)$ almost surely.
By taking a union over $\cW \cap \Q$
and by continuity of $f_{W \mid A}$ and $f_W$,
this implies that $f_{W \mid A}(w) = f_W(w)$
for all $w \in \cW$
almost surely. Thus
%
\begin{align*}
\E\left[
k_h(W_{i j},w) \mid A_i
\right]
&=
\int_{\cW}
k_h(s,w)
f_{W \mid A}(s \mid A_i)
\diff{s}
=
\int_{\cW}
k_h(s,w)
f_W(s)
\diff{s}
=
\E\left[
k_h(W_{i j},w)
\right]
\end{align*}
%
for all $w \in \cW$ almost surely.
Hence $l_i(w) = 0$ and so $L_n(w) = 0$
for all $w \in \cW$ almost surely.

\proofparagraph{no degeneracy}

Suppose $\Dl > 0$.
As $f_{W|A}(\cdot \mid a)$ is $C_\rH$-Lipschitz
for all $a \in \cA$ and since $|k_h| \leq C_\rk/h$,
%
\begin{align*}
&\sup_{w \in \cW}
\left|
\E[k_h(W_{i j},w) \mid A_i]
- f_{W \mid A}(w \mid A_i)
\right| \\
&\quad=
\sup_{w \in \cW}
\left|
\int_{\cW}
k_h(s,w)
f_{W \mid A}(s \mid A_i)
\diff{s}
- f_{W \mid A}(w \mid A_i)
\right| \\
&\quad=
\sup_{w \in \cW}
\left|
\int_{\cW \cap [w \pm h]}
k_h(s,w)
\left(
f_{W \mid A}(s \mid A_i)
- f_{W \mid A}(w \mid A_i)
\right)
\diff{s}
\right| \\
&\quad\leq
2h
\frac{C_\rk}{h}
C_\rH h
\leq
2 C_\rk C_\rH h
\end{align*}
%
almost surely.
Therefore, since $f_{W \mid A}(w \mid a) \leq C_\rd$,
we have
%
\begin{align*}
\sup_{w \in \cW}
\left|
\Var\big[
\E[k_h(W_{i j},w) \mid A_i]
\big]
- \Var\left[
f_{W \mid A}(w \mid A_i)
\right]
\right|
&\leq
16 C_\rk C_\rH C_\rd h
\end{align*}
%
whenever $h$ is small enough that
$2 C_\rk C_\rH h \leq C_\rd$. Thus
%
\begin{align*}
\inf_{w \in \cW} \Var\big[\E[k_h(W_{i j},w) \mid A_i]\big]
&\geq
\inf_{w \in \cW}\Var[f_{W \mid A}(w \mid A_i)]
- 16 C_\rk C_\rH C_\rd h.
\end{align*}
%
Therefore, if $\Dl > 0$, then eventually
$\inf_{w \in \cW} \Var\big[\E[k_h(W_{i j},w) \mid A_i]\big] \geq \Dl/2$.
Finally,
%
\begin{align*}
\inf_{w \in \cW}\Var[L_n(w)]
&=
\frac{4}{n}
\inf_{w \in \cW}
\Var\big[\E[k_h(W_{i j},w) \mid A_i]\big]
\geq
\frac{2 \Dl}{n}.
\end{align*}

\proofparagraph{partial degeneracy}

Since $f_{W \mid A}(w \mid A_i)$
is bounded by $C_\rd$ and $C_\rH$-Lipschitz in $w$,
we have that
$\Var[f_{W \mid A}(w \mid A_i)]$
is continuous on $\cW$.
Thus if $\Dl = 0$,
there is at least one point $w \in \cW$
for which
$\Var[f_{W \mid A}(w \mid A_i)] = 0$
by compactness.
Let $w$ be any such degenerate point.
Then by the previous part,
%
\begin{align*}
\Var[L_n(w)] =
\frac{4}{n} \Var\big[\E[k_h(W_{i j},w) \mid A_i]\big]
&\leq
64 C_\rk C_\rH C_\rd \frac{h}{n}.
\end{align*}
%
If conversely $w$ is not a degenerate point
then
$\Var[f_{W \mid A}(w \mid A_i)] > 0$
so eventually
%
\begin{align*}
\Var[L_n(w)]
= \frac{4}{n}
\Var\big[\E[k_h(W_{i j},w) \mid A_i]\big]
&\geq
\frac{2}{n}
\Var[f_{W \mid A}(w \mid A_i)].
\end{align*}
\end{proof}

\begin{proof}[Lemma~\ref{lem:kernel_uniform_concentration}]

We establish VC-type properties of function
classes and apply empirical process theory.

\proofparagraph{establishing VC-type classes}

Consider the following function classes:
%
\begin{align*}
\cF_1
&=
\Big\{
W_{i j} \mapsto
k_h(W_{i j},w)
: w \in \cW
\Big\}, \\
\cF_2
&=
\Big\{
(A_i, A_j) \mapsto
\E\big[ k_h(W_{i j},w) \mid A_i, A_j \big]
: w \in \cW
\Big\}, \\
\cF_3
&=
\Big\{
A_i \mapsto
\E\big[ k_h(W_{i j},w) \mid A_i \big]
: w \in \cW
\Big\}.
\end{align*}
%
For $\cF_1$, take $0 < \varepsilon \leq \Leb(\cW)$
and $\cW_\varepsilon$ an $\varepsilon$-cover of $\cW$
of cardinality at most $\Leb(\cW)/\varepsilon$. As
%
\begin{align*}
\sup_{s, w, w' \in \cW}
\left|
\frac{k_h(s,w) - k_h(s,w')}
{w-w'}
\right|
&\leq
\frac{C_\mathrm{L}}{h^2}
\end{align*}
%
almost surely,
we see that
%
\begin{align*}
\sup_\Q
N\left(\cF_1, \rho_\Q,
\frac{C_\mathrm{L}}{h^2} \varepsilon \right)
&\leq
N\left(\cF_1, \|\cdot\|_\infty,
\frac{C_\mathrm{L}}{h^2} \varepsilon \right)
\leq
\frac{\Leb(\cW)}{\varepsilon},
\end{align*}
%
where $\Q$ ranges over Borel
probability measures on $\cW$.
Since
$\frac{C_\rk}{h}$
is an envelope for $\cF_1$,
%
\begin{align*}
\sup_\Q
N\left(\cF_1, \rho_\Q,
\frac{C_\rk}{h} \varepsilon \right)
&\leq
\frac{C_\rL}{C_\rk}
\frac{\Leb(\cW)}{h \varepsilon}.
\end{align*}
%
Thus for all $\varepsilon \in (0,1]$,
%
\begin{align*}
\sup_\Q
N\left(\cF_1, \rho_\Q,
\frac{C_\rk}{h} \varepsilon \right)
&\leq
\frac{C_\rL}{C_\rk}
\frac{\Leb(\cW) \vee 1}{h \varepsilon}
\leq
(C_1/(h\varepsilon))^{C_2},
\end{align*}
%
where
$C_1 = \frac{C_\rL}{C_\rk} (\Leb(\cW) \vee 1)$
and $C_2 = 1$.
Next, $\cF_2$ forms a smoothly parameterized class of functions
since for $w,w' \in \cW$ we have
by the uniform Lipschitz properties of
$f_{W \mid AA}(\cdot \mid A_i, A_j)$ and
$k_h(s, \cdot)$,
with $|w-w'| \leq h$,
%
\begin{align*}
&\left|
\E\big[ k_h(W_{i j},w) \mid A_i, A_j \big]
- \E\big[ k_h(W_{i j},w') \mid A_i, A_j \big]
\right| \\
&\quad=
\left|
\int_{[w \pm h] \cap \cW}
k_h(s,w)
f_{W \mid AA}(s \mid A_i, A_j)
\diff{s}
- \int_{[w' \pm h] \cap \cW}
k_h(s,w')
f_{W \mid AA}(s \mid A_i, A_j)
\diff{s}
\right| \\
&\quad=
\left|
\int_{[w \pm 2h] \cap \cW}
\big(
k_h(s,w)
- k_h(s,w')
\big)
f_{W \mid AA}(s \mid A_i, A_j)
\diff{s}
\right| \\
&\quad=
\left|
\int_{[w \pm 2h] \cap \cW}
\big(
k_h(s,w)
- k_h(s,w')
\big)
\big(
f_{W \mid AA}(s \mid A_i, A_j)
- f_{W \mid AA}(w \mid A_i, A_j)
\big)
\diff{s}
\right| \\
&\quad\leq
4h
\frac{C_\rL}{h^2}
|w-w'|
2 C_\rH h
\leq
8 C_\rL C_\rH
|w-w'|
\leq
C_3
|w-w'|,
\end{align*}
%
where $C_3 = 8 C_\rL C_\rH$.
The same holds for $|w-w'| > h$
as the Lipschitz property is local.
By taking $\E[\, \cdot \mid A_i]$,
it can be seen
by the contraction property of conditional expectation that
the same holds for the
singly-conditioned terms:
%
\begin{align*}
\left|
\E\big[ k_h(W_{i j},w) \mid A_i \big]
- \E\big[ k_h(W_{i j},w') \mid A_i \big]
\right|
&\leq
C_3
|w-w'|.
\end{align*}
%
Therefore $\cF_3$ is also smoothly parameterized
in exactly the same manner.
Let
%
\begin{align*}
C_4
&=
\sup_{w \in \cW}
\esssup_{A_i, A_j}
\big|
\E\big[ k_h(W_{i j},w) \mid A_i, A_j \big]
\big| \\
&=
\sup_{w \in \cW}
\esssup_{A_i, A_j}
\left|
\int_{[w \pm h] \cap \cW}
k_h(s,w)
f_{W \mid AA}(s \mid A_i, A_j)
\diff{s}
\right| \\
&\leq 2h \frac{C_\rk}{h} C_\rd
\leq 2 C_\rk C_\rd.
\end{align*}
%
For $\varepsilon \in (0,1]$,
take an $(\varepsilon C_4/C_3)$-cover of $\cW$
of cardinality at most $C_3 \Leb(\cW) / (\varepsilon C_4)$.
By the above parameterization properties,
this cover induces an
$\varepsilon C_4$-cover for both $\cF_2$ and $\cF_3$:
%
\begin{align*}
\sup_\Q
N\big(\cF_2, \rho_\Q, \varepsilon C_4 \big)
&\leq
N\big(\cF_2, \|\cdot\|_\infty, \varepsilon C_4 \big)
\leq
C_3 \Leb(\cW) / (\varepsilon C_4), \\
\sup_\Q
N\big(\cF_3, \rho_\Q, \varepsilon C_4 \big)
&\leq
N\big(\cF_3, \|\cdot\|_\infty, \varepsilon C_4 \big)
\leq
C_3 \Leb(\cW) / (\varepsilon C_4).
\end{align*}
%
Hence $\cF_1$, $\cF_2$, and $\cF_3$
form VC-type classes with envelopes
$F_1 = C_\rk / h$ and $F_2 = F_3 = C_4$:
%
\begin{align*}
\sup_\Q
N\left(\cF_1, \rho_\Q,
\varepsilon C_\rk / h \right)
&\leq
(C_1/(h\varepsilon))^{C_2},
&\sup_\Q
N\big(\cF_2, \rho_\Q, \varepsilon C_4 \big)
&\leq
(C_1/\varepsilon)^{C_2}, \\
\sup_\Q
N\big(\cF_3, \rho_\Q, \varepsilon C_4 \big)
&\leq
(C_1/\varepsilon)^{C_2},
\end{align*}
%
for some constants $C_1 \geq e$ and $C_2 \geq 1$,
where we augment the constants if necessary.

\proofparagraph{controlling $L_n$}

Observe that
$\sqrt{n}L_n$
is the empirical process of the i.i.d.\ variables $A_i$
indexed by $\cF_3$.
We apply Lemma~\ref{lem:kernel_app_maximal_vc_inid}
with $\sigma = C_4$:
%
\begin{align*}
\E \left[
\sup_{w \in \cW}
\big| \sqrt{n} L_
n(w) \big|
\right]
&\lesssim
C_4
\sqrt{C_2 \log C_1}
+
\frac{C_4 C_2 \log C_1}
{\sqrt{n}}
\lesssim 1.
\end{align*}
%
By Lemma~\ref{lem:kernel_trichotomy},
the left hand side is zero whenever
$\Du = 0$,
so we can also write
%
\begin{align*}
\E \left[
\sup_{w \in \cW}
\big| \sqrt{n} L_n(w) \big|
\right]
&\lesssim
\Du.
\end{align*}

\proofparagraph{controlling $Q_n$}

Observe that $n Q_n$
is the completely degenerate second-order U-process
of the i.i.d.\ variables $A_i$
indexed by $\cF_2$.
This function class is again uniformly bounded and VC-type,
so applying the U-process maximal inequality from
Lemma~\ref{lem:kernel_app_uprocess_maximal}
yields with $\sigma = C_4$
%
\begin{align*}
\E \left[
\sup_{w \in \cW}
\big| n Q_n(w) \big|
\right]
&\lesssim
C_4
C_2 \log C_1
+
\frac{C_4 (C_2 \log C_1)^2}
{\sqrt{n}}
\lesssim 1.
\end{align*}

\proofparagraph{controlling $E_n$}

Conditional on $\bA_n$,
note that $n E_n$
is the empirical process of the conditionally
i.n.i.d.\ variables $W_{i j}$
indexed by $\cF_1$.
We apply Lemma~\ref{lem:kernel_app_maximal_vc_inid}
conditionally with
%
\begin{align*}
\sigma^2
&=
\sup_{w \in \cW}
\E\Big[
\big(
k_h(W_{i j},w)
- \E[k_h(W_{i j},w) \mid A_i, A_j]
\big)^2
\mid A_i, A_j
\Big]
\leq
\sup_{w \in \cW}
\E\Big[
k_h(W_{i j},w)^2
\mid A_i, A_j
\Big] \\
&\leq
\sup_{w \in \cW}
\int_{[w \pm h] \cap \cW}
k_h(s,w)^2
f_{W \mid AA}(s \mid A_i, A_j)
\diff{s}
\leq 2h \frac{C_\rk^2}{h^2}
\lesssim 1/h
\end{align*}
%
and noting that we have
a sample size of
$\frac{1}{2}n(n-1)$,
giving
%
\begin{align*}
\E \left[
\sup_{w \in \cW}
\big| n E_n(w) \big|
\right]
&\lesssim
\sigma
\sqrt{C_2 \log \big((C_1/h) F_1 / \sigma \big)}
+
\frac{F_1 C_2 \log \big((C_1/h) F_1 / \sigma\big)}
{n} \\
&\lesssim
\frac{1}{\sqrt h}
\sqrt{C_2 \log \big((C_1/h) (C_\rk/h) \sqrt h \big)}
+
\frac{(C_\rk/h) C_2 \log \big((C_1/h) (C_\rk/h) \sqrt h \big)}
{n} \\
&\lesssim
\sqrt{\frac{\log 1/h}{h}}
+
\frac{\log \big(1/h\big)}
{n h}
\lesssim
\sqrt{\frac{\log n}{h}},
\end{align*}
%
where the last line follows by the bandwidth assumption
of $\frac{\log n}{n^2h} \to 0$.
\end{proof}

\begin{proof}[Theorem~\ref{thm:kernel_uniform_consistency}]
This follows from Theorem~\ref{thm:kernel_bias}
and Lemma~\ref{lem:kernel_uniform_concentration}.
\end{proof}

Before proving Theorem~\ref{thm:kernel_minimax}
we first give a lower bound result
for parametric point estimation in
Lemma~\ref{lem:kernel_app_neyman_pearson_bernoulli}.

\begin{lemma}[A Neyman--Pearson result for Bernoulli random variables]
\label{lem:kernel_app_neyman_pearson_bernoulli}

Recall that the Bernoulli distribution
$\Ber(\theta)$
places mass $\theta$ at $1$ and mass
$1-\theta$ at $0$.
Define $\P_\theta^n$ as the law of
$(A_1, A_2, \ldots, A_n, V)$,
where $A_1, \ldots, A_n$
are i.i.d.\ $\Ber(\theta)$,
and $V$ is an $\R^d$-valued random variable
for some $d \geq 1$
which is independent of the $A$ variables
and with a fixed distribution that does not depend on $\theta$.
Let $\theta_0 = \frac{1}{2}$
and $\theta_{1,n} = \frac{1}{2} + \frac{1}{\sqrt{8n}}$.
Then for any estimator $\tilde \theta_n$
which is a function of
$(A_1, A_2, \ldots, A_n, V)$ only,
%
\begin{align*}
\P_{\theta_0}^n \left(
\big| \tilde \theta_n - \theta_0 \big|
\geq \frac{1}{\sqrt{32n}}
\right)
+ \P_{\theta_{1,n}}^n \left(
\big| \tilde \theta_n - \theta_{1,n} \big|
\geq \frac{1}{\sqrt{32n}}
\right)
\geq \frac{1}{2}.
\end{align*}

\end{lemma}

\begin{proof}[Lemma~\ref{lem:kernel_app_neyman_pearson_bernoulli}]

Let $f: \{0,1\}^n \to \{0,1\}$
be any function.
Considering this function as a statistical test,
the Neyman--Pearson lemma and Pinsker's inequality
\citep{gine2021mathematical}
give
%
\begin{align*}
\P_{\theta_0}^n \big(
f=1
\big)
+\P_{\theta_{1,n}}^n \big(
f=0
\big)
&\geq
1-
\TV\left(
\P_{\theta_0}^n,
\P_{\theta_{1,n}}^n
\right)
\geq
1-
\sqrt{
\frac{1}{2}
\KL \left(
\P_{\theta_0}^n
\bigm\|
\P_{\theta_{1,n}}^n
\right)} \\
&=
1-
\sqrt{
\frac{n}{2}
\KL \left(
\Ber(\theta_0)
\bigm\|
\Ber(\theta_{1,n})
\right)
+ \frac{n}{2}
\KL \left(
V
\bigm\|
V
\right)} \\
&=
1-
\sqrt{
\frac{n}{2}
\KL \left(
\Ber(\theta_0)
\bigm\|
\Ber(\theta_{1,n})
\right)},
\end{align*}
%
where $\TV$ is the total variation distance
and $\KL$ is the Kullback--Leibler divergence.
In the penultimate line
we used the tensorization of Kullback--Leibler divergence
\citep{gine2021mathematical},
noting that the law of $V$ is fixed and hence does not contribute.
We now evaluate this Kullback--Leibler divergence at the specified
parameter values.
%
\begin{align*}
\P_{\theta_0}^n \big(
f=1
\big)
+\P_{\theta_{1,n}}^n \big(
f=0
\big)
&\geq
1-
\sqrt{
\frac{n}{2}
\KL \left(
\Ber(\theta_0)
\bigm\|
\Ber(\theta_{1,n})
\right)} \\
&=
1-
\sqrt{\frac{n}{2}}
\sqrt{
\theta_0 \log \frac{\theta_0}{\theta_{1,n}}
+ (1 - \theta_0) \log \frac{1 - \theta_0}{1 - \theta_{1,n}}} \\
&=
1-
\sqrt{\frac{n}{2}}
\sqrt{
\frac{1}{2} \log \frac{1/2}{1/2 + 1/\sqrt{8n}}
+ \frac{1}{2} \log \frac{1/2}{1/2 - 1/\sqrt{8n}}} \\
&=
1-
\frac{\sqrt n}{2}
\sqrt{\log \frac{1}{1 - 1/(2n)}}
\geq
1-
\frac{\sqrt n}{2}
\sqrt{\frac{1}{n}}
=
\frac{1}{2},
\end{align*}
%
where in the penultimate line we used that
$\log \frac{1}{1-x} \leq 2x$
for $x \in [0,1/2]$.
Now define a test $f$ by
$f = 1$ if $\tilde \theta_n > \frac{1}{2} + \frac{1}{\sqrt{32n}}$
and $f=0$ otherwise,
to see
%
\begin{align*}
\P_{\theta_0}^n \left(
\tilde \theta_n > \frac{1}{2} + \frac{1}{\sqrt{32n}}
\right)
+ \P_{\theta_{1,n}}^n \left(
\tilde \theta_n \leq \frac{1}{2} + \frac{1}{\sqrt{32n}}
\right)
\geq \frac{1}{2}.
\end{align*}
%
By the triangle inequality,
recalling that
$\theta_0 = \frac{1}{2}$
and $\theta_{1,n} = \frac{1}{2} + \frac{1}{\sqrt{8n}}$,
we have
%
\begin{align*}
\left\{
\tilde \theta_n > \frac{1}{2} + \frac{1}{\sqrt{32n}}
\right\}
&\subseteq
\left\{
\left| \tilde \theta_n - \theta_0 \right|
\geq \frac{1}{\sqrt{32n}}
\right\} \\
\left\{
\tilde \theta_n \leq \frac{1}{2} + \frac{1}{\sqrt{32n}}
\right\}
&\subseteq
\left\{
\left| \tilde \theta_n - \theta_{1,n} \right|
\geq \frac{1}{\sqrt{32n}}
\right\}.
\end{align*}
%
Thus by the monotonicity of measures,
%
\begin{align*}
\P_{\theta_0}^n \left(
\big| \tilde \theta_n - \theta_0 \big|
\geq \frac{1}{\sqrt{32n}}
\right)
+ \P_{\theta_{1,n}}^n \left(
\big| \tilde \theta_n - \theta_{1,n} \big|
\geq \frac{1}{\sqrt{32n}}
\right)
\geq \frac{1}{2}.
\end{align*}
\end{proof}

\begin{proof}[Theorem~\ref{thm:kernel_minimax}]

\proofparagraph{lower bound for $\cP$}

By translation and scaling of the data,
we may assume without loss of generality that $\cW = [-1,1]$.
We may also assume that $C_\rH \leq 1/2$,
since reducing $C_\rH$ can only shrink the class of distributions.
Define the dyadic distribution $\P_\theta$
with parameter $\theta \in [1/2, 1]$
as follows:
$A_1, \ldots, A_n$ are i.i.d.\ $\Ber(\theta)$, while
$V_{i j}$ for $1 \leq i < j \leq n$ are i.i.d.\
and independent of $\bA_n$.
The distribution of $V_{i j}$ is given by its density function
$f_V(v) = \frac{1}{2} + C_\rH v$ on $[-1,1]$.
Finally, generate
$W_{i j} = W(A_i, A_j, V_{i j}) \vcentcolon=
(2 A_i A_j - 1) V_{i j}$.
Note that the function $W$ does not depend on $\theta$.
The conditional and marginal densities of $W_{i j}$ are
for $w \in [-1,1]$
%
\begin{align*}
f_{W \mid AA}(w \mid A_i, A_j)
&=
\begin{cases}
\frac{1}{2} + C_\rH w & \text{if } A_i = A_j = 1, \\
\frac{1}{2} - C_\rH w & \text{if } A_i = 0 \text{ or } A_j = 0, \\
\end{cases} \\
f_{W \mid A}(w \mid A_i)
&=
\begin{cases}
\frac{1}{2} + (2 \theta - 1) C_\rH w
& \text{if } A_i = 1, \\
\frac{1}{2} - C_\rH w & \text{if } A_i = 0 , \\
\end{cases} \\
f_W(w)&= \frac{1}{2} + (2\theta^2 - 1) C_\rH w.
\end{align*}
%
Clearly,
$f_W \in \cH^\beta_{C_\rH}(\cW)$ and
$f_{W \mid AA}(\cdot \mid a, a') \in \cH^1_{C_\rH}(\cW)$.
Also
$\sup_{w \in \cW} \|f_{W \mid A}(w \mid \cdot\,)\|_\TV \leq 1$.
Therefore
$\P_\theta$ satisfies Assumption~\ref{ass:kernel_data}
and so
$\big\{\P_\theta : \theta \in [1/2, 1] \big\} \subseteq \cP$.

Note that $f_W(1) = \frac{1}{2} + (2\theta^2 - 1) C_\rH $,
so $\theta^2 = \frac{1}{2 C_\rH}(f_W(1) - 1/2 + C_\rH)$.
Thus if $\tilde f_W$ is some density estimator
depending only on the data $\bW_n$,
we define the parameter estimator
%
\begin{align*}
\tilde \theta_n^2
&\vcentcolon=
\frac{1}{2 C_\rH}\left(
\tilde f_W(1) - \frac{1}{2} + C_\rH
\right)
\vee 0.
\end{align*}
%
This gives the inequality
%
\begin{align*}
\big|
\tilde \theta_n^2 - \theta^2
\big|
&=
\left|
\frac{1}{2 C_\rH}\left(
\tilde f_W(1) - \frac{1}{2} + C_\rH
\right)
\vee 0
-
\frac{1}{2 C_\rH}\left(
f_W(1) - \frac{1}{2} + C_\rH
\right)
\right| \\
&\leq
\frac{1}{2 C_\rH}
\sup_{w \in \cW}
\left|
\tilde f_W(w) - f_W(w)
\right|.
\end{align*}
%
Therefore, since also $\tilde \theta \geq 0$
and $\theta \geq \frac{1}{2}$,
%
\begin{align*}
\big|
\tilde \theta_n - \theta
\big|
&=
\frac{\big|\tilde \theta_n^2 - \theta^2\big|}
{\tilde \theta_n + \theta}
\leq
\frac{1}{C_\rH}
\sup_{w \in \cW}
\left|
\tilde f_W(w) - f_W(w)
\right|.
\end{align*}
%
Now we apply the point estimation lower bound from
Lemma~\ref{lem:kernel_app_neyman_pearson_bernoulli},
setting $\theta_0 = \frac{1}{2}$
and $\theta_{1,n} = \frac{1}{2} + \frac{1}{\sqrt{8n}}$,
noting that the estimator
$\tilde \theta_n$
is a function of $\bW_n$ only,
thus is a function of $\bA_n$ and
$\bV_n$ only and so satisfies the conditions.
%
\begin{align*}
&\P_{\theta_0} \left(
\sup_{w \in \cW} \big| \tilde f_W(w) - f^{(0)}_W(w) \big|
\geq \frac{1}{C\sqrt{n}}
\right)
+ \P_{\theta_{1,n}} \left(
\sup_{w \in \cW} \big| \tilde f_W(w) - f^{(1)}_W(w) \big|
\geq \frac{1}{C\sqrt{n}}
\right) \\
&\quad\geq
\P_{\theta_0} \left(
\big| \tilde \theta_n - \theta_0 \big|
\geq \frac{1}{C C_\rH \sqrt{n}}
\right)
+ \P_{\theta_{1,n}} \left(
\big| \tilde \theta_n - \theta_{1,n} \big|
\geq \frac{1}{C C_\rH \sqrt{n}}
\right) \\
&\quad\geq
\P_{\theta_0} \left(
\big| \tilde \theta_n - \theta_0 \big|
\geq \frac{1}{\sqrt{32n}}
\right)
+ \P_{\theta_{1,n}} \left(
\big| \tilde \theta_n - \theta_{1,n} \big|
\geq \frac{1}{\sqrt{32n}}
\right)
\geq
\frac{1}{2},
\end{align*}
%
where we set $C \geq \frac{\sqrt{32}}{C_\rH}$.
Therefore we deduce that
%
\begin{align*}
\inf_{\tilde f_W}
\sup_{\P \in \cP}
\P\left(
\sup_{w \in \cW}
\big|
\tilde f_W(w) - f_W(w)
\big|
\geq
\frac{1}{C \sqrt n}
\right)
\geq \frac{1}{4}
\end{align*}
%
and so
%
\begin{align*}
\inf_{\tilde f_W}
\sup_{\P \in \cP}
\E_\P\left[
\sup_{w \in \cW}
\big|
\tilde f_W(w) - f_W(w)
\big|
\right]
\geq \frac{1}{4 C \sqrt{n}}.
\end{align*}

\proofparagraph{lower bound for $\cP_\rd$}

For the subclass of totally degenerate distributions,
we rely on the main theorem
from \citet{khasminskii1978lower}.
Let $\cP_0$ be the subclass of $\cP_\rd$
consisting of the distributions which satisfy
$A_1 = \cdots = A_n = 0$
and $W_{i j} \vcentcolon= A_i + A_j + V_{i j} = V_{i j}$,
so that $W_{i j}$ are i.i.d.\ with common density $f_W = f_V$.
Define the class
%
\begin{align*}
\cF
&=
\left\{
f \text{ density function on } \R, \
f \in \cH^\beta_{C_\rH}(\cW)
\right\}.
\end{align*}
%
Write $\E_f$ for the expectation under $W_{i j}$ having density $f$.
Then by \citet{khasminskii1978lower},
%
\begin{align*}
\liminf_{n \to \infty}
\inf_{\tilde f_W}
\sup_{f \in \cF}
\E_f\left[
\left( \frac{n^2}{\log n} \right)^{\frac{\beta}{2\beta + 1}}
\sup_{w \in \cW}
\big| \tilde f_W(w) - f_W(w) \big|
\right]
> 0,
\end{align*}
%
where $\tilde f_W$ is any
density estimator
depending only on the $\frac{1}{2}n(n-1)$ i.i.d.\ data samples $\bW_n$.
Now every density function in
$\cH^\beta_{C_\rH}(\cW)$
corresponds to a distribution in
$\cP_0$ and therefore to a distribution in $\cP_\rd$.
Thus for large enough $n$ and
some positive constant $C$,
%
\begin{align*}
\inf_{\tilde f_W}
\sup_{\P \in \cP_\rd}
\E_\P\left[
\sup_{w \in \cW}
\big| \tilde f_W(w) - f_W(w) \big|
\right]
\geq
\frac{1}{C}
\left( \frac{\log n}{n^2} \right)^{\frac{\beta}{2\beta + 1}}.
\end{align*}

\proofparagraph{upper bounds}

The upper bounds follow by
using a dyadic kernel density estimator $\hat f_W$
with a boundary bias-corrected
Lipschitz kernel of order $p \geq \beta$ and a bandwidth of $h$.
Theorem~\ref{thm:kernel_bias} gives
%
\begin{align*}
\sup_{\P \in \cP}
\sup_{w \in \cW}
\big|
\E_\P\big[\hat f_W(w)\big]
- f_W(w)
\big|
\leq
\frac{4C_\rk C_\rH}{\flbeta !}
h^\beta.
\end{align*}
%
Then,
treating the degenerate and non-degenerate cases separately
and noting that all inequalities hold uniformly over
$\cP$ and $\cP_\rd$,
the proof of Lemma~\ref{lem:kernel_uniform_concentration}
shows that
%
\begin{align*}
\sup_{\P \in \cP}
\E_\P\left[
\sup_{w \in \cW}
\big|\hat f_W(w) - \E_\P[\hat f_W(w)]\big|
\right]
&\lesssim
\frac{1}{\sqrt n}
+ \sqrt{\frac{\log n}{n^2h}}, \\
\sup_{\P \in \cP_\rd}
\E_\P\left[
\sup_{w \in \cW}
\big|\hat f_W(w) - \E_\P[\hat f_W(w)]\big|
\right]
&\lesssim
\sqrt{\frac{\log n}{n^2h}}.
\end{align*}
%
Thus combining these yields that
%
\begin{align*}
\sup_{\P \in \cP}
\E_\P\left[
\sup_{w \in \cW}
\big|\hat f_W(w) - f_W(w)\big|
\right]
&\lesssim
h^\beta
+ \frac{1}{\sqrt n}
+ \sqrt{\frac{\log n}{n^2h}}, \\
\sup_{\P \in \cP_\rd}
\E_\P\left[
\sup_{w \in \cW}
\big|\hat f_W(w) - f_W(w)\big|
\right]
&\lesssim
h^\beta
+ \sqrt{\frac{\log n}{n^2h}}.
\end{align*}
%
Set $h = \left( \frac{\log n}{n^2} \right)^{\frac{1}{2\beta+1}}$
and note that $\beta \geq 1$ implies that
$\left(\frac{\log n}{n^2} \right)^{\frac{\beta}{2\beta+1}}
\ll \frac{1}{\sqrt n}$.
So for $C > 0$,
%
\begin{align*}
\sup_{\P \in \cP}
\E_\P\left[
\sup_{w \in \cW}
\big|\hat f_W(w) - f_W(w)\big|
\right]
&\lesssim
\frac{1}{\sqrt n}
+ \left(
\frac{\log n}{n^2}
\right)^{\frac{\beta}{2\beta+1}}
\leq
\frac{C}{\sqrt n}, \\
\sup_{\P \in \cP_\rd}
\E_\P\left[
\sup_{w \in \cW}
\big|\hat f_W(w) - f_W(w)\big|
\right]
&\leq
C\left(
\frac{\log n}{n^2}
\right)^{\frac{\beta}{2\beta+1}}.
\end{align*}
\end{proof}

\begin{proof}[Lemma~\ref{lem:kernel_app_covariance_structure}]

We write $k_{i j}$ for $k_h(W_{i j},w)$
and $k_{i j}'$ for $k_h(W_{i j},w')$, in the interest of brevity.
%
\begin{align*}
\Sigma_n(w,w')
&=
\E\Big[
\big(
\hat f_W(w)
- \E[\hat f_W(w)]
\big)
\big(
\hat f_W(w')
- \E[\hat f_W(w')]
\big)
\Big] \\
&=
\E\left[
\left(
\frac{2}{n(n-1)}
\sum_{i<j}
\big(
k_{i j} - \E k_{i j}
\big)
\right)
\left(
\frac{2}{n(n-1)}
\sum_{r<s}
\big(
k_{rs}' - \E k_{rs}'
\big)
\right)
\right] \\
&=
\frac{4}{n^2(n-1)^2}
\sum_{i<j}
\sum_{r<s}
\E\left[
\big(
k_{i j} - \E k_{i j}
\big)
\big(
k_{rs}' - \E k_{rs}'
\big)
\right] \\
&=
\frac{4}{n^2(n-1)^2}
\sum_{i<j}
\sum_{r<s}
\Cov\left[
k_{i j},
k_{rs}'
\right].
\end{align*}
%
Note first that
for $i,j,r,s$ all distinct,
$k_{i j}$ is independent of $k_{rs}'$
and so the covariance is zero.
By a counting argument,
it can be seen that
there are
$n(n-1)/2$
summands where
$|\{i,j,r,s\}| = 2$,
and
$n(n-1)(n-2)$
summands where
$|\{i,j,r,s\}| = 3$.
Therefore, since the samples
are identically distributed,
the value of the summands
depends only on the number of distinct indices
and we have the decomposition
%
\begin{align*}
\Sigma_n(w,w')
&=
\frac{4}{n^2(n-1)^2}
\bigg(
\frac{n(n-1)}{2}
\Cov[k_{i j}, k_{i j}']
+ n(n-1)(n-2)
\Cov[k_{i j}, k_{i r}']
\bigg) \\
&=
\frac{2}{n(n-1)}
\Cov[k_{i j}, k_{i j}']
+ \frac{4(n-2)}{n(n-1)}
\Cov[k_{i j}, k_{i r}'],
\end{align*}
%
giving the first representation.
To obtain the second representation,
note that since
$W_{i j}$ and $W_{i r}$
are independent conditional
on $A_i$,
%
\begin{align*}
\Cov\big[
k_{i j}
k_{i r}'
\big]
&=
\E\big[
k_{i j}
k_{i r}'
\big]
-
\E[k_{i j}]
\E[k_{i r}']
=
\E\big[
\E\big[
k_{i j}
k_{i r}'
\mid A_i
\big]
\big]
-
\E[k_{i j}]
\E[k_{i r}'] \\
&=
\E\big[
\E[k_{i j} \mid A_i]
\E[k_{i r}' \mid A_i]
\big]
-
\E[k_{i j}]
\E[k_{i r}']
=
\Cov\big[
\E[k_{i j} \mid A_i],
\E[k_{i r}' \mid A_i]
\big].
\end{align*}
\end{proof}

\begin{proof}[Lemma~\ref{lem:kernel_variance_bounds}]

By Lemma~\ref{lem:kernel_app_covariance_structure},
the diagonal elements of $\Sigma_n$ are
%
\begin{align*}
\Sigma_n(w,w)
&=
\frac{2}{n(n-1)}
\Var\big[
k_h(W_{i j},w)
\big]
+
\frac{4(n-2)}{n(n-1)}
\Var\big[
\E[k_h(W_{i j},w) \mid A_i]
\big].
\end{align*}
%
We bound each of the two terms separately.
Firstly, note that since $k_h$ is bounded by $C_\rk/h$,
%
\begin{align*}
\Var\big[
k_h(W_{i j},w)
\big]
\leq
\E\big[
k_h(W_{i j},w)^2
\big]
=
\int_{\cW \cap [w \pm h]}
k_h(s,w)^2
f_W(s)
\diff{s}
\leq 2 C_\rk^2 / h.
\end{align*}
%
Since $\big|\E[k_h(W_{i j},w)]\big|
= \big|\int_{[w \pm h] \cap \cW} k_h(s,w) f_W(s) \diff{s}\big|
\leq 2 C_\rk C_\rd$, Jensen's inequality shows
%
\begin{align*}
\Var\big[
k_h(W_{i j},w)
\big]
&\geq
\int_{\cW \cap [w \pm h]}
k_h(s,w)^2
f_W(s)
\diff{s}
- 4 C_\rk^2 C_\rd^2 \\
&\geq
\inf_{w \in \cW} f_W(w)
\frac{1}{2h}
\left(
\int_{\cW \cap [w \pm h]}
k_h(s,w)
\diff{s}
\right)^2
- 4 C_\rk^2 C_\rd^2 \\
&\geq
\frac{1}{2h}
\inf_{w \in \cW} f_W(w)
- 4 C_\rk^2 C_\rd^2
\geq
\frac{1}{4h}
\inf_{w \in \cW} f_W(w)
\end{align*}
%
for small enough $h$, noting that this is trivial if the infimum is zero.
For the other term,
%
\begin{align*}
\Var\big[
\E[k_h(W_{i j},w) \mid A_i]
\big]
&\leq
\Var\big[
f_{W \mid A}(w \mid A_i)
\big]
+ 16 C_\rH C_\rk C_\rd h
\leq
2 \Du^2
\end{align*}
%
for small enough $h$, by a result from
the proof of Lemma~\ref{lem:kernel_trichotomy}.
Also
%
\begin{align*}
\Var\big[
\E[k_h(W_{i j},w) \mid A_i]
\big]
&\geq
\Var\big[
f_{W \mid A}(w \mid A_i)
\big]
- 16 C_\rH C_\rk C_\rd h
\geq
\frac{\Dl^2}{2}
\end{align*}
%
for small enough $h$.
Combining these four inequalities yields
that for all large enough $n$,
%
\begin{align*}
&\frac{2}{n(n-1)}
\frac{1}{4h}
\inf_{w \in \cW} f_W(w)
+ \frac{4(n-2)}{n(n-1)}
\frac{\Dl^2}{2}
\leq
\inf_{w \in \cW} \Sigma_n(w,w) \\
&\qquad\leq
\sup_{w \in \cW} \Sigma_n(w,w)
\leq
\frac{2}{n(n-1)}
\frac{2 C_\rk^2}{h}
+ \frac{4(n-2)}{n(n-1)}
2 \Du^2,
\end{align*}
%
so that
%
\begin{align*}
\frac{\Dl^2}{n}
+ \frac{1}{n^2h}
\inf_{w \in \cW} f_W(w)
&\lesssim
\inf_{w \in \cW} \Sigma_n(w,w)
\leq
\sup_{w \in \cW} \Sigma_n(w,w)
\lesssim
\frac{\Du^2}{n}
+ \frac{1}{n^2h}.
\end{align*}
\end{proof}

\begin{proof}[Lemma~\ref{lem:kernel_app_strong_approx_Ln}]

For the strong approximation,
apply the KMT corollary from
Lemma~\ref{lem:kernel_app_kmt_corollary}.
Define
%
\begin{align*}
k_h^A(a, w) = 2\E[k_h(W_{i j},w) \mid A_i = a],
\end{align*}
%
which are of bounded variation in $a$ uniformly over $w$ since
%
\begin{align*}
&\sup_{w \in \cW} \|k_h^A(\cdot,w)\|_\T
= 2\sup_{w \in \cW}
\sup_{m \in \N}
\sup_{a_0 \leq \cdots \leq a_m}
\sum_{i=1}^m
\big|k_h^A(a_i,w) - k_h^A(a_{i-1},w)\big| \\
&\quad=
2\sup_{w \in \cW}
\sup_{m \in \N}
\sup_{a_0 \leq \cdots \leq a_m}
\sum_{i=1}^m
\left|
\int_{[w \pm h] \cap \cW}
k_h(s,w)
\big(
f_{W \mid A}(s \mid a_i)
- f_{W \mid A}(s \mid a_{i-1})
\big)
\diff{s}
\right| \\
&\quad\leq
2 \sup_{w \in \cW}
\int_{[w \pm h] \cap \cW}
|k_h(s,w)|
\sup_{m \in \N}
\sup_{a_0 \leq \cdots \leq a_m}
\sum_{i=1}^m
\big|
f_{W \mid A}(s \mid a_i)
- f_{W \mid A}(s \mid a_{i-1})
\big|
\diff{s} \\
&\quad\leq
2 \sup_{w \in \cW}
\int_{[w \pm h] \cap \cW}
|k_h(s,w)|
\,
\big\|
f_{W \mid A}(w \mid \cdot)
\big\|_\TV
\diff{s} \\
&\quad\leq
4 C_\rk \sup_{w \in \cW}
\big\|
f_{W \mid A}(w \mid \cdot)
\big\|_\TV
\lesssim
\Du,
\end{align*}
%
where the last line is by observing that the total variation
is zero whenever $\Du = 0$.
Hence by Lemma~\ref{lem:kernel_app_kmt_corollary}
there exist (on some probability space)
$n$ independent copies of $A_i$,
denoted $A_i'$
and a centered Gaussian process $Z_n^{L\prime}$
such that if we define
%
\begin{align*}
L_n'(w)
&=
\frac{1}{n}
\sum_{i=1}^n
\big(k_h^A(A_i',w) -
\E[k_h^A(A_i',w)]\big),
\end{align*}
%
then for positive constants
$C_1, C_2, C_3$,
by defining the processes as zero outside $\cW$
we have
%
\begin{align*}
\P\left(
\sup_{w \in \cW}
\Big|\sqrt{n} L_n'(w) - Z_n^{L\prime}(w)\Big|
> \Du \frac{t + C_1 \log n}{\sqrt n}
\right)
\leq C_2 e^{-C_3 t}.
\end{align*}
%
Integrating tail probabilities shows that
%
\begin{align*}
\E\left[
\sup_{w \in \cW}
\Big|\sqrt{n} L_n'(w) - Z_n^{L\prime}(w)\Big|
\right]
&\leq
\Du \frac{C_1 \log n}{\sqrt n}
+ \int_0^\infty
\frac{\Du}{\sqrt n}
C_2 e^{-C_3 t}
\diff{t}
\lesssim
\frac{\Du \log n}{\sqrt n}.
\end{align*}
%
Further,
$Z_n^{L\prime}$ has the
same covariance structure as $G_n^{L\prime}$ in the
sense that for all $w, w' \in \cW$,
%
\begin{align*}
\E\big[Z_n^{L\prime}(w) Z_n^{L\prime}(w')\big]
= \E\big[G_n^{L\prime}(w) G_n^{L\prime}(w')\big],
\end{align*}
%
and clearly $L_n'$
is equal in distribution to $L_n$.
To obtain the trajectory regularity property of
$Z_n^{L\prime}$,
note that it was shown in the proof of
Lemma~\ref{lem:kernel_uniform_concentration}
that for all $w,w' \in \cW$,
%
\begin{align*}
\left|
k_h^A(A_i,w)
- k_h^A(A_i,w')
\right|
&\leq
C
|w-w'|
\end{align*}
%
for some constant $C > 0$.
Therefore, since the $A_i$ are i.i.d.,
%
\begin{align*}
&\E\left[
\big|
Z_n^{L\prime}(w)
- Z_n^{L\prime}(w')
\big|^2
\right]^{1/2}
=
\sqrt{n}
\E\left[
\big|
L_n(w)
- L_n(w')
\big|^2
\right]^{1/2} \\
&\quad=
\sqrt{n}
\E\left[
\left|
\frac{1}{n}
\sum_{i=1}^n
\Big(
k_h^A(A_i,w)
- k_h^A(A_i,w')
- \E\big[k_h^A(A_i,w)]
+ \E\big[k_h^A(A_i,w')]
\Big)
\right|^2
\right]^{1/2} \\
&\quad=
\E\left[
\Big|
k_h^A(A_i,w)
- k_h^A(A_i,w')
- \E\big[k_h^A(A_i,w)]
+ \E\big[k_h^A(A_i,w')]
\Big|^2
\right]^{1/2}
\lesssim
|w-w'|.
\end{align*}
%
Therefore, by
the regularity result for Gaussian processes in
Lemma~\ref{lem:kernel_app_gaussian_process_maximal},
with $\delta_n \in (0, 1/2]$:
%
\begin{align*}
\E\left[
\sup_{|w-w'| \leq \delta_n}
\big|
Z_n^{L\prime}(w)
- Z_n^{L\prime}(w')
\big|
\right]
&\lesssim
\int_0^{\delta_n}
\sqrt{\log 1/\varepsilon}
\diff{\varepsilon}
\lesssim
\delta_n \sqrt{\log 1/\delta_n}
\lesssim
\Du
\delta_n \sqrt{\log 1/\delta_n},
\end{align*}
%
where the last inequality is because
$Z_n^{L\prime} \equiv 0$ whenever $\Du = 0$.
There is a modification of $Z_n^{L\prime}$
with continuous trajectories
by Kolmogorov's continuity criterion
\citep[Theorem~2.9]{legall2016brownian}.
Note that $L_n'$ is $\bA_n'$-measurable
and so by Lemma~\ref{lem:kernel_app_kmt_corollary}
we can assume that $Z_n^{L\prime}$
depends only on $\bA_n'$ and some
random noise which is independent of
$(\bA_n', \bV_n')$.
Finally, in order to have
$\bA_n', \bV_n', L_n'$, and $Z_n^{L\prime}$
all defined on the same probability space,
we note that $\bA_n$ and $\bV_n$ are random vectors
while $L_n'$ and $Z_n^{L\prime}$
are stochastic processes
with continuous sample paths
indexed on
the compact interval $\cW$.
Hence the Vorob'ev--Berkes--Philipp theorem
(Lemma~\ref{lem:kernel_app_vbp})
allows us to ``glue'' them together
in the desired way
on another new probability space, giving
$\big(\bA_n', \bV_n', L_n', Z_n^{L\prime}\big)$,
retaining the single prime notation for clarity.
\end{proof}

\begin{proof}[Lemma~\ref{lem:kernel_strong_approx_Ln}]
See Lemma~\ref{lem:kernel_app_strong_approx_Ln}
\end{proof}

\begin{proof}[Lemma~\ref{lem:kernel_app_conditional_strong_approx_En}]

We apply Lemma~\ref{lem:kernel_app_yurinskii_corollary} conditional on
$\bA_n$. While this lemma is not in its current form
stated for conditional distributions,
the Yurinskii coupling on which it depends can be readily extended
by following the proof of \citet[Lemma~38]{belloni2019conditional},
using a conditional version of Strassen's theorem
\cite[Theorem~B.2]{chen2020jackknife}.
Care must similarly be taken in embedding the conditionally Gaussian vectors
into a conditionally Gaussian process, using the
Vorob'ev--Berkes--Philipp theorem (Lemma~\ref{lem:kernel_app_vbp}).

By the mutual independence of $A_i$ and $V_{i j}$,
we have that the observations
$W_{i j}$ are independent
(but not necessarily identically distributed)
conditionally on $\bA_n$.
Note that
$\sup_{s,w \in \cW} |k_h(s,w)| \lesssim M_n = h^{-1}$
and
$\E[k_h(W_{i j},w)^2 \mid \bA_n] \lesssim \sigma_n^2 = h^{-1}$.
The following uniform Lipschitz condition holds
with $l_{n,\infty} = C_\rL h^{-2}$,
by the Lipschitz property of the kernels:
%
\begin{align*}
\sup_{s,w,w' \in \cW}
\left|
\frac{k_h(s, w) - k_h(s, w')}
{w-w'}
\right|
\leq
l_{n,\infty}.
\end{align*}
%
Also, the following $L^2$ Lipschitz condition holds
uniformly with $l_{n,2} = 2 C_\rL \sqrt{C_\rd} h^{-3/2}$:
%
\begin{align*}
&\E\big[
\big|
k_h(W_{i j}, w) - k_h(W_{i j}, w')
\big|^2
\mid \bA_n
\big]^{1/2} \\
&\quad\leq
\frac{C_\rL}{h^2}
|w-w'|
\left(
\int_{([w \pm h] \cup [w' \pm h]) \cap \cW}
f_{W \mid AA}(s \mid \bA_n)
\diff{s}
\right)^{1/2} \\
&\quad\leq
\frac{C_\rL}{h^2}
|w-w'|
\sqrt{4h C_\rd}
\leq
l_{n,2}
|w-w'|.
\end{align*}
%
So we apply
Lemma~\ref{lem:kernel_app_yurinskii_corollary}
conditionally on $\bA_n$
to the $\frac{1}{2}n(n-1)$ observations,
noting that
%
\begin{align*}
\sqrt{n^2h} E_n(w)
=
\sqrt{\frac{2 n h}{n-1}}
\sqrt{\frac{2}{n(n-1)}}
\sum_{i=1}^{n-1}
\sum_{j=i+1}^{n}
\Big(
k_h(W_{i j},w)
- \E[k_h(W_{i j},w) \mid A_i, A_j]
\Big),
\end{align*}
%
to deduce that for $t_n > 0$ there exist
(an enlarged probability space)
conditionally mean-zero
and conditionally Gaussian processes
$\tilde Z_n^{E\prime}(w)$
with the same conditional covariance structure as
$\sqrt{n^2 h} E_n(w)$ and
satisfying
%
\begin{align*}
&\P\left(
\sup_{w \in \cW}
\big|
\sqrt{n^2h} E_n(w) - \tilde Z_n^{E\prime}(w)
\big|
> t_n
\Bigm\vert \bA_n'
\right) \\
&\quad=
\P\left(
\sup_{w \in \cW}
\left|
\sqrt{\frac{n(n-1)}{2}} E_n(w)
- \sqrt{\frac{n-1}{2 n h}} \tilde Z_n^{E\prime}(w)
\right|
> \sqrt{\frac{n-1}{2 n h}}
t_n
\Bigm\vert \bA_n'
\right) \\
&\quad\lesssim
\frac{
\sigma_n
\sqrt{\Leb(\cW)}
\sqrt{\log n}
\sqrt{M_n + \sigma_n\sqrt{\log n}}
}{n^{1/2} t_n^2 / h}
\sqrt{
l_{n,2}
\sqrt{\log n}
+ \frac{l_{n,\infty}}{n}
\log n} \\
&\quad\lesssim
\frac{
h^{-1/2}
\sqrt{\log n}
\sqrt{h^{-1} + h^{-1/2} \sqrt{\log n}}
}{n^{1/2} t_n^2 / h}
\sqrt{
h^{-3/2}
\sqrt{\log n}
+ \frac{h^{-2}}{n}
\log n} \\
&\quad\lesssim
\sqrt{\frac{\log n}{n}}
\frac{
\sqrt{1 + \sqrt{h \log n}}
}{t_n^2}
\sqrt{
\sqrt{\frac{\log n}{h^3}}
\left( 1 + \sqrt{\frac{\log n}{n^2 h}} \right)
} \\
&\quad\lesssim
\sqrt{\frac{\log n}{n}}
\frac{ 1 }{t_n^2}
\left(
\frac{\log n}{h^3}
\right)^{1/4}
\lesssim
t_n^{-2}
n^{-1/2}
h^{-3/4}
(\log n)^{3/4},
\end{align*}
%
where we used
$h \lesssim 1 / \log n$
and $\frac{\log n}{n^2 h} \lesssim 1$.
To obtain the trajectory regularity property of
$\tilde Z_n^{E\prime}$,
note that
for $w, w' \in \cW$,
by conditional independence,
%
\begin{align*}
&\E\left[
\big|
\tilde Z_n^{E\prime}(w)
- \tilde Z_n^{E\prime}(w')
\big|^2
\mid \bA_n'
\right]^{1/2}
=
\sqrt{n^2h} \,
\E\left[
\big|
E_n(w)
- E_n(w')
\big|^2
\mid \bA_n
\right]^{1/2} \\
&\quad\lesssim
\sqrt{n^2h} \,
\E\left[
\left|
\frac{2}{n(n-1)}
\sum_{i=1}^{n-1}
\sum_{j=i+1}^{n}
\Big(
k_h(W_{i j},w)
- k_h(W_{i j},w')
\Big)
\right|^2
\Bigm\vert \bA_n
\right]^{1/2} \\
&\quad\lesssim
\sqrt{h} \,
\E\left[
\big|
k_h(W_{i j},w)
- k_h(W_{i j},w')
\big|^2
\bigm\vert \bA_n
\right]^{1/2}
\lesssim
h^{-1} |w-w'|.
\end{align*}
%
So by the regularity result for Gaussian processes in
Lemma~\ref{lem:kernel_app_gaussian_process_maximal},
with $\delta_n \in (0, 1/(2h)]$:
%
\begin{align*}
\E\left[
\sup_{|w-w'| \leq \delta_n}
\big|
\tilde Z_n^{E\prime}(w)
- \tilde Z_n^{E\prime}(w')
\big|
\mid \bA_n'
\right]
&\lesssim
\int_0^{\delta_n/h}
\sqrt{\log (\varepsilon^{-1} h^{-1})}
\diff{\varepsilon}
\lesssim
\frac{\delta_n}{h}
\sqrt{\log \frac{1}{h\delta_n}},
\end{align*}
%
and there exists a modification with continuous trajectories.
Finally, in order to have $\bA_n', \bV_n', E_n'$, and $\tilde Z_n^{E\prime}$
all defined on the same probability space,
we note that $\bA_n$ and $\bV_n$ are random vectors
while $E_n'$ and $\tilde Z_n^{E\prime}$ are stochastic processes
with continuous sample paths indexed on the compact interval $\cW$.
Hence the Vorob'ev--Berkes--Philipp theorem (Lemma~\ref{lem:kernel_app_vbp})
allows us to ``glue together'' $\big(\bA_n, \bV_n, E_n\big)$
and $\big(E_n', \tilde Z_n^{E\prime}\big)$
in the desired way on another new probability space, giving
$\big(\bA_n', \bV_n', E_n', \tilde Z_n^{E\prime}\big)$,
retaining the single prime notation for clarity.

The trajectories of the conditionally Gaussian processes
$\tilde Z_n^{E\prime}$ depend on the choice of $t_n$,
necessitating the use of a divergent sequence $R_n$ to establish
bounds in probability.
\end{proof}

\begin{proof}[Lemma~\ref{lem:kernel_conditional_strong_approx_En}]
See Lemma~\ref{lem:kernel_app_conditional_strong_approx_En}
\end{proof}

\begin{proof}[Lemma~\ref{lem:kernel_app_unconditional_strong_approx_En}]

\proofparagraph{defining $Z_n^{E\dprime}$}

Pick $\delta_n \to 0$
with $\log 1/\delta_n \lesssim \log n$.
Let $\cW_\delta$ be a $\delta_n$-covering of $\cW$
with cardinality $\Leb(\cW)/\delta_n$
which is also a $\delta_n$-packing.
Let $\tilde Z_{n,\delta}^{E\prime}$
be the restriction of $\tilde Z_n^{E\prime}$
to $\cW_\delta$.
Let
$\tilde \Sigma_n^E(w, w') =
\E\big[\tilde Z_n^{E\prime}(w) \tilde Z_n^{E\prime}(w')
\mid \bA_n' \big]$
be the conditional covariance function of $\tilde Z_n^{E\prime}$,
and define
$\Sigma_n^E(w,w') = \E\big[\tilde \Sigma_n^E(w,w')\big]$.
Let $\tilde \Sigma^E_{n,\delta}$ and $\Sigma^E_{n,\delta}$
be the restriction matrices of
$\tilde \Sigma^E_n$ and $\Sigma^E_n$
to $\cW_\delta \times \cW_\delta$,
noting that, as (conditional) covariance matrices,
these are
(almost surely)
positive semi-definite.

Let $N \sim \cN(0, I_{|\cW_\delta|})$
be independent of $\bA_n'$,
and define using the matrix square root
$\tilde Z_{n,\delta}^{E\dprime}
= \big(\tilde \Sigma^E_{n,\delta})^{1/2} N$,
which has the same distribution as
$\tilde Z_{n,\delta}^{E\prime}$,
conditional on $\bA_n'$.
Extend it using
the Vorob'ev--Berkes--Philipp theorem
(Lemma~\ref{lem:kernel_app_vbp})
to the compact interval $\cW$,
giving a conditionally Gaussian process
$\tilde Z_n^{E\dprime}$
which has the same distribution as
$\tilde Z_{n}^{E\prime}$,
conditional on $\bA_n'$.
Define
$Z_{n,\delta}^{E\dprime} = \big(\Sigma^E_{n,\delta})^{1/2} N$,
noting that this is independent of $\bA_n'$,
and extend it using
the Vorob'ev--Berkes--Philipp theorem
(Lemma~\ref{lem:kernel_app_vbp})
to a Gaussian process
$Z_n^{E\dprime}$ on the compact interval $\cW$,
which is independent of $\bA_n'$
and has covariance structure given by
$\Sigma_n^E$.

\proofparagraph{closeness of $Z_n^{E\dprime}$ and
$\tilde Z_n^{E\dprime}$ on the mesh}

Note that conditionally on $\bA_n'$,
$\tilde Z_{n,\delta}^{E\dprime} - Z_{n,\delta}^{E\dprime}$
is a length-$|\cW_\delta|$
Gaussian random vector with covariance matrix
$\big(
\big(\tilde \Sigma^E_{n,\delta}\big)^{1/2}
- \big(\Sigma^E_{n,\delta}\big)^{1/2}
\big)^2$.
So by the Gaussian maximal inequality in
Lemma~\ref{lem:kernel_app_gaussian_vector_maximal}
applied conditionally on $\bA_n'$,
%
\begin{align*}
\E\left[
\max_{w \in \cW_\delta}
\big|\tilde Z_n^{E\dprime}(w) - Z_n^{E\dprime}(w)\big|
\Bigm| \bA_n'
\right]
&\lesssim
\sqrt{\log n}
\left\|
\tilde\Sigma^E_{n,\delta}
- \Sigma^E_{n,\delta}
\right\|_2^{1/2},
\end{align*}
%
since $\log |\cW_\delta| \lesssim \log n$.
Next, we apply some U-statistic theory to
$\tilde\Sigma^E_{n,\delta} - \Sigma^E_{n,\delta}$,
with the aim of applying the
matrix concentration result
for second-order U-statistics
presented in Lemma~\ref{lem:kernel_app_ustat_matrix_concentration}.
Firstly, we note that
since
the conditional covariance structures of
$\tilde Z_n^{E\prime}$ and $\sqrt{n^2h} E_n$
are equal in distribution,
we have,
writing $E_n(\cW_\delta)$
for the vector $\big(E_n(w) : w \in \cW_\delta\big)$
and similarly for $k_h(W_{i j}, \cW_\delta)$,
%
\begin{align*}
\tilde\Sigma^E_{n,\delta}
&=
n^2h \E[E_n(\cW_\delta) E_n(\cW_\delta)^\T \mid \bA_n] \\
&=
n^2h
\frac{4}{n^2(n-1)^2}
\sum_{i=1}^{n-1}
\sum_{j=i+1}^{n}
\E\left[
\Big(
k_h(W_{i j}, \cW_\delta)
- \E\left[
k_h(W_{i j}, \cW_\delta)
\mid \bA_n
\right]
\Big)
\right. \\
&\qquad\left.
\times\Big(
k_h(W_{i j}, \cW_\delta)
- \E\left[
k_h(W_{i j}, \cW_\delta)
\mid \bA_n
\right]
\Big)^\T
\bigm\vert \bA_n
\right] \\
&=
\frac{4h}{(n-1)^2}
\sum_{i=1}^{n-1}
\sum_{j=i+1}^{n}
u(A_i, A_j),
\end{align*}
%
where we
define the random
$|\cW_\delta| \times |\cW_\delta|$
matrices
%
\begin{align*}
u(A_i, A_j)
&=
\E\!\left[
k_h(W_{i j}, \cW_\delta)
k_h(W_{i j}, \cW_\delta)^\T
\mid \bA_n
\right]
-
\E\!\left[
k_h(W_{i j}, \cW_\delta)
\mid \bA_n
\right]
\E\!\left[
k_h(W_{i j}, \cW_\delta)
\mid \bA_n
\right]^\T.
\end{align*}
%
Let $u(A_i) = \E[u(A_i, A_j) \mid A_i]$ and
$u = \E[u(A_i, A_j)]$.
The decomposition
$\tilde \Sigma^E_{n,\delta} - \Sigma^E_{n,\delta}
= \tilde L +\tilde Q$
holds by Lemma~\ref{lem:kernel_app_general_hoeffding}, where
%
\begin{align*}
\tilde L
&=
\frac{4h}{n-1}
\sum_{i=1}^n
\big(
u(A_i) - u
\big),
&\tilde Q
&=
\frac{4h}{(n-1)^2}
\sum_{i=1}^{n-1}
\sum_{j=i+1}^{n}
\big(
u(A_i, A_j) - u(A_i) - u(A_j) + u
\big).
\end{align*}
%
Next, we seek an almost sure upper bound on
$\|u(A_i, A_j)\|_2$.
Since this is a symmetric matrix,
we have by H{\"o}lder's inequality
%
\begin{align*}
\|u(A_i, A_j)\|_2
&\leq
\|u(A_i, A_j)\|_1^{1/2}
\|u(A_i, A_j)\|_\infty^{1/2}
=
\max_{1 \leq k \leq |\cW_\delta|}
\sum_{l=1}^{|\cW_\delta|}
|u(A_i, A_j)_{kl}|.
\end{align*}
%
The terms on the right hand side can be bounded as follows,
writing $w, w'$ for the $k$th and $l$th
points in $\cW_\delta$ respectively:
%
\begin{align*}
|u(A_i, A_j)_{kl}|
&=
\big|
\E\left[
k_h(W_{i j}, w)
k_h(W_{i j}, w')
\mid \bA_n
\right]
-
\E\left[
k_h(W_{i j}, w)
\mid \bA_n
\right]
\E\left[
k_h(W_{i j}, w')
\mid \bA_n
\right]
\big| \\
&\lesssim
\E\left[
|
k_h(W_{i j}, w)
k_h(W_{i j}, w')
|
\mid \bA_n
\right]
+
\E\left[
|
k_h(W_{i j}, w)
|
\mid \bA_n
\right]
\E\left[
|
k_h(W_{i j}, w')
|
\mid \bA_n
\right] \\
&\lesssim
h^{-1}
\I\big\{ |w-w'| \leq 2h \big\}
+ 1
\lesssim
h^{-1}
\I\big\{ |k-l| \leq 2h/\delta_n \big\}
+ 1,
\end{align*}
%
where we used that
$|w-w'| \geq |k-l| \delta_n$
because $\cW_\delta$
is a $\delta_n$-packing.
Hence
%
\begin{align*}
\|u(A_i, A_j)\|_2
&\leq
\max_{1 \leq k \leq |\cW_\delta|}
\sum_{l=1}^{|\cW_\delta|}
|u(A_i, A_j)_{kl}|
\lesssim
\max_{1 \leq k \leq |\cW_\delta|}
\sum_{l=1}^{|\cW_\delta|}
\Big(
h^{-1}
\I\big\{ |k-l| \leq 2h/\delta_n \big\}
+ 1
\Big) \\
&\lesssim
1/\delta_n
+ 1/h
+ |\cW_\delta|
\lesssim
1/\delta_n
+ 1/h.
\end{align*}
%
Clearly, the same bound holds for
$\|u(A_i)\|_2$ and $\|u\|_2$, by Jensen's inequality.
Therefore, applying the matrix Bernstein inequality
(Lemma~\ref{lem:kernel_app_matrix_bernstein})
to the zero-mean matrix $\tilde L$ gives
%
\begin{align*}
\E\left[
\left\|
\tilde L
\right\|_2
\right]
&\lesssim
\frac{h}{n}
\left(\frac{1}{\delta_n} + \frac{1}{h} \right)
\left(
\log |\cW_\delta| + \sqrt{n \log |\cW_\delta|}
\right)
\lesssim
\left(\frac{h}{\delta_n} + 1 \right)
\sqrt{\frac{\log n}{n}}.
\end{align*}
%
The matrix U-statistic concentration inequality
(Lemma~\ref{lem:kernel_app_ustat_matrix_concentration})
with $\tilde Q$ gives
%
\begin{align*}
\E\left[
\big\|
\tilde Q
\big\|_2
\right]
&\lesssim
\frac{h}{n^2}
n
\left(\frac{1}{\delta_n} + \frac{1}{h} \right)
\left(
\log |\cW_\delta|
\right)^{3/2}
\lesssim
\left(\frac{h}{\delta_n} + 1 \right)
\frac{(\log n)^{3/2}}{n}.
\end{align*}
%
Hence taking a marginal expectation
and applying Jensen's inequality,
%
\begin{align*}
&\E\left[
\max_{w \in \cW_\delta}
\big|\tilde Z_n^{E\dprime}(w) - Z_n^{E\dprime}(w)\big|
\right] \\
&\quad\lesssim
\sqrt{\log n} \
\E\left[
\left\|
\tilde\Sigma^E_{n,\delta} - \Sigma^E_{n,\delta}
\right\|_2^{1/2}
\right]
\lesssim
\sqrt{\log n} \
\E\left[
\left\|
\tilde\Sigma^E_{n,\delta} - \Sigma^E_{n,\delta}
\right\|_2
\right]^{1/2} \\
&\quad\lesssim
\sqrt{\log n} \
\E\left[
\left\|
\tilde L
+ \tilde Q
\right\|_2
\right]^{1/2}
\lesssim
\sqrt{\log n} \
\E\left[
\left\|
\tilde L
\right\|_2
+ \left\|
\tilde Q
\right\|_2
\right]^{1/2} \\
&\quad\lesssim
\sqrt{\log n}
\left(
\left(\frac{h}{\delta_n} + 1 \right)
\sqrt{\frac{\log n}{n}}
+ \left(\frac{h}{\delta_n} + 1 \right)
\frac{(\log n)^{3/2}}{n}
\right)^{1/2} \\
&\quad\lesssim
\sqrt{\frac{h}{\delta_n} + 1}
\frac{(\log n)^{3/4}}{n^{1/4}}.
\end{align*}

\proofparagraph{regularity of $Z_n^E$ and $\tilde Z_n^{E\prime}$}

Define the semimetrics
%
\begin{align*}
\rho(w, w')^2
&=
\E\left[
\big|Z_n^{E\dprime}(w) - Z_n^{E\dprime}(w')\big|^2
\right],
&\tilde\rho(w, w')^2
&=
\E\left[
\big|\tilde Z_n^{E\dprime}(w) - \tilde Z_n^{E\dprime}(w')\big|^2
\mid \bA_n
\right].
\end{align*}
%
We bound $\tilde \rho$ as follows,
since $\tilde Z_n^{E\dprime}$ and $\sqrt{n^2h} E_n$
have the same conditional covariance structure:
%
\begin{align*}
\tilde\rho(w, w')
&=
\E\left[
\big|\tilde Z_n^{E\dprime}(w) - \tilde Z_n^{E\dprime}(w')\big|^2
\mid \bA_n'
\right]^{1/2} \\
&=
\sqrt{n^2 h} \,
\E\left[
\big|E_n(w) - E_n(w')\big|^2
\mid \bA_n'
\right]^{1/2}
\lesssim
h^{-1}
|w-w'|,
\end{align*}
%
uniformly in $\bA_n'$,
where the last line was shown in
the proof of Lemma~\ref{lem:kernel_app_conditional_strong_approx_En}.
Note that also
%
\begin{align*}
\rho(w, w')
&=
\sqrt{\E[\tilde \rho(w,w')^2]}
\lesssim
h^{-1}
|w-w'|.
\end{align*}
%
Thus Lemma~\ref{lem:kernel_app_gaussian_process_maximal}
applies directly to $Z_n^E$
and conditionally to $\tilde Z_n^{E\prime}$,
with $\delta_n \in (0, 1/(2h)]$,
demonstrating that
%
\begin{align*}
\E\left[
\sup_{|w-w'| \leq \delta_n}
\big|\tilde Z_n^{E\dprime}(w) - \tilde Z_n^{E\dprime}(w')\big|
\bigm\vert \bA_n'
\right]
&\lesssim
\int_0^{\delta_n / h}
\sqrt{\log (1 / (\varepsilon h))}
\diff{\varepsilon}
\lesssim
\frac{\delta_n}{h}
\sqrt{\log \frac{1}{h \delta_n}}, \\
\E\left[
\sup_{|w-w'| \leq \delta_n}
|Z_n^{E\dprime}(w) - Z_n^{E\dprime}(w')|
\right]
&\lesssim
\int_0^{\delta_n / h}
\sqrt{\log (1 / (\varepsilon h))}
\diff{\varepsilon}
\lesssim
\frac{\delta_n}{h}
\sqrt{\log \frac{1}{h \delta_n}}.
\end{align*}
%
Continuity of trajectories follows from this.

\proofparagraph{conclusion}

We use the previous parts to deduce that
%
\begin{align*}
&\E\left[
\sup_{w \in \cW}
\big|\tilde Z_n^{E\dprime}(w) - Z_n^{E\dprime}(w)\big|
\right] \\
&\quad\lesssim
\E\left[
\max_{w \in \cW_\delta}
\big|\tilde Z_n^{E\dprime}(w) - Z_n^{E\dprime}(w)\big|
\right] \\
&\qquad+
\E\left[
\sup_{|w-w'| \leq \delta_n}
\left\{
\big|\tilde Z_n^{E\dprime}(w) - \tilde Z_n^{E\dprime}(w')\big|
+ \big|Z_n^{E\dprime}(w) - Z_n^{E\dprime}(w')\big|
\right\}
\right] \\
&\quad\lesssim
\sqrt{\frac{h}{\delta_n} + 1}
\frac{(\log n)^{3/4}}{n^{1/4}}
+ \frac{\delta_n \sqrt{\log n}}{h}.
\end{align*}
%
Setting
$\delta_n = h \left( \frac{\log n}{n} \right)^{1/6}$
gives
%
\begin{align*}
\E\left[
\sup_{w \in \cW}
\big|\tilde Z_n^{E\dprime}(w) - Z_n^{E\dprime}(w)\big|
\right]
&\lesssim
n^{-1/6} (\log n)^{2/3}.
\end{align*}
%
Independence of $Z_n^{E\dprime}$ and $\bA_n''$
follows by applying the
Vorob'ev--Berkes--Philipp theorem (Lemma~\ref{lem:kernel_app_vbp}),
conditionally on $\bA_n'$, to the variables
$\big(\bA_n', \tilde Z_n^{E\prime}\big)$ and
$\big(\tilde Z_n^{E\dprime}, Z_n^{E\dprime}\big)$.
\end{proof}

\begin{proof}[Lemma~\ref{lem:kernel_unconditional_strong_approx_En}]
See Lemma~\ref{lem:kernel_app_unconditional_strong_approx_En}
\end{proof}

\begin{proof}[Theorem~\ref{thm:kernel_app_strong_approx_fW}]

We add together the strong approximations
for the $L_n$ and $E_n$ terms,
and then add an independent Gaussian process
to account for the variance of $Q_n$.

\proofparagraph{gluing together the strong approximations}

Let $\big(\bA_n', \bV_n', L_n', Z_n^{L\prime}\big)$
be the strong approximation for $L_n$
derived in Lemma~\ref{lem:kernel_app_strong_approx_Ln}.
Let $\big(\bA_n'', \bV_n'', E_n'', \tilde Z_n^{E\dprime}\big)$
and
$\big(\bA_n''', \bV_n''', \tilde Z_n^{E\tprime}, Z_n^{E\tprime}\big)$
be the conditional and unconditional strong approximations for $E_n$
given in Lemmas~\ref{lem:kernel_app_conditional_strong_approx_En}
and \ref{lem:kernel_app_unconditional_strong_approx_En}
respectively.
The first step is to define copies of these variables
and processes on the same probability space.
This is achieved by applying the
Vorob'ev--Berkes--Philipp theorem (Lemma~\ref{lem:kernel_app_vbp}).
Dropping the prime notation for clarity, we construct
$\big(\bA_n, \bV_n, L_n, Z_n^L, E_n, \tilde Z_n^E, Z_n^E\big)$
with the following properties:
%
\begin{enumerate}[label=(\roman*)]

\item
$\sup_{w \in \cW}
\big| \sqrt{n} L_n(w) - Z_n^L(w)\big|
\lesssim_\P n^{-1/2} \log n$,

\item
$\sup_{w \in \cW}
\big|\sqrt{n^2h} E_n(w) - \tilde Z^E_n(w) \big|
\lesssim_\P n^{-1/4} h^{-3/8} (\log n)^{3/8} R_n$,

\item
$\sup_{w \in \cW}
\big| \tilde Z^E_n(w) - Z^E_n(w) \big|
\lesssim_\P n^{-1/6} (\log n)^{2/3}$,

\item
$Z_n^L$ is independent of $Z_n^E$.

\end{enumerate}
%
Note that the independence of
$Z_n^L$ and $Z_n^E$
follows since $Z_n^L$
depends only on $\bA_n$ and some independent random noise,
while $Z_n^E$ is independent of $\bA_n$.
Therefore $(Z_n^L, Z_n^E)$ are jointly Gaussian.
To get the strong approximation result
for $\hat f_W$,
define the Gaussian process
%
\begin{align*}
Z_n^f(w)
&=
\frac{1}{\sqrt n} Z_n^L(w)
+ \frac{1}{n} Z_n^Q(w)
+ \frac{1}{\sqrt{n^2h}} Z_n^E(w),
\end{align*}
%
where $Z_n^Q(w)$ is a mean-zero Gaussian process
independent of everything else
with covariance
%
\begin{align*}
\E\big[
Z_n^Q(w)
Z_n^Q(w')
\big]
&=
n^2 \E\big[
Q_n(w)
Q_n(w')
\big].
\end{align*}
%
As shown in the proof of
Lemma~\ref{lem:kernel_uniform_concentration},
the process
$Q_n(w)$ is uniformly Lipschitz
and uniformly bounded in $w$.
Thus by Lemma~\ref{lem:kernel_app_gaussian_process_maximal},
we have
$\E\big[\sup_{w \in \cW}
|Z_n^Q(w)|\big]
\lesssim 1$.
Therefore the uniform approximation error is given by
%
\begin{align*}
&
\sup_{w \in \cW}
\big|
\hat f_W(w) - \E[\hat f_W(w)]
- Z_n^f(w)
\big|
\\
&\quad=
\sup_{w \in \cW}
\left|
\frac{1}{\sqrt n} Z_n^L(w)
+ \frac{1}{n} Z_n^Q(w)
+ \frac{1}{\sqrt{n^2h}} Z_n^E(w)
- \Big(
L_n(w) + Q_n(w) + E_n(w)
\Big)
\right| \\
&\quad\leq
\sup_{w \in \cW}
\bigg(
\frac{1}{\sqrt n}
\left|
Z_n^L(w) - \sqrt{n} L_n(w)
\right|
+ \frac{1}{\sqrt{n^2h}}
\left|
\tilde Z_n^E(w) - \sqrt{n^2h} E_n(w)
\right| \\
&\qquad+
\frac{1}{\sqrt{n^2h}}
\left|
Z_n^E(w) - \tilde Z_n^E(w)
\right|
\big| Q_n(w) \big|
+ \frac{1}{n}
\big| Z_n^Q(w) \big|
\bigg) \\
&\quad\lesssim_\P
n^{-1} \log n
+ n^{-5/4} h^{-7/8} (\log n)^{3/8} R_n
+ n^{-7/6} h^{-1/2} (\log n)^{2/3}.
\end{align*}

\proofparagraph{covariance structure}

Since $L_n$, $Q_n$, and $E_n$
are mutually orthogonal in $L^2$
(as shown in Lemma~\ref{lem:kernel_hoeffding}),
we have the following covariance
structure:
%
\begin{align*}
\E\big[Z_n^f(w) Z_n^f(w')\big]
&=
\frac{1}{n} \E\big[ Z_n^L(w) Z_n^L(w') \big]
+ \frac{1}{n^2} \E\big[ Z_n^Q(w) Z_n^Q(w') \big]
+ \frac{1}{n^2h} \E\big[ Z_n^E(w) Z_n^E(w') \big] \\
&=
\E\big[ L_n(w) L_n(w') \big]
+ \E\big[ Q_n(w) Q_n(w') \big]
+ \E\big[ E_n(w) E_n(w') \big] \\
&=
\E\big[
\big(\hat f_W(w) - \E[\hat f_W(w)]\big)
\big(\hat f_W(w') - \E[\hat f_W(w')]\big)
\big].
\end{align*}

\proofparagraph{trajectory regularity}

The trajectory regularity of the process
$Z_n^f$ follows directly by adding the regularities
of the processes $\frac{1}{\sqrt n} Z_n^L$,
$\frac{1}{n} Z_n^Q$, and $\frac{1}{\sqrt{n^2h}} Z_n^E$.
Similarly, $Z_n^f$ has continuous trajectories.
\end{proof}

\begin{proof}[Theorem~\ref{thm:kernel_strong_approx_Tn}]

Define $Z_n^T(w) = \frac{Z_n^f(w)}{\sqrt{\Sigma_n(w,w)}}$ so that
%
\begin{align*}
\left| T_n(w) - Z_n^T(w) \right|
&= \frac{\big| \hat f_W(w) - f_W(w) - Z_n^f(w) \big|}
{\sqrt{\Sigma_n(w,w)}}.
\end{align*}
%
By Theorems~\ref{thm:kernel_app_strong_approx_fW} and \ref{thm:kernel_bias},
the numerator can be bounded above by
%
\begin{align*}
&\sup_{w \in \cW}
\left|
\hat f_W(w) - f_W(w)
-
Z_n^f(w)
\right| \\
&\quad\leq
\sup_{w \in \cW}
\left|
\hat f_W(w)
- \E\big[\hat f_W(w)\big]
-
Z_n^f(w)
\right|
+ \sup_{w \in \cW}
\left|
\E\big[\hat f_W(w)\big]
- f_W(w)
\right| \\
&\quad\lesssim_\P
n^{-1} \log n
+ n^{-5/4} h^{-7/8} (\log n)^{3/8} R_n
+ n^{-7/6} h^{-1/2} (\log n)^{2/3}
+ h^{p \wedge \beta}.
\end{align*}
%
By Lemma~\ref{lem:kernel_variance_bounds}
with $\inf_\cW f_W(w) > 0$,
the denominator is bounded below by
%
\begin{align*}
\inf_{w \in \cW}
\sqrt{\Sigma_n(w,w)}
&\gtrsim
\frac{\Dl}{\sqrt n} + \frac{1}{\sqrt{n^2h}},
\end{align*}
%
and the result follows.
\end{proof}

\begin{proof}[Theorem~\ref{thm:kernel_infeasible_ucb}]

Note that the covariance structure of $Z_n^T$ is given by
%
\begin{align*}
\Cov\big[
Z_n^T(w),
Z_n^T(w')
\big]
&=
\frac{\Sigma_n(w,w')}
{\sqrt{\Sigma_n(w,w) \Sigma_n(w',w')}}.
\end{align*}
%
We apply an anti-concentration result
to establish that all quantiles of
$\sup_{w \in \cW} \big|Z_n^T(w)\big|$ exist.
To do this, we must first establish regularity
properties of $Z_n^T$.

\proofparagraph{$L^2$ regularity of $Z_n^T$}

Writing $k_{i j}'$ for $k_h(W_{i j},w')$ etc.,
note that by Lemma~\ref{lem:kernel_app_covariance_structure},
%
\begin{align*}
&\big|
\Sigma_n(w,w')
-
\Sigma_n(w, w'')
\big| \\
&\quad=
\left|
\frac{2}{n(n-1)}
\Cov\big[
k_{i j},
k_{i j}'
\big]
+
\frac{4(n-2)}{n(n-1)}
\Cov\big[
k_{i j},
k_{i r}'
\big]
\right. \\
&\left.
\quad\qquad-
\frac{2}{n(n-1)}
\Cov\big[
k_{i j},
k_{i j}''
\big]
-
\frac{4(n-2)}{n(n-1)}
\Cov\big[
k_{i j},
k_{i r}''
\big]
\right| \\
&\quad\leq
\frac{2}{n(n-1)}
\Big|
\Cov\big[
k_{i j},
k_{i j}' - k_{i j}''
\big]
\Big|
+
\frac{4(n-2)}{n(n-1)}
\Big|
\Cov\big[
k_{i j},
k_{i r}' - k_{i r}''
\big]
\Big| \\
&\quad\leq
\frac{2}{n(n-1)}
\|k_{i j}\|_\infty
\|k_{i j}' - k_{i j}''\|_\infty
+
\frac{4(n-2)}{n(n-1)}
\|k_{i j}\|_\infty
\|k_{i r}' - k_{i r}''\|_\infty \\
&\quad\leq
\frac{4}{n h^3}
C_\rk C_\rL
|w'-w''|
\lesssim
n^{-1}h^{-3} |w'-w''|
\end{align*}
%
uniformly in $w, w', w'' \in \cW$.
Therefore, by Lemma~\ref{lem:kernel_variance_bounds},
with $\delta_n \leq n^{-2} h^2$,
we have
%
\begin{align*}
\inf_{|w-w'| \leq \delta_n}
\Sigma_n(w,w')
&\gtrsim
\frac{\Dl^2}{n}
+ \frac{1}{n^2h}
- n^{-1} h^{-3} \delta_n
\gtrsim
\frac{\Dl^2}{n}
+ \frac{1}{n^2h}
- \frac{1}{n^3h}
\gtrsim
\frac{\Dl^2}{n}
+ \frac{1}{n^2h}, \\
\sup_{|w-w'| \leq \delta_n}
\Sigma_n(w,w')
&\lesssim
\frac{\Du^2}{n}
+ \frac{1}{n^2h}
+ n^{-1} h^{-3} \delta_n
\lesssim
\frac{\Du^2}{n}
+ \frac{1}{n^2h}
+ \frac{1}{n^3h}
\lesssim
\frac{\Du^2}{n}
+ \frac{1}{n^2h}.
\end{align*}
%
The $L^2$
regularity of $Z_n^T$
is
%
\begin{align*}
\E\left[
\big(
Z_n^T(w) - Z_n^T(w')
\big)^2
\right]
&=
2 - 2
\frac{\Sigma_n(w,w')}
{\sqrt{\Sigma_n(w,w) \Sigma_n(w',w')}}.
\end{align*}
%
Applying the elementary result
that for $a,b,c > 0$,
%
\begin{align*}
1 - \frac{a}{\sqrt{b c}}
&=
\frac{b(c-a) + a(b-a)}
{\sqrt{b c}\big(\sqrt{b c} + a\big)},
\end{align*}
%
with $a = \Sigma_n(w,w')$,
$b = \Sigma_n(w,w)$,
and $c = \Sigma_n(w',w')$,
and noting $|c-a| \lesssim n^{-1} h^{-3} |w-w'|$
and $|b-a| \lesssim n^{-1} h^{-3} |w-w'|$ and
$\frac{\Dl^2}{n} + \frac{1}{n^2h}
\lesssim a,b,c \lesssim \frac{\Du^2}{n} + \frac{1}{n^2h}$,
yields
%
\begin{align*}
\E\left[
\big(
Z_n^T(w) - Z_n^T(w')
\big)^2
\right]
&\lesssim
\frac{(\Du^2/n + 1/(n^2h))n^{-1}h^{-3}|w-w'|}
{(\Dl^2/n + 1/(n^2h))^2} \\
&\lesssim
\frac{n^{2} h^{-4}|w-w'|}
{n^{-4}h^{-2}}
\lesssim
n^2 h^{-2} |w-w'|.
\end{align*}
%
Thus the semimetric
induced by $Z_n^T$ on $\cW$ is
%
\begin{align*}
\rho(w,w')
&\vcentcolon=
\E\left[
\big(
Z_n^T(w) - Z_n^T(w')
\big)^2
\right]^{1/2}
\lesssim
n h^{-1} \sqrt{|w-w'|}.
\end{align*}

\proofparagraph{trajectory regularity of $Z_n^T$}

By the bound on $\rho$ from the previous part,
we deduce the covering number bound
%
\begin{align*}
N(\varepsilon, \cW, \rho)
&\lesssim
N\big(
\varepsilon,
\cW,
n h^{-1} \sqrt{|\cdot|}
\big)
\lesssim
N\big(
n^{-1} h \varepsilon,
\cW,
\sqrt{|\cdot|}
\big) \\
&\lesssim
N\big(
n^{-2} h^2 \varepsilon^2,
\cW,
|\cdot|
\big)
\lesssim
n^2 h^{-2} \varepsilon^{-2}.
\end{align*}
%
Now apply the Gaussian process regularity result from
Lemma~\ref{lem:kernel_app_gaussian_process_maximal}.
%
\begin{align*}
\E\left[
\sup_{\rho(w,w') \leq \delta}
\big| Z_n^T(w) - Z_n^T(w') \big|
\right]
&\lesssim
\int_0^{\delta}
\sqrt{\log N(\varepsilon, \cW, \rho)}
\diff{\varepsilon}
\lesssim
\int_0^{\delta}
\sqrt{\log (n^2 h^{-2} \varepsilon^{-2})}
\diff{\varepsilon} \\
&\lesssim
\int_0^{\delta}
\left(
\sqrt{\log n}
+ \sqrt{\log 1/\varepsilon}
\right)
\diff{\varepsilon}
\lesssim
\delta
\left(
\sqrt{\log n}
+ \sqrt{\log 1/\delta}
\right),
\end{align*}
%
and so
%
\begin{align*}
\E\left[
\sup_{|w-w'| \leq \delta_n}
\big| Z_n^T(w) - Z_n^T(w') \big|
\right]
&\lesssim
\E\left[
\sup_{\rho(w,w') \leq n h^{-1} \delta_n^{1/2}}
\big| Z_n^T(w) - Z_n^T(w') \big|
\right]
\lesssim
n h^{-1}
\sqrt{\delta_n \log n},
\end{align*}
%
whenever $1/\delta_n$
is at most polynomial in $n$.

\proofparagraph{existence of the quantile}

Apply the Gaussian anti-concentration
result from Lemma~\ref{lem:kernel_app_anticoncentration},
noting that $Z_n^T$ is separable,
mean-zero, and has unit variance:
%
\begin{align*}
\sup_{t \in \R}
\P\left(
\left|
\sup_{w \in \cW}
\big| Z_n^T(w) \big|
- t
\right|
\leq 2\varepsilon_n
\right)
&\leq
8 \varepsilon_n
\left(
1 + \E\left[
\sup_{w \in \cW}
\big| Z_n^T(w) \big|
\right]
\right).
\end{align*}
%
To bound the supremum on the right hand side,
apply the Gaussian process maximal inequality from
Lemma~\ref{lem:kernel_app_gaussian_process_maximal}
with
$\sigma \leq 1$ and
$N(\varepsilon, \cW, \rho) \lesssim n^2 h^{-2} \varepsilon^{-2}$:
%
\begin{align*}
\E\left[
\sup_{w \in \cW}
\big|Z_n^T(w)\big|
\right]
&\lesssim
1
+ \int_0^{2}
\sqrt{\log (n^2 h^{-2} \varepsilon^{-2})}
\diff{\varepsilon}
\lesssim
\sqrt{\log n}.
\end{align*}
%
Therefore
%
\begin{align*}
\sup_{t \in \R}
\P\left(
\left|
\sup_{w \in \cW}
\big| Z_n^T(w) \big|
- t
\right|
\leq \varepsilon
\right)
&\lesssim
\varepsilon
\sqrt{\log n}.
\end{align*}
%
Letting $\varepsilon \to 0$
shows that the distribution function of
$\sup_{w \in \cW} \big|Z_n^T(w)\big|$
is continuous,
and therefore all of its quantiles exist.

\proofparagraph{validity of the infeasible uniform confidence band}

Under Assumption~\ref{ass:kernel_rates} and with a
sufficiently slowly diverging sequence $R_n$,
the strong approximation rate established in
Theorem~\ref{thm:kernel_strong_approx_Tn} is
%
\begin{align*}
&\sup_{w \in \cW} \left| T_n(w) - Z_n^T(w) \right| \\
&\quad\lesssim_\P
\frac{
n^{-1/2} \log n
+ n^{-3/4} h^{-7/8} (\log n)^{3/8} R_n
+ n^{-2/3} h^{-1/2} (\log n)^{2/3}
+ n^{1/2} h^{p \wedge \beta}}
{\Dl + 1/\sqrt{n h}}
\ll \frac{1}{\sqrt{\log n}}.
\end{align*}
%
So by Lemma~\ref{lem:kernel_app_slow_convergence}, take $\varepsilon_n$ such
that
%
\begin{align*}
\P \left(
\sup_{w \in \cW} \left| T_n(w) - Z_n^T(w) \right|
> \varepsilon_n
\right)
&\leq
\varepsilon_n \sqrt{\log n}
\end{align*}
%
and $\varepsilon_n \sqrt{\log n} \to 0$.
So by the previously established anti-concentration result,
%
\begin{align*}
&\P\left(
\left|
\hat f_W(w) - f_W(w)
\right|
\leq
q_{1-\alpha}
\sqrt{\Sigma_n(w,w)}
\textup{ for all }
w \in \cW
\right) \\
&\quad=
\P\left(
\sup_{w \in \cW}
\left| T_n(w) \right|
\leq
q_{1-\alpha}
\right) \\
&\quad\leq
\P\left(
\sup_{w \in \cW}
\left| Z_n^T(w) \right|
\leq
q_{1-\alpha}
+ \varepsilon_n
\right)
+ \P \left(
\sup_{w \in \cW} \left| T_n(w) - Z_n^T(w) \right|
> \varepsilon_n
\right) \\
&\quad\leq
\P\left(
\sup_{w \in \cW}
\left|
Z_n^T(w)
\right|
\leq
q_{1-\alpha}
\right)
+ \P\left(
\left|
\sup_{w \in \cW}
\big| Z_n^T(w) \big|
- q_{1-\alpha}
\right|
\leq \varepsilon_n
\right)
+ \varepsilon_n \sqrt{\log n} \\
&\quad\leq
1 - \alpha
+ 2 \varepsilon_n \sqrt{\log n}.
\end{align*}
%
The lower bound follows analogously:
%
\begin{align*}
&\P\left(
\left|
\hat f_W(w) - f_W(w)
\right|
\leq
q_{1-\alpha}
\sqrt{\Sigma_n(w,w)}
\textup{ for all }
w \in \cW
\right) \\
&\quad\geq
\P\left(
\sup_{w \in \cW}
\left| Z_n^T(w) \right|
\leq
q_{1-\alpha}
- \varepsilon_n
\right)
- \varepsilon_n \sqrt{\log n} \\
&\quad\geq
\P\left(
\sup_{w \in \cW}
\left|
Z_n^T(w)
\right|
\leq
q_{1-\alpha}
\right)
- \P\left(
\left|
\sup_{w \in \cW}
\big| Z_n^T(w) \big|
- q_{1-\alpha}
\right|
\leq \varepsilon_n
\right)
- \varepsilon_n \sqrt{\log n} \\
&\quad\leq
1 - \alpha
- 2 \varepsilon_n \sqrt{\log n}.
\end{align*}
%
Finally, we apply $\varepsilon_n \sqrt{\log n} \to 0$
to see
%
\begin{align*}
\left|
\P\left(
\left|
\hat f_W(w) - f_W(w)
\right|
\leq
q_{1-\alpha}
\sqrt{\Sigma_n(w,w)}
\textup{ for all }
w \in \cW
\right)
- (1 - \alpha)
\right|
&\to 0.
\end{align*}
\end{proof}

Before proving
Lemma~\ref{lem:kernel_app_covariance_estimation},
we provide the following useful
concentration inequality.
This is essentially a corollary of the
U-statistic concentration inequality given in
Theorem~3.3 in \citet{gine2000exponential}.

\begin{lemma}[A concentration inequality]
\label{lem:kernel_app_dyadic_concentration}

Let $X_{i j}$ be mutually independent for $1 \leq i < j \leq n$
taking values in a measurable space $\cX$.
Let $h_1$, $h_2$ be measurable functions from $\cX$ to $\R$
satisfying the following for all $i$ and $j$.
%
\begin{align*}
\E\big[h_1(X_{i j})\big]
&= 0,
&\E\big[h_2(X_{i j})\big]
&=0, \\
\E\big[h_1(X_{i j})^2\big]
&\leq \sigma^2,
&\E\big[h_2(X_{i j})^2\big]
&\leq \sigma^2, \\
\big|h_1(X_{i j})\big|
&\leq M,
&\big|h_2(X_{i j})\big|
&\leq M.
\end{align*}
%
Consider the sum
%
\begin{align*}
S_n
&=
\sum_{1 \leq i < j < r \leq n}
h_1(X_{i j})
h_2(X_{i r}).
\end{align*}
%
Then $S_n$ satisfies the concentration inequality
%
\begin{align*}
\P\big(
|S_n| \geq t
\big)
&\leq
C \exp\left(
-\frac{1}{C}
\min \left\{
\frac{t^2}{n^3 \sigma^4},
\frac{t}{\sqrt{n^3 \sigma^4}},
\frac{t^{2/3}}{(n M \sigma)^{2/3}},
\frac{t^{1/2}}{M}
\right\}
\right)
\end{align*}
%
for some universal constant
$C > 0$
and for all $t>0$.

\end{lemma}

\begin{proof}[Lemma~\ref{lem:kernel_app_dyadic_concentration}]

We proceed in three main steps.
Firstly, we write $S_n$ as a second-order U-statistic
where we use double indices instead of single indices.
Then we use a decoupling result to introduce extra independence.
Finally, a concentration result is applied
to the decoupled U-statistic.

\proofparagraph{writing $S_n$ as a second-order U-statistic}

Note that we can write $S_n$ as
the second-order U-statistic
%
\begin{align*}
S_n
&=
\sum_{1 \leq i < j \leq n}
\sum_{1 \leq q < r \leq n}
h_{i j q r}
(X_{i j}, X_{qr}),
\end{align*}
%
where
%
\begin{align*}
h_{i j q r}
(a,b)
&=
h_1(a) h_2(b) \,
\I\{j<r,\, q=i\}.
\end{align*}
%
Although this may look like a fourth-order
U-statistic,
it is in fact second-order.
This is due to independence of the variables
$X_{i j}$,
and by treating $(i,j)$ as a single index.

\proofparagraph{decoupling}

By the decoupling result of Theorem~1
from \citet{delapena1995decoupling}, there exists a universal
constant $C_1 > 0$ satisfying
%
$\P\big( |S_n| \geq t \big)
\leq C_1 \P\big( C_1 |\tilde S_n| \geq t \big)$,
%
where
%
$\tilde S_n = \sum_{1 \leq i < j \leq n} \sum_{1 \leq q < r \leq n}
h_{i j q r} (X_{i j}, X'_{qr})$,
%
with $(X'_{i j})$
an independent copy of $(X_{i j})$.

\proofparagraph{U-statistic concentration}

The U-statistic kernel $h_{i j q r}(X_{i j}, X'_{qr})$
is totally degenerate in that
%
$ \E[h_{i j q r}(X_{i j}, X'_{qr}) \mid X_{i j}]
= \E[h_{i j q r}(X_{i j}, X'_{qr}) \mid X'_{qr}] = 0$.
%
Define and bound the following quantities:
%
\pagebreak
%
\begin{align*}
A
&=
\max_{i j q r}
\|h_{i j q r}(X_{i j}, X'_{qr})\|_\infty
\leq M^2, \\
B
&=
\max
\left\{
\left\|
\sum_{1 \leq i < j \leq n}
\E\Big[
h_{i j q r}(X_{i j}, X'_{qr})^2
\mid X_{i j}
\Big]
\right\|_\infty,
\left\|
\sum_{1 \leq q < r \leq n}
\E\Big[
h_{i j q r}(X_{i j}, X'_{qr})^2
\mid X'_{qr}
\Big]
\right\|_\infty
\right\}^{1/2} \\
&=
\max
\left\{
\left\|
\sum_{1 \leq i < j \leq n}
h_1(X_{i j})^2
\E\big[
h_2(X_{qr}')^2
\big]
\I\{j<r, q=i\}
\right\|_\infty,
\right. \\
&\left.
\qquad\qquad\quad
\left\|
\sum_{1 \leq q < r \leq n}
\E\big[
h_1(X_{i j})^2
\big]
h_2(X_{qr}')^2
\I\{j<r, q=i\}
\right\|_\infty
\right\}^{1/2} \\
&\leq
\max
\left\{
n^2 M^2 \sigma^2,
n M^2 \sigma^2
\right\}^{1/2}
=
n M \sigma, \\
C
&=
\left(
\sum_{1 \leq i < j \leq n}
\sum_{1 \leq q < r \leq n}
\!\E\big[
h_{i j q r}(X_{i j}, X'_{qr})^2
\big]
\right)^{\!1/2}
\!\!\!\!\! = \left(
\sum_{1 \leq i < j < r \leq n}
\!\!\E\big[
h_1(X_{i j})^2
h_2(X_{i r}')^2
\big]
\right)^{\!1/2}
\!\!\!\!\! \leq
\sqrt{n^3 \sigma^4}, \\
D
&=
\sup_{f,g} \Bigg\{
\sum_{1 \leq i < j \leq n}
\sum_{1 \leq q < r \leq n}
\E\big[
h_{i j q r}(X_{i j}, X'_{qr})
f_{i j}(X_{i j})
g_{qr}(X'_{qr})
\big]
\ : \\
&\qquad\qquad\quad
\sum_{1 \leq i < j \leq n}
\E\big[f_{i j}(X_{i j})^2\big]
\leq 1, \
\sum_{1 \leq q < r \leq n}
\E\big[g_{qr}(X'_{qr})^2\big]
\leq 1
\Bigg\} \\
&=
\sup_{f,g} \Bigg\{
\sum_{1 \leq i < j < r \leq n}
\E\big[
h_1(X_{i j})
f_{i j}(X_{i j})
\big]
\E\big[
h_2(X'_{i r})
g_{i r}(X'_{i r})
\big]
\ : \\
&\qquad\qquad\quad
\sum_{1 \leq i < j \leq n}
\E\big[f_{i j}(X_{i j})^2\big]
\leq 1, \
\sum_{1 \leq q < r \leq n}
\E\big[g_{qr}(X'_{qr})^2\big]
\leq 1
\Bigg\} \\
&\leq
\sup_{f,g} \Bigg\{
\sum_{1 \leq i < j < r \leq n}
\E\big[ h_1(X_{i j})^2 \big]^{1/2}
\E\big[ f_{i j}(X_{i j})^2 \big]^{1/2}
\E\big[ h_2(X_{i r}')^2 \big]^{1/2}
\E\big[ g_{i r}(X'_{i r})^2 \big]^{1/2}
\ : \\
&\qquad\qquad\quad
\sum_{1 \leq i < j \leq n}
\E\big[f_{i j}(X_{i j})^2\big]
\leq 1, \
\sum_{1 \leq q < r \leq n}
\E\big[g_{qr}(X'_{qr})^2\big]
\leq 1
\Bigg\} \\
&\leq
\sigma^2
\sup_{f,g} \Bigg\{
\sum_{1 \leq i < j \leq n}
\E\big[ f_{i j}(X_{i j})^2 \big]^{1/2}
\sum_{1 \leq r \leq n }
\E\big[ g_{i r}(X'_{i r})^2 \big]^{1/2}
\ : \\
&\qquad\qquad\quad
\sum_{1 \leq i < j \leq n}
\E\big[f_{i j}(X_{i j})^2\big]
\leq 1, \
\sum_{1 \leq q < r \leq n}
\E\big[g_{qr}(X'_{qr})^2\big]
\leq 1
\Bigg\} \\
&\leq
\sigma^2
\sup_{f,g} \Bigg\{
\Bigg(
n^2
\sum_{1 \leq i < j \leq n}
\E\big[ f_{i j}(X_{i j})^2 \big]
\Bigg)^{1/2}
\Bigg(
n
\sum_{1 \leq r \leq n }
\E\big[ g_{i r}(X'_{i r})^2 \big]
\Bigg)^{1/2}
\ : \\
&\qquad\qquad\quad
\sum_{1 \leq i < j \leq n}
\E\big[f_{i j}(X_{i j})^2\big]
\leq 1, \
\sum_{1 \leq q < r \leq n}
\E\big[g_{qr}(X'_{qr})^2\big]
\leq 1
\Bigg\}
\leq
\sqrt{n^3 \sigma^4}.
\end{align*}
%
By Theorem~3.3 in \citet{gine2000exponential},
for some universal constant $C_2 > 0$ and for all $t > 0$,
%
\begin{align*}
\P\left(
|\tilde S_n| \geq t
\right)
&\leq
C_2 \exp\left(
-\frac{1}{C_2}
\min \left\{
\frac{t^2}{C^2},
\frac{t}{D},
\frac{t^{2/3}}{B^{2/3}},
\frac{t^{1/2}}{A^{1/2}}
\right\}
\right) \\
&\leq
C_2 \exp\left(
-\frac{1}{C_2}
\min \left\{
\frac{t^2}{n^3 \sigma^4},
\frac{t}{\sqrt{n^3 \sigma^4}},
\frac{t^{2/3}}{(n M \sigma)^{2/3}},
\frac{t^{1/2}}{M}
\right\}
\right).
\end{align*}

\proofparagraph{Conclusion}

By the previous parts
and absorbing constants into a new constant $C > 0$,
we therefore have
%
\begin{align*}
\P\left(
|S_n| \geq t
\right)
&\leq
C_1 \P\left(
C_1 |\tilde S_n| \geq t
\right) \\
&\leq
C_1 C_2 \exp\left(
-\frac{1}{C_2}
\min \left\{
\frac{t^2}{n^3 \sigma^4 C_1^2},
\frac{t}{\sqrt{n^3 \sigma^4 C_1}},
\frac{t^{2/3}}{(n M \sigma C_1)^{2/3}},
\frac{t^{1/2}}{M C_1^{1/2}}
\right\}
\right) \\
&\leq
C \exp\left(
-\frac{1}{C}
\min \left\{
\frac{t^2}{n^3 \sigma^4},
\frac{t}{\sqrt{n^3 \sigma^4}},
\frac{t^{2/3}}{(n M \sigma)^{2/3}},
\frac{t^{1/2}}{M}
\right\}
\right).
\end{align*}
\end{proof}

\begin{proof}[Lemma~\ref{lem:kernel_app_covariance_estimation}]

Throughout this proof we will write
$k_{i j}$ for $k_h(W_{i j},w)$ and
$k_{i j}'$ for $k_h(W_{i j},w')$,
in the interest of brevity.
Similarly, we write $S_{i j r}$ to denote $S_{i j r}(w,w')$.
The estimand and estimator are reproduced below for clarity.
%
\begin{align*}
\Sigma_n(w,w')
&=
\frac{2}{n(n-1)}
\E[k_{i j} k_{i j}']
+ \frac{4(n-2)}{n(n-1)}
\E[k_{i j} k_{i r}']
- \frac{4n-6}{n(n-1)}
\E[k_{i j}]
\E[k_{i j}'] \\
\hat \Sigma_n(w,w')
&=
\frac{2}{n(n-1)}
\frac{2}{n(n-1)}
\sum_{i<j}
k_{i j}
k_{i j}'
+
\frac{4(n-2)}{n(n-1)}
\frac{6}{n(n-1)(n-2)}
\sum_{i<j<r}
S_{i j r} \\
&\quad-
\frac{4n-6}{n(n-1)}
\hat f_W(w)
\hat f_W(w'),
\end{align*}
%
where
%
$S_{i j r} = \frac{1}{6}
\big( k_{i j} k_{i r}'
+ k_{i j} k_{jr}'
+ k_{i r} k_{i j}'
+ k_{i r} k_{jr}'
+ k_{jr} k_{i j}'
+ k_{jr} k_{i r}'
\big).$
%
We will prove uniform consistency of each of the three terms separately.

\proofparagraph{uniform consistency of the $\hat f_W(w) \hat f_W(w')$ term}

By boundedness of $f_W$ and
Theorem~\ref{thm:kernel_uniform_consistency},
$\hat f_W$ is uniformly bounded in probability.
Noting that
$\E[\hat f_W(w)] = \E[k_{i j}]$
and by Lemma~\ref{lem:kernel_variance_bounds},
%
\begin{align*}
&\sup_{w,w' \in \cW}
\left|
\frac{
\hat f_W(w) \hat f_W(w')
- \E\big[k_{i j}\big] \E\big[k_{i j'}\big]}
{\sqrt{\Sigma_n(w,w) + \Sigma_n(w',w')}}
\right|
=
\sup_{w,w' \in \cW}
\left|
\frac{
\hat f_W(w) \hat f_W(w')
- \E\big[\hat f_W(w)\big] \E\big[\hat f_W(w')\big]}
{\sqrt{\Sigma_n(w,w) + \Sigma_n(w',w')}}
\right| \\
&\quad\leq
\sup_{w,w' \in \cW}
\left|
\frac{\hat f_W(w) - \E\big[\hat f_W(w)\big]}
{\sqrt{\Sigma_n(w,w)}}
\hat f_W(w')
+ \frac{\hat f_W(w') - \E\big[\hat f_W(w')\big]}
{\sqrt{\Sigma_n(w',w')}}
\E\big[\hat f_W(w)]
\right| \\
&\quad\lesssim_\P
\sup_{w \in \cW}
\left|
\frac{\hat f_W(w) - \E\big[\hat f_W(w)\big]}
{\sqrt{\Sigma_n(w,w)}}
\right| \\
&\quad\lesssim_\P
\sup_{w \in \cW}
\left|
\frac{L_n(w)}
{\sqrt{\Sigma_n(w,w)}}
\right|
+ \sqrt{n^2h} \sup_{w \in \cW} \left| Q_n(w) \right|
+ \sqrt{n^2h} \sup_{w \in \cW} \left| E_n(w) \right| \\
&\quad\lesssim_\P
\sup_{w \in \cW}
\left|
\frac{L_n(w)}
{\sqrt{\Sigma_n(w,w)}}
\right|
+ \sqrt{n^2h} \frac{1}{n}
+ \sqrt{n^2h} \sqrt{\frac{\log n}{n^2h}}
\lesssim_\P
\sup_{w \in \cW}
\left|
\frac{L_n(w)}
{\sqrt{\Sigma_n(w,w)}}
\right|
+ \sqrt{\log n}.
\end{align*}
%
Now consider the function class
%
\begin{align*}
\cF
&=
\left\{
a \mapsto
\frac{
\E\big[k_h(W_{i j},w) \mid A_i = a \big]
- \E\big[k_h(W_{i j},w) \big]}
{\sqrt{n \Sigma_n(w,w)}}:
w \in \cW
\right\},
\end{align*}
%
noting that
%
\begin{align*}
\frac{L_n(w)}
{\Sigma_n(w,w)^{1/2}}
&=
\frac{1}{\sqrt n}
\sum_{i=1}^n
g_w(A_i)
\end{align*}
%
is an empirical process evaluated at
$g_w \in \cF$.
By the lower bound on $\Sigma_n(w,w)$
from Lemma~\ref{lem:kernel_variance_bounds}
with $\inf_\cW f_W(w) > 0$ and since $n h \gtrsim \log n$,
the class $\cF$ has a constant envelope function
given by $F(a) \lesssim \sqrt{n h}$.
Clearly, $M = \sup_a F(a) \lesssim \sqrt{n h}$.
Also by definition of $\Sigma_n$
and orthogonality of $L_n$, $Q_n$, and $E_n$,
we have
$\sup_{f \in \cF} \E[f(A_i)^2] \leq \sigma^2 = 1$.
To verify a VC-type condition on $\cF$
we need to establish the regularity of the process.
By Lipschitz properties
of $L_n$ and $\Sigma_n$
derived in the proofs of Lemma~\ref{lem:kernel_uniform_concentration}
and Theorem~\ref{thm:kernel_infeasible_ucb}
respectively,
we have
%
\begin{align*}
\left|
\frac{L_n(w)}
{\sqrt{\Sigma_n(w,w)}}
- \frac{L_n(w')}
{\sqrt{\Sigma_n(w',w')}}
\right|
&\lesssim
\frac{\big|L_n(w) - L_n(w')\big|}
{\sqrt{\Sigma_n(w,w)}}
+
\left| L_n(w') \right|
\left|
\frac{1}
{\sqrt{\Sigma_n(w,w)}}
- \frac{1}
{\sqrt{\Sigma_n(w',w')}}
\right| \\
&\lesssim
\sqrt{n^2h}
|w-w'|
+\left|
\frac{\Sigma_n(w,w) - \Sigma_n(w',w')}
{\Sigma_n(w,w)\sqrt{\Sigma_n(w',w')}}
\right| \\
&\lesssim
\sqrt{n^2h}
|w-w'|
+ (n^2h)^{3/2}
\left|
\Sigma_n(w,w) - \Sigma_n(w',w')
\right| \\
&\lesssim
\sqrt{n^2h}
|w-w'|
+ (n^2h)^{3/2}
n^{-1} h^{-3}
|w-w'|
\lesssim
n^4 |w-w'|,
\end{align*}
%
uniformly over $w,w' \in \cW$. By compactness of $\cW$ we have the covering
number bound
%
$N(\cF, \|\cdot\|_\infty, \varepsilon) \lesssim
N(\cW, |\cdot|, n^{-4} \varepsilon) \lesssim n^4 \varepsilon^{-1}$.
%
Thus by Lemma~\ref{lem:kernel_app_maximal_vc_inid},
%
\begin{align*}
\E \left[
\sup_{w \in \cW}
\left|
\frac{L_n(w)}
{\sqrt{\Sigma_n(w,w)}}
\right|
\right]
&\lesssim
\sqrt{\log n}
+ \frac{\sqrt{n h} \log n}{\sqrt{n}}
\lesssim
\sqrt{\log n}.
\end{align*}
%
Therefore
%
\begin{align*}
\sup_{w,w' \in \cW}
\left|
\frac{
\hat f_W(w) \hat f_W(w')
- \E\big[k_{i j}\big] \E\big[k_{i j'}\big]}
{\sqrt{\Sigma_n(w,w) + \Sigma_n(w',w')}}
\right|
&\lesssim_\P
\sqrt{\log n}.
\end{align*}

\proofparagraph{decomposition of the $S_{i j r}$ term}

We first decompose the $S_{i j r}$ term into two parts,
and obtain a pointwise concentration result for each.
This is extended to a uniform concentration result
by considering the regularity of the covariance estimator process.
Note that
$\E[S_{i j r}] = \E[k_{i j} k_{i r}']$,
and hence
%
\begin{align*}
&\frac{6}{n(n-1)(n-2)}
\sum_{i<j<r}
\big(
S_{i j r}
- \E[k_{i j} k_{i r}']
\big) \\
&\quad=
\frac{6}{n(n-1)(n-2)}
\sum_{i<j<r}
S_{i j r}^{(1)}
+ \frac{6}{n(n-1)(n-2)}
\sum_{i<j<r}
S_{i j r}^{(2)},
\end{align*}
%
where $S_{i j r}^{(1)} = S_{i j r} - \E[S_{i j r} \mid \bA_n]$
and $S_{i j r}^{(2)} = \E[S_{i j r} \mid \bA_n] - \E[S_{i j r}]$.

\proofparagraph{pointwise concentration of the $S_{i j r}^{(1)}$ term}

By symmetry in $i, j$, and $r$
it is sufficient to consider only the first summand
in the definition of $S_{i j r}$.
By conditional independence properties,
we have the decomposition
%
\begin{align}
\nonumber
&\frac{6}{n(n-1)(n-2)}
\sum_{i<j<r}
\Big(
k_{i j}k_{i r}'
- \E[k_{i j}k_{i r}' \mid \bA_n]
\Big) \\
\nonumber
&\quad=
\frac{6}{n(n-1)(n-2)}
\sum_{i<j<r}
\Big(
k_{i j}k_{i r}'
- \E[k_{i j} \mid \bA_n]
\E[k_{i r}' \mid \bA_n]
\Big) \\
\nonumber
&\quad=
\frac{6}{n(n-1)(n-2)}
\sum_{i<j<r}
\Big(
\big(
k_{i j}
- \E[k_{i j} \mid \bA_n]
\big)
\big(
k_{i r}'
- \E[k_{i r}' \mid \bA_n]
\big) \\
\nonumber
&\qquad+
\big(
k_{i j}
- \E[k_{i j} \mid \bA_n]
\big)
\E[k_{i r}' \mid \bA_n]
+ \big(
k_{i r}'
- \E[k_{i r}' \mid \bA_n]
\big)
\E[k_{i j} \mid \bA_n]
\Big) \\
\label{eq:kernel_app_Sijr1_decomp1}
&\quad=
\frac{6}{n(n-1)(n-2)}
\sum_{i<j<r}
\Big(
k_{i j}
- \E[k_{i j} \mid \bA_n]
\Big)
\Big(
k_{i r}'
- \E[k_{i r}' \mid \bA_n]
\Big) \\
\label{eq:kernel_app_Sijr1_decomp2}
&\qquad+
\frac{2}{(n-1)(n-2)}
\sum_{i=1}^{n-2}
\sum_{j=i+1}^{n-1}
\Big(
k_{i j}
- \E[k_{i j} \mid \bA_n]
\Big)
\cdot \frac{3}{n}
\sum_{r=j+1}^n
\E[k_{i r}' \mid \bA_n] \\
\label{eq:kernel_app_Sijr1_decomp3}
&\qquad+
\frac{2}{(n-1)(n-2)}
\sum_{i=1}^{n-2}
\sum_{r=i+2}^n
\Big(
k_{i r}'
- \E[k_{i r}' \mid \bA_n]
\Big)
\cdot \frac{3}{n}
\sum_{j=i+1}^{r-1}
\E[k_{i j} \mid \bA_n].
\end{align}
%
For the term in \eqref{eq:kernel_app_Sijr1_decomp1},
note that conditional on $\bA_n$,
we have that
$k_{i j} - \E[k_{i j} \mid \bA_n]$
are conditionally mean-zero
and conditionally independent,
as the only randomness is from $\bV_n$.
Also
$\Var[k_{i j} \mid \bA_n] \lesssim \sigma^2 := 1/h$
and
$|k_{i j}| \lesssim M := 1/h$
uniformly.
The same is true for $k_{i j}'$.
Thus by Lemma~\ref{lem:kernel_app_dyadic_concentration}
for some universal constant $C_1 > 0$:
%
\begin{align*}
&\P\left(
\left|
\sum_{i<j<r}
\Big(
k_{i j}
- \E[k_{i j} \mid \bA_n]
\Big)
\Big(
k_{i r}'
- \E[k_{i r}' \mid \bA_n]
\Big)
\right|
> t
\biggm\vert \bA_n
\right) \\
&\quad\leq
C_1 \exp\left(
-\frac{1}{C_1}
\min \left\{
\frac{t^2}{n^3 \sigma^4},
\frac{t}{\sqrt{n^3 \sigma^4}},
\frac{t^{2/3}}{(n M \sigma)^{2/3}},
\frac{t^{1/2}}{M}
\right\}
\right) \\
&\quad\leq
C_1 \exp\left(
-\frac{1}{C_1}
\min \left\{
\frac{t^2 h^2}{n^3},
\frac{t h}{\sqrt{n^3}},
\frac{t^{2/3} h}{n^{2/3}},
t^{1/2} h
\right\}
\right),
\end{align*}
%
and therefore
with $t \geq 1$
and since
$n h \gtrsim \log n$,
introducing and adjusting a new
constant $C_2$ where necessary,
%
\begin{align*}
&\P\left(
\left|
\frac{6}{n(n-1)(n-2)}
\sum_{i<j<r}
\Big(
k_{i j}
- \E[k_{i j} \mid \bA_n]
\Big)
\Big(
k_{i r}'
- \E[k_{i r}' \mid \bA_n]
\Big)
\right|
> t
\frac{\log n}{\sqrt{n^3 h^2}}
\Bigm\vert \bA_n
\right) \\
&\quad\leq
\P\left(
\left|
\sum_{i<j<r}
\Big(
k_{i j}
- \E[k_{i j} \mid \bA_n]
\Big)
\Big(
k_{i r}'
- \E[k_{i r}' \mid \bA_n]
\Big)
\right|
> t
n^{3/2} h^{-1} \log n / 24
\Bigm\vert \bA_n
\right) \\
&\quad\leq
C_2 \exp\left(
-\frac{1}{C_2}
\min \left\{
(t \log n)^2,
t \log n,
(t \log n)^{2/3} (n h)^{1/3},
(t n h \log n)^{1/2} n^{1/4}
\right\}
\right) \\
&\quad\leq
C_2 \exp\left(
-\frac{1}{C_2}
\min \left\{
t \log n,
t \log n,
t^{2/3} \log n,
t^{1/2} n^{1/4} \log n
\right\}
\right) \\
&\quad=
C_2 \exp\left(
-\frac{t^{2/3} \log n}{C_2}
\right)
=
C_2
n^{-t^{2/3} / C_2}.
\end{align*}
%
Now for the term
in \eqref{eq:kernel_app_Sijr1_decomp2},
note that
$\frac{3}{n} \sum_{r=j+1}^n \E[k_{i r}' \mid \bA_n]$
is $\bA_n$-measurable and bounded uniformly in $i,j$.
Also, using the previously established conditional variance
and almost sure bounds on $k_{i j}$,
Bernstein's inequality
(Lemma~\ref{lem:kernel_app_bernstein})
applied conditionally
gives for some constant $C_3 > 0$
%
\begin{align*}
&\P\left(
\Bigg|
\frac{2}{(n-1)(n-2)}
\sum_{i=1}^{n-2}
\sum_{j=i+1}^{n-1}
\Big(
k_{i j}
- \E[k_{i j} \mid \bA_n]
\Big)
\cdot \frac{3}{n}
\sum_{r=j+1}^n
\E[k_{i r}' \mid \bA_n]
\Bigg|
> t
\sqrt{\frac{\log n}{n^2h}}
\Bigm\vert \bA_n
\right) \\
&\qquad\leq
2 \exp \left( -
\frac{t^2 n^2 \log n / (n^2h)}
{C_3/(2h) + C_3 t \sqrt{\log n / (n^2h)} / (2h)}
\right) \\
&\qquad=
2 \exp \left( -
\frac{t^2 \log n}
{C_3/2 + C_3 t \sqrt{\log n / (n^2h)} / 2}
\right)
\leq
2 \exp \left( -
\frac{t^2 \log n}{C_3}
\right)
=
2 n^{-t^2 / C_3}.
\end{align*}
%
The term in \eqref{eq:kernel_app_Sijr1_decomp3}
is controlled in exactly the same way.
Putting these together, noting the symmetry in $i,j,r$
and taking a marginal expectation,
we obtain the unconditional pointwise concentration inequality
%
\begin{align*}
\P\left(
\Bigg|
\frac{6}{n(n-1)(n-2)}
\sum_{i<j<r}
S_{i j r}^{(1)}
\Bigg|
> t
\frac{\log n}{\sqrt{n^3h^2}}
+ t \sqrt{\frac{\log n}{n^2h}}
\right)
&\leq
C_2 n^{-t^{2/3} / C_2}
+ 4 n^{-t^2 / (4C_3)}.
\end{align*}
%
Multiplying by
$\big(\Sigma_n(w,w) + \Sigma_n(w',w')\big)^{-1/2} \lesssim \sqrt{n^2h}$
gives (adjusting constants if necessary)
%
\begin{align*}
&\P\left(
\Bigg|
\frac{6}{n(n-1)(n-2)}
\sum_{i<j<r}
\frac{S_{i j r}^{(1)}}
{\sqrt{\Sigma_n(w,w) + \Sigma_n(w',w')}}
\Bigg|
> t \frac{\log n}{\sqrt{n h}}
+ t \sqrt{\log n}
\right) \\
&\quad\leq
C_2 n^{-t^{2/3} / C_2}
+ 4 n^{-t^2 / (4C_3)}.
\end{align*}

\proofparagraph{pointwise concentration of the $S_{i j r}^{(2)}$ term}

We apply the U-statistic concentration inequality from
Lemma~\ref{lem:kernel_app_ustat_concentration}.
Note that the terms
$\E[S_{i j r} \mid \bA_n]$
are permutation-symmetric functions of
the random variables
$A_i, A_j$, and $A_r$ only,
making $S_{i j r}^{(2)}$ the summands of
a (non-degenerate) mean-zero third-order U-statistic.
While we could apply a third-order Hoeffding decomposition
here to achieve degeneracy,
it is unnecessary as Lemma~\ref{lem:kernel_app_ustat_concentration}
is general enough to deal with the non-degenerate case directly.
The quantity of interest here is
%
\begin{align*}
\frac{6}{n(n-1)(n-2)}
\sum_{i<j<r}
S_{i j r}^{(2)}
&=
\frac{6}{n(n-1)(n-2)}
\sum_{i<j<r}
\Big(
\E[S_{i j r} \mid \bA_n]
- \E[S_{i j r}]
\Big).
\end{align*}
%
Note that by conditional independence,
%
\begin{align*}
\big|
\E\big[
k_{i j}k_{i r} \mid \bA_n
\big]
\big|
&=
\big|
\E\big[
k_{i j} \mid \bA_n
\big]
\E\big[
k_{i r} \mid \bA_n
\big]
\big|
\lesssim 1,
\end{align*}
%
and similarly for the other summands in $S_{i j r}$,
giving the almost sure bound
$|S_{i j r}^{(2)}| \lesssim 1$. Also,
%
\begin{align*}
\Var\big[ \E[k_{i j} \mid A_i] \E[k_{i r}' \mid A_i] \big]
&\lesssim
\Var\big[\E[k_{i j} \mid A_i]\big]
+ \Var\big[\E[k_{i r}' \mid A_i]\big] \\
&\lesssim
n \Var[L_n(w)] + n \Var[L_n(w')] \\
&\lesssim
n \Sigma_n(w,w) + n \Sigma_n(w',w')
\end{align*}
%
and similarly for the other summands in $S_{i j r}$,
giving the conditional variance bound
%
\begin{align*}
\E[\E[S_{i j r}^{(2)} \mid A_i]^2] \lesssim
n \Sigma_n(w,w) + n \Sigma_n(w',w').
\end{align*}
%
So Lemma~\ref{lem:kernel_app_ustat_concentration}
and Lemma~\ref{lem:kernel_variance_bounds}
give the pointwise concentration inequality
%
\begin{align*}
&\P\left(
\Bigg|
\frac{6}{n(n-1)(n-2)}
\sum_{i<j<r}
S_{i j r}^{(2)}
\Bigg|
> t \sqrt{\log n} \sqrt{\Sigma_n(w,w) + \Sigma_n(w',w')}
\right) \\
&\quad\leq
4 \exp \left(
- \frac{n t^2 (\Sigma_n(w,w) + \Sigma_n(w',w')) \log n}
{C_4 (n\Sigma_n(w,w) + n\Sigma_n(w',w'))
+ C_4 t \sqrt{\Sigma_n(w,w) + \Sigma_n(w',w')}\sqrt{\log n}}
\right) \\
&\quad\leq
4 \exp \left(
- \frac{t^2 \log n}
{C_4
+ C_4 t (\Sigma_n(w,w) + \Sigma_n(w',w'))^{-1/2} \sqrt{\log n} / n}
\right) \\
&\quad\leq
4 \exp \left(
- \frac{t^2 \log n}
{C_4
+ C_4 t \sqrt{h}}
\right)
\leq
4 n^{-t^2 / C_4}
\end{align*}
%
for some universal constant $C_4 > 0$
(which may change from line to line),
since the order of this U-statistic is fixed at three.

\proofparagraph{concentration of the $S_{i j r}$ term on a mesh}

Pick $\delta_n \to 0$
with $\log 1/\delta_n \lesssim \log n$.
Let $\cW_\delta$ be a $\delta_n$-covering of $\cW$
with cardinality $O(1/\delta_n)$.
Then $\cW_\delta \times \cW_\delta$
is a $2\delta_n$-covering of $\cW \times \cW$
with cardinality $O(1/\delta_n^2)$,
under the Manhattan metric
$d\big((w_1, w_1'), (w_2, w_2')\big)
= |w_1 - w_2| + |w_1' - w_2'|$.
By the previous parts,
we have that for fixed $w$ and $w'$:
%
\begin{align*}
&\P\Bigg(
\Bigg|
\frac{6}{n(n-1)(n-2)}
\sum_{i<j<r}
\frac{S_{i j r}(w,w') - \E[S_{i j r}(w,w')]}
{\sqrt{\Sigma_n(w,w) + \Sigma_n(w',w')}}
\Bigg|
> t \frac{\log n}{\sqrt{n h}}
+ 2t \sqrt{\log n}
\Bigg) \\
&\quad\leq
C_2 n^{-t^{2/3} / C_2}
+ 4 n^{-t^2 / (4C_3)}
+ 4 n^{-t^2 / C_4}.
\end{align*}
%
Taking a union bound over $\cW_\delta \times \cW_\delta$,
noting that $n h \gtrsim \log n$
and adjusting constants gives
%
\begin{align*}
&\P\Bigg(
\sup_{w, w' \in \cW_\delta}
\Bigg|
\frac{6}{n(n-1)(n-2)}
\sum_{i<j<r}
\frac{S_{i j r}(w,w') - \E[S_{i j r}(w,w')]}
{\sqrt{\Sigma_n(w,w) + \Sigma_n(w',w')}}
\Bigg|
> t \sqrt{\log n}
\Bigg) \\
&\quad\lesssim
\delta_n^{-2}
\Big(
C_2 n^{-t^{2/3} / C_2}
+ 4 n^{-t^2 / (4C_3)}
+ 4 n^{-t^2 / C_4}
\Big)
\lesssim
\delta_n^{-2}
n^{-t^{2/3} / C_5},
\end{align*}
%
for some constant $C_5 > 0$.

\proofparagraph{regularity of the $S_{i j r}$ term}

Next we bound the fluctuations in $S_{i j r}(w,w')$.
Writing $k_{i j}(w)$ for $k_h(W_{i j},w)$,
note that
%
\begin{align*}
\big|
k_{i j}(w_1)
k_{i r}(w_1')
- k_{i j}(w_2)
k_{i r}(w_2')
\big|
&\lesssim
\frac{1}{h}
\big| k_{i j}(w_1) - k_{i j}(w_2) \big|
+
\frac{1}{h}
\big| k_{i r}(w_1') - k_{i r}(w_2') \big| \\
&\lesssim
\frac{1}{h^3}
\Big(
|w_1 - w_2|
+ |w_1' - w_2'|
\Big),
\end{align*}
%
by the Lipschitz property of the kernel,
and similarly for the other summands in $S_{i j r}$.
Therefore,
%
\begin{align*}
\sup_{|w_1-w_2| \leq \delta_n}
\sup_{|w_1'-w_2'| \leq \delta_n}
\big|
S_{i j r}(w_1, w_1')
- S_{i j r}(w_2, w_2')
\big|
&\lesssim
\delta_n h^{-3}.
\end{align*}
%
Also as noted in the proof of Theorem~\ref{thm:kernel_infeasible_ucb},
%
\begin{align*}
\sup_{|w_1-w_2| \leq \delta_n}
\sup_{|w_1'-w_2'| \leq \delta_n}
\big|
\Sigma_n(w_1,w_1')
-
\Sigma_n(w_2, w_2')
\big|
&\lesssim
\delta_n n^{-1}h^{-3}.
\end{align*}
%
Therefore, since $\sqrt{\Sigma_n(w,w)} \gtrsim \sqrt{n^2h}$
and $|S_{i j r}| \lesssim h^{-2}$,
using
$\frac{a}{\sqrt b} - \frac{c}{\sqrt d}
= \frac{a-c}{\sqrt b} + c \frac{d-b}{\sqrt{b d} \sqrt{b+d}}$,
%
\begin{align*}
&\sup_{|w_1-w_2| \leq \delta_n}
\sup_{|w_1'-w_2'| \leq \delta_n}
\left|
\frac{S_{i j r}(w_1, w_1')}
{\sqrt{\Sigma_n(w_1,w_1) + \Sigma_n(w_1',w_1')}}
- \frac{S_{i j r}(w_2, w_2')}
{\sqrt{\Sigma_n(w_2,w_2) + \Sigma_n(w_2',w_2')}}
\right| \\
&\quad\lesssim
\delta_n h^{-3} \sqrt{n^2h}
+ h^{-2} \delta_n n^{-1} h^{-3} (n^2h)^{3/2}
\lesssim
\delta_n n h^{-5/2}
+ \delta_n n^{2} h^{-7/2}
\lesssim
\delta_n n^{6},
\end{align*}
%
where in the last line we use that
$1/h \lesssim n$.

\proofparagraph{uniform concentration of the $S_{i j r}$ term}

By setting
$\delta_n = n^{-6} \sqrt{\log n}$,
the fluctuations can be at most $\sqrt{\log n}$,
so we have for $t \geq 1$
%
\begin{align*}
&\P\Bigg(
\sup_{w, w' \in \cW}
\Bigg|
\frac{6}{n(n-1)(n-2)}
\sum_{i<j<r}
\frac{S_{i j r}(w,w') - \E[S_{i j r}(w,w')]}
{\sqrt{\Sigma_n(w,w) + \Sigma_n(w',w')}}
\Bigg|
> 2t \sqrt{\log n}
\Bigg) \\
&\quad\lesssim
\delta_n^{-2}
n^{-t^{2/3} / C_5}
\lesssim
n^{12-t^{2/3} / C_5}.
\end{align*}
%
This converges to zero for any sufficiently large $t$, so
%
\begin{align*}
\sup_{w, w' \in \cW}
\Bigg|
\frac{6}{n(n-1)(n-2)}
\sum_{i<j<r}
\frac{S_{i j r}(w,w') - \E[S_{i j r}(w,w')]}
{\sqrt{\Sigma_n(w,w) + \Sigma_n(w',w')}}
\Bigg|
&\lesssim_\P
\sqrt{\log n}.
\end{align*}

\proofparagraph{decomposition of the $k_{i j}k_{i j}'$ term}

We move on to the final term in
the covariance estimator.
We have the decomposition
%
\begin{align*}
\frac{2}{n(n-1)}
\sum_{i<j}
\Big(
k_{i j} k_{i j}'
- \E\big[k_{i j} k_{i j}']
\Big)
&=
\frac{2}{n(n-1)}
\sum_{i<j}
S_{i j}^{(1)}
+
\frac{2}{n(n-1)}
\sum_{i<j}
S_{i j}^{(2)},
\end{align*}
%
where
%
\begin{align*}
S_{i j}^{(1)}
&=
k_{i j} k_{i j}'
- \E\big[ k_{i j} k_{i j}' \mid \bA_n \big],
&S_{i j}^{(2)}
&=
\E\big[ k_{i j} k_{i j}' \mid \bA_n \big]
- \E\big[ k_{i j} k_{i j}' \big].
\end{align*}

\proofparagraph{pointwise concentration of the $S_{i j}^{(1)}$ term}

Conditioning on $\bA_n$,
the variables $S_{i j}^{(1)}$
are conditionally independent
and conditionally mean-zero.
They further satisfy
$|S_{i j}^{(1)}| \lesssim h^{-2}$
and the conditional variance bound
$\E\big[\big( S_{i j}^{(1)} \big)^2 \mid \bA_n \big] \lesssim h^{-3}$.
Therefore applying Bernstein's inequality
(Lemma~\ref{lem:kernel_app_bernstein})
conditional on $\bA_n$,
we obtain the pointwise in $w,w'$
concentration inequality
%
\begin{align*}
&\P\left(
\Bigg|
\frac{2}{n(n-1)}
\sum_{i<j}
S_{i j}^{(1)}
\Bigg|
> t
\sqrt{\frac{\log n}{n^2h^3}}
\Bigm\vert \bA_n
\right) \\
&\quad\leq
2 \exp\left(
- \frac{t^2 n^2 \log n / (n^2h^3)}
{C_6 h^{-3} / 2 + C_6 t h^{-2} \sqrt{\log n / (n^2h^3)} / 2}
\right) \\
&\quad\leq
2 \exp\left(
- \frac{t^2 \log n}
{C_6 / 2 + C_6 t \sqrt{\log n / (n^2h)} / 2}
\right)
\leq
2 \exp\left( - \frac{t^2 \log n}{C_6} \right)
= 2 n^{-t^2 / C_6},
\end{align*}
%
where $C_6$ is a universal positive constant.

\proofparagraph{pointwise concentration of the $S_{i j}^{(2)}$ term}

We apply the U-statistic concentration inequality from
Lemma~\ref{lem:kernel_app_ustat_concentration}.
Note that $S_{i j}^{(2)}$
are permutation-symmetric functions of
the random variables
$A_i$ and $A_j$ only,
making them the summands of
a (non-degenerate) mean-zero second-order U-statistic.
Note that
$\big|S_{i j}^{(2)}\big| \lesssim h^{-1}$
and so trivially
$\E\big[\E[S_{i j}^{(2)} \mid A_i ]^2 \big] \lesssim h^{-2}$.
Thus by Lemma~\ref{lem:kernel_app_ustat_concentration},
since the order of this U-statistic is fixed at two,
for some universal positive constant $C_7$ we have
%
\begin{align*}
\P\left(
\Bigg|
\frac{2}{n(n-1)}
\sum_{i<j}
S_{i j}^{(2)}
\Bigg|
> t
\sqrt{\frac{\log n}{n h^2}}
\right)
&\leq
2 \exp\left(
- \frac{t^2 n \log n / (n h^2)}
{C_7 h^{-2} / 2 + C_7 t h^{-1} \sqrt{\log n / (n h^2)} / 2}
\right) \\
&\leq
2 \exp\left(
- \frac{t^2 \log n}
{C_7 / 2 + C_7 t \sqrt{\log n / n} / 2}
\right) \\
&\leq
2 \exp\left(
- \frac{t^2 \log n}{C_7}
\right)
=
2 n^{-t^2 / C_7}.
\end{align*}

\proofparagraph{concentration of the $k_{i j}k_{i j}'$ term on a mesh}

As before, use a union bound
on the mesh $\cW_\delta \times \cW_\delta$.
%
\begin{align*}
&\P\left(
\sup_{w,w' \in \cW_\delta}
\left|
\frac{2}{n(n-1)}
\sum_{i<j}
\Big(
k_{i j} k_{i j}'
- \E\big[k_{i j} k_{i j}']
\Big)
\right|
> t \sqrt{\frac{\log n}{n^2h^3}}
+ t \sqrt{\frac{\log n}{n h^2}}
\right) \\
&\ \leq
\P\!\left(
\!\sup_{w,w' \in \cW_\delta}
\Bigg|
\frac{2}{n(n-1)}
\sum_{i<j}
S_{i j}^{(1)}
\Bigg|
> t
\sqrt{\frac{\log n}{n^2h^3}}
\right)
\! + \P\!\left(
\!\sup_{w,w' \in \cW_\delta}
\Bigg|
\frac{2}{n(n-1)}
\sum_{i<j}
S_{i j}^{(2)}
\Bigg|
> t
\sqrt{\frac{\log n}{n h^2}}
\right) \\
&\ \lesssim
\delta_n^{-2} n^{-t^2 / C_6}
+ \delta_n^{-2} n^{-t^2 / C_7}.
\end{align*}

\proofparagraph{regularity of the $k_{i j}k_{i j}'$ term}

As for the $S_{i j r}$ term,
%
$\big| k_{i j}(w_1) k_{i j}(w_1') - k_{i j}(w_2) k_{i j}(w_2') \big|
\lesssim \frac{1}{h^3} \Big( |w_1 - w_2| + |w_1' - w_2'| \Big)$.

\proofparagraph{uniform concentration of the $k_{i j}k_{i j}'$ term}

Setting $\delta_n = h^3\sqrt{\log n / (n h^2)}$,
the fluctuations are at most $\sqrt{\log n / (n h^2)}$,
so for $t \geq 1$
%
\begin{align*}
&\P\left(
\sup_{w,w' \in \cW}
\left|
\frac{2}{n(n-1)}
\sum_{i<j}
\Big(
k_{i j} k_{i j}'
- \E\big[k_{i j} k_{i j}']
\Big)
\right|
> t \sqrt{\frac{\log n}{n^2h^3}}
+ 2t \sqrt{\frac{\log n}{n h^2}}
\right) \\
&\quad\leq
\P\left(
\sup_{w,w' \in \cW_\delta}
\left|
\frac{2}{n(n-1)}
\sum_{i<j}
\Big(
k_{i j} k_{i j}'
- \E\big[k_{i j} k_{i j}']
\Big)
\right|
> t \sqrt{\frac{\log n}{n^2h^3}}
+ t \sqrt{\frac{\log n}{n h^2}}
\right) \\
&\qquad+
\P\left(
\sup_{|w_1-w_2| \leq \delta_n}
\sup_{|w_1'-w_2'| \leq \delta_n}
\big|
k_{i j}(w_1)
k_{i j}(w_1')
- k_{i j}(w_2)
k_{i j}(w_2')
\big|
> t \sqrt{\frac{\log n}{n h^2}}
\right) \\
&\quad\lesssim
\delta_n^{-2} n^{-t^2 / C_6}
+ \delta_n^{-2} n^{-t^2 / C_7}
\lesssim
n^{1-t^2 / C_6} h^{-4}
+ n^{1-t^2 / C_7} h^{-4}
\lesssim
n^{5-t^2 / C_8},
\end{align*}
%
where $C_8 > 0$ is a constant and
in the last line we use $1/h \lesssim n$.
This converges to zero for any sufficiently large $t$,
so by Lemma~\ref{lem:kernel_variance_bounds} we have
%
\begin{align*}
\sup_{w,w' \in \cW}
\left|
\frac{2}{n(n-1)}
\sum_{i<j}
\frac{k_{i j} k_{i j}' - \E\big[k_{i j} k_{i j}']}
{\sqrt{\Sigma_n(w,w) + \Sigma_n(w',w')}}
\right|
&\lesssim_\P
\left(
\!\sqrt{\frac{\log n}{n^2h^3}}
+ \sqrt{\frac{\log n}{n h^2}}
\right)
\sqrt{n^2h}
\lesssim_\P
\sqrt{\frac{n \log n}{h}}.
\end{align*}

\proofparagraph{conclusion}

By the uniform bounds derived in the previous parts,
and with $n h \gtrsim \log n$, we conclude that
%
\begin{align*}
&\sup_{w,w' \in \cW}
\left|
\frac{\hat \Sigma_n(w,w') - \Sigma_n(w,w')}
{\sqrt{\Sigma_n(w,w) + \Sigma_n(w',w')}}
\right|
\leq
\frac{2}{n(n-1)}
\sup_{w,w' \in \cW}
\left|
\frac{2}{n(n-1)}
\!\sum_{i<j}\!
\frac{k_{i j} k_{i j}' - \E\big[k_{i j} k_{i j}']}
{\sqrt{\Sigma_n(w,w) + \Sigma_n(w',w')}}
\right| \\
&\qquad+
\frac{4(n-2)}{n(n-1)}
\sup_{w,w' \in \cW}
\left|
\frac{6}{n(n-1)(n-2)}
\sum_{i<j<r}
\frac{S_{i j r} - \E\big[k_{i j} k_{i r}']}
{\sqrt{\Sigma_n(w,w) + \Sigma_n(w',w')}}
\right| \\
&\qquad+
\frac{4n-6}{n(n-1)}
\sup_{w,w' \in \cW}
\left|
\frac{\hat f_W(w) \hat f_W(w') - \E[k_{i j}] \E[k_{i j}']}
{\sqrt{\Sigma_n(w,w) + \Sigma_n(w',w')}}
\right| \\
&\quad\lesssim_\P
\sqrt{\frac{\log n}{n^3h}}
+ \frac{\sqrt{\log n}}{n}
+ \frac{\sqrt{\log n}}{n}
\lesssim_\P
\frac{\sqrt{\log n}}{n}.
\end{align*}
\end{proof}

\begin{proof}[Lemma~\ref{lem:kernel_app_alternative_covariance_estimator}]

Write $k_{i j}$ for $k_h(W_{i j},w)$
if $i<j$ and $k_h(W_{j i},w)$ if $j<i$,
and use a prime to denote evaluation at $w'$.
Thus we write $S_i(w) = \frac{1}{n-1} \sum_{j \neq i} k_{i j}$.
Let $\sum_{i \neq j \neq r}$ indicate all indices are distinct.
%
\begin{align*}
\frac{4}{n^2}
\sum_{i=1}^n
S_i(w) S_i(w')
&=
\frac{4}{n^2}
\sum_{i=1}^n
\frac{1}{n-1}
\sum_{j \neq i}
k_{i j}
\frac{1}{n-1}
\sum_{r \neq i}
k_{i r}'
=
\frac{4}{n^2(n-1)^2}
\sum_{i=1}^n
\sum_{j \neq i}
\sum_{r \neq i}
k_{i j}
k_{i r}' \\
&=
\frac{4}{n^2(n-1)^2}
\sum_{i=1}^n
\sum_{j \neq i}
\left(
\sum_{r \neq i, r \neq j}
k_{i j}
k_{i r}'
+ k_{i j}
k_{i j}'
\right) \\
&=
\frac{4}{n^2(n-1)^2}
\sum_{i \neq j \neq r}
k_{i j}
k_{i r}'
+ \frac{4}{n^2(n-1)^2}
\sum_{i \neq j}
k_{i j}
k_{i j}' \\
&=
\frac{24}{n^2(n-1)^2}
\sum_{i < j < r}
S_{i j r}(w,w')
+ \frac{8}{n^2(n-1)^2}
\sum_{i < j}
k_{i j}
k_{i j}' \\
&=
\hat \Sigma_n(w,w')
+ \frac{4}{n^2(n-1)^2}
\sum_{i < j}
k_{i j}
k_{i r}'
+ \frac{4n-6}{n(n-1)}
\hat f
\hat f'.
\end{align*}
%
\end{proof}

\begin{proof}[Lemma~\ref{lem:kernel_app_sdp}]

Firstly, we prove that the true covariance function
$\Sigma_n$
is feasible for the optimization problem
\eqref{eq:kernel_app_sdp} in the sense that it satisfies the constraints.
As a covariance function, it is symmetric and positive semi-definite.
The Lipschitz constraint is established in the proof of
Theorem~\ref{thm:kernel_infeasible_ucb}:
%
\begin{align*}
\big| \Sigma_n(w,w') - \Sigma_n(w, w'') \big|
&\leq
\frac{4}{n h^3}
C_\rk
C_\rL
|w'-w''|
\end{align*}
%
for all $w,w',w'' \in \cW$.
Denote the (random) objective function
in \eqref{eq:kernel_app_sdp} by
%
\begin{align*}
\objective(M) = \sup_{w,w' \in \cW}
\left|
\frac{M(w,w') - \hat\Sigma_n(w,w')}
{\sqrt{\hat \Sigma_n(w,w) + \hat \Sigma_n(w',w')}}
\right|.
\end{align*}
%
By Lemma~\ref{lem:kernel_app_covariance_estimation}
with $w = w'$ we deduce that
$\sup_{w \in \cW}
\left|\frac{\hat \Sigma_n(w,w)}{\Sigma_n(w,w)} - 1\right|
\lesssim_\P \sqrt{h \log n}$
and so
%
\begin{align*}
\objective(\Sigma_n)
&= \sup_{w,w' \in \cW}
\left|
\frac{\hat\Sigma_n(w,w') - \Sigma_n}
{\sqrt{\Sigma_n(w,w) + \Sigma_n(w',w')}}
\right|
\sqrt{\frac{\Sigma_n(w,w) + \Sigma_n(w',w')}
{\hat \Sigma_n(w,w) + \hat \Sigma_n(w',w')}} \\
&\lesssim_\P
\frac{\sqrt{\log n}}{n}
\left(
1 - \frac{\big|\hat \Sigma_n(w,w) - \Sigma_n(w,w)\big|}
{\Sigma_n(w,w)}
- \frac{\big|\hat \Sigma_n(w',w') - \Sigma_n(w',w')\big|}
{\Sigma_n(w',w')}
\right)^{-1/2} \\
&\lesssim_\P
\frac{\sqrt{\log n}}{n}
\left(
1 - \sqrt{h \log n}
\right)^{-1/2}
\lesssim_\P
\frac{\sqrt{\log n}}{n}.
\end{align*}
%
Since the objective function
is non-negative and because we have established
at least one feasible function $M$ with
an almost surely finite objective value,
we can conclude the following.
Let $\objective^* = \inf_M \objective(M)$,
where the infimum is over feasible functions $M$.
Then for all $\varepsilon > 0$
there exists a feasible function $M_\varepsilon$ with
$\objective(M_\varepsilon) \leq \objective^* + \varepsilon$,
and we call such a solution $\varepsilon$-optimal.
Let $\hat \Sigma_n^+$ be an $n^{-1}$-optimal solution.
Then
%
\begin{align*}
\objective(\hat \Sigma_n^+)
&\leq \objective^* + n^{-1}
\leq \objective(\Sigma_n) + n^{-1}.
\end{align*}
%
Thus by the triangle inequality,
%
\begin{align*}
\sup_{w,w' \in \cW}
\left|
\frac{\hat \Sigma_n^+(w,w') - \Sigma_n(w,w')}
{\sqrt{\Sigma_n(w,w) + \Sigma_n(w',w')}}
\right|
&\leq
\objective(\hat \Sigma_n^+)
+ \objective(\Sigma_n)
\leq 2 \, \objective(\Sigma_n) + n^{-1}
\lesssim_\P
\frac{\sqrt{\log n}}{n}.
\end{align*}
\end{proof}

\begin{proof}[Lemma~\ref{lem:kernel_app_variance_estimator_bounds}]

Since $\hat \Sigma_n^+$ is positive semi-definite,
we must have $\hat \Sigma_n^+(w,w) \geq 0$.
Now Lemma~\ref{lem:kernel_app_sdp}
implies that for all $\varepsilon \in (0,1)$
there exists a $C_\varepsilon$ such that
%
\begin{align*}
&\P\left(
\Sigma_n(w,w) - C_\varepsilon \frac{\sqrt{\log n}}{n} \sqrt{\Sigma_n(w,w)}
\leq
\hat \Sigma_n^+(w,w)
\right.
\\
&\left.
\qquad\leq
\Sigma_n(w,w) + C_\varepsilon \frac{\sqrt{\log n}}{n}
\sqrt{\Sigma_n(w,w)},
\quad \forall w \in \cW
\right)
\geq 1-\varepsilon.
\end{align*}
%
Consider the function
$g_a(t) = t - a \sqrt{t}$
and note that it is increasing on $\{t \geq a^2/4\}$.
Applying this with $t = \Sigma_n(w,w)$
and $a = \frac{\sqrt{\log n}}{n}$,
noting that by Lemma~\ref{lem:kernel_variance_bounds} we have
$t = \Sigma_n(w,w) \gtrsim \frac{1}{n^2h}
\gg \frac{\log n}{4n^2} = a^2/4$,
shows that for $n$ large enough,
%
\begin{align*}
\inf_{w \in \cW} \Sigma_n(w,w)
- \frac{\sqrt{\log n}}{n} \sqrt{\inf_{w \in \cW} \Sigma_n(w,w)}
\lesssim_\P
\inf_{w \in \cW}\hat \Sigma_n^+(w,w), \\
\sup_{w \in \cW}\hat \Sigma_n^+(w,w)
\lesssim_\P
\sup_{w \in \cW} \Sigma_n(w,w)
+ \frac{\sqrt{\log n}}{n} \sqrt{\sup_{w \in \cW} \Sigma_n(w,w)}.
\end{align*}
%
Applying the bounds from Lemma~\ref{lem:kernel_variance_bounds}
yields
%
\begin{align*}
\frac{\Dl^2}{n} + \frac{1}{n^2h}
- \frac{\sqrt{\log n}}{n}
\left( \frac{\Dl}{\sqrt n} + \frac{1}{\sqrt{n^2h}} \right)
\lesssim_\P
\inf_{w \in \cW}\hat \Sigma_n^+(w,w), \\
\sup_{w \in \cW}\hat \Sigma_n^+(w,w)
\lesssim_\P
\frac{\Du^2}{n} + \frac{1}{n^2h}
+ \frac{\sqrt{\log n}}{n}
\left( \frac{\Du}{\sqrt n} + \frac{1}{\sqrt{n^2h}} \right)
\end{align*}
%
and so
%
\begin{align*}
\frac{\Dl^2}{n} + \frac{1}{n^2h}
\lesssim_\P
\inf_{w \in \cW}\hat \Sigma_n^+(w,w)
\leq
\sup_{w \in \cW}\hat \Sigma_n^+(w,w)
\lesssim_\P
\frac{\Du^2}{n} + \frac{1}{n^2h}.
\end{align*}
\end{proof}

\begin{proof}[Lemma~\ref{lem:kernel_sdp}]
See Lemma~\ref{lem:kernel_app_covariance_estimation}
and Lemma~\ref{lem:kernel_app_sdp}.
\end{proof}

\begin{proof}[Lemma~\ref{lem:kernel_app_studentized_t_statistic}]
%
We have
%
\begin{align*}
&\sup_{w \in \cW}
\left| \hat T_n(w) - T_n(w) \right|
=
\sup_{w \in \cW}
\bigg\{
\left|
\hat f_W(w) - f_W(w)
\right|
\cdot
\bigg|
\frac{1}
{\hat\Sigma_n^+(w,w)^{1/2}}
-
\frac{1}{\Sigma_n(w,w)^{1/2}}
\bigg|
\bigg\} \\
&\quad\leq
\sup_{w \in \cW}
\left|
\frac{\hat f_W(w) - \E\big[\hat f_W(w)\big]}
{\sqrt{\Sigma_n(w,w)}}
+ \frac{\E\big[\hat f_W(w)\big] - f_W(w)}
{\sqrt{\Sigma_n(w,w)}}
\right|
\cdot \sup_{w \in \cW}
\left|
\frac{\hat\Sigma_n^+(w,w) - \Sigma_n(w,w)}
{\sqrt{\Sigma_n(w,w) \hat\Sigma_n^+(w,w)}}
\right|.
\end{align*}
%
Now from the proof of Lemma~\ref{lem:kernel_app_covariance_estimation} we
have that
$\sup_{w \in \cW} \left|
\frac{\hat f_W(w) - \E\big[\hat f_W(w)\big]}
{\sqrt{\Sigma_n(w,w)}} \right|
\lesssim_\P \sqrt{\log n}$,
while Theorem~\ref{thm:kernel_bias} gives
$\sup_{w \in \cW} \big| \E\big[\hat f_W(w)\big] - f_W(w) \big|
\lesssim h^{p \wedge \beta}$.
By Lemma~\ref{lem:kernel_variance_bounds},
note that
$\sup_{w \in \cW} \Sigma_n(w,w)^{-1/2}
\lesssim \frac{1}{\Dl/\sqrt{n} + 1/\sqrt{n^2h}}$, and
$\sup_{w \in \cW} \hat \Sigma_n^+(w,w)^{-1/2}
\lesssim_\P \frac{1}{\Dl/\sqrt{n} + 1/\sqrt{n^2h}}$
by Lemma~\ref{lem:kernel_app_variance_estimator_bounds}.
Thus, applying Lemma~\ref{lem:kernel_app_sdp} to control the
covariance estimation error,
%
\begin{align*}
\sup_{w \in \cW}
\left| \hat T_n(w) - T_n(w) \right|
&\lesssim_\P
\left(
\sqrt{\log n} + \frac{h^{p \wedge \beta}}{\Dl/\sqrt{n} + 1/\sqrt{n^2h}}
\right)
\frac{\sqrt{\log n}}{n}
\frac{1}{\Dl/\sqrt{n} + 1/\sqrt{n^2h}} \\
&\lesssim_\P
\sqrt{\frac{\log n}{n}}
\left(
\sqrt{\log n} + \frac{\sqrt n h^{p \wedge \beta}}
{\Dl + 1/\sqrt{n h}}
\right)
\frac{1}{\Dl + 1/\sqrt{n h}}.
\end{align*}
\end{proof}

\begin{proof}[%
Lemma~\ref{lem:kernel_app_distributional_approx_feasible_gaussian}]

Firstly, note that $\hat Z_n^T$ exists
by noting that $\hat \Sigma_n^+(w,w')$ and therefore also
$\frac{\hat \Sigma_n^+(w,w')}
{\sqrt{\hat \Sigma_n^+(w,w) \hat \Sigma_n^+(w',w')}}$
are positive semi-definite
functions and appealing to the
Kolmogorov consistency theorem \citep{gine2021mathematical}.
To obtain the desired Kolmogorov--Smirnov result we discretize and
use the Gaussian--Gaussian comparison result found in
Lemma~3.1 in \citet{chernozhukov2013gaussian}.

\proofparagraph{bounding the covariance discrepancy}

Define the maximum discrepancy in the (conditional) covariances
of $\hat Z_n^T$ and $Z_n^T$ by
%
\begin{align*}
\Delta
&\vcentcolon=
\sup_{w, w' \in \cW}
\left|
\frac{\hat \Sigma_n^+(w,w')}
{\sqrt{\hat \Sigma_n^+(w,w) \hat \Sigma_n^+(w',w')}}
- \frac{\Sigma_n(w,w')}
{\sqrt{\Sigma_n(w,w) \Sigma_n(w',w')}}
\right|.
\end{align*}
%
This variable can be bounded in probability
in the following manner.
First note that by the Cauchy--Schwarz inequality
for covariances,
$|\Sigma_n(w,w')| \leq
\sqrt{\Sigma_n(w,w) \Sigma_n(w',w')}$.
Hence
%
\begin{align*}
\Delta
&\leq
\sup_{w, w' \in \cW}
\left\{
\left|
\frac{\hat \Sigma_n^+(w,w') - \Sigma_n(w,w')}
{\sqrt{\hat \Sigma_n^+(w,w) \hat \Sigma_n^+(w',w')}}
\right|
+ \left|
\frac{\sqrt{\hat \Sigma_n^+(w,w) \hat \Sigma_n^+(w',w')}
- \sqrt{\Sigma_n(w,w) \Sigma_n(w',w')}}
{\sqrt{\hat \Sigma_n^+(w,w) \hat \Sigma_n^+(w',w')}}
\right|
\right\} \\
&\leq
\sup_{w, w' \in \cW}
\left\{
\sqrt{\frac{\Sigma_n(w,w) + \Sigma_n(w',w')}
{\hat \Sigma_n^+(w,w) \hat \Sigma_n^+(w',w')}}
\left|
\frac{\hat \Sigma_n^+(w,w') - \Sigma_n(w,w')}
{\sqrt{\Sigma_n(w,w) + \Sigma_n(w',w')}}
\right|
\right\} \\
&\quad+
\sup_{w, w' \in \cW}
\left|
\frac{\hat \Sigma_n^+(w,w)\hat \Sigma_n^+(w',w')
- \Sigma_n(w,w) \Sigma_n(w',w')}
{\sqrt{\hat \Sigma_n^+(w,w) \hat \Sigma_n^+(w',w')
\Sigma_n(w,w) \Sigma_n(w',w')}}
\right|.
\end{align*}
%
For the first term, note that
$\inf_{w \in \cW} \hat \Sigma_n^+(w,w)
\gtrsim \frac{\Dl^2}{n} + \frac{1}{n^2h}$
by Lemma~\ref{lem:kernel_app_variance_estimator_bounds} and also
$\sup_{w \in \cW}
\left|\frac{\hat \Sigma_n(w,w)}{\Sigma_n(w,w)} - 1\right|
\lesssim_\P \sqrt{h \log n}$
by the proof of Lemma~\ref{lem:kernel_app_sdp}.
Thus by Lemma~\ref{lem:kernel_app_sdp},
%
\begin{align*}
&\sup_{w, w' \in \cW}
\left\{
\sqrt{\frac{\Sigma_n(w,w) + \Sigma_n(w',w')}
{\hat \Sigma_n^+(w,w) \hat \Sigma_n^+(w',w')}}
\left|
\frac{\hat \Sigma_n^+(w,w') - \Sigma_n(w,w')}
{\sqrt{\Sigma_n(w,w) + \Sigma_n(w',w')}}
\right|
\right\} \\
&\quad\lesssim_\P
\frac{\sqrt{\log n}}{n}
\frac{1}{\Dl/\sqrt{n} + 1/\sqrt{n^2h}}
\lesssim_\P
\sqrt{\frac{\log n}{n}}
\frac{1}{\Dl + 1/\sqrt{n h}}.
\end{align*}
%
For the second term, we have by the same bounds
%
\begin{align*}
&\sup_{w, w' \in \cW}
\left|
\frac{\hat \Sigma_n^+(w,w) \hat \Sigma_n^+(w',w')
- \Sigma_n(w,w) \Sigma_n(w',w')}
{\sqrt{\hat \Sigma_n^+(w,w) \hat \Sigma_n^+(w',w')
\Sigma_n(w,w) \Sigma_n(w',w')}}
\right| \\
&\quad\leq
\sup_{w, w' \in \cW}
\left\{
\frac{\big| \hat \Sigma_n^+(w,w) - \Sigma_n(w,w)\big|
\hat \Sigma_n^+(w',w')}
{\sqrt{\hat \Sigma_n^+(w,w) \hat \Sigma_n^+(w',w')
\Sigma_n(w,w) \Sigma_n(w',w')}}
\right\} \\
&\qquad+
\sup_{w, w' \in \cW}
\left\{
\frac{\big| \hat \Sigma_n^+(w',w') - \Sigma_n(w',w')\big|
\Sigma_n(w,w)}
{\sqrt{\hat \Sigma_n^+(w,w) \hat \Sigma_n^+(w',w')
\Sigma_n(w,w) \Sigma_n(w',w')}}
\right\} \\
&\quad\leq
\sup_{w, w' \in \cW}
\left\{
\frac{\big| \hat \Sigma_n^+(w,w) - \Sigma_n(w,w)\big|}
{\sqrt{\Sigma_n(w,w)}}
\frac{\sqrt{\hat \Sigma_n^+(w',w')}}
{\sqrt{\hat \Sigma_n^+(w,w) \Sigma_n(w',w')}}
\right\} \\
&\qquad+
\!\sup_{w, w' \in \cW}\!
\left\{
\frac{\big| \hat \Sigma_n^+(w',w') - \Sigma_n(w',w')\big|}
{\sqrt{\Sigma_n(w',w')}}
\frac{\sqrt{\Sigma_n(w,w)}}
{\sqrt{\hat \Sigma_n^+(w,w) \hat \Sigma_n^+(w',w')}}
\right\}
\lesssim_\P
\sqrt{\frac{\log n}{n}}
\frac{1}{\Dl + 1/\sqrt{n h}}.
\end{align*}
%
Therefore
$\Delta \lesssim_\P \sqrt{\frac{\log n}{n}} \frac{1}{\Dl + 1/\sqrt{n h}}$.

\proofparagraph{Gaussian comparison on a mesh}

Let $\cW_\delta$ be a $\delta_n$-covering of $\cW$
with cardinality $O(1/\delta_n)$,
where $1/\delta_n$ is at most polynomial in $n$.
The scaled (conditionally) Gaussian
processes $Z_n^T$ and $\hat Z_n^T$
both have pointwise (conditional) variances of 1.
Therefore, by Lemma~3.1 in \citet{chernozhukov2013gaussian},
%
\begin{align*}
\sup_{t \in \R}
\left|
\P\left(
\sup_{w \in \cW_\delta}
Z_n^T(w)
\leq t
\right)
- \P\left(
\sup_{w \in \cW_\delta}
\hat Z_n^T(w)
\leq t
\Bigm\vert \bW_n
\right)
\right|
&\lesssim
\Delta^{1/3}
\Big(
1 \vee \log \frac{1}{\Delta \delta_n}
\Big)^{2/3}
\end{align*}
%
uniformly in the data. By the previous part and
since $x (\log 1/x)^2$ is increasing on $\big(0, e^{-2}\big)$,
%
\begin{align*}
&\sup_{t \in \R}
\left|
\P\left(
\sup_{w \in \cW_\delta}
Z_n^T(w)
\leq t
\right)
- \P\left(
\sup_{w \in \cW_\delta}
\hat Z_n^T(w)
\leq t
\Bigm\vert \bW_n
\right)
\right| \\
&\quad\lesssim_\P
\left(
\sqrt{\frac{\log n}{n}}
\frac{1}{\Dl + 1/\sqrt{n h}}
\right)^{1/3}
(\log n)^{2/3}
\lesssim_\P
\frac{n^{-1/6}(\log n)^{5/6}}
{\Dl^{1/3} + (n h)^{-1/6}}.
\end{align*}

\proofparagraph{trajectory regularity of $Z_n^T$}

In the proof of Theorem~\ref{thm:kernel_infeasible_ucb}
we established that $Z_n^T$ satisfies the regularity property
%
\begin{align*}
\E\left[
\sup_{|w-w'| \leq \delta_n}
\big| Z_n^T(w) - Z_n^T(w') \big|
\right]
&\lesssim
n h^{-1}
\sqrt{\delta_n \log n},
\end{align*}
%
whenever $1/\delta_n$
is at most polynomial in $n$.

\proofparagraph{conditional $L^2$ regularity of $\hat Z_n^T$}

By Lemma~\ref{lem:kernel_app_sdp},
with $n h \gtrsim \log n$,
we have
uniformly in $w,w'$,
%
\begin{align*}
\big|
\hat \Sigma_n^+(w,w')
- \hat \Sigma_n^+(w,w)
\big|
&\lesssim
n^{-1} h^{-3} |w-w'|.
\end{align*}
%
Taking
$\delta_n \leq n^{-2} h^2$,
Lemma~\ref{lem:kernel_app_variance_estimator_bounds}
gives
%
\begin{align*}
\inf_{|w-w'| \leq \delta_n}
\hat \Sigma_n^+(w,w')
\gtrsim
\frac{\Dl^2}{n}
+ \frac{1}{n^2h}
- n^{-1} h^{-3} \delta_n
\gtrsim
\frac{\Dl^2}{n}
+ \frac{1}{n^2h}
- \frac{1}{n^3h}
\gtrsim
\frac{\Dl^2}{n}
+ \frac{1}{n^2h}.
\end{align*}
%
The conditional $L^2$
regularity of $\hat Z_n^T$ is
%
\begin{align*}
\E\left[
\big(
\hat Z_n^T(w) - \hat Z_n^T(w')
\big)^2
\bigm\vert \bW_n
\right]
&=
2 - 2
\frac{\hat \Sigma_n^+(w,w')}
{\sqrt{\hat \Sigma_n^+(w,w) \hat \Sigma_n^+(w',w')}}.
\end{align*}
%
Applying the same elementary result as for $Z_n^T$
in the proof of Theorem~\ref{thm:kernel_infeasible_ucb} yields
%
\begin{align*}
\E\left[
\big(
\hat Z_n^T(w) - \hat Z_n^T(w')
\big)^2
\bigm\vert \bW_n
\right]
&\lesssim_\P
n^2 h^{-2} |w-w'|.
\end{align*}
%
Thus the conditional semimetric
induced by $\hat Z_n^T$ on $\cW$ is
%
\begin{align*}
\hat\rho(w,w')
&\vcentcolon=
\E\left[
\big(
\hat Z_n^T(w) - \hat Z_n^T(w')
\big)^2
\bigm\vert \bW_n
\right]^{1/2}
\lesssim_\P
n h^{-1} \sqrt{|w-w'|}.
\end{align*}

\proofparagraph{conditional trajectory regularity of $\hat Z_n^T$}

As for $Z_n^T$ in the proof of Theorem~\ref{thm:kernel_infeasible_ucb},
we apply Lemma~\ref{lem:kernel_app_gaussian_process_maximal},
now conditionally, to obtain
%
\begin{align*}
\E\left[
\sup_{|w-w'| \leq \delta_n}
\left| \hat Z_n^T(w) - \hat Z_n^T(w') \right|
\Bigm\vert \bW_n
\right]
&\lesssim_\P
n h^{-1}
\sqrt{\delta_n \log n},
\end{align*}
%
whenever $1/\delta_n$
is at most polynomial in $n$.

\proofparagraph{uniform Gaussian comparison}

Now we use the trajectory regularity properties to
extend the Gaussian--Gaussian comparison result from a finite mesh
to all of $\cW$.
Write the previously established
approximation rate as
%
\begin{align*}
r_n
&=
\frac{n^{-1/6}(\log n)^{5/6}}
{\Dl^{1/3} + (n h)^{-1/6}}.
\end{align*}
%
Take $\varepsilon_n > 0$ and observe that
uniformly in $t \in \R$,
%
\begin{align*}
&\P\left(
\sup_{w \in \cW}
\big| \hat Z_n^T(w) \big|
\leq t
\Bigm\vert \bW_n
\right) \\
&\quad\leq
\P\left(
\sup_{w \in \cW_\delta}
\big| \hat Z_n^T(w) \big|
\leq t + \varepsilon_n
\Bigm\vert \bW_n
\right)
+ \P\left(
\sup_{|w-w'| \leq \delta_n}
\left|
\hat Z_n^T(w)
- \hat Z_n^T(w')
\right|
\geq \varepsilon_n
\Bigm\vert \bW_n
\right) \\
&\quad\leq
\P\left(
\sup_{w \in \cW_\delta}
\big| Z_n^T(w) \big|
\leq t + \varepsilon_n
\right)
+ O_\P(r_n)
+ \P\left(
\sup_{|w-w'| \leq \delta_n}
\left|
\hat Z_n^T(w)
- \hat Z_n^T(w')
\right|
\geq \varepsilon_n
\Bigm\vert \bW_n
\right) \\
&\quad\leq
\P\left(
\sup_{w \in \cW}
\big| Z_n^T(w) \big|
\leq t + 2\varepsilon_n
\right)
+ O_\P(r_n)
+ \P\left(
\sup_{|w-w'| \leq \delta_n}
\left|
Z_n^T(w)
- Z_n^T(w')
\right|
\geq \varepsilon_n
\right) \\
&\qquad+
\P\left(
\sup_{|w-w'| \leq \delta_n}
\left|
\hat Z_n^T(w)
- \hat Z_n^T(w')
\right|
\geq \varepsilon_n
\Bigm\vert \bW_n
\right) \\
&\quad\leq
\P\left(
\sup_{w \in \cW}
\big| Z_n^T(w) \big|
\leq t + 2\varepsilon_n
\right)
+ O_\P(r_n)
+ O_\P(\varepsilon_n^{-1} n h^{-1} \sqrt{\delta_n \log n}) \\
&\quad\leq
\P\left(
\sup_{w \in \cW}
\big| Z_n^T(w) \big|
\leq t
\right)
+ \P\left(
\left|
\sup_{w \in \cW}
\big| Z_n^T(w) \big|
- t
\right|
\leq 2\varepsilon_n
\right) \\
&\qquad+
O_\P(r_n)
+ O_\P(\varepsilon_n^{-1} n h^{-1} \sqrt{\delta_n \log n}).
\end{align*}
%
The converse inequality is obtained
analogously as follows:
%
\begin{align*}
&\P\left(
\sup_{w \in \cW}
\big| \hat Z_n^T(w) \big|
\leq t
\Bigm\vert \bW_n
\right) \\
&\quad\geq
\P\left(
\sup_{w \in \cW_\delta}
\big| \hat Z_n^T(w) \big|
\leq t - \varepsilon_n
\Bigm\vert \bW_n
\right)
- \P\left(
\sup_{|w-w'| \leq \delta_n}
\left|
\hat Z_n^T(w)
- \hat Z_n^T(w')
\right|
\geq \varepsilon_n
\Bigm\vert \bW_n
\right) \\
&\quad\geq
\P\left(
\sup_{w \in \cW_\delta}
\big| Z_n^T(w) \big|
\leq t - \varepsilon_n
\right)
- O_\P(r_n)
- \P\left(
\sup_{|w-w'| \leq \delta_n}
\left|
\hat Z_n^T(w)
- \hat Z_n^T(w')
\right|
\geq \varepsilon_n
\Bigm\vert \bW_n
\right) \\
&\quad\geq
\P\left(
\sup_{w \in \cW}
\big| Z_n^T(w) \big|
\leq t - 2\varepsilon_n
\right)
- O_\P(r_n)
- \P\left(
\sup_{|w-w'| \leq \delta_n}
\left|
Z_n^T(w)
- Z_n^T(w')
\right|
\geq \varepsilon_n
\right) \\
&\qquad-
\P\left(
\sup_{|w-w'| \leq \delta_n}
\left|
\hat Z_n^T(w)
- \hat Z_n^T(w')
\right|
\geq \varepsilon_n
\Bigm\vert \bW_n
\right) \\
&\quad\geq
\P\left(
\sup_{w \in \cW}
\big| Z_n^T(w) \big|
\leq t - 2\varepsilon_n
\right)
- O_\P(r_n)
- O_\P(\varepsilon_n^{-1} n h^{-1} \sqrt{\delta_n \log n}) \\
&\quad\geq
\P\left(
\sup_{w \in \cW}
\big| Z_n^T(w) \big|
\leq t
\right)
- \P\left(
\left|
\sup_{w \in \cW}
\big| Z_n^T(w) \big|
- t
\right|
\leq 2\varepsilon_n
\right) \\
&\qquad-
O_\P(r_n)
- O_\P(\varepsilon_n^{-1} n h^{-1} \sqrt{\delta_n \log n}).
\end{align*}
%
Combining these uniform upper and lower bounds gives
%
\begin{align*}
&\sup_{t \in \R}
\left|
\P\left(
\sup_{w \in \cW}
\big| \hat Z_n^T(w) \big|
\leq t
\Bigm\vert \bW_n
\right)
-
\P\left(
\sup_{w \in \cW}
\big| Z_n^T(w) \big|
\leq t
\right)
\right| \\
&\qquad\lesssim_\P
\sup_{t \in \R}
\P\left(
\left|
\sup_{w \in \cW}
\big| Z_n^T(w) \big|
- t
\right|
\leq 2\varepsilon_n
\right)
+ r_n
+ \varepsilon_n^{-1} n h^{-1/2} \delta_n^{1/2} \sqrt{\log n}.
\end{align*}
%
For the remaining term, apply anti-concentration
for $Z_n^T$ from the proof of Theorem~\ref{thm:kernel_infeasible_ucb}:
%
\begin{align*}
\sup_{t \in \R}
\P\left(
\left|
\sup_{w \in \cW}
\big| Z_n^T(w) \big|
- t
\right|
\leq \varepsilon
\right)
&\lesssim
\varepsilon
\sqrt{\log n}.
\end{align*}
%
Therefore
%
\begin{align*}
&\sup_{t \in \R}
\left|
\P\left(
\sup_{w \in \cW}
\big| \hat Z_n^T(w) \big|
\leq t
\Bigm\vert \bW_n
\right)
-
\P\left(
\sup_{w \in \cW}
\big| Z_n^T(w) \big|
\leq t
\right)
\right| \\
&\qquad\lesssim_\P
\varepsilon_n \sqrt{\log n}
+ r_n
+ \varepsilon_n^{-1} n h^{-1/2} \delta_n^{1/2} \sqrt{\log n}.
\end{align*}
%
Taking $\varepsilon = r_n / \sqrt{\log n}$
and then $\delta_n = n^{-2} h r_n^2 \varepsilon_n^2 / \log n$
yields
%
\begin{align*}
\left|
\P\left(
\sup_{w \in \cW}
\big| \hat Z_n^T(w) \big|
\leq t
\Bigm\vert \bW_n
\right)
-
\P\left(
\sup_{w \in \cW}
\big| Z_n^T(w) \big|
\leq t
\right)
\right|
&\lesssim_\P
r_n =
\frac{n^{-1/6}(\log n)^{5/6}}
{\Dl^{1/3} + (n h)^{-1/6}}.
\end{align*}
\end{proof}

\begin{proof}[Lemma~\ref{lem:kernel_app_feasible_gaussian_approx}]

\proofparagraph{Kolmogorov--Smirnov approximation}

Let $Z_n^T$ and $\hat Z_n^T$ be defined
as in the proof of
Lemma~\ref{lem:kernel_app_distributional_approx_feasible_gaussian}.
Write
%
\begin{align*}
r_n
&=
\frac{n^{-1/6}(\log n)^{5/6}}
{\Dl^{1/3} + (n h)^{-1/6}}
\end{align*}
%
for the rate of approximation from
Lemma~\ref{lem:kernel_app_distributional_approx_feasible_gaussian}.
For any $\varepsilon_n > 0$ and uniformly in $t \in \R$:
%
\begin{align*}
&\P\left(
\sup_{w \in \cW}
\left|
\hat Z_n^T(w)
\right|
\leq t
\Bigm\vert \bW_n
\right)
\leq
\P\left(
\sup_{w \in \cW}
\left|
Z_n^T(w)
\right|
\leq t
\right)
+
O_\P(r_n) \\
&\quad\leq
\P\left(
\sup_{w \in \cW}
\left|
Z_n^T(w)
\right|
\leq t - \varepsilon_n
\right)
+
\P\left(
\left|
\sup_{w \in \cW}
\big|
Z_n^T(w)
\big|
-t
\right|
\leq \varepsilon_n
\right)
+
O_\P(r_n) \\
&\quad\leq
\P\left(
\sup_{w \in \cW}
\left| \hat T_n(w) \right|
\leq t
\right)
+
\P\left(
\sup_{w \in \cW}
\left| \hat T_n(w) - Z_n^T(w) \right|
\geq \varepsilon_n
\right) \\
&\qquad+
\P\left(
\left|
\sup_{w \in \cW}
\big|
Z_n^T(w)
\big|
-t
\right|
\leq \varepsilon_n
\right)
+
O_\P(r_n) \\
&\quad\leq
\P\left(
\sup_{w \in \cW}
\left| \hat T_n(w) \right|
\leq t
\right)
+
\P\left(
\sup_{w \in \cW}
\left| \hat T_n(w) - Z_n^T(w) \right|
\geq \varepsilon_n
\right)
+ \varepsilon_n \sqrt{\log n}
+ O_\P(r_n),
\end{align*}
%
where in the last line we used the anti-concentration result
from Lemma~\ref{lem:kernel_app_anticoncentration}
applied to $Z_n^T$,
as in the proof of
Lemma~\ref{lem:kernel_app_distributional_approx_feasible_gaussian}.
The corresponding lower bound is as follows:
%
\begin{align*}
&\P\left(
\sup_{w \in \cW}
\left|
\hat Z_n^T(w)
\right|
\leq t
\Bigm\vert \bW_n
\right)
\geq
\P\left(
\sup_{w \in \cW}
\left|
Z_n^T(w)
\right|
\leq t
\right)
-
O_\P(r_n) \\
&\quad\geq
\P\left(
\sup_{w \in \cW}
\left|
Z_n^T(w)
\right|
\leq t + \varepsilon_n
\right)
-
\P\left(
\left|
\sup_{w \in \cW}
\big|
Z_n^T(w)
\big|
-t
\right|
\leq \varepsilon_n
\right)
-
O_\P(r_n) \\
&\quad\geq
\P\left(
\sup_{w \in \cW}
\left| \hat T_n(w) \right|
\leq t
\right)
-
\P\left(
\sup_{w \in \cW}
\left| \hat T_n(w) - Z_n^T(w) \right|
\geq \varepsilon_n
\right) \\
&\qquad-
\P\left(
\left|
\sup_{w \in \cW}
\big|
Z_n^T(w)
\big|
-t
\right|
\leq \varepsilon_n
\right)
-
O_\P(r_n) \\
&\quad\geq
\P\left(
\sup_{w \in \cW}
\left| \hat T_n(w) \right|
\leq t
\right)
-
\P\left(
\sup_{w \in \cW}
\left| \hat T_n(w) - Z_n^T(w) \right|
\geq \varepsilon_n
\right)
- \varepsilon_n \sqrt{\log n}
- O_\P(r_n).
\end{align*}

\proofparagraph{$t$-statistic approximation}

To control the remaining term,
note that by
Theorem~\ref{thm:kernel_strong_approx_Tn}
and Lemma~\ref{lem:kernel_app_studentized_t_statistic},
%
\begin{align*}
&\sup_{w \in \cW}
\left| \hat T_n(w) - Z_n^T(w) \right| \\
&\quad\leq
\sup_{w \in \cW}
\left| \hat T_n(w) - T_n(w) \right|
+ \sup_{w \in \cW}
\left| T_n(w) - Z_n^T(w) \right| \\
&\quad\lesssim_\P
\sqrt{\frac{\log n}{n}}
\left(
\sqrt{\log n} + \frac{\sqrt n h^{p \wedge \beta}}
{\Dl + 1/\sqrt{n h}}
\right)
\frac{1}{\Dl + 1/\sqrt{n h}} \\
&\qquad+
\frac{
n^{-1/2} \log n
+ n^{-3/4} h^{-7/8} (\log n)^{3/8} R_n
+ n^{-2/3} h^{-1/2} (\log n)^{2/3}
+ n^{1/2} h^{p \wedge \beta}}
{\Dl + 1/\sqrt{n h}}
\end{align*}
%
and denote this last quantity by $r_n'$.
Then for any $\varepsilon_n \gg r_n'$,
we have
%
\begin{align*}
\sup_{t \in \R}
\left|
\P\left(
\sup_{w \in \cW}
\left| \hat T_n(w) \right|
\leq t
\right)
- \P\left(
\sup_{w \in \cW}
\left|
\hat Z_n^T(w)
\right|
\leq t
\Bigm\vert \bW_n
\right)
\right|
&\lesssim_\P
\varepsilon_n \sqrt{\log n}
+ r_n
+ o(1).
\end{align*}

\proofparagraph{rate analysis}

This rate is $o_\P(1)$
with an appropriate choice of $\varepsilon_n$ whenever
$r_n \to 0$ and $r_n' \sqrt{\log n} \to 0$,
by Lemma~\ref{lem:kernel_app_slow_convergence}, along with
a slowly diverging sequence $R_n$. Explicitly, we require the following.
%
\begin{align*}
\frac{n^{-1/2} (\log n)^{3/2}}{\Dl + 1/\sqrt{n h}}
&\to 0,
&\frac{h^{p \wedge \beta} \log n}{\Dl^2 + (n h)^{-1}}
&\to 0, \\
\frac{n^{-1/2} (\log n)^{3/2}}
{\Dl + 1/\sqrt{n h}}
&\to 0,
&\frac{n^{-3/4} h^{-7/8} (\log n)^{7/8}}
{\Dl + 1/\sqrt{n h}}
&\to 0, \\
\frac{n^{-2/3} h^{-1/2} (\log n)^{7/6}}
{\Dl + 1/\sqrt{n h}}
&\to 0,
&\frac{n^{1/2} h^{p \wedge \beta} (\log n)^{1/2}}
{\Dl + 1/\sqrt{n h}}
&\to 0, \\
\frac{n^{-1/6}(\log n)^{5/6}}
{\Dl^{1/3} + (n h)^{-1/6}}
&\to 0.
\end{align*}
%
Using the fact that $h \lesssim n^{-\varepsilon}$
for some $\varepsilon > 0$
and removing trivial statements leaves us with
%
\begin{align*}
\frac{n^{-3/4} h^{-7/8} (\log n)^{7/8}}
{\Dl + 1/\sqrt{n h}}
&\to 0,
&\frac{n^{1/2} h^{p \wedge \beta} (\log n)^{1/2}}
{\Dl + 1/\sqrt{n h}}
&\to 0.
\end{align*}
%
We analyze these based on the degeneracy
and verify that they hold under Assumption~\ref{ass:kernel_rates}.
%
\begin{enumerate}[label=(\roman*)]

\item No degeneracy:
if $\Dl > 0$ then we need
%
\begin{align*}
n^{-3/4} h^{-7/8} (\log n)^{7/8}
&\to 0,
&n^{1/2} h^{p \wedge \beta} (\log n)^{1/2}
&\to 0.
\end{align*}
%
These reduce to
$n^{-6/7} \log n \ll h
\ll (n \log n)^{-\frac{1}{2(p \wedge \beta)}}$.

\item Partial or total degeneracy:
if $\Dl = 0$ then we need
%
\begin{align*}
n^{-1/4} h^{-3/8} (\log n)^{7/8}
&\to 0,
&n h^{(p \wedge \beta) + 1/2} (\log n)^{1/2}
&\to 0.
\end{align*}
%
These reduce to
$n^{-2/3} (\log n)^{7/3} \ll h
\ll (n^2 \log n)^{-\frac{1}{2(p \wedge \beta) + 1}}$.
%
\end{enumerate}

\end{proof}

\begin{proof}[Theorem~\ref{thm:kernel_ucb}]

\proofparagraph{existence of the conditional quantile}

We argue as in the proof of
Lemma~\ref{lem:kernel_app_distributional_approx_feasible_gaussian},
now also conditioning on the data.
In particular, using the anti-concentration result from
Lemma~\ref{lem:kernel_app_anticoncentration},
the regularity property of $\hat Z_n^T$,
and the Gaussian process maximal inequality from
Lemma~\ref{lem:kernel_app_gaussian_process_maximal},
we see that for any $\varepsilon > 0$,
%
\begin{align*}
\sup_{t \in \R}
\P\left(
\left|
\sup_{w \in \cW}
\big| \hat Z_n^T(w) \big|
- t
\right|
\leq 2\varepsilon
\Bigm\vert \bW_n
\right)
&\leq
8 \varepsilon
\left(
1 + \E\left[
\sup_{w \in \cW}
\big| \hat Z_n^T(w) \big|
\Bigm\vert \bW_n
\right]
\right)
\lesssim \varepsilon \sqrt{\log n}.
\end{align*}
%
Thus letting $\varepsilon \to 0$
shows that the conditional distribution function of
$\sup_{w \in \cW} \big|\hat Z_n^T(w)\big|$
is continuous,
and therefore all of its conditional quantiles exist.

\proofparagraph{validity of the confidence band}

Define the following (conditional) distribution functions.
%
\begin{align*}
F_Z(t \mid \bW_n)
&=
\P\left(
\sup_{w \in \cW}
\left| \hat Z_n^T(w) \right|
\leq t
\Bigm\vert \bW_n
\right),
&F_T(t)
&=
\P\left(
\sup_{w \in \cW}
\left| \hat T_n(w) \right|
\leq t
\right),
\end{align*}
%
along with their well-defined right-quantile functions,
%
\begin{align*}
F_Z^{-1}(p \mid \bW_n)
&=
\sup
\big\{
t \in \R
\, : \,
F_Z(t \mid \bW_n)
= p
\big\},
&F_T^{-1}(p)
&=
\sup
\big\{
t \in \R
\, : \,
F_T(t)
= p
\big\}.
\end{align*}
%
Note that
$t \leq F_Z^{-1}(p \mid \bW_n)$
if and only if
$F_Z(t \mid \bW_n) \leq p$.
Take $\alpha \in (0,1)$ and
define the quantile
$\hat q_{1-\alpha} = F_Z^{-1}(1-\alpha \mid \bW_n)$,
so that
$F_Z(\hat q_{1-\alpha} \mid \bW_n) = 1-\alpha$.
By Lemma~\ref{lem:kernel_app_feasible_gaussian_approx},
%
\begin{align*}
\sup_{t \in \R}
\big|
F_Z(t \mid \bW_n) - F_T(t)
\big|
&=
o_\P(1).
\end{align*}
%
Thus by Lemma~\ref{lem:kernel_app_slow_convergence},
this can be replaced by
%
\begin{align*}
\P\left(
\sup_{t \in \R} \big| F_Z(t \mid \bW_n) - F_T(t) \big|
> \varepsilon_n
\right)
&\leq \varepsilon_n
\end{align*}
%
for some $\varepsilon_n \to 0$.
Therefore
%
\begin{align*}
\P\left(
\sup_{w \in \cW}
\left|
\hat T_n(w)
\right|
\leq
\hat q_{1-\alpha}
\right)
&=
\P\left(
\sup_{w \in \cW}
\left|
\hat T_n(w)
\right|
\leq
F_Z^{-1}(1-\alpha \mid \bW_n)
\right) \\
&=
\P\left(
F_Z\left(
\sup_{w \in \cW}
\left|
\hat T_n(w)
\right|
\Bigm\vert \bW_n
\right)
\leq
1 - \alpha
\right) \\
&\leq
\P\left(
F_T\left(
\sup_{w \in \cW}
\left|
\hat T_n(w)
\right|
\right)
\leq
1 - \alpha + \varepsilon_n
\right)
+ \varepsilon_n
\leq 1 - \alpha + 3\varepsilon_n,
\end{align*}
%
where we used the fact that for any
real-valued random variable $X$ with distribution function $F$,
we have
$\big|\P\big(F(X) \leq t\big) - t\big| \leq \Delta$,
where $\Delta$ is the size of the
largest jump discontinuity in $F$.
By uniform integrability,
$\sup_{t \in \R} \big| F_Z(t) - F_T(t) \big| = o(\varepsilon_n)$.
Since $F_Z$ has no jumps,
we must have $\Delta \leq \varepsilon_n$ for $F_T$.
Finally, a lower bound is constructed in an analogous manner,
giving
%
\begin{align*}
\P\left(
\sup_{w \in \cW}
\left| \hat T_n(w) \right|
\leq
\hat q_{1-\alpha}
\right)
&\geq
1 - \alpha - 3\varepsilon_n.
\end{align*}
%
\end{proof}

\begin{proof}[Lemma~\ref{lem:kernel_app_counterfactual_bias}]

Writing
$k_{i j} = k_h(W_{i j}^1, w)$,
$\psi_i = \psi(X_i^1)$,
$\hat\psi_i = \hat\psi(X_i^1)$,
and $\kappa_{i j} = \kappa(X_i^0, X_i^1, X_j^1)$,
%
\begin{align*}
\E\big[\hat f_W^{1 \triangleright 0}(w)\big]
&=
\E\left[
\frac{2}{n(n-1)}
\sum_{i<j}
\hat \psi_i
\hat \psi_j
k_{i j}
\right] \\
&=
\frac{2}{n(n-1)(n-2)}
\sum_{i < j}
\sum_{r \notin \{i,j\}}
\E\left[
k_{i j}
\Big(
\psi_i
\psi_j
+\psi_i
\kappa_{r j}
+\psi_j
\kappa_{r i}
\Big)
\right]
+ O\left(\frac{1}{n}\right) \\
&=
\E\left[
k_{i j}
\psi_i
\psi_j
\right]
+ O\left(\frac{1}{n}\right)
=
\E\big[
\psi_i
\psi_j
\E\left[
k_h(W_{i j}^1, w)
\mid X_i^1, X_j^1
\right]
\big]
+ O\left(\frac{1}{n}\right) \\
&=
\E\big[
\psi_i
\psi_j
f_{W \mid XX}^1(w \mid X_i^1, X_j^1)
+ O_\P\left( h^{p \wedge \beta} \right)
\big]
+ O\left(\frac{1}{n}\right) \\
&=
f_W^{1 \triangleright 0}(w)
+ O\left( h^{p \wedge \beta} + \frac{1}{n}\right)
\end{align*}
%
uniformly in $w$, by the proof of
Theorem~\ref{thm:kernel_bias} and H{\"o}lder continuity
of $f_{W \mid XX}^1$.
%
\end{proof}

\begin{proof}[Lemma~\ref{lem:kernel_app_counterfactual_hoeffding}]
%
\begin{align*}
\hat f_W^{1 \triangleright 0}(w)
&=
\frac{2}{n(n-1)}
\sum_{i < j}
\hat \psi_i
\hat \psi_j
k_{i j} \\
&=
\frac{2}{n(n-1)}
\sum_{i < j}
\left(
\psi_i
+ \frac{1}{n}
\sum_{r=1}^n \kappa_{r i}
\right)
\left(
\psi_j
+ \frac{1}{n}
\sum_{r=1}^n \kappa_{r j}
\right)
k_{i j}
+ O_\P\left(\frac{1}{n}\right) \\
&=
\frac{2}{n(n-1)}
\sum_{i < j}
\psi_i
\psi_j
k_{i j}
+ \frac{2}{n(n-1)}
\sum_{i < j}
\psi_i
\frac{1}{n}
\sum_{r \notin \{i,j\}}^n \kappa_{r j}
k_{i j} \\
&\quad+
\frac{2}{n(n-1)}
\sum_{i < j}
\psi_j
\frac{1}{n}
\sum_{r \notin \{i,j\}}^n \kappa_{r i}
k_{i j}
+ O_\P\left(\frac{1}{n}\right) \\
&=
\frac{2}{n(n-1)(n-2)}
\sum_{i < j}
\sum_{r \notin \{i,j\}}
k_{i j}
\Big(
\psi_i
\psi_j
+\psi_i
\kappa_{r j}
+\psi_j
\kappa_{r i}
\Big)
+ O_\P\left(\frac{1}{n}\right) \\
&=
\frac{6}{n(n-1)(n-2)}
\sum_{i < j < r}
v_{i j r}
+ O_\P\left(\frac{1}{n}\right)
\end{align*}
%
where
%
\begin{align*}
v_{i j r}
&=
\frac{1}{3}
k_{i j} \Big(\psi_i \psi_j +\psi_i \kappa_{r j} +\psi_j \kappa_{r i} \Big)
+ \frac{1}{3}
k_{i r} \Big(\psi_i \psi_r +\psi_i \kappa_{jr} +\psi_r \kappa_{j i} \Big) \\
&\quad+
\frac{1}{3}
k_{jr} \Big(\psi_j \psi_r +\psi_j \kappa_{i r} +\psi_r \kappa_{i j} \Big)
\end{align*}
%
So by the Hoeffding decomposition for third-order U-statistics,
%
\begin{align*}
\hat f_W^{1 \triangleright 0}(w)
&=
u
+ \frac{3}{n}
\sum_{i=1}^n
u_i
+ \frac{6}{n(n-1)}
\sum_{i=1}^{n-1}
\sum_{j=i+1}^n
u_{i j}
+ \frac{6}{n(n-1)(n-2)}
\sum_{i=1}^{n-2}
\sum_{j=i+1}^{n-1}
\sum_{r=j+1}^n
u_{i j r} \\
&\quad+
\frac{6}{n(n-1)(n-2)}
\sum_{i=1}^{n-2}
\sum_{j=i+1}^{n-1}
\sum_{r=j+1}^n
\big(v_{i j r} - u_{i j r}\big)
+ O_\P\left( \frac{1}{n} \right) \\
&=
\E\big[\hat f_W^{1 \triangleright 0}(w) \big]
+ L_n^{1 \triangleright 0}(w)
+ Q_n^{1 \triangleright 0}(w)
+ T_n^{1 \triangleright 0}(w)
+ E_n^{1 \triangleright 0}(w)
+ O_\P\left( \frac{1}{n} \right).
\end{align*}
%
Noting that $\psi_i$, $\kappa_{i j}$
and $\E[k_{i j} \mid A_i^1, A_j^1]$
are all bounded and that
$\E[k_{i j} \mid A_i^1, A_j^1]$
is Lipschitz in $w$,
we deduce by
Lemma~\ref{lem:kernel_app_uprocess_maximal}
and Proposition~2.3 of
\citet{arcones1993limit} that
$\sup_{w \in \cW} |Q_n^{1 \triangleright 0}(w)
+ T_n^{1 \triangleright 0}(w)| \lesssim_\P \frac 1n$.
%
\end{proof}

\begin{proof}[Lemma~\ref{lem:kernel_app_counterfactual_uniform_consistency}]

By Lemma~\ref{lem:kernel_app_maximal_vc_inid},
$\sup_{w \in \cW} \big|L_n^{1 \triangleright 0}(w)\big|
\lesssim_\P \frac{1}{\sqrt n}$.
In the proof of Lemma~\ref{lem:kernel_app_counterfactual_hoeffding}
the terms $v_{i j r} - u_{i j r}$ depend only on
$V_{i j}$, $V_{i r}$, and $V_{jr}$
after conditioning on $\bA_n^1$, $\bX_n^0$, and $\bX_n^1$.
Thus $E_n^{1 \triangleright 0}(w)$ is a degenerate second-order
U-statistic so
$\sup_{w \in \cW} \big|E_n^{1 \triangleright 0}(w)\big|
\lesssim_\P \sqrt{\frac{\log n}{n^2h}}$
by Lemma~\ref{lem:kernel_app_uprocess_maximal}.
%
\end{proof}

\begin{proof}[Lemma~\ref{lem:kernel_app_counterfactual_sa}]

Note that
$L_n^{1 \triangleright 0}(w)
= \frac 3n \sum_{i=1}^n l_i^{1 \triangleright 0}(w)$
where $l_i^{1 \triangleright 0}(w)$ depends only on
$A_i^1$, $X_i^0$, and $X_i^1$.
Let $\gamma: \cX \times \cX \to \{1, \ldots, |\cX|^2\}$
be a bijection and
define $\logistic(x) = \frac{1}{1+e^{-x}}$.
Let
$\tilde A_i = \logistic(A_i^1) + \gamma(X_i^0, X_i^1)$
so that
$A_i^1 = \logistic^{-1}\big(\tilde A_i
- \lfloor \tilde A_i \rfloor\big)$
and
$(X_i^0, X_i^1) = \gamma^{-1}(\lfloor \tilde A_i \rfloor)$.
Thus
$l_i^{1 \triangleright 0}(w)$ is a bounded-variation function
of $\tilde A_i$, uniformly in $w$, and so as in
Lemma~\ref{lem:kernel_app_strong_approx_Ln} we have that
on an appropriately enlarged probability space,
%
\begin{align*}
\E\left[
\sup_{w \in \cW}
\left|
\sqrt n L_n^{1 \triangleright 0}(w)
- Z_n^{L, 1 \triangleright 0}(w)
\right|
\right]
\lesssim
\frac{\log n}{\sqrt n}
\end{align*}
%
where $Z_n^{L, 1 \triangleright 0}$ is a mean-zero
Gaussian process with the same covariance as
$\sqrt n L_n^{1 \triangleright 0}$.
For $E_n^{1 \triangleright 0}(w)$,
we first construct a strong approximation conditional on
$\bA_n$ and $\bX_n$ as shown in
Lemma~\ref{lem:kernel_app_conditional_strong_approx_En}
and deduce an unconditional strong approximation as in
Lemma~\ref{lem:kernel_app_unconditional_strong_approx_En} to see
%
\begin{align*}
\sup_{w \in \cW}
\left|
\sqrt{n^2h} E_n^{1 \triangleright 0}(w)
- Z_n^{E, 1 \triangleright 0}(w)
\right|
\lesssim_\P
n^{-1/4} h^{-3/8} (\log n)^{3/8} R_n
+ n^{-1/6} (\log n)^{2/3}
\end{align*}
%
where $Z_n^{E, 1 \triangleright 0}$ is a mean-zero
Gaussian process with the same covariance as
$\sqrt{n^2h} E_n^{1 \triangleright 0}$.
Arguing as in the proof of Theorem~\ref{thm:kernel_app_strong_approx_fW}
shows that the Gaussian processes are independent
and can be summed to yield a single strong approximation.
%
\end{proof}

\begin{proof}[Lemma~\ref{lem:kernel_app_counterfactual_covariance_structure}]

Arguing by mean-zero properties and conditional independence,
%
\begin{align*}
&\Sigma_n^{1 \triangleright 0}(w,w')
= \Cov\left[
\hat f_W^{1 \triangleright 0}(w),
\hat f_W^{1 \triangleright 0}(w')
\right] \\
&\quad=
\frac{1}{n^2(n-1)^2(n-2)^2}
\sum_{i \neq j}
\sum_{r \notin \{i,j\}}
\sum_{i' \neq j'}
\sum_{r' \notin \{i',j'\}}
\!\!\!\!\E\Big[
\!\Big(
k_{i j} \psi_i \psi_j
- \E[k_{i j} \psi_i \psi_j]
+ k_{i j}\psi_i \kappa_{r j}
+ k_{i j}\psi_j \kappa_{r i}
\Big) \\
&\qquad\qquad\times
\Big(
k_{i' j'}' \psi_{i'} \psi_{j'}
- \E[k_{i j}' \psi_i \psi_j]
+ k_{i' j'}'\psi_{i'} \kappa_{r' j'}
+ k_{i' j'}'\psi_{j'} \kappa_{r' i'}
\Big)
\Big]
+ O\left( \frac{1}{n^{3/2}} + \frac{1}{\sqrt{n^4h}} \right) \\
&\quad=
\frac{2}{n^2}
\E\left[
k_{i j} \psi_i \psi_j
k_{i j}' \psi_i \psi_j
\right]
+ \frac{4}{n}
\E\left[
k_{i j} \psi_i \psi_j
k_{i r}' \psi_i \psi_r
\right]
- \frac{4}{n}
\E\left[
k_{i j} \psi_i \psi_j
\right]
\E\left[
k_{i j}' \psi_i \psi_j
\right] \\
&\qquad+
\frac{4}{n}
\E\left[
k_{i j}\psi_i \kappa_{i' j}
k_{i' j'}' \psi_{i'} \psi_{j'}
\right]
+ \frac{4}{n}
\E\left[
k_{i j} \psi_{i} \psi_{j}
k_{i' j'}'\psi_{i'} \kappa_{i j'}
\right]
+ \frac{4}{n}
\E\left[
k_{i j} k'_{i' j'}
\psi_i \psi_{i'}
\kappa_{r j} \kappa_{r j'}
\right] \\
&\qquad+
O\left( \frac{1}{n^{3/2}} + \frac{1}{\sqrt{n^4h}} \right) \\
&\quad=
\frac{4}{n}
\E\left[
\Big(
\psi_i
\E\big[
k_{i j} \psi_j
\mid i
\big]
+ \E\left[
k_{r j} \psi_r \kappa_{i j}
\mid i
\right]
\Big)
\Big(
\psi_i
\E\big[
k_{i j}' \psi_j
\mid i
\big]
+ \E\left[
k_{r j}' \psi_r \kappa_{i j}
\mid i
\right]
\Big)
\right] \\
&\qquad+
\frac{2}{n^2}
\E\left[
k_{i j} k_{i j}'
\psi_i^2 \psi_j^2
\right]
- \frac{4}{n}
\E\left[
k_{i j} \psi_i \psi_j
\right]
\E\left[
k_{i j}' \psi_i \psi_j
\right]
+ O\left( \frac{1}{n^{3/2}} + \frac{1}{\sqrt{n^4h}} \right),
\end{align*}
%
where all indices are distinct.
%
\end{proof}

\begin{proof}[Lemma~\ref{lem:kernel_app_counterfactual_infeasible_t_statistic}]
The proof is exactly the same as the proof of
Theorem~\ref{thm:kernel_strong_approx_Tn}.
\end{proof}

\begin{proof}[Theorem~\ref{thm:kernel_app_counterfactual_infeasible_ucb}]
This proof proceeds in the same manner as the proof of
Theorem~\ref{thm:kernel_infeasible_ucb}.
\end{proof}

\chapter[Supplement to Yurinskii's Coupling for Martingales]%
{Supplement to Yurinskii's \\ Coupling for Martingales}
\label{app:yurinskii}

\section{Proofs of main results}
\label{sec:yurinskii_app_proofs}

\subsection{Preliminary lemmas}

We give a sequence of preliminary lemmas which are useful for establishing our
main results. Firstly, we present a conditional version of Strassen's theorem
for the $\ell^p$-norm \citep[Theorem~B.2]{chen2020jackknife}, stated for
completeness as Lemma~\ref{lem:yurinskii_app_strassen}.

\begin{lemma}[A conditional Strassen theorem for the
\texorpdfstring{$\ell^p$}{lp}-norm]%
\label{lem:yurinskii_app_strassen}
%
Let $(\Omega, \cH, \P)$ be a probability space supporting the $\R^d$-valued
variable $X$ for some $d \geq 1$. Let $\cH'$ be a countably generated
sub-$\sigma$-algebra of $\cH$ and suppose there is a $\Unif[0,1]$ random
variable on $(\Omega, \cH, \P)$, independent of the $\sigma$-algebra
generated by $X$ and $\cH'$. Take a regular conditional distribution
$F(\cdot \mid \cH')$ satisfying the following. Firstly, $F(A \mid \cH')$ is
an $\cH'$-measurable variable for all Borel sets $A \in \cB(\R^d)$.
Secondly, $F(\cdot \mid \cH')(\omega)$ is a Borel probability measure on
$\R^d$ for all $\omega \in \Omega$. Taking $\eta, \rho > 0$ and
$p \in [1, \infty]$, with $\E^*$ the outer expectation, if
%
\begin{align*}
\E^* \left[
\sup_{A \in \cB(\R^d)}
\Big\{
\P \big( X \in A \mid \cH' \big)
- F \big( A_p^\eta \mid \cH' \big)
\Big\}
\right]
\leq \rho,
\end{align*}
%
where $A_p^\eta = \{x \in \R^d : \|x - A\|_p \leq \eta\}$
and $\|x - A\|_p = \inf_{x' \in A} \|x - x'\|_p$,
then there exists an $\R^d$-valued random variable $Y$
with $Y \mid \cH' \sim F(\cdot \mid \cH')$
and $\P \left( \|X-Y\|_p > \eta \right) \leq \rho$.
%
\end{lemma}

\begin{proof}[Lemma~\ref{lem:yurinskii_app_strassen}]
By Theorem~B.2 in \citet{chen2020jackknife}, noting that the $\sigma$-algebra
generated by $Z$ is countably generated and using the metric induced by the
$\ell^p$-norm.
\end{proof}

Next, we present in Lemma~\ref{lem:yurinskii_app_smooth_approximation} an
analytic result
concerning the smooth approximation of Borel set indicator functions, similar
to that given in \citet[Lemma~39]{belloni2019conditional}.

\begin{lemma}[Smooth approximation of Borel indicator functions]%
\label{lem:yurinskii_app_smooth_approximation}
Let $A \subseteq \R^d$ be a Borel set and $Z \sim \cN(0, I_d)$.
For $\sigma, \eta > 0$ and $p \in [1, \infty]$, define
%
\begin{align*}
g_{A\eta}(x)
&=
\left( 1 - \frac{\|x-A^\eta\|_p}{\eta} \right) \vee 0
& &\text{and}
&f_{A\eta\sigma}(x)
&=
\E\big[g_{A\eta}(x + \sigma Z) \big].
\end{align*}
%
Then $f$ is infinitely differentiable
and with $\varepsilon = \P(\|Z\|_p > \eta / \sigma)$,
for all $k \geq 0$,
any multi-index $\kappa = (\kappa_1,\dots, \kappa_d)\in\N^d$,
and all $x,y \in \R^d$,
we have $|\partial^\kappa f_{A\eta\sigma}(x)| \leq
\frac{\sqrt{\kappa!}}{\sigma^{|\kappa|}}$ and
%
\begin{align*}
&\Bigg|
f_{A\eta\sigma}(x+y) - \sum_{|\kappa| = 0}^k
\frac{1}{\kappa!}
\partial^\kappa f_{A\eta\sigma}(x)
y^\kappa
\Bigg|
\leq
\frac{\|y\|_p \|y\|_2^k}{\sigma^k \eta \sqrt{k!}}, \\
&(1 - \varepsilon) \I\big\{x \in A\big\}
\leq f_{A\eta\sigma}(x)
\leq \varepsilon + (1 - \varepsilon)
\I\big\{x \in A^{3\eta}\big\}.
\end{align*}
%
\end{lemma}

\begin{proof}[Lemma~\ref{lem:yurinskii_app_smooth_approximation}]
Drop subscripts on $g_{A\eta}$ and $f_{A \eta \sigma}$.
By Taylor's theorem with Lagrange remainder, for $t \in [0,1]$,
%
\begin{align*}
\Bigg|
f(x + y)
- \sum_{|\kappa|=0}^{k}
\frac{1}{\kappa!}
\partial^{\kappa} f(x)
y^\kappa
\Bigg|
\leq
\Bigg|
\sum_{|\kappa|=k}
\frac{y^\kappa}{\kappa!}
\big(
\partial^{\kappa} f(x + t y)
- \partial^{\kappa} f(x)
\big)
\Bigg|.
\end{align*}
%
Now with $\phi(x) = \frac{1}{\sqrt{2 \pi}} e^{-x^2/2}$,
%
\begin{align*}
f(x)
&=
\E\big[g(x + \sigma W) \big]
=
\int_{\R^d}
g(x + \sigma u)
\prod_{j=1}^{d}
\phi(u_j)
\diff u
=
\frac{1}{\sigma^d}
\int_{\R^d}
g(u)
\prod_{j=1}^{d}
\phi \left( \frac{u_j-x_j}{\sigma} \right)
\diff u
\end{align*}
%
and since the integrand is bounded, we exchange differentiation and
integration to compute
%
\begin{align}
\nonumber
\partial^\kappa
f(x)
&=
\frac{1}{\sigma^{d+|\kappa|}}
\int_{\R^d}
g(u)
\prod_{j=1}^{d}
\partial^{\kappa_j}
\phi \left( \frac{u_j-x_j}{\sigma} \right)
\diff u
= \left( \frac{-1}{\sigma} \right)^{|\kappa|}
\int_{\R^d}
g(x + \sigma u)
\prod_{j=1}^{d}
\partial^{\kappa_j}
\phi(u_j)
\diff u \\
\label{eq:yurinskii_app_smoothing_derivative}
&=
\left( \frac{-1}{\sigma} \right)^{|\kappa|}
\E \Bigg[
g(x + \sigma Z)
\prod_{j=1}^{d}
\frac{\partial^{\kappa_j}\phi(Z_j)}{\phi(Z_j)}
\Bigg],
\end{align}
%
where $Z \sim \cN(0, I_d)$.
Recalling that $|g(x)| \leq 1$ and applying the Cauchy--Schwarz inequality,
%
\begin{align*}
\left|
\partial^\kappa
f(x)
\right|
&\leq
\frac{1}{\sigma^{|\kappa|}}
\prod_{j=1}^{d}
\E \left[
\left(
\frac{\partial^{\kappa_j}\phi(Z_j)}{\phi(Z_j)}
\right)^2
\right]^{1/2}
\leq
\frac{1}{\sigma^{|\kappa|}}
\prod_{j=1}^{d}
\sqrt{\kappa_j!}
=
\frac{\sqrt{\kappa!}}{\sigma^{|\kappa|}},
\end{align*}
%
as the expected square of the Hermite polynomial of degree
$\kappa_j$ against the standard Gaussian measure is $\kappa_j!$. By the
reverse triangle inequality, $|g(x + t y) - g(x)| \leq t \|y\|_p / \eta$,
so by \eqref{eq:yurinskii_app_smoothing_derivative},
%
\begin{align*}
&\left|
\sum_{|\kappa|=k}
\frac{y^\kappa}{\kappa!}
\big(
\partial^{\kappa} f(x + t y)
- \partial^{\kappa} f(x)
\big)
\right| \\
&\quad=
\left|
\sum_{|\kappa|=k}
\frac{y^\kappa}{\kappa!}
\frac{1}{\sigma^{|\kappa|}}
\E \Bigg[
\big(
g(x + t y + \sigma Z)
- g(x + \sigma Z)
\big)
\prod_{j=1}^{d}
\frac{\partial^{\kappa_j}\phi(Z_j)}{\phi(Z_j)}
\Bigg]
\right| \\
&\quad\leq
\frac{t \|y\|_p}{\sigma^k \eta}
\, \E \left[
\Bigg|
\sum_{|\kappa|=k}
\frac{y^\kappa}{\kappa!}
\prod_{j=1}^{d}
\frac{\partial^{\kappa_j}\phi(Z_j)}{\phi(Z_j)}
\Bigg|
\right].
\end{align*}
%
Therefore, by the Cauchy--Schwarz inequality,
%
\begin{align*}
&\Bigg(
\sum_{|\kappa|=k}
\frac{y^\kappa}{\kappa!}
\big(
\partial^{\kappa} f(x + t y)
- \partial^{\kappa} f(x)
\big)
\Bigg)^2
\leq
\frac{t^2 \|y\|_p^2}{\sigma^{2k} \eta^2}
\, \E \left[
\Bigg(
\sum_{|\kappa|=k}
\frac{y^\kappa}{\kappa!}
\prod_{j=1}^{d}
\frac{\partial^{\kappa_j} \phi(Z_j)}{\phi(Z_j)}
\Bigg)^2
\right] \\
&\quad=
\frac{t^2 \|y\|_p^2}{\sigma^{2k} \eta^2}
\sum_{|\kappa|=k}
\sum_{|\kappa'|=k}
\frac{y^{\kappa + \kappa'}}{\kappa! \kappa'!}
\prod_{j=1}^{d}
\, \E \left[
\frac{\partial^{\kappa_j} \phi(Z_j)}{\phi(Z_j)}
\frac{\partial^{\kappa'_j} \phi(Z_j)}{\phi(Z_j)}
\right].
\end{align*}
%
Orthogonality of Hermite polynomials gives zero if
$\kappa_j \neq \kappa'_j$. By the multinomial theorem,
%
\begin{align*}
\left|
f(x + y)
- \sum_{|\kappa|=0}^{k}
\frac{1}{\kappa!}
\partial^{\kappa} f(x)
y^\kappa
\right|
&\leq
\frac{\|y\|_p}{\sigma^k \eta}
\Bigg(
\sum_{|\kappa|=k}
\frac{y^{2 \kappa}}{\kappa!}
\Bigg)^{1/2}
\leq
\frac{\|y\|_p}{\sigma^k \eta \sqrt{k!}}
\Bigg(
\sum_{|\kappa|=k}
\frac{k!}{\kappa!}
y^{2 \kappa}
\Bigg)^{1/2} \\
&\leq
\frac{\|y\|_p \|y\|_2^k}{\sigma^k \eta \sqrt{k!}}.
\end{align*}
%
For the final result, since
$f(x) = \E \left[ g(x + \sigma Z) \right]$ and
$\I\big\{x \in A^\eta\big\}\leq g(x)\leq \I\big\{x \in A^{2\eta}\big\}$,
%
\begin{align*}
f(x)
&\leq
\P \left( x + \sigma Z \in A^{2 \eta} \right) \\
&\leq
\P \left( \|Z\|_p > \frac{\eta}{\sigma} \right)
+ \I \left\{ x \in A^{3 \eta} \right\}
\P \left( \|Z\|_p \leq \frac{\eta}{\sigma} \right)
= \varepsilon
+ (1 - \varepsilon) \I \left\{ x \in A^{3 \eta} \right\}, \\
f(x)
&\geq
\P \left( x + \sigma Z \in A^{\eta} \right)
\geq
\I \left\{ x \in A \right\}
\P \left( \|Z\|_p \leq \frac{\eta}{\sigma} \right)
= (1 - \varepsilon) \I \left\{ x \in A \right\}.
\end{align*}
%
\end{proof}

We provide a useful Gaussian inequality in
Lemma~\ref{lem:yurinskii_app_gaussian_useful}
which helps bound the $\beta_{\infty,k}$ moment terms appearing in several
places throughout the analysis.

\begin{lemma}[A useful Gaussian inequality]%
\label{lem:yurinskii_app_gaussian_useful}

Let $X \sim \cN(0, \Sigma)$
where $\sigma_j^2 = \Sigma_{j j} \leq \sigma^2$ for all $1 \leq j \leq d$.
Then
%
\begin{align*}
\E\left[
\|X\|_2^2
\|X\|_\infty
\right]
&\leq
4 \sigma \sqrt{\log 2d}
\,\sum_{j=1}^d \sigma_j^2
&&\text{and}
&\E\left[
\|X\|_2^3
\|X\|_\infty
\right]
&\leq
8 \sigma \sqrt{\log 2d}
\,\bigg( \sum_{j=1}^d \sigma_j^2 \bigg)^{3/2}.
\end{align*}
%
\end{lemma}

\begin{proof}[Lemma~\ref{lem:yurinskii_app_gaussian_useful}]

By Cauchy--Schwarz, with $k \in \{2,3\}$, we have
$\E\left[\|X\|_2^{k} \|X\|_\infty \right]
\leq \E\big[\|X\|_2^{2k} \big]^{1/2} \E\big[\|X\|_\infty^2 \big]^{1/2}$.
For the first term, by H{\"o}lder's inequality and the even
moments of the normal distribution,
%
\begin{align*}
\E\big[\|X\|_2^4 \big]
&=
\E\Bigg[
\bigg(
\sum_{j=1}^d X_j^2
\bigg)^2
\Bigg]
=
\sum_{j=1}^d \sum_{k=1}^d
\E\big[
X_j^2 X_k^2
\big]
\leq
\bigg(
\sum_{j=1}^d
\E\big[X_j^4 \big]^{\frac{1}{2}}
\bigg)^2
=
3 \bigg(
\sum_{j=1}^d
\sigma_j^2
\bigg)^2, \\
\E\big[\|X\|_2^6 \big]
&=
\sum_{j=1}^d \sum_{k=1}^d \sum_{l=1}^d
\E\big[
X_j^2 X_k^2 X_l^2
\big]
\leq
\bigg(
\sum_{j=1}^d
\E\big[X_j^6 \big]^{\frac{1}{3}}
\bigg)^3
=
15 \bigg(
\sum_{j=1}^d
\sigma_j^2
\bigg)^3.
\end{align*}
%
For the second term, by Jensen's inequality and the $\chi^2$ moment
generating function,
%
\begin{align*}
\E\big[\|X\|_\infty^2 \big]
&=
\E\left[
\max_{1 \leq j \leq d}
X_j^2
\right]
\leq
4 \sigma^2
\log
\sum_{j=1}^d
\E\Big[
e^{X_j^2 / (4\sigma^2)}
\Big]
\leq
4 \sigma^2
\log
\sum_{j=1}^d
\sqrt{2}
\leq
4 \sigma^2
\log 2 d.
\end{align*}
%
\end{proof}

We provide an $\ell^p$-norm tail probability bound for Gaussian variables in
Lemma~\ref{lem:yurinskii_app_gaussian_pnorm}, motivating the definition of the
term
$\phi_p(d)$.

\begin{lemma}[Gaussian \texorpdfstring{$\ell^p$}{lp}-norm bound]%
\label{lem:yurinskii_app_gaussian_pnorm}
Let $X \sim \cN(0, \Sigma)$ where $\Sigma \in \R^{d \times d}$
is a positive semi-definite matrix. Then we have that
$\E\left[ \|X\|_p \right] \leq
\phi_p(d) \max_{1 \leq j \leq d} \sqrt{\Sigma_{j j}}$
with $\phi_p(d) = \sqrt{pd^{2/p} }$ for $p \in [1,\infty)$
and $\phi_\infty(d) = \sqrt{2\log 2d}$.
\end{lemma}

\begin{proof}[Lemma~\ref{lem:yurinskii_app_gaussian_pnorm}]

For $p \in [1, \infty)$,
as each $X_j$ is Gaussian, we have
$\big(\E\big[|X_j|^p\big]\big)^{1/p}
\leq \sqrt{p\, \E[X_j^2]}
= \sqrt{p \Sigma_{j j}}$.
So
%
\begin{align*}
\E\big[\|X\|_p\big]
&\leq
\Bigg(\sum_{j=1}^d \E \big[ |X_j|^p \big] \Bigg)^{1/p}
\leq \Bigg(\sum_{j=1}^d p^{p/2} \Sigma_{j j}^{p/2} \Bigg)^{1/p}
\leq \sqrt{p d^{2/p}}
\max_{1\leq j\leq d}
\sqrt{\Sigma_{j j}}
\end{align*}
%
by Jensen's inequality.
For $p=\infty$,
with $\sigma^2 = \max_j \Sigma_{j j}$,
for $t>0$,
%
\begin{align*}
\E\big[\|X\|_\infty \big]
&\leq
t
\log
\sum_{j=1}^d
\E\Big[
e^{|X_j| / t}
\Big]
\leq
t
\log
\sum_{j=1}^d
\E\Big[
2 e^{X_j / t}
\Big]
\leq t \log \Big(2 d e^{\sigma^2/(2t^2)}\Big)
\leq t \log 2 d + \frac{\sigma^2}{2t},
\end{align*}
%
again by Jensen's inequality.
Setting $t = \frac{\sigma}{\sqrt{2 \log 2d}}$ gives
$\E\big[\|X\|_\infty \big] \leq \sigma \sqrt{2 \log 2d}$.
%
\end{proof}

We give a Gaussian--Gaussian $\ell^p$-norm approximation
as Lemma~\ref{lem:yurinskii_app_feasible_gaussian}, useful for
ensuring approximations remain valid upon substituting
an estimator for the true variance matrix.

\begin{lemma}[Gaussian--Gaussian approximation in
\texorpdfstring{$\ell^p$}{lp}-norm]%
\label{lem:yurinskii_app_feasible_gaussian}

Let $\Sigma_1, \Sigma_2 \in \R^{d \times d}$ be positive semi-definite
and take $Z \sim \cN(0, I_d)$.
For $p \in [1, \infty]$ we have
%
\begin{align*}
\P\left(
\left\|
\left(\Sigma_1^{1/2} - \Sigma_2^{1/2}\right) Z
\right\|_p
> t
\right)
&\leq
2 d \exp \left(
\frac{-t^2}
{2 d^{2/p} \big\|\Sigma_1^{1/2} - \Sigma_2^{1/2}\big\|_2^2}
\right).
\end{align*}

\end{lemma}

\begin{proof}[Lemma~\ref{lem:yurinskii_app_feasible_gaussian}]

Let $\Sigma \in \R^{d \times d}$ be positive semi-definite
and write $\sigma^2_j = \Sigma_{j j} $.
For $p \in [1, \infty)$ by a union bound and
Gaussian tail probabilities,
%
\begin{align*}
\P\left(\big\| \Sigma^{1/2} Z \big\|_p > t \right)
&=
\P\Bigg(
\sum_{j=1}^d
\left|
\left(
\Sigma^{1/2} Z
\right)_j
\right|^p
> t^p \Bigg)
\leq
\sum_{j=1}^d
\P\Bigg(
\left|
\left(
\Sigma^{1/2} Z
\right)_j
\right|^p
> \frac{t^p \sigma_j^p}{\|\sigma\|_p^p}
\Bigg) \\
&=
\sum_{j=1}^d
\P\Bigg(
\left|
\sigma_j Z_j
\right|^p
> \frac{t^p \sigma_j^p}{\|\sigma\|_p^p}
\Bigg)
=
\sum_{j=1}^d
\P\left(
\left| Z_j \right|
> \frac{t}{\|\sigma\|_p}
\right)
\leq
2 d \, \exp\left( \frac{-t^2}{2 \|\sigma\|_p^2} \right).
\end{align*}
%
The same result holds for $p = \infty$ since
%
\begin{align*}
\P\left(\big\| \Sigma^{1/2} Z \big\|_\infty > t \right)
&=
\P\left(
\max_{1 \leq j \leq d}
\left|
\left(
\Sigma^{1/2} Z
\right)_j
\right|
> t \right)
\leq
\sum_{j=1}^d
\P\left(
\left|
\left(
\Sigma^{1/2} Z
\right)_j
\right|
> t
\right) \\
&=
\sum_{j=1}^d
\P\left(
\left|
\sigma_j Z_j
\right|
> t
\right)
\leq
2 \sum_{j=1}^d
\exp\left( \frac{-t^2}{2 \sigma_j^2} \right)
\leq
2 d
\exp\left( \frac{-t^2}{2 \|\sigma\|_\infty^2} \right).
\end{align*}
%
Now we apply this to the matrix
$\Sigma = \big(\Sigma_1^{1/2} - \Sigma_2^{1/2}\big)^2$.
For $p \in [1, \infty)$,
%
\begin{align*}
\|\sigma\|_p^p
&=
\sum_{j=1}^d (\Sigma_{j j})^{p/2}
=
\sum_{j=1}^d
\Big(\big(\Sigma_1^{1/2} - \Sigma_2^{1/2}\big)^2\Big)_{j j}^{p/2}
\leq
d \max_{1 \leq j \leq d}
\Big(\big(\Sigma_1^{1/2} - \Sigma_2^{1/2}\big)^2\Big)_{j j}^{p/2} \\
&\leq
d \, \Big\|\big(\Sigma_1^{1/2} - \Sigma_2^{1/2}\big)^2\Big\|_2^{p/2}
=
d \, \big\|\Sigma_1^{1/2} - \Sigma_2^{1/2}\big\|_2^p
\end{align*}
%
Similarly, for $p = \infty$ we have
%
\begin{align*}
\|\sigma\|_\infty
&=
\max_{1 \leq j \leq d}
(\Sigma_{j j})^{1/2}
=
\max_{1 \leq j \leq d}
\Big(\big(\Sigma_1^{1/2} - \Sigma_2^{1/2}\big)^2\Big)_{j j}^{1/2}
\leq
\big\|\Sigma_1^{1/2} - \Sigma_2^{1/2}\big\|_2.
\end{align*}
%
Thus for all $p \in [1, \infty]$ we have
$\|\sigma\|_p \leq
d^{1/p} \big\|\Sigma_1^{1/2} - \Sigma_2^{1/2}\big\|_2$,
with $d^{1/\infty} = 1$. Hence
%
\begin{align*}
\P\left(
\left\|
\left(\Sigma_1^{1/2} - \Sigma_2^{1/2}\right) Z
\right\|_p
> t
\right)
&\leq
2 d \exp \left( \frac{-t^2}{2 \|\sigma\|_p^2} \right)
\leq
2 d \exp \left(
\frac{-t^2}
{2 d^{2/p} \big\|\Sigma_1^{1/2} - \Sigma_2^{1/2}\big\|_2^2}
\right).
\end{align*}
%
\end{proof}

We give a variance bound and an exponential inequality for $\alpha$-mixing
variables.

\begin{lemma}[Variance bounds for
\texorpdfstring{$\alpha$}{alpha}-mixing random variables]
\label{lem:yurinskii_app_variance_mixing}

Let $X_1, \ldots, X_n$ be
real-valued $\alpha$-mixing random
variables with mixing coefficients $\alpha(j)$.
Then
%
\begin{enumerate}[label=(\roman*)]

\item
\label{it:yurinskii_app_variance_mixing_bounded}
If for constants $M_i$ we have
$|X_i| \leq M_i$ a.s.\ then
%
\begin{align*}
\Var\left[
\sum_{i=1}^n X_i
\right]
&\leq
4 \sum_{j=1}^\infty \alpha(j)
\sum_{i=1}^n M_i^2.
\end{align*}

\item
\label{it:yurinskii_app_variance_mixing_exponential}
If $\alpha(j) \leq e^{-2j / C_\alpha}$ then
for any $r>2$ there is a constant
$C_r$ depending only on $r$ with
%
\begin{align*}
\Var\left[
\sum_{i=1}^n X_i
\right]
&\leq
C_r C_\alpha
\sum_{i=1}^n
\E\big[|X_i|^r\big]^{2/r}.
\end{align*}
\end{enumerate}
%
\end{lemma}

\begin{proof}[Lemma~\ref{lem:yurinskii_app_variance_mixing}]

Define
$\alpha^{-1}(t) =
\inf\{j \in \N : \alpha(j) \leq t\}$
and $Q_i(t) = \inf\{s \in \R : \P(|X_i| > s) \leq t\}$.
By Corollary~1.1 in \citet{rio2017asymptotic}
and H{\"o}lder's inequality for $r > 2$,
%
\begin{align*}
\Var\left[
\sum_{i=1}^n X_i
\right]
&\leq
4 \sum_{i=1}^n
\int_0^1 \alpha^{-1}(t)
Q_i(t)^2 \diff{t} \\
&\leq
4 \sum_{i=1}^n
\left(
\int_0^1 \alpha^{-1}(t)^{\frac{r}{r-2}} \diff{t}
\right)^{\frac{r-2}{r}}
\left(
\int_0^1 |Q_i(t)|^r \diff{t}
\right)^{\frac{2}{r}}
\diff{t}.
\end{align*}
%
Now note that if $U \sim \Unif[0,1]$ then
$Q_i(U)$ has the same distribution as $X_i$.
Therefore
%
\begin{align*}
\Var\left[
\sum_{i=1}^n X_i
\right]
&\leq
4
\left(
\int_0^1 \alpha^{-1}(t)^{\frac r{r-2}} \diff{t}
\right)^{\frac{r-2}r}
\sum_{i=1}^n
\E[|X_i|^r]^{\frac 2 r}.
\end{align*}
%
If $\alpha(j) \leq e^{-2j/C_\alpha}$ then
$\alpha^{-1}(t) \leq \frac{-C_\alpha \log t}{2}$
so, for some constant
$C_r$ depending only on $r$,
%
\begin{align*}
\Var\left[
\sum_{i=1}^n X_i
\right]
\leq
2 C_\alpha
\left(
\int_0^1 (-\log t)^{\frac r{r-2}} \diff{t}
\right)^{\frac{r-2} r}
\sum_{i=1}^n
\E[|X_i|^r]^{\frac 2 r}
\leq
C_r C_\alpha
\sum_{i=1}^n
\E[|X_i|^r]^{\frac 2 r}.
\end{align*}
%
Alternatively, if for constants $M_i$ we have
$|X_i| \leq M_i$ a.s.\ then
%
\begin{align*}
\Var\left[
\sum_{i=1}^n X_i
\right]
&\leq
4 \int_0^1 \alpha^{-1}(t)
\diff{t}
\sum_{i=1}^n M_i^2
\leq
4 \sum_{j=1}^\infty \alpha(j)
\sum_{i=1}^n M_i^2.
\end{align*}
%
\end{proof}

\begin{lemma}[Exponential concentration inequalities for
\texorpdfstring{$\alpha$}{alpha}-mixing random variables]
\label{lem:yurinskii_app_exponential_mixing}

Let $X_1, \ldots, X_n$ be zero-mean real-valued
variables with $\alpha$-mixing coefficients
$\alpha(j) \leq e^{-2 j / C_\alpha}$.

\begin{enumerate}[label=(\roman*)]

\item
\label{it:yurinskii_app_exponential_mixing_bounded}
Suppose $|X_i| \leq M$ a.s.\ for $1 \leq i \leq n$.
Then for all $t > 0$ there is a constant $C_1$ with
%
\begin{align*}
\P\left(
\left|
\sum_{i=1}^n
X_i
\right|
> C_1 M \big( \sqrt{n t}
+ (\log n)(\log \log n) t \big)
\right)
&\leq
C_1 e^{-t}.
\end{align*}
%
\item
\label{it:yurinskii_app_exponential_mixing_bernstein}
If further $\sum_{j=1}^n |\Cov[X_i, X_j]| \leq \sigma^2$,
then for all $t > 0$ there is a constant $C_2$ with
%
\begin{align*}
\P\left(
\left|
\sum_{i=1}^n
X_i
\right|
\geq C_2 \big( (\sigma \sqrt n + M) \sqrt t
+ M (\log n)^2 t \big)
\right)
&\leq
C_2 e^{-t}.
\end{align*}

\end{enumerate}

\end{lemma}

\begin{proof}[Lemma~\ref{lem:yurinskii_app_exponential_mixing}]

\begin{enumerate}[label=(\roman*)]

\item
By Theorem~1 in \citet{merlevede2009bernstein},
%
\begin{align*}
\P\left(
\left|
\sum_{i=1}^n
X_i
\right|
> t
\right)
&\leq
\exp\left(
-\frac{C_1 t^2}{n M^2 + Mt (\log n)(\log\log n)}
\right).
\end{align*}
%
Replace $t$ by
$M \sqrt{n t} + M (\log n)(\log \log n) t$.

\item
By Theorem~2 in \citet{merlevede2009bernstein},
%
\begin{align*}
\P\left(
\left|
\sum_{i=1}^n
X_i
\right|
> t
\right)
&\leq
\exp\left(
-\frac{C_2 t^2}{n\sigma^2 + M^2 + Mt (\log n)^2}
\right).
\end{align*}
%
Replace $t$ by
$\sigma \sqrt n \sqrt t + M \sqrt t + M (\log n)^2 t$.
\end{enumerate}
%
\end{proof}

\subsection{Main results}

To establish Theorem~\ref{thm:yurinskii_sa_dependent}, we first
give the analogous result
for martingales as Lemma~\ref{lem:yurinskii_app_sa_martingale}. Our approach is
similar to
that used in modern versions of Yurinskii's coupling for independent data, as
in Theorem~1 in \citet{lecam1988} and Theorem~10 in Chapter~10 of
\citet{pollard2002user}. The proof of
Lemma~\ref{lem:yurinskii_app_sa_martingale} relies on
constructing a ``modified'' martingale, which is close to the original
martingale, but which has an $\cH_0$-measurable terminal quadratic variation.

\begin{lemma}[Strong approximation for vector-valued martingales]%
\label{lem:yurinskii_app_sa_martingale}

Let $X_1, \ldots, X_n$ be $\R^d$-valued
square-integrable random vectors
adapted to a countably generated
filtration $\cH_0, \ldots, \cH_n$.
Suppose that
$\E[X_i \mid \cH_{i-1}] = 0$ for all $1 \leq i \leq n$
and define $S = \sum_{i=1}^n X_i$.
Let $V_i = \Var[X_i \mid \cH_{i-1}]$ and
$\Omega = \sum_{i=1}^n V_i - \Sigma$
where $\Sigma$ is a positive semi-definite
$\cH_0$-measurable $d \times d$ random matrix.
For each $\eta > 0$ and $p \in [1,\infty]$
there is $T \mid \cH_0 \sim \cN(0, \Sigma)$ with
%
\begin{align*}
\P\big(\|S-T\|_p > 5\eta\big)
&\leq
\inf_{t>0}
\left\{
2 \P\big( \|Z\|_p > t \big)
+ \min\left\{
\frac{\beta_{p,2} t^2}{\eta^3},
\frac{\beta_{p,3} t^3}{\eta^4}
+ \frac{\pi_3 t^3}{\eta^3}
\right\}
\right\} \\
\nonumber
&\quad+
\inf_{M \succeq 0}
\big\{ 2\gamma(M) + \delta_p(M,\eta)
+ \varepsilon_p(M, \eta)\big\},
\end{align*}
%
where the second infimum is over all positive semi-definite
$d \times d$ non-random matrices, and
%
\begin{align*}
\beta_{p,k}
&=
\sum_{i=1}^n \E\left[\| X_i \|^k_2 \| X_i \|_p
+ \|V_i^{1/2} Z_i \|^k_2 \|V_i^{1/2} Z_i \|_p \right],
\qquad\gamma(M)
= \P\big(\Omega \npreceq M\big), \\
\delta_p(M,\eta)
&=
\P\left(
\big\|\big((\Sigma +M)^{1/2}- \Sigma^{1/2}\big) Z\big\|_p
\geq \eta
\right),
\qquad\pi_3
=
\sum_{i=1}^{n+m}
\sum_{|\kappa| = 3}
\E \Big[ \big|
\E \left[ X_i^\kappa \mid \cH_{i-1} \right]
\big| \Big], \\
\varepsilon_p(M, \eta)
&=
\P\left(\big\| (M - \Omega)^{1/2} Z \big\|_p\geq \eta, \
\Omega \preceq M\right),
\end{align*}
%
for $k \in \{2,3\}$, with $Z, Z_1,\dots ,Z_n$ i.i.d.\ standard Gaussian
on $\R^d$ independent of $\cH_n$.
\end{lemma}

\begin{proof}[Lemma~\ref{lem:yurinskii_app_sa_martingale}]

\proofparagraph{constructing a modified martingale}

Take $M \succeq 0$ a fixed positive semi-definite
$d \times d$ matrix.
We start by constructing a new martingale based on $S$
whose quadratic variation is $\Sigma + M$.
Take $m \geq 1$ and define
%
\begin{align*}
H_k
&=
\Sigma
+ M
- \sum_{i=1}^{k} V_i,
\qquad\qquad\qquad\qquad\tau
=
\sup \big\{ k\in\{0,1,\dots,n\} : H_k \succeq 0 \big\}, \\
\tilde X_i
&=
X_i\I\{i \leq \tau\}
+ \frac{1}{\sqrt{m}} H_\tau^{1/2} Z_i\I\{n+1 \leq i \leq n+m\},
\qquad\qquad\tilde S
=
\sum_{i=1}^{n+m} \tilde X_i,
\end{align*}
%
where $Z_{n+1}, \ldots, Z_{n+m}$ is an i.i.d.\
sequence of standard Gaussian vectors in $\R^d$
independent of $\cH_n$,
noting that $H_0 = \Sigma + M \succeq 0$ a.s.
Define the filtration
$\tilde \cH_0, \ldots, \tilde \cH_{n+m}$,
where $\tilde \cH_i = \cH_i$ for $0 \leq i \leq n$
and is the $\sigma$-algebra generated by
$\cH_n$ and $Z_{n+1}, \dots, Z_{i}$ for $n+1 \leq i\leq n+m$.
Observe that $\tau$ is a stopping time with respect to $\tilde\cH_i$
because $H_{i+1} - H_i = -V_{i+1} \preceq 0$ almost surely,
so $\{\tau \leq i\} = \{H_{i+1} \nsucceq 0\}$ for $0\leq i<n$.
This depends only on $V_1, \dots, V_{i+1}$ and $\Sigma$
which are $\tilde\cH_i$-measurable.
Similarly, $\{\tau = n\} = \{H_n \succeq 0\} \in \tilde\cH_{n-1}$.
Let $\tilde V_i = V_i \I\{i\leq\tau\}$ for
$1\leq i\leq n$ and
$\tilde V_i = H_\tau/m$ for $n+1\leq i\leq n+m$.
Note that $\tilde X_i$ is $\tilde \cH_i$-measurable
and $\tilde V_i$ is $\tilde \cH_{i-1}$-measurable.
Further, $\E \left[ \tilde X_i \mid \tilde \cH_{i-1} \right] = 0$ and
$\E \left[ \tilde X_i \tilde X_i^\T \mid \tilde \cH_{i-1} \right]
= \tilde V_i$.

\proofparagraph{bounding the difference between the original and
modified martingales}

By the triangle inequality,
%
\begin{align*}
\|S - \tilde S \|_p
&\leq
\left\| \sum_{i=\tau+1}^n X_i \right\|_p
+ \left\| \frac{1}{\sqrt{m}} \sum_{i=n+1}^m H_\tau^{1/2} Z_i \right\|_p.
\end{align*}
%
The first term on the right vanishes on
$\{\tau = n\} = \{H_n \succeq 0\} = \{\Omega \preceq M\}$.
For the second term, note that
$\tfrac{1}{\sqrt{m}}\sum_{i=n+1}^m H_\tau^{1/2} Z_i$
is distributed as $H_\tau^{1/2}Z$,
where $Z$ is an independent standard Gaussian variable.
Also
$\P\big( \| H_\tau^{1/2} Z \|_p > \eta \big)
\leq \P\big( \| H_n^{1/2} Z \|_p > \eta,\, \Omega \preceq M)
+ \P\big( \Omega \npreceq M \big)$,
so
%
\begin{align*}%
\label{eq:yurinskii_app_approx_modified_original}
\P\big( \| S - \tilde S \|_p > \eta\big)
&\leq
2 \P\big(\Omega \npreceq M \big)
+ \P\big( \| (M-\Omega)^{1/2}Z \|_p > \eta,\,
\Omega \preceq M \big)
= 2 \gamma(M) + \varepsilon_p(M, \eta).
\end{align*}

\proofparagraph{strong approximation of the modified martingale}

Let $\tilde Z_1, \ldots, \tilde Z_{n+m}$ be i.i.d.\ $\cN(0, I_d)$
and independent of $\tilde \cH_{n+m}$.
Define $\check X_i = \tilde V_i^{1/2} \tilde Z_i$
and $\check S = \sum_{i=1}^{n+m} \check X_i$.
Fix a Borel set $A \subseteq \R^d$ and $\sigma, \eta > 0$ and
let $f = f_{A\eta\sigma}$ be the function defined in
Lemma~\ref{lem:yurinskii_app_smooth_approximation}.
By the Lindeberg method, write the telescoping sum
%
\begin{align*}
\E\Big[f\big(\tilde S\big) - f\big(\check S\big)
\mid \cH_0 \Big]
&=
\sum_{i=1}^{n+m}
\E\Big[ f\big(Y_i + \tilde X_i\big)
- f\big(Y_i + \check X_i\big)
\mid \cH_0 \Big]
\end{align*}
%
where
$Y_i = \sum_{j=1}^{i-1} \tilde X_j + \sum_{j=i+1}^{n+m} \check X_j$.
By Lemma~\ref{lem:yurinskii_app_smooth_approximation} we have for $k \geq 0$
%
\begin{align*}
&\Bigg|
\E\big[
f(Y_i + \tilde X_i)
- f(Y_i + \check X_i)
\mid \cH_0
\big]
- \sum_{|\kappa| = 0}^k
\frac{1}{\kappa!}
\E \left[
\partial^\kappa f(Y_i)
\left( \tilde X_i^\kappa - \check X_i^\kappa \right)
\bigm| \cH_0
\right]
\Bigg| \\
&\quad\leq
\frac{1}{\sigma^k \eta \sqrt{k!}}
\E \left[
\|\tilde X_i\|_p \|\tilde X_i\|_2^k
+ \|\check X_i\|_p \|\check X_i\|_2^k
\bigm| \cH_0
\right].
\end{align*}
%
With $k \in \{2, 3\}$, we bound each summand.
With $|\kappa| = 0$ we have
$\tilde X_i^\kappa = \check X_i^\kappa$,
so consider $|\kappa| = 1$.
Noting that $\sum_{i=1}^{n+m} \tilde V_i = \Sigma + M$, define
%
\begin{align*}
\tilde Y_i
&=
\sum_{j=1}^{i-1} \tilde X_j
+ \tilde Z_i
\Bigg(\sum_{j=i+1}^{n+m} \tilde V_j\Bigg)^{1/2}
=
\sum_{j=1}^{i-1} \tilde X_j
+ \tilde Z_i
\Bigg(\Sigma + M - \sum_{j=1}^{i} \tilde V_j\Bigg)^{1/2}
\end{align*}
%
and let $\check \cH_i$ be the $\sigma$-algebra generated by
$\tilde \cH_{i-1}$ and $\tilde Z_i$.
Note that $\tilde Y_i$ is $\check \cH_i$-measurable
and that $Y_i$ and $\tilde Y_i$
have the same distribution conditional on $\tilde \cH_{n+m}$. So
%
\begin{align*}
&\sum_{|\kappa| = 1}
\frac{1}{\kappa!}
\E\left[
\partial^\kappa f(Y_i)
\big( \tilde X_i^\kappa - \check X_i^\kappa \big)
\bigm| \cH_0
\right]
= \E \left[
\nabla f(Y_i)^\T
\big( \tilde X_i - \tilde V_i^{1/2} \tilde Z_i \big)
\bigm| \cH_0
\right] \\
&\quad=
\E \left[
\nabla f(\tilde Y_i)^\T \tilde X_i
\bigm| \cH_0
\right]
- \E \left[
\nabla f(Y_i)^\T \tilde V_i^{1/2} \tilde Z_i
\bigm| \cH_0
\right] \\
&\quad=
\E \left[
\nabla f(\tilde Y_i)^\T
\E \left[
\tilde X_i
\mid \check \cH_i
\right]
\bigm| \cH_0
\right]
- \E \left[
\tilde Z_i
\right]
\E \left[
\nabla f(Y_i)^\T \tilde V_i^{1/2}
\bigm| \cH_0
\right] \\
&\quad=
\E \left[
\nabla f(\tilde Y_i)^\T
\E \left[
\tilde X_i
\mid \tilde \cH_{i-1}
\right]
\bigm| \cH_0
\right]
- 0
= 0.
\end{align*}
%
Next, if $|\kappa| = 2$ then
%
\begin{align*}
&\sum_{|\kappa| = 2}
\frac{1}{\kappa!}
\E \left[
\partial^\kappa f(Y_i)
\left( \tilde X_i^\kappa - \check X_i^\kappa \right)
\bigm| \cH_0
\right] \\
&\quad=
\frac{1}{2}
\E \left[
\tilde X_i^\T \nabla^2 f(Y_i) \tilde X_i
- \tilde Z_i^\T \tilde V_i^{1/2} \nabla^2 f(Y_i)
\tilde V_i^{1/2} \tilde Z_i
\bigm| \cH_0
\right] \\
&\quad=
\frac{1}{2}
\E \left[
\E \left[
\Tr \nabla^2 f(\tilde Y_i) \tilde X_i \tilde X_i^\T
\bigm| \check \cH_i
\right]
\bigm| \cH_0
\right]
- \frac{1}{2}
\E \left[
\Tr \tilde V_i^{1/2} \nabla^2 f(Y_i) \tilde V_i^{1/2}
\bigm| \cH_0
\right]
\E \left[
\tilde Z_i \tilde Z_i^\T
\right] \\
&\quad=
\frac{1}{2}
\E \left[
\Tr \nabla^2 f(Y_i)
\E \left[
\tilde X_i \tilde X_i^\T
\bigm| \tilde \cH_{i-1}
\right]
\bigm| \cH_0
\right]
- \frac{1}{2}
\E \left[
\Tr \nabla^2 f(Y_i) \tilde V_i
\bigm| \cH_0
\right]
= 0.
\end{align*}
%
Finally, if $|\kappa| = 3$, then since
$\check X_i \sim \cN(0, \tilde V_i)$
conditional on $\tilde \cH_{n+m}$, we have by symmetry of the Gaussian
distribution and Lemma~\ref{lem:yurinskii_app_smooth_approximation},
%
\begin{align*}
&
\left|
\sum_{|\kappa| = 3}
\frac{1}{\kappa!}
\E \left[
\partial^\kappa f(Y_i)
\left( \tilde X_i^\kappa - \check X_i^\kappa \right)
\bigm| \cH_0
\right]
\right|
\\
&\quad=
\left|
\sum_{|\kappa| = 3}
\frac{1}{\kappa!}
\left(
\E \left[
\partial^\kappa f(\tilde Y_i)
\E \left[ \tilde X_i^\kappa \mid \check \cH_i \right]
\bigm| \cH_0
\right]
- \E \left[
\partial^\kappa f(Y_i) \,
\E \left[
\check X_i^\kappa
\bigm| \tilde \cH_{n+m}
\right]
\bigm| \cH_0
\right]
\right)
\right|
\\
&\quad=
\left|
\sum_{|\kappa| = 3}
\frac{1}{\kappa!}
\E \left[
\partial^\kappa f(Y_i) \,
\E \left[ \tilde X_i^\kappa \mid \tilde \cH_{i-1} \right]
\bigm| \cH_0
\right]
\right|
\leq
\frac{1}{\sigma^3}
\sum_{|\kappa| = 3}
\E \left[
\left|
\E \left[ \tilde X_i^\kappa \mid \tilde \cH_{i-1} \right]
\right|
\bigm| \cH_0
\right].
\end{align*}
%
Combining these and summing over $i$ with $k=2$ shows
%
\begin{align*}
\E\left[
f\big(\tilde S\big) - f\big(\check S\big)
\bigm| \cH_0
\right]
&\leq
\frac{1}{\sigma^2 \eta \sqrt{2}}
\sum_{i=1}^{n+m}
\E \left[
\|\tilde X_i\|_p \|\tilde X_i\|_2^2
+ \|\check X_i\|_p \|\check X_i\|_2^2
\bigm| \cH_0
\right]
\end{align*}
%
On the other hand, taking $k = 3$ gives
%
\begin{align*}
\E\left[
f\big(\tilde S\big) - f\big(\check S\big)
\bigm| \cH_0
\right]
&\leq
\frac{1}{\sigma^3 \eta \sqrt{6}}
\sum_{i=1}^{n+m}
\E \left[
\|\tilde X_i\|_p \|\tilde X_i\|_2^3
+ \|\check X_i\|_p \|\check X_i\|_2^3
\bigm| \cH_0
\right] \\
&\quad+
\frac{1}{\sigma^3}
\sum_{i=1}^{n+m}
\sum_{|\kappa| = 3}
\E \left[
\left|
\E \left[ \tilde X_i^\kappa \mid \tilde \cH_{i-1} \right]
\right|
\bigm| \cH_0
\right].
\end{align*}
%
For $1 \leq i \leq n$ we have
$\|\tilde X_i\| \leq \|X_i\|$
and $\|\check X_i\| \leq \|V_i^{1/2} \tilde Z_i\|$.
For $n+1 \leq i \leq n+m$ we have
$\tilde X_i = H_\tau^{1/2} Z_i / \sqrt m$
and $\check X_i = H_\tau^{1/2} \tilde Z_i / \sqrt m$
which are equal in distribution given $\cH_0$.
So with
%
\begin{align*}
\tilde \beta_{p,k}
&=
\sum_{i=1}^{n}
\E \left[
\|X_i\|_p \|X_i\|_2^k
+ \|V_i^{1/2} Z_i\|_p \|V_i^{1/2} Z_i\|_2^k
\bigm| \cH_0
\right],
\end{align*}
%
we have, since $k \in \{2,3\}$,
%
\begin{align*}
&\sum_{i=1}^{n+m}
\E \left[
\|\tilde X_i\|_p \|\tilde X_i\|_2^k
+ \|\check X_i\|_p \|\check X_i\|_2^k
\bigm| \cH_0
\right]
\leq
\tilde\beta_{p,k}
+ \frac{2}{\sqrt m}
\E \left[
\|H_\tau^{1/2} Z\|_p \|H_\tau^{1/2} Z\|_2^k
\bigm| \cH_0
\right].
\end{align*}
%
Since $H_i$ is weakly decreasing under the
semi-definite partial order, we have
$H_\tau \preceq H_0 = \Sigma + M$
implying that $|(H_\tau)_{j j}| \leq \|\Sigma + M\|_{\max}$ and
$\E\big[|(H_\tau^{1/2} Z)_j|^3 \mid \cH_0 \big]
\leq \sqrt{8/\pi}\, \|\Sigma + M\|_{\max}^{3/2}$.
Hence as $p \geq 1$ and $k \in \{2,3\}$,
%
\begin{align*}
\E\left[
\|H_\tau^{1/2}Z\|_p
\|H_\tau^{1/2}Z\|_2^k
\bigm| \cH_0
\right]
&\leq
\E\left[\|H_\tau^{1/2} Z\|_1^{k+1}
\bigm| \cH_0
\right]
\leq
d^{k+1} \max_{1\leq j\leq d}
\E\left[|(H_\tau^{1/2} Z)_j|^{k+1}
\bigm| \cH_0
\right] \\
&\leq 3 d^4 \,
\|\Sigma + M\|_{\max}^{(k+1)/2}
\leq 6 d^4 \,
\|\Sigma \|_{\max}^{(k+1)/2}
+ 6 d^4 \|M\|.
\end{align*}
%
Assuming some $X_i$ is not identically zero so
the result is non-trivial,
and supposing that $\Sigma$ is bounded a.s.\
(replacing $\Sigma$ by $\Sigma \cdot \I\{\|\Sigma\|_{\max} \leq C\}$
for an appropriately large $C$ if necessary),
take $m$ large enough that
%
\begin{align}
\label{eq:yurinskii_app_bound_extra_terms}
\frac{2}{\sqrt m}
\E \left[
\|H_\tau^{1/2} Z\|_p \|H_\tau^{1/2} Z\|_2^k
\bigm| \cH_0
\right]
\leq
\frac{1}{4}
\beta_{p,k}.
\end{align}
%
Further, if $|\kappa| = 3$ then
$\big|\E \big[
\tilde X_i^\kappa \mid \tilde \cH_{i-1} \big]\big|
\leq \big| \E \left[ X_i^\kappa \mid \cH_{i-1} \right]\big|$
for $1 \leq i \leq n$
while by symmetry of the Gaussian distribution
$\E \left[ \tilde X_i^\kappa \mid \tilde \cH_{i-1} \right] = 0$
for $n+1 \leq i \leq n+m$.
Hence with
%
\begin{align*}
\tilde \pi_3
&=
\sum_{i=1}^{n+m}
\sum_{|\kappa| = 3}
\E \Big[ \big|
\E \left[ X_i^\kappa \mid \cH_{i-1} \right]
\big| \mid \cH_0 \Big],
\end{align*}
%
we have
%
\begin{align*}
\E\left[
f\big(\tilde S\big) - f\big(\check S\big)
\bigm| \cH_0
\right]
&\leq
\min \left\{
\frac{3 \tilde \beta_{p,2}}{4 \sigma^2 \eta}
+ \frac{\beta_{p,2}}{4 \sigma^2 \eta},
\frac{3 \tilde \beta_{p,3}}{4 \sigma^3 \eta}
+ \frac{\beta_{p,3}}{4 \sigma^3 \eta}
+ \frac{\tilde \pi_3}{\sigma^3}
\right\}.
\end{align*}
%
Along with Lemma~\ref{lem:yurinskii_app_smooth_approximation}, and with
$\sigma = \eta / t$ and $\varepsilon = \P(\|Z\|_p > t)$,
we conclude that
%
\begin{align*}
&\P(\tilde S \in A \mid \cH_0)
=
\E\big[\I\{\tilde S \in A\} - f(\tilde S)
\mid \cH_0
\big]
+ \E\big[f(\tilde S) - f\big(\check S\big)
\mid \cH_0
\big]
+ \E \big[f\big(\check S\big)
\mid \cH_0
\big] \\
&\,\leq
\varepsilon\P(\tilde S \in A
\mid \cH_0)
+ \min \! \left\{
\frac{3 \tilde \beta_{p,2}}{4 \sigma^2 \eta}
+ \frac{\beta_{p,2}}{4 \sigma^2 \eta},
\frac{3 \tilde \beta_{p,3}}{4 \sigma^3 \eta}
+ \frac{\beta_{p,3}}{4 \sigma^3 \eta}
+ \frac{\tilde \pi_3}{\sigma^3}
\right\}
+
\varepsilon
+ (1 - \varepsilon) \P\big(\check S \in A_p^{3\eta}
\mid \cH_0
\big) \\
&\,\leq
\P\big( \check S \in A_p^{3\eta}
\mid \cH_0
\big)
+ 2 \P(\|Z\|_p > t)
+ \min\!\left\{
\frac{3 \tilde \beta_{p,2} t^2}{4 \eta^3}
+ \frac{\beta_{p,2} t^2}{4 \eta^3},
\frac{3 \tilde \beta_{p,3} t^3}{4 \eta^4}
+ \frac{\beta_{p,3} t^3}{4 \eta^4}
+ \frac{\tilde \pi_3 t^3}{\eta^3}
\right\}.
\end{align*}
%
Taking a supremum and an outer expectation yields
with $\beta_{p,k} = \E\big[\tilde \beta_{p,k}\big]$
and $\pi_3 = \E[\tilde \pi_3]$,
%
\begin{align*}
&\E^* \left[
\sup_{A \in \cB(\R^d)}
\left\{
\P(\tilde S \in A \mid \cH_0)
- \P\big( \check S \in A_p^{3\eta} \mid \cH_0 \big)
\right\}
\right] \\
&\quad\leq
2 \P(\|Z\|_p > t)
+ \min \left\{
\frac{\beta_{p,2} t^2}{\eta^3},
\frac{\beta_{p,3} t^3}{\eta^4}
+ \frac{\pi_3 t^3}{\eta^3}
\right\}.
\end{align*}
%
Finally, since
$\check S = \sum_{i=1}^n \tilde V_i^{1/2} \tilde Z_i
\sim \cN(0,\Sigma + M)$ conditional on $\cH_0$,
the conditional Strassen theorem
in Lemma~\ref{lem:yurinskii_app_strassen}
ensures the existence of $\tilde S$ and
$\tilde T \mid \cH_0 \sim \cN(0, \Sigma + M)$
such that
%
\begin{align}
\label{eq:yurinskii_app_approx_modified_martingale}
\P\left(\|\tilde S-\tilde T\|_p>3\eta\right)
&\leq
\inf_{t>0}
\left\{
2 \P(\|Z\|_p > t)
+ \min \left\{
\frac{\beta_{p,2} t^2}{\eta^3},
\frac{\beta_{p,3} t^3}{\eta^4} + \frac{\pi_3 t^3}{\eta^3}
\right\}
\right\},
\end{align}
%
since the infimum is attained by continuity of $\|Z\|_p$.

\proofparagraph{conclusion}

We show how to write
$\tilde T = (\Sigma + M)^{1/2} W$
where $W \sim \cN(0,I_d)$
and use this representation to construct
$T \mid \cH_0 \sim \cN(0, \Sigma)$.
By the spectral theorem, let $\Sigma + M = U \Lambda U^\T$
where $U$ is a $d \times d$ orthogonal random matrix
and $\Lambda$ is a diagonal $d \times d$ random matrix with
diagonal entries satisfying
$\lambda_1 \geq \cdots \geq \lambda_r > 0$
and $\lambda_{r+1} = \cdots = \lambda_d = 0$
where $r = \rank (\Sigma + M)$.
Let $\Lambda^+$ be the Moore--Penrose pseudo-inverse of $\Lambda$
(obtained by inverting its non-zero elements) and define
$W = U (\Lambda^+)^{1/2} U^\T \tilde T + U \tilde W$, where
the first $r$ elements of $\tilde W$ are zero
and the last $d-r$ elements are i.i.d.\ $\cN(0,1)$
independent from $\tilde T$.
Then, it is easy to check that
$W \sim \cN(0, I_d)$ and that
$\tilde T = (\Sigma + M)^{1/2} W$.
Now define $T = \Sigma^{1/2} W$ so
%
\begin{equation}%
\label{eq:yurinskii_app_approx_target}
\P\big(\|T - \tilde T\|_p > \eta\big)
= \P\big(\big\|\big((\Sigma + M)^{1/2}
- \Sigma^{1/2} \big) W \big\|_p>\eta \big)
= \delta_p(M, \eta).
\end{equation}
%
Finally
\eqref{eq:yurinskii_app_approx_modified_original},
\eqref{eq:yurinskii_app_approx_modified_martingale},
\eqref{eq:yurinskii_app_approx_target},
the triangle inequality,
and a union bound conclude the proof since
by taking an infimum over $M \succeq 0$,
and by possibly reducing the constant of $1/4$ in
\eqref{eq:yurinskii_app_bound_extra_terms} to account for
this infimum being potentially unattainable,
%
\begin{align*}
\P\big(\|S-T\|_p > 5\eta\big)
&\leq
\P\big(\|\tilde S - \tilde T \|_p > 3\eta \big)
+\P\big(\|S - \tilde S \|_p > \eta\big)
+\P\big(\|T - \tilde T \|_p > \eta\big) \\
&\leq
\inf_{t>0}
\left\{
2 \P\big( \|Z\|_p > t \big)
+ \min\left\{
\frac{\beta_{p,2} t^2}{\eta^3},
\frac{\beta_{p,3} t^3}{\eta^4}
+ \frac{\pi_3 t^3}{\eta^3}
\right\}
\right\} \\
&\quad+
\inf_{M \succeq 0}
\big\{ 2\gamma(M) + \delta_p(M,\eta)
+ \varepsilon_p(M, \eta)\big\}.
\end{align*}
%
\end{proof}

Lemma~\ref{lem:yurinskii_app_sa_martingale} and the martingale approximation
immediately yield Theorem~\ref{thm:yurinskii_sa_dependent}.

\begin{proof}[Theorem~\ref{thm:yurinskii_sa_dependent}]
Apply Lemma~\ref{lem:yurinskii_app_sa_martingale} to
the martingale $\sum_{i=1}^{n} \tilde X_i$,
noting that $S - \sum_{i=1}^{n} \tilde X_i = U$.
\end{proof}

Bounding the quantities
in Theorem~\ref{thm:yurinskii_sa_dependent} gives a
user-friendly version as Proposition~\ref{pro:yurinskii_sa_simplified}.

\begin{proof}[Proposition~\ref{pro:yurinskii_sa_simplified}]

Set $M = \nu^2 I_d$ and
bound the terms appearing
the main inequality in Proposition~\ref{pro:yurinskii_sa_simplified}.

\proofparagraph{bounding $\P( \|Z\|_p > t )$}

By Markov's inequality and Lemma~\ref{lem:yurinskii_app_gaussian_pnorm},
we have
$\P( \|Z\|_p > t ) \leq \E[\|Z\|_p] / t \leq \phi_p(d) / t$.

\proofparagraph{bounding $\gamma(M)$}

With $M = \nu^2 I_d$,
by Markov's inequality,
$\gamma(M) = \P\big(\Omega \npreceq M\big)
= \P\big(\|\Omega\|_2 > \nu^2 \big)
\leq \nu^{-2} \E[\|\Omega\|_2]$.

\proofparagraph{bounding $\delta(M, \eta)$}

By Markov's inequality and Lemma~\ref{lem:yurinskii_app_gaussian_pnorm},
using
$\max_j |M_{j j}| \leq \|M\|_2$
for $M \succeq 0$,
%
\begin{align*}
\delta_{p}(M,\eta)
&= \P\left(
\big\|\big((\Sigma +M)^{1/2}- \Sigma^{1/2}\big) Z\big\|_p
\geq \eta
\right)
\leq \frac{\phi_p(d)} {\eta}
\E \left[
\big\|(\Sigma +M)^{1/2}- \Sigma^{1/2}\big\|_2
\right].
\end{align*}
%
For semi-definite matrices
the eigenvalue operator commutes with smooth matrix functions so
%
\begin{align*}
\|(\Sigma +M)^{1/2}- \Sigma^{1/2}\|_2
&=
\max_{1 \leq j \leq d}
\left|
\sqrt{\lambda_j(\Sigma) + \nu^2} - \sqrt{\lambda_j(\Sigma)}
\right|
\leq \nu
\end{align*}
%
and hence $\delta_{p}(M,\eta) \leq \phi_p(d)\nu / \eta$.

\proofparagraph{bounding $\varepsilon(M, \eta)$}

Note that $(M -\Omega)^{1/2}Z$ is a centered Gaussian
conditional on $\cH_n$,
on the event $\{\Omega \preceq M\}$.
We thus have by Markov's inequality,
Lemma~\ref{lem:yurinskii_app_gaussian_pnorm},
and Jensen's inequality that
%
\begin{align*}
\varepsilon_p(M, \eta)
&= \P\left(\big\| (M - \Omega)^{1/2} Z \big\|_p\geq \eta, \
\Omega \preceq M\right)
\leq
\frac{1}{\eta}
\E\left[
\I\{\Omega \preceq M\}
\E\left[
\big\| (M - \Omega)^{1/2} Z \big\|_p
\mid \cH_n
\right]
\right] \\
&\leq
\frac{\phi_p(d)}{\eta}
\E\left[
\I\{\Omega \preceq M\}
\max_{1 \leq j \leq d}
\sqrt{(M - \Omega)_{j j}}
\right]
\leq
\frac{\phi_p(d)}{\eta}
\E\left[
\sqrt{\|M - \Omega\|_2}
\right] \\
&\leq
\frac{\phi_p(d)}{\eta}
\E\left[
\sqrt{\|\Omega\|_2} + \nu
\right]
\leq
\frac{\phi_p(d)}{\eta}
\left(\sqrt{\E[\|\Omega\|_2]} + \nu \right).
\end{align*}
%
Thus by Theorem~\ref{thm:yurinskii_sa_dependent} and the previous parts,
%
\begin{align*}
\P\big(\|S-T\|_p > 6\eta\big)
&\leq
\inf_{t>0}
\left\{
2 \P\big(\|Z\|_p>t\big)
+ \min\left\{
\frac{\beta_{p,2} t^2}{\eta^3},
\frac{\beta_{p,3} t^3}{\eta^4}
+ \frac{\pi_3 t^3}{\eta^3}
\right\}
\right\} \\
&\quad+
\inf_{M \succeq 0}
\big\{ 2\gamma(M) + \delta_p(M,\eta)
+ \varepsilon_p(M, \eta)\big\}
+\P\big(\|U\|_p>\eta\big) \\
&\leq
\inf_{t>0}
\left\{
\frac{2 \phi_p(d)}{t}
+ \min\left\{
\frac{\beta_{p,2} t^2}{\eta^3},
\frac{\beta_{p,3} t^3}{\eta^4}
+ \frac{\pi_3 t^3}{\eta^3}
\right\}
\right\} \\
&\quad+
\inf_{\nu > 0}
\left\{ \frac{2\E \left[ \|\Omega\|_2 \right]}{\nu^2}
+ \frac{2 \phi_p(d) \nu}{\eta}
\right\}
+ \frac{\phi_p(d) \sqrt{\E \left[ \|\Omega\|_2 \right]}}{\eta}
+\P\big(\|U\|_p>\eta\big).
\end{align*}
%
Set $t = 2^{1/3} \phi_p(d)^{1/3} \beta_{p,2}^{-1/3} \eta$
and $\nu = \E[\|\Omega\|_2]^{1/3} \phi_p(d)^{-1/3} \eta^{1/3}$,
then replace $\eta$ with $\eta / 6$ to see
%
\begin{align*}
\P\big(\|S-T\|_p > 6\eta\big)
&\leq
24 \left(
\frac{\beta_{p,2} \phi_p(d)^2}{\eta^3}
\right)^{1/3}
+ 17 \left(
\frac{\E \left[ \|\Omega\|_2 \right] \phi_p(d)^2}{\eta^2}
\right)^{1/3}
+\P\left(\|U\|_p>\frac{\eta}{6}\right).
\end{align*}
%
Whenever $\pi_3 = 0$ we can set
$t = 2^{1/4} \phi_p(d)^{1/4} \beta_{p,3}^{-1/4} \eta$,
and with $\nu$ as above we obtain
%
\begin{align*}
\P\big(\|S-T\|_p > \eta\big)
&\leq
24 \left(
\frac{\beta_{p,3} \phi_p(d)^3}{\eta^4}
\right)^{1/4}
+ 17 \left(
\frac{\E \left[ \|\Omega\|_2 \right] \phi_p(d)^2}{\eta^2}
\right)^{1/3}
+\P\left(\|U\|_p>\frac{\eta}{6}\right).
\end{align*}
%
\end{proof}

After establishing Proposition~\ref{pro:yurinskii_sa_simplified},
Corollaries~\ref{cor:yurinskii_sa_mixingale},
\ref{cor:yurinskii_sa_martingale},
and \ref{cor:yurinskii_sa_indep} follow easily.

\begin{proof}[Corollary~\ref{cor:yurinskii_sa_mixingale}]
Proposition~\ref{pro:yurinskii_sa_simplified} with
$\P ( \|U\|_p > \frac{\eta}{6} )
\leq \frac{6}{\eta} \sum_{i=1}^{n} c_i (\zeta_{i} + \zeta_{n-i+1})$.
\end{proof}

\begin{proof}[Corollary~\ref{cor:yurinskii_sa_martingale}]
By Proposition~\ref{pro:yurinskii_sa_simplified}
with $U=0$ a.s.
\end{proof}

\begin{proof}[Corollary~\ref{cor:yurinskii_sa_indep}]
By Corollary~\ref{cor:yurinskii_sa_martingale}
with $\Omega=0$ a.s.
\end{proof}

We conclude this section with a discussion expanding on the comments made
in Remark~\ref{rem:yurinskii_coupling_bounds_probability} on deriving bounds in
probability from Yurinskii's coupling. Consider for illustration the
independent data second-order result given in
Corollary~\ref{cor:yurinskii_sa_indep}: for each $\eta > 0$,
there exists $T_n \mid \cH_0 \sim \cN(0, \Sigma)$ satisfying
%
\begin{align*}
\P\big(\|S_n-T_n\|_p > \eta\big)
&\leq
24 \left(
\frac{\beta_{p,2} \phi_p(d)^2}{\eta^3}
\right)^{1/3},
\end{align*}
%
where here we make explicit the dependence on the sample size $n$ for clarity.
The naive approach to converting this into a probability bound for
$\|S_n-T_n\|_p$ is to select $\eta$ to ensure the right-hand side is
of order $1$, arguing that the probability can then be made arbitrarily
small by taking, in this case, $\eta$ to be a large enough multiple of
$\beta_{p,2}^{1/3} \phi_p(d)^{2/3}$. However, the somewhat subtle mistake is
in neglecting the fact that the realization of the coupling variable $T_n$
will in general depend on $\eta$, rendering the resulting
bound invalid.
As an explicit example of this phenomenon, take $\eta > 1$ and suppose
$\|S_n - T_n(\eta)\| = \eta$ with probability $1 - 1/\eta$ and
$\|S_n - T_n(\eta)\| = n$ with probability $1/\eta$.
Then $\P\big(\|S_n - T_n(\eta)\| > \eta\big) = 1/\eta$
but it is not true for any $\eta$ that $\|S_n - T_n(\eta)\| \lesssim_\P 1$.

We propose in Remark~\ref{rem:yurinskii_coupling_bounds_probability} the
following fix.
Instead of selecting $\eta$ to ensure the right-hand side is of order $1$,
we instead choose it so the bound converges (slowly) to zero. This is
easily achieved by taking the naive and incorrect bound and multiplying
by some divergent sequence $R_n$. The resulting inequality reads,
in the case of Corollary~\ref{cor:yurinskii_sa_indep} with
$\eta = \beta_{p,2}^{1/3} \phi_p(d)^{2/3} R_n$,
%
\begin{align*}
\P\Big(\|S_n-T_n\|_p >
\beta_{p,2}^{1/3} \phi_p(d)^{2/3} R_n
\Big)
&\leq
\frac{24}{R_n}
\to 0.
\end{align*}
%
We thus recover, for the price of a rate which is slower by an arbitrarily
small amount, a valid upper bound in probability, as we can immediately
conclude that
%
\begin{align*}
\|S_n-T_n\|_p
\lesssim_\P
\beta_{p,2}^{1/3} \phi_p(d)^{2/3} R_n.
\end{align*}

\subsection{Strong approximation for martingale empirical processes}

We begin by presenting some calculations omitted from the main text
relating to the motivating example of kernel density estimation with
i.i.d.\ data.
First, the bias is bounded as
%
\begin{align*}
\big| \E \big[ \hat g(x) \big] - g(x) \big|
&=
\left|
\int_{\frac{-x}{h}}^{\frac{1-x}{h}}
K(\xi)
\diff \xi
- 1
\right|
\leq
2 \int_{\frac{a}{h}}^\infty
\frac{1}{\sqrt{2 \pi}}
e^{-\frac{\xi^2}{2}}
\diff \xi
\leq
\frac{h}{a}
\sqrt{\frac{2}{\pi}}
e^{-\frac{a^2}{2 h^2}}.
\end{align*}
%
Next, we do the calculations necessary to apply
Corollary~\ref{cor:yurinskii_sa_indep}.
Define $k_{i j} = \frac{1}{n h} K \left( \frac{X_i - x_j}{h} \right)$ and
$k_i = (k_{i j} : 1 \leq j \leq N)$.
Then $\|k_i\|_\infty \leq \frac{1}{n h \sqrt{2 \pi}}$ a.s.\ and
$\E[\|k_i\|_2^2] \leq \frac{N}{n^2 h} \int_{-\infty}^\infty K(\xi)^2 \diff \xi
\leq \frac{N}{2 n^2 h \sqrt{\pi}}$.
Let $V = \Var[k_i] \in \R^{N \times N}$,
so assuming that $1/h \geq \log 2 N$,
by Lemma~\ref{lem:yurinskii_app_gaussian_useful} we bound
%
\begin{align*}
\beta_{\infty,2}
&=
n \E\left[\| k_i \|^2_2 \| k_i \|_\infty
\right]
+ n \E \left[ \|V^{1/2} Z \|^2_2 \|V^{1/2} Z \|_\infty \right]
\leq
\frac{N}{\sqrt{8} n^2 h^2 \pi}
+ \frac{4 N \sqrt{\log 2 N}}{\sqrt{8} n^2 h^{3/2} \pi^{3/4}}
\leq
\frac{N}{n^2 h^2}.
\end{align*}
%
Finally, we verify the stochastic continuity bounds.
By the Lipschitz property of $K$, it is easy to show that
for $x,x' \in \cX$ we have
$\left|\frac{1}{h} K \left( \frac{X_i - x}{h} \right)
- \frac{1}{h} K \left( \frac{X_i - x'}{h} \right)\right|
\lesssim \frac{|x-x'|}{h^2}$ almost surely, and also that
$\E \Big[ \left|\frac{1}{h} K \left( \frac{X_i - x}{h} \right)
- \frac{1}{h} K \left( \frac{X_i - x'}{h} \right)\right|^2 \Big]
\lesssim \frac{|x-x'|^2}{h^3}$.
By chaining with the Bernstein--Orlicz norm and polynomial covering numbers,
%
\begin{align*}
\sup_{|x-x'| \leq \delta}
\big\|S(x) - S(x')\big\|_\infty
\lesssim_\P
\delta
\sqrt{\frac{\log n}{n h^3}}
\end{align*}
%
whenever $\log(N/h) \lesssim \log n$ and $n h \gtrsim \log n$.
By a Gaussian process maximal inequality
\citep[Corollary~2.2.8]{van1996weak}
the same bound holds for $T(x)$ with
%
\begin{align*}
\sup_{|x-x'| \leq \delta}
\big\|T(x) - T(x')\big\|_\infty
\lesssim_\P
\delta
\sqrt{\frac{\log n}{n h^3}}.
\end{align*}

\begin{proof}[Lemma~\ref{lem:yurinskii_kde_eigenvalue}]

For $x, x' \in [a, 1-a]$, the scaled covariance function
of this nonparametric estimator is
%
\begin{align*}
n h\, \Cov\big[\hat g(x), \hat g(x')\big]
&=
\frac{1}{h}
\E \left[
K \left( \frac{X_i - x}{h} \right)
K \left( \frac{X_i - x'}{h} \right)
\right] \\
&\quad-
\frac{1}{h}
\E \left[
K \left( \frac{X_i - x}{h} \right)
\right]
\E \left[
K \left( \frac{X_i - x'}{h} \right)
\right] \\
&=
\frac{1}{2 \pi}
\int_{\frac{-x}{h}}^{\frac{1-x}{h}}
\exp \left( - \frac{t^2}{2} \right)
\exp \left( - \frac{1}{2} \left( t + \frac{x - x'}{h} \right)^2 \right)
\diff t
- h I(x) I(x')
\end{align*}
%
where
$I(x) = \frac{1}{\sqrt 2 \pi} \int_{-x/h}^{(1-x)/h} e^{-t^2/2} \diff t$.
Completing the square and a substitution gives
%
\begin{align*}
n h\, \Cov\big[\hat g(x), \hat g(x')\big]
&=
\frac{1}{2 \pi}
\exp \left( - \frac{1}{4} \left( \frac{x-x'}{h} \right)^2 \right)
\int_{\frac{-x-x'}{2h}}^{\frac{2-x-x'}{2h}}
\exp \left(-t^2\right)
\diff t
- h I(x) I(x').
\end{align*}
%
Now we show that since $x, x'$ are not too close to the boundary
of $[0,1]$,
the limits in the above integral can be replaced by $\pm \infty$.
Note that $\frac{-x-x'}{2h} \leq \frac{-a}{h}$
and $\frac{2-x-x'}{2h} \geq \frac{a}{h}$ so
%
\begin{align*}
\int_{-\infty}^{\infty}
\exp \left(-t^2\right)
\diff t
- \int_{\frac{-x-x'}{2h}}^{\frac{2-x-x'}{2h}}
\exp \left(-t^2\right)
\diff t
\leq
2 \int_{a/h}^\infty
\exp \left(-t^2\right)
\diff t
\leq
\frac{h}{a}
\exp \left(- \frac{a^2}{h^2}\right).
\end{align*}
%
Therefore, since
$\int_{-\infty}^{\infty} e^{-t^2} \diff t = \sqrt \pi$,
%
\begin{align*}
\left|
n h\, \Cov\big[\hat g(x), \hat g(x')\big]
- \frac{1}{2 \sqrt \pi}
\exp \left( - \frac{1}{4} \left( \frac{x-x'}{h} \right)^2 \right)
+ h I(x) I(x')
\right|
\leq
\frac{h}{2 \pi a}
\exp \left(- \frac{a^2}{h^2}\right).
\end{align*}
%
Define the $N \times N$ matrix
$\tilde\Sigma_{i j} = \frac{1}{2 \sqrt \pi}
\exp \left( - \frac{1}{4} \left( \frac{x_i-x_j}{h} \right)^2 \right)$.
By \citet[Proposition~2.4,
Proposition~2.5, and Equation~2.10]{baxter1994norm},
with
$\cB_k = \big\{b \in \R^\Z :
\sum_{i \in \Z} \I\{b_i \neq 0\} \leq k \big\}$,
%
\begin{align*}
\inf_{k \in \N}
\inf_{b \in \R^k}
\frac{\sum_{i=1}^k \sum_{j=1}^k b_i b_j \, e^{-\lambda(i-j)^2}}
{\sum_{i=1}^k b_i^2}
=
\sqrt{\frac{\pi}{\lambda}}
\sum_{i=-\infty}^{\infty}
\exp \left( - \frac{(\pi e + 2 \pi i)^2}{4 \lambda} \right).
\end{align*}
%
We use Riemann sums,
noting that $\pi e + 2 \pi x = 0$ at
$x = -e/2 \approx -1.359$.
Consider the substitutions
$\Z \cap (-\infty, -3] \mapsto (-\infty, -2]$,
$\{-2, -1\} \mapsto \{-2, -1\}$, and
$\Z \cap [0, \infty) \mapsto [-1, \infty)$.
%
\begin{align*}
\sum_{i \in \Z}
e^{-(\pi e + 2 \pi i)^2 / 4 \lambda}
&\leq
\int_{-\infty}^{-2}
e^{ - (\pi e + 2 \pi x)^2/4 \lambda}
\diff x
+ e^{- (\pi e - 4 \pi)^2/4 \lambda} \\
&\quad+
e^{ - (\pi e - 2 \pi)^2 / 4 \lambda}
+ \int_{-1}^{\infty}
e^{ -(\pi e + 2 \pi x)^2 / 4 \lambda}
\diff x.
\end{align*}
%
Now use the substitution $t = \frac{\pi e + 2 \pi x}{2 \sqrt \lambda}$
and suppose $\lambda < 1$, yielding
%
\begin{align*}
\sum_{i \in \Z}
e^{-(\pi e + 2 \pi i)^2 / 4 \lambda}
&\leq
\frac{\sqrt \lambda}{\pi}
\int_{-\infty}^{\frac{\pi e - 4 \pi}{2 \sqrt \lambda}}
e^{-t^2}
\diff t
+ e^{- (\pi e - 4 \pi)^2/4 \lambda}
+ e^{ - (\pi e - 2 \pi)^2 / 4 \lambda}
+ \frac{\sqrt \lambda}{\pi}
\int_{\frac{\pi e - 2 \pi}{2 \sqrt \lambda}}^{\infty}
e^{-t^2}
\diff t \\
&\leq
\left( 1 + \frac{1}{\pi} \frac{\lambda}{4 \pi - \pi e} \right)
e^{-(\pi e - 4 \pi)^2 / 4 \lambda}
+
\left( 1 + \frac{1}{\pi} \frac{\lambda}{\pi e - 2 \pi} \right)
e^{- (\pi e - 2 \pi)^2 / 4 \lambda} \\
&\leq
\frac{13}{12}
e^{-(\pi e - 4 \pi)^2 / 4 \lambda}
+
\frac{8}{7}
e^{- (\pi e - 2 \pi)^2 / 4 \lambda}
\leq
\frac{9}{4}
\exp \left( - \frac{5}{4 \lambda} \right).
\end{align*}
%
Therefore
%
\begin{align*}
\inf_{k \in \N}
\inf_{b \in \cB_k}
\frac{\sum_{i \in \Z} \sum_{j \in \Z} b_i b_j \, e^{-\lambda(i-j)^2}}
{\sum_{i \in \Z} b_i^2}
< \frac{4}{\sqrt \lambda}
\exp \left( - \frac{5}{4 \lambda} \right)
< 4 e^{-1/\lambda}.
\end{align*}
%
From this and since
$\tilde\Sigma_{i j} = \frac{1}{2 \sqrt \pi} e^{-\lambda(i-j)^2}$
with $\lambda = \frac{1}{4(N-1)^2 h^2} \leq \frac{\delta^2}{h^2}$,
for each $h$ and some $\delta \leq h$,
we have $\lambda_{\min}(\tilde\Sigma) \leq 2 e^{-h^2/\delta^2}$.
Recall that
%
\begin{align*}
\left|
\Sigma_{i j}
- \tilde\Sigma_{i j}
+ h I(x_i) I(x_j)
\right|
\leq
\frac{h}{2 \pi a}
\exp \left(- \frac{a^2}{h^2}\right).
\end{align*}
%
For any positive semi-definite $N \times N$ matrices $A$ and $B$
and vector $v$ we have $\lambda_{\min}(A - v v^\T) \leq \lambda_{\min}(A)$
and $\lambda_{\min}(B) \leq \lambda_{\min}(A) + \|B-A\|_2
\leq \lambda_{\min}(A) + N \|B-A\|_{\max}$.
Hence with $I_i = I(x_i)$,
%
\begin{align*}
\lambda_{\min}(\Sigma)
&\leq
\lambda_{\min}(\tilde\Sigma - h I I^\T)
+ \frac{N h}{2 \pi a}
\exp \left(- \frac{a^2}{h^2}\right)
\leq
2 e^{-h^2/\delta^2}
+ \frac{h}{\pi a \delta}
e^{-a^2 / h^2}.
\end{align*}
\end{proof}

\begin{proof}[Proposition~\ref{pro:yurinskii_emp_proc}]

Let $\cF_\delta$ be a $\delta$-cover of $(\cF, d)$.
Using a union bound, we can write
%
\begin{align*}
&\P\left(\sup_{f \in \cF}
\big| S(f) - T(f) \big|
\geq 2t + \eta \right)
\leq
\P\left(\sup_{f \in \cF_\delta}
\big| S(f) - T(f) \big|
\geq \eta \right) \\
&\qquad\qquad+
\P\left(\sup_{d(f,f') \leq \delta}
\big| S(f) - S(f') \big|
\geq t \right)
+ \P\left(\sup_{d(f,f') \leq \delta}
\big| T(f) - T(f') \big|
\geq t \right).
\end{align*}

\proofparagraph{bounding the difference on $\cF_\delta$}

We apply Corollary~\ref{cor:yurinskii_sa_martingale}
with $p = \infty$ to the
martingale difference sequence
$\cF_\delta(X_i) = \big(f(X_i) : f \in \cF_\delta\big)$
which takes values in $\R^{|\cF_\delta|}$.
Square integrability can be assumed otherwise
$\beta_\delta = \infty$.
Note $\sum_{i=1}^n \cF_\delta(X_i) = S(\cF_\delta)$
and $\phi_\infty(\cF_\delta) \leq \sqrt{2 \log 2 |\cF_\delta|}$.
Therefore there exists a conditionally Gaussian vector $T(\cF_\delta)$
with the same covariance structure as $S(\cF_\delta)$
conditional on $\cH_0$ satisfying
%
\begin{align*}
\P\left(
\sup_{f \in \cF_\delta}
\big| S(f) - T(f) \big|
\geq \eta
\right)
&\leq
\frac{24\beta_\delta^{\frac{1}{3}}
(2\log 2 |\cF_\delta|)^{\frac{1}{3}}}{\eta}
+ 17\left(\frac{\sqrt{2 \log 2 |\cF_\delta|}
\sqrt{\E\left[\|\Omega_\delta\|_2\right]}}{\eta }\right)^{\frac{2}{3}}.
\end{align*}

\proofparagraph{bounding the fluctuations in $S(f)$}

Since $\big\| S(f) - S(f') \big\|_\psi \leq L d(f,f')$,
by Theorem~2.2.4 in \citet{van1996weak}
%
\begin{align*}
\left\|
\sup_{d(f,f') \leq \delta}
\big| S(f) - S(f') \big|
\right\|_\psi
&\leq
C_\psi L
\left(
\int_0^\delta
\psi^{-1}(N_\varepsilon) \diff{\varepsilon}
+ \delta \psi^{-1}(N_\delta^2)
\right)
= C_\psi L J_\psi(\delta).
\end{align*}
%
Then, by Markov's inequality and the definition of the Orlicz norm,
%
\begin{align*}
\P\left(
\sup_{d(f,f') \leq \delta}
\big| S(f) - S(f') \big|
\geq t
\right)
&\leq
\psi\left(\frac{t}{C_\psi L J_\psi(\delta)} \right)^{-1}.
\end{align*}

\proofparagraph{bounding the fluctuations in $T(f)$}

By the Vorob'ev--Berkes--Philipp theorem
\citep{dudley1999uniform},
$T(\cF_\delta)$ extends to a conditionally Gaussian process $T(f)$.
Firstly, since
$\bigvvvert T(f) - T(f') \bigvvvert_2 \leq L d(f,f')$
conditionally on $\cH_0$,
and $T(f)$ is a conditional Gaussian process, we have
$\big\| T(f) - T(f') \big\|_{\psi_2} \leq 2 L d(f,f')$
conditional on $\cH_0$
by \citet[Chapter~2.2, Complement~1]{van1996weak},
where $\psi_2(x) = \exp(x^2) - 1$.
Thus again by Theorem~2.2.4 in \citet{van1996weak},
again conditioning on $\cH_0$,
%
\begin{align*}
\left\|
\sup_{d(f,f') \leq \delta}
\big| T(f) - T(f') \big|
\right\|_{\psi_2}
&\leq
C_1 L
\int_0^\delta
\sqrt{\log N_\varepsilon} \diff{\varepsilon}
= C_1 L J_2(\delta)
\end{align*}
%
for some universal constant $C_1 > 0$,
where we used $\psi_2^{-1}(x) = \sqrt{\log(1+x)}$
and monotonicity of covering numbers.
Then by Markov's inequality and the definition of the Orlicz norm,
%
\begin{align*}
\P\left(
\sup_{d(f,f') \leq \delta}
\big| T(f) - T(f') \big|
\geq t
\right)
&\leq
\left(
\exp\left(
\frac{t^2}{C_1^2 L^2 J_2(\delta)^2}
\right) - 1
\right)^{-1}
\!\vee 1
\leq
2 \exp\left(
\frac{-t^2}{C_1^2 L^2 J_2(\delta)^2}
\right).
\end{align*}
%

\proofparagraph{conclusion}

The result follows by scaling $t$ and $\eta$
and enlarging constants if necessary.
%
\end{proof}

\subsection{Applications to nonparametric regression}

\begin{proof}[Proposition~\ref{pro:yurinskii_series}]

Proceed according to the decomposition in
Section~\ref{sec:yurinskii_series}.
By stationarity and Lemma~SA-2.1 in
\citet{cattaneo2020large},
we have $\sup_w \|p(w)\|_1 \lesssim 1$
and also $\|H\|_1 \lesssim n/k$
and $\|H^{-1}\|_1 \lesssim k/n$.

\proofparagraph{bounding $\beta_{\infty,2}$ and $\beta_{\infty,3}$}

Set $X_i = p(W_i) \varepsilon_i$
so $S = \sum_{i=1}^n X_i$,
and set $\sigma^2_i = \sigma^2(W_i)$ and
$V_i = \Var[X_i \mid \cH_{i-1}] = \sigma_i^2 p(W_i) p(W_i)^\T$.
Recall from Corollary~\ref{cor:yurinskii_sa_martingale} that for
$r \in \{2,3\}$,
%
\begin{align*}
\beta_{\infty,r}
= \sum_{i=1}^n \E\left[\| X_i \|^r_2 \| X_i \|_\infty
+ \|V_i^{1/2} Z_i \|^r_2 \|V_i^{1/2} Z_i \|_\infty \right]
\end{align*}
%
with $Z_i \sim \cN(0,1)$ i.i.d.\ and independent of $V_i$.
For the first term, we use
$\sup_w \|p(w)\|_2 \lesssim 1$
and bounded third moments of $\varepsilon_i$:
%
\begin{align*}
\E\left[ \| X_i \|^r_2 \| X_i \|_\infty \right]
&\leq
\E\left[ |\varepsilon_i|^3 \| p(W_i) \|^{r+1}_2 \right]
\lesssim 1.
\end{align*}
%
For the second term, apply Lemma~\ref{lem:yurinskii_app_gaussian_useful}
conditionally on
$\cH_n$ with $\sup_w \|p(w)\|_2 \lesssim 1$ to see
%
\begin{align*}
&\E\left[ \|V_i^{1/2} Z_i \|^r_2 \|V_i^{1/2} Z_i \|_\infty \right]
\lesssim
\sqrt{\log 2k} \
\E\left[
\max_{1 \leq j \leq k}
(V_i)_{j j}^{1/2}
\bigg( \sum_{j=1}^k (V_i)_{j j} \bigg)^{r/2}
\right] \\
&\quad\lesssim
\sqrt{\log 2k} \
\E\left[
\sigma_i^{r+1}
\max_{1 \leq j \leq k}
p(W_i)_j
\bigg(
\sum_{j=1}^k
p(W_i)_{j}^2
\bigg)^{r/2}
\right]
\lesssim
\sqrt{\log 2k} \
\E\left[
\sigma_i^{r+1}
\right]
\lesssim
\sqrt{\log 2k}.
\end{align*}
%
Putting these together yields
%
$\beta_{\infty,2} \lesssim n \sqrt{\log 2k}$
and $\beta_{\infty,3} \lesssim n \sqrt{\log 2k}$.

\proofparagraph{bounding $\Omega$}

Set $\Omega = \sum_{i=1}^n \big(V_i - \E[V_i] \big)$ so
%
\begin{align*}
\Omega
&= \sum_{i=1}^n
\big(\sigma_i^2 p(W_i)p(W_i)^\T - \E\left[ \sigma_i^2 p(W_i)p(W_i)^\T
\right]\big).
\end{align*}
%
Observe that $\Omega_{j l}$ is the sum of a zero-mean
strictly stationary $\alpha$-mixing sequence and so $\E[\Omega_{j l}^2]
\lesssim n$ by
Lemma~\ref{lem:yurinskii_app_variance_mixing}%
\ref{it:yurinskii_app_variance_mixing_bounded}.
Since the basis functions
satisfy Assumption~3 in \citet{cattaneo2020large}, $\Omega$ has a bounded
number of non-zero entries in each row, so by Jensen's inequality
%
\begin{align*}
\E\left[
\|\Omega\|_2
\right]
&\leq
\E\left[
\|\Omega\|_\rF
\right]
\leq
\left(
\sum_{j=1}^k
\sum_{l=1}^k
\E\left[
\Omega_{j l}^2
\right]
\right)^{1/2}
\lesssim \sqrt{n k}.
\end{align*}
%

\proofparagraph{strong approximation}

By Corollary~\ref{cor:yurinskii_sa_martingale} and the previous parts,
with any sequence $R_n \to \infty$,
%
\begin{align*}
\|S - T \|_\infty
&\lesssim_\P
\beta_{\infty,2}^{1/3} (\log 2k)^{1/3} R_n
+ \sqrt{\log 2k} \sqrt{\E[\|\Omega\|_2]} R_n \\
&\lesssim_\P
n^{1/3} \sqrt{\log 2k} R_n
+ (n k)^{1/4} \sqrt{\log 2k} R_n.
\end{align*}
%
If further $\E \left[ \varepsilon_i^3 \mid \cH_{i-1} \right] = 0$ then
the third-order version of Corollary~\ref{cor:yurinskii_sa_martingale}
applies since
%
\begin{align*}
\pi_3
&=
\sum_{i=1}^{n}
\sum_{|\kappa| = 3}
\E \Big[ \big|
\E [ X_i^\kappa \mid \cH_{i-1} ]
\big| \Big]
=
\sum_{i=1}^{n}
\sum_{|\kappa| = 3}
\E \Big[ \big|
p(W_i)^\kappa \,
\E [ \varepsilon_i^3 \mid \cH_{i-1} ]
\big| \Big]
= 0,
\end{align*}
%
giving
%
\begin{align*}
\|S - T \|_\infty
&\lesssim_\P
\beta_{\infty,3}^{1/4} (\log 2k)^{3/8} R_n
+ \sqrt{\log 2k} \sqrt{\E[\|\Omega\|_2]} R_n
\lesssim_\P
(n k)^{1/4} \sqrt{\log 2k} R_n.
\end{align*}
%
By H{\"o}lder's inequality and with
$\|H^{-1}\|_1 \lesssim k/n$ we have
%
\begin{align*}
\sup_{w \in \cW}
\left|
p(w)^\T H^{-1} S
- p(w)^\T H^{-1} T
\right|
&\leq
\sup_{w \in \cW}
\|p(w)\|_1
\|H^{-1}\|_1
\| S - T \|_\infty
\lesssim
n^{-1} k
\| S - T \|_\infty.
\end{align*}

\proofparagraph{convergence of $\hat H$}

We have
$\hat H - H = \sum_{i=1}^n \big(p(W_i)p(W_i)^\T - \E\left[
p(W_i)p(W_i)^\T \right]\big)$.
Observe that $(\hat H - H)_{j l}$ is the sum of
a zero-mean strictly stationary $\alpha$-mixing sequence and so
$\E[(\hat H - H)_{j l}^2] \lesssim n$ by
Lemma~\ref{lem:yurinskii_app_variance_mixing}%
\ref{it:yurinskii_app_variance_mixing_bounded}.
Since the basis
functions satisfy Assumption~3 in \citet{cattaneo2020large},
$\hat H-H$ has a
bounded number of non-zero entries in each row and so by Jensen's inequality
%
\begin{align*}
\E\left[
\|\hat H-H\|_1
\right]
&=
\E\left[
\max_{1 \leq i \leq k}
\sum_{j=1}^k
\big|(\hat H-H)_{i j}\big|
\right]
\leq
\E\left[
\sum_{1 \leq i \leq k}
\Bigg(
\sum_{j=1}^k
|(\hat H-H)_{i j}|
\Bigg)^2
\right]^{\frac{1}{2}}
\lesssim \sqrt{n k}.
\end{align*}

\proofparagraph{bounding the matrix term}

Note $\|\hat H^{-1}\|_1 \leq \|H^{-1}\|_1
+ \|\hat H^{-1}\|_1 \|\hat H-H\|_1 \|H^{-1}\|_1$
so by the previous part, we deduce
%
\begin{align*}
\|\hat H^{-1}\|_1
\leq
\frac{\|H^{-1}\|_1}
{1 - \|\hat H-H\|_1 \|H^{-1}\|_1}
\lesssim_\P
\frac{k/n}
{1 - \sqrt{n k}\, k/n}
\lesssim_\P
\frac{k}{n}
\end{align*}
%
as $k^3 / n \to 0$. Note that by the martingale structure, since
$p(W_i)$ is bounded and supported on a region with volume at most of the order
$1/k$, and as $W_i$ has a Lebesgue density,
%
\begin{align*}
\Var[T_j]
&=
\Var[S_j]
=
\Var\left[
\sum_{i=1}^n \varepsilon_i p(W_i)_j
\right]
=
\sum_{i=1}^n
\E\left[
\sigma_i^2 p(W_i)_j^2
\right]
\lesssim
\frac{n}{k}.
\end{align*}
%
So by the Gaussian maximal inequality in
Lemma~\ref{lem:yurinskii_app_gaussian_pnorm},
$\|T\|_\infty \lesssim_\P \sqrt{\frac{n \log 2k}{k}}$.
Since $k^3/n \to 0$,
%
\begin{align*}
\sup_{w \in \cW}
\left|
p(w)^\T (\hat H^{-1} - H^{-1}) S
\right|
&\leq
\sup_{w \in \cW}
\|p(w)^\T\|_1
\|\hat H^{-1}\|_1
\|\hat H - H\|_1
\|H^{-1}\|_1
\|S - T\|_\infty \\
&\quad+
\sup_{w \in \cW}
\|p(w)^\T\|_1
\|\hat H^{-1}\|_1
\|\hat H - H\|_1
\|H^{-1}\|_1
\|T\|_\infty \\
&\lesssim_\P
\frac{k^2}{n^2}
\sqrt{n k}
\!\left(
n^{1/3} \sqrt{\log 2k}
+ (n k)^{1/4} \sqrt{\log 2k}
\right)
\!+ \frac{k^2}{n^2}
\sqrt{n k}
\sqrt{\frac{n \log 2k}{k}} \\
&\lesssim_\P
\frac{k^2}{n}
\sqrt{\log 2k}.
\end{align*}
%

\proofparagraph{conclusion of the main result}

By the previous parts,
with $G(w) = p(w)^\T H^{-1} T$,
%
\begin{align*}
&\sup_{w \in \cW}
\left|
\hat\mu(w) - \mu(w)
- p(w)^\T H^{-1} T
\right| \\
&\quad=
\sup_{w \in \cW}
\left|
p(w)^\T H^{-1} (S - T)
+ p(w)^\T (\hat H^{-1} - H^{-1}) S
+ \Bias(w)
\right| \\
&\quad\lesssim_\P
\frac{k}{n}
\|S - T\|_\infty
+ \frac{k^2}{n} \sqrt{\log 2k}
+ \sup_{w \in \cW} |\Bias(w)| \\
&\quad\lesssim_\P
\frac{k}{n}
\left( n^{1/3} \sqrt{\log 2k} + (n k)^{1/4} \sqrt{\log 2k} \right) R_n
+ \frac{k^2}{n} \sqrt{\log 2k}
+ \sup_{w \in \cW} |\Bias(w)| \\
&\quad\lesssim_\P
n^{-2/3} k \sqrt{\log 2k} R_n
+ n^{-3/4} k^{5/4} \sqrt{\log 2k} R_n
+ \frac{k^2}{n} \sqrt{\log 2k}
+ \sup_{w \in \cW} |\Bias(w)| \\
&\quad\lesssim_\P
n^{-2/3} k \sqrt{\log 2k} R_n
+ \sup_{w \in \cW} |\Bias(w)|
\end{align*}
%
since $k^3/n \to 0$.
If further $\E \left[ \varepsilon_i^3 \mid \cH_{i-1} \right] = 0$ then
%
\begin{align*}
\sup_{w \in \cW}
\left|
\hat\mu(w) - \mu(w)
- p(w)^\T H^{-1} T
\right|
&\lesssim_\P
\frac{k}{n}
\|S - T\|_\infty
+ \frac{k^2}{n} \sqrt{\log 2k}
+ \sup_{w \in \cW} |\Bias(w)| \\
&\lesssim_\P
n^{-3/4} k^{5/4} \sqrt{\log 2k} R_n
+ \sup_{w \in \cW} |\Bias(w)|.
\end{align*}
%
Finally, we verify the variance bounds for the Gaussian process.
With $\sigma^2(w)$ bounded above,
%
\begin{align*}
\Var[G(w)]
&=
p(w)^\T H^{-1}
\Var\left[ \sum_{i=1}^n p(W_i) \varepsilon_i \right]
H^{-1} p(w) \\
&=
p(w)^\T H^{-1}
\E\left[\sum_{i=1}^n p(W_i) p(W_i)^\T \sigma^2(W_i) \right]
H^{-1} p(w) \\
&\lesssim
\|p(w)\|_2^2 \|H^{-1}\|_2^2
\|H\|_2
\lesssim
k/n.
\end{align*}
%
Similarly, since $\sigma^2(w)$ is bounded away from zero,
%
\begin{align*}
\Var[G(w)]
&\gtrsim
\|p(w)\|_2^2 \|H^{-1}\|_2^2
\|H^{-1}\|_2^{-1}
\gtrsim
k/n.
\end{align*}

\proofparagraph{bounding the bias}

We delegate the task of carefully deriving bounds on the bias to
\citet{cattaneo2020large}, who provide a high-level assumption on the
approximation error in Assumption~4 and then use it to derive bias bounds in
Section~3 of the form $\sup_{w \in \cW} |\Bias(w)| \lesssim_\P k^{-\gamma}$.
This assumption is then verified for B-splines, wavelets, and piecewise
polynomials in their supplemental appendix.

\end{proof}

\begin{proof}[Proposition~\ref{pro:yurinskii_series_feasible}]
\proofparagraph{infeasible supremum approximation}

Provided that the bias is negligible,
for all $s > 0$ we have
%
\begin{align*}
&\sup_{t \in \R}
\left|
\P\left(
\sup_{w \in \cW}
\left|
\frac{\hat\mu(w)-\mu(w)}{\sqrt{\rho(w,w)}}
\right| \leq t
\right)
-
\P\left(
\sup_{w \in \cW}
\left|
\frac{G(w)}{\sqrt{\rho(w,w)}}
\right| \leq t
\right)
\right| \\
&\quad\leq
\sup_{t \in \R}
\P\left(
t \leq
\sup_{w \in \cW}
\left|
\frac{G(w)}{\sqrt{\rho(w,w)}}
\right|
\leq t + s
\right)
+
\P\left(
\sup_{w \in \cW}
\left|
\frac{\hat\mu(w)-\mu(w)-G(w)}{\sqrt{\rho(w,w)}}
\right| > s
\right).
\end{align*}
%
By the Gaussian anti-concentration result given as Corollary~2.1 in
\citet{chernozhukov2014anti} applied to a discretization of $\cW$, the first
term is at most $s \sqrt{\log n}$ up to a constant factor, and the second
term converges to zero whenever
$\frac{1}{s} \left( \frac{k^3 (\log k)^3}{n} \right)^{1/6} \to 0$.
Thus a suitable value of $s$ exists whenever $\frac{k^3(\log n)^6}{n} \to 0$.

\proofparagraph{feasible supremum approximation}

By \citet[Lemma~3.1]{chernozhukov2013gaussian} and discretization,
with $\rho(w,w') = \E[\hat\rho(w,w')]$,
%
\begin{align*}
&\sup_{t \in \R}
\left|
\P\left(
\sup_{w \in \cW}
\left|
\frac{\hat G(w)}{\sqrt{\hat\rho(w,w)}}
\right|
\leq t \biggm| \bW, \bY
\right)
- \P\left(
\left|
\frac{G(w)}{\sqrt{\rho(w,w)}}
\right|
\leq t
\right)
\right| \\
&\quad\lesssim_\P
\sup_{w,w' \in \cW}
\left|
\frac{\hat\rho(w,w')}
{\sqrt{\hat\rho(w,w)\hat\rho(w',w')}}
- \frac{\rho(w,w')}
{\sqrt{\rho(w,w)\rho(w',w')}}
\right|^{1/3}
(\log n)^{2/3} \\
&\quad\lesssim_\P
\left(\frac n k \right)^{1/3}
\sup_{w,w' \in \cW} |\hat\rho(w,w') - \rho(w,w')|^{1/3}
(\log n)^{2/3} \\
&\quad\lesssim_\P
\left( \frac{n (\log n)^2}{k} \right)^{1/3}
\sup_{w,w' \in \cW}
\left|
p(w)^\T \hat H^{-1}
\left(
\hat{V}[S]
- \Var[S]
\right)
\hat H^{-1} p(w')
\right|^{1/3} \\
&\quad\lesssim_\P
\left( \frac{k (\log n)^2}{n} \right)^{1/3}
\left\|
\hat{V}[S]
- \Var[S]
\right\|_2^{1/3},
\end{align*}
%
and vanishes in probability whenever
$\frac{k (\log n)^2}{n}
\big\| \hat{V}[S] - \Var[S] \big\|_2 \to_\P 0$.
For the plug-in estimator,
%
\begin{align*}
&\left\|
\hat{V}[S]
- \Var[S]
\right\|_2
=
\left\|
\sum_{i=1}^n
p(W_i) p(W_i^\T)
\hat\sigma^2(W_i)
- n \E\left[
p(W_i) p(W_i^\T)
\sigma^2(W_i)
\right]
\right\|_2 \\
&\quad\lesssim_\P
\sup_{w \in \cW}
|\hat{\sigma}^2(w)-\sigma^2(w)|
\, \big\| \hat H \big\|_2 \\
&\qquad+
\left\|
\sum_{i=1}^n
p(W_i) p(W_i^\T)
\sigma^2(W_i)
- n \E\left[
p(W_i) p(W_i^\T)
\sigma^2(W_i)
\right]
\right\|_2 \\
&\quad\lesssim_\P
\frac{n}{k}
\sup_{w \in \cW}
|\hat{\sigma}^2(w)-\sigma^2(w)|
+ \sqrt{n k},
\end{align*}
%
where the second term is bounded by the same argument
used to bound $\|\hat H - H\|_1$.
Thus, the feasible approximation is valid whenever
$(\log n)^2 \sup_{w \in \cW}
|\hat{\sigma}^2(w)-\sigma^2(w)| \to_\P 0$
and $\frac{k^3 (\log n)^4}{n} \to 0$.
The validity of the uniform confidence band follows immediately.
%
\end{proof}

\begin{proof}[Proposition~\ref{pro:yurinskii_local_poly}]

We apply Proposition~\ref{pro:yurinskii_emp_proc}
with the metric $d(f_w, f_{w'}) = \|w-w'\|_2$
and the function class
%
\begin{align*}
\cF
&=
\left\{
(W_i, \varepsilon_i) \mapsto
e_1^\T H(w)^{-1} K_h(W_i-w) p_h(W_i-w)
\varepsilon_i
:\ w \in \cW
\right\},
\end{align*}
%
with $\psi$ chosen as a suitable Bernstein Orlicz function.

\proofparagraph{bounding $H(w)^{-1}$}

Recall that
$H(w) = \sum_{i=1}^n \E[K_h(W_i-w) p_h(W_i-w)p_h(W_i-w)^\T]$
and let $a(w) \in \R^k$ with $\|a(w)\|_2 = 1$.
Since the density of $W_i$ is bounded away from zero on $\cW$,
%
\begin{align*}
a(w)^\T H(w) a(w)
&=
n \E\left[
\big( a(w)^\T p_h(W_i-w) \big)^2
K_h(W_i-w)
\right] \\
&\gtrsim
n \int_\cW
\big( a(w)^\T p_h(u-w) \big)^2
K_h(u-w)
\diff{u}
\gtrsim
n \int_{\frac{\cW-w}{h}}
\big( a(w)^\T p(u) \big)^2
K(u)
\diff{u}.
\end{align*}
%
This is continuous in $a(w)$ on the compact set
$\|a(w)\|_2 = 1$
and $p(u)$ forms a polynomial basis so
$a(w)^\T p(u)$ has finitely many zeroes.
Since $K(u)$ is compactly supported
and $h \to 0$,
the above integral is eventually strictly positive
for all $x \in \cW$,
and hence is bounded below uniformly in $w \in \cW$
by a positive constant.
Therefore
$\sup_{w \in \cW} \|H(w)^{-1}\|_2 \lesssim 1/n$.

\proofparagraph{bounding $\beta_\delta$}

Let $\cF_\delta$ be a $\delta$-cover of $(\cF, d)$
with cardinality $|\cF_\delta| \asymp \delta^{-m}$
and let
$\cF_\delta(W_i, \varepsilon_i)
= \big(f(W_i, \varepsilon_i) : f\in \cF_\delta\big)$.
Define the truncated errors
$\tilde\varepsilon_i =
\varepsilon_i\I\{-a \log n \leq \varepsilon_i \leq b \log n\}$
and note that
$\E\big[e^{|\varepsilon_i|/C_\varepsilon}\big] < \infty$
implies that
$\P(\exists i: \tilde\varepsilon_i \neq \varepsilon_i)
\lesssim n^{1-(a \vee b)/C_\varepsilon}$.
Hence, by choosing $a$ and $b$ large enough,
with high probability, we can replace all
$\varepsilon_i$ by $\tilde\varepsilon_i$.
Further, it is always possible to increase either $a$ or $b$
along with some randomization to ensure that
$\E[\tilde\varepsilon_i] = 0$.
Since $K$ is bounded and compactly supported,
$W_i$ has a bounded density and
$|\tilde\varepsilon_i| \lesssim \log n$,
%
\begin{align*}
\bigvvvert
f(W_i, \tilde\varepsilon_i)
\bigvvvert_2
&=
\E\left[
\left|
e_1^\T H(w)^{-1} K_h(W_i-w) p_h(W_i-w)
\tilde\varepsilon_i
\right|^2
\right]^{1/2} \\
&\leq
\E\left[
\|H(w)^{-1}\|_2^2
K_h(W_i-w)^2
\|p_h(W_i-w)\|_2^2
\sigma^2(W_i)
\right]^{1/2} \\
&\lesssim
n^{-1}
\E\left[
K_h(W_i-w)^2
\right]^{1/2}
\lesssim
n^{-1}
h^{-m / 2}, \\
\bigvvvert
f(W_i, \tilde\varepsilon_i)
\bigvvvert_\infty
&\leq
\bigvvvert
\|H(w)^{-1}\|_2
K_h(W_i-w)
\|p_h(W_i-w)\|_2
|\tilde\varepsilon_i|
\bigvvvert_\infty \\
&\lesssim
n^{-1}
\bigvvvert
K_h(W_i-w)
\bigvvvert_\infty
\log n
\lesssim
n^{-1}
h^{-m}
\log n.
\end{align*}
%
Therefore
%
\begin{align*}
\E\left[
\|\cF_\delta(W_i, \tilde\varepsilon_i)\|_2^2
\|\cF_\delta(W_i, \tilde\varepsilon_i)\|_\infty
\right]
&\leq
\!\sum_{f\in\cF_\delta}
\!\bigvvvert f(W_i, \tilde\varepsilon_i) \bigvvvert_2^2
\max_{f\in\cF_\delta}
\bigvvvert f(W_i, \tilde\varepsilon_i) \bigvvvert_\infty
\!\lesssim
n^{-3} \delta^{-m} h^{-2m} \log n.
\end{align*}
%
Let
$V_i(\cF_\delta) =
\E\big[\cF_\delta(W_i, \tilde\varepsilon_i)
\cF_\delta(W_i, \tilde\varepsilon_i)^\T
\mid \cH_{i-1}\big]$
and $Z_i \sim \cN(0, I_d)$ be i.i.d.\ and
independent of $\cH_n$.
Note that
$V_i(f,f) = \E[f(W_i, \tilde\varepsilon_i)^2 \mid W_i]
\lesssim n^{-2} h^{-2m}$
and
$\E[V_i(f,f)] = \E[f(W_i, \tilde\varepsilon_i)^2]
\lesssim n^{-2} h^{-m}$.
Thus by Lemma~\ref{lem:yurinskii_app_gaussian_useful},
%
\begin{align*}
\E\left[
\big\| V_i(\cF_\delta)^{1/2} Z_i \big\|^2_2
\big\| V_i(\cF_\delta)^{1/2} Z_i \big\|_\infty
\right]
&=
\E\left[
\E\left[
\big\| V_i(\cF_\delta)^{1/2} Z_i \big\|^2_2
\big\| V_i(\cF_\delta)^{1/2} Z_i \big\|_\infty
\mid \cH_n
\right]
\right] \\
&\leq
4 \sqrt{\log 2|\cF_\delta|}
\,\E\Bigg[
\max_{f \in \cF_\delta} \sqrt{V_i(f,f)}
\sum_{f \in \cF_\delta} V_i(f,f)
\Bigg] \\
&\lesssim
n^{-3}
h^{-2m}
\delta^{-m}
\sqrt{\log(1/\delta)}.
\end{align*}
%
Thus since $\log(1/\delta) \asymp \log(1/h) \asymp\log n$,
%
\begin{align*}
\beta_\delta
&=
\sum_{i=1}^n
\E\left[
\|\cF_\delta(W_i, \tilde\varepsilon_i)\|_2^2
\|\cF_\delta(W_i, \tilde\varepsilon_i)\|_\infty
+ \big\| V_i(\cF_\delta)^{1/2} Z_i \big\|^2_2
\big\| V_i(\cF_\delta)^{1/2} Z_i \big\|_\infty
\right]
\lesssim
\frac{\log n}
{n^2 h^{2m} \delta^m}.
\end{align*}

\proofparagraph{bounding $\Omega_\delta$}

Let $C_K>0$ be the radius of a $\ell^2$-ball
containing the support of $K$
and note that
%
\begin{align*}
\left|
V_i(f,f')
\right|
&=
\Big|
\E\Big[
e_1^\T H(w)^{-1}
p_h(W_i-w)
e_1^\T H(w')^{-1}
p_h(W_i-w') \\
&\qquad\times
K_h(W_i-w)
K_h(W_i-w')
\tilde\varepsilon_i^2
\Bigm| \cH_{i-1}
\Big]
\Big| \\
&\lesssim
n^{-2}
K_h(W_i-w)
K_h(W_i-w') \\
&\lesssim
n^{-2}
h^{-m}
K_h(W_i-w)
\I\{\|w-w'\|_2 \leq 2 C_K h\}.
\end{align*}
%
Since $W_i$ are $\alpha$-mixing
with $\alpha(j) < e^{-2j / C_\alpha}$,
Lemma~\ref{lem:yurinskii_app_variance_mixing}%
\ref{it:yurinskii_app_variance_mixing_exponential}
with $r=3$ gives
%
\begin{align*}
&\Var\left[
\sum_{i=1}^n V_i(f,f')
\right] \\
&\quad\lesssim
\sum_{i=1}^n
\E\left[
|V_i(f,f')|^3
\right] ^{2/3}
\lesssim
n^{-3} h^{-2m}
\E\left[
K_h(W_i-w)^3
\right] ^{2/3}
\I\{\|w-w'\|_2 \leq 2 C_K h\} \\
&\quad\lesssim
n^{-3} h^{-2m}
(h^{-2m})^{2/3}
\I\{\|w-w'\|_2 \leq 2 C_K h\} \\
&\quad\lesssim
n^{-3} h^{-10m/3}
\I\{\|w-w'\|_2 \leq 2 C_K h\}.
\end{align*}
%
Therefore, by Jensen's inequality,
%
\begin{align*}
\E\big[ \|\Omega_\delta\|_2 \big]
&\leq
\E\big[ \|\Omega_\delta\|_\rF \big]
\leq
\E\Bigg[
\sum_{f,f' \in \cF_\delta}
(\Omega_\delta)_{f,f'}^2
\Bigg]^{1/2}
\leq
\Bigg(
\sum_{f,f' \in \cF_\delta}
\Var\left[
\sum_{i=1}^n V_i(f,f')
\right]
\Bigg)^{1/2} \\
&\lesssim
n^{-3/2} h^{-5m/3}
\Bigg(
\sum_{f,f' \in \cF_\delta}
\I\{\|w-w'\|_2 \leq 2 C_K h\}
\Bigg)^{1/2} \\
&\lesssim
n^{-3/2} h^{-5m/3}
\big(h^{m} \delta^{-2m} \big)^{1/2}
\lesssim
n^{-3/2}
h^{-7m/6}
\delta^{-m}.
\end{align*}
%
Note that we could have used
$\|\cdot\|_1$ rather than $\|\cdot\|_\rF$,
but this term is negligible either way.

\proofparagraph{regularity of the stochastic processes}

For each $f, f' \in \cF$,
define the mean-zero and $\alpha$-mixing random variables
%
\begin{align*}
u_i(f,f')
&=
e_1^\T
\big(
H(w)^{-1} K_h(W_i-w) p_h(W_i-w)
- H(w')^{-1} K_h(W_i-w') p_h(W_i-w')
\big)
\tilde\varepsilon_i.
\end{align*}
%
Note that for all $1 \leq j \leq k$,
by the Lipschitz property of the kernel and monomials,
%
\begin{align*}
&\left|
K_h(W_i-w) - K_h(W_i-w')
\right| \\
&\quad\lesssim
h^{-m-1}
\|w-w'\|_2
\big(
\I\{\|W_i-w\| \leq C_K h\}
+ \I\{\|W_i-w'\| \leq C_K h\}
\big), \\
&\left|
p_h(W_i-w)_j - p_h(W_i-w')_j
\right|
\lesssim
h^{-1}
\|w-w'\|_2,
\end{align*}
%
to deduce that for any $1 \leq j,l \leq k$,
%
\begin{align*}
\big| H(w)_{j l} - H(w')_{j l} \big|
&=
\big|
n \E\big[
K_h(W_i-w) p_h(W_i-w)_j p_h(W_i-w)_l \\
&\qquad-
K_h(W_i-w') p_h(W_i-w')_j p_h(W_i-w')_l
\big]
\big| \\
&\leq
n\E\left[
\left|
K_h(W_i-w) - K_h(W_i-w')
\right|
\left|
p_h(W_i-w)_j
p_h(W_i-w)_l
\right|
\right] \\
&\quad+
n\E\left[
\left|
p_h(W_i-w)_j - p_h(W_i-w')_j
\right|
\left|
K_h(W_i-w')
p_h(W_i-w)_l
\right|
\right] \\
&\quad+
n\E\left[
\left|
p_h(W_i-w)_l - p_h(W_i-w')_l
\right|
\left|
K_h(W_i-w')
p_h(W_i-w')_j
\right|
\right] \\
&\lesssim
n h^{-1}\|w-w'\|_2.
\end{align*}
%
Therefore, as the dimension of the matrix $H(w)$ is fixed,
%
\begin{align*}
\big\| H(w)^{-1} - H(w')^{-1} \big\|_2
&\leq
\big\| H(w)^{-1}\big\|_2
\big\| H(w')^{-1}\big\|_2
\big\| H(w) - H(w') \big\|_2
\lesssim
\frac{\|w-w'\|_2}{n h}.
\end{align*}
%
Hence
%
\begin{align*}
\big| u_i(f,f') \big|
&\leq
\big\|
H(w)^{-1} K_h(W_i-w) p_h(W_i-w)
- H(w')^{-1} K_h(W_i-w') p_h(W_i-w')
\tilde\varepsilon_i
\big\|_2 \\
&\leq
\big\| H(w)^{-1} - H(w')^{-1} \big\|_2
\big\| K_h(W_i-w) p_h(W_i-w)
\tilde\varepsilon_i
\big\|_2 \\
&\quad+
\big| K_h(W_i-w) - K_h(W_i-w') \big|
\big\| H(w')^{-1} p_h(W_i-w)
\tilde\varepsilon_i
\big\|_2 \\
&\quad+
\big\| p_h(W_i-w) - p_h(W_i-w') \big\|_2
\big\| H(w')^{-1} K_h(W_i-w')
\tilde\varepsilon_i \big\|_2 \\
&\lesssim
\frac{\|w-w'\|_2}{n h}
\big| K_h(W_i-w) \tilde\varepsilon_i \big|
+ \frac{1}{n}
\big| K_h(W_i-w) - K_h(W_i-w') \big|
\,|\tilde\varepsilon_i| \\
&\lesssim
\frac{\|w-w'\|_2 \log n}{n h^{m+1}},
\end{align*}
%
and from the penultimate line, we also deduce that
%
\begin{align*}
\Var[u_i(f,f')]
&\lesssim
\frac{\|w-w'\|_2^2}{n^2h^2}
\E\left[
K_h(W_i-w)^2 \sigma^2(X_i)
\right] \\
&\quad+
\frac{1}{n^2}
\E\left[
\big( K_h(W_i-w) - K_h(W_i-w') \big)^2
\sigma^2(X_i)
\right]
\lesssim
\frac{\|w-w'\|_2^2}{n^2h^{m+2}}.
\end{align*}
%
Further, $\E[u_i(f,f') u_j(f,f')] = 0$ for $i \neq j$ so
by Lemma~\ref{lem:yurinskii_app_exponential_mixing}%
\ref{it:yurinskii_app_exponential_mixing_bernstein},
for a constant $C_1>0$,
%
\begin{align*}
\P\left(
\Big| \sum_{i=1}^n u_i(f,f') \Big|
\geq \frac{C_1 \|w-w'\|_2}{\sqrt n h^{m/2+1}}
\left(
\sqrt{t}
+ \sqrt{\frac{(\log n)^2}{n h^m}} \sqrt t
+ \sqrt{\frac{(\log n)^6}{n h^m}} t
\right)
\right)
&\leq
C_1 e^{-t}.
\end{align*}
%
Therefore, adjusting the constant if necessary
and since $n h^{m} \gtrsim (\log n)^7$,
%
\begin{align*}
\P\left(
\Big| \sum_{i=1}^n u_i(f,f') \Big|
\geq
\frac{C_1 \|w-w'\|_2}{\sqrt{n} h^{m/2+1}}
\left(
\sqrt{t} + \frac{t}{\sqrt{\log n}}
\right)
\right)
&\leq
C_1 e^{-t}.
\end{align*}
%
\Citet[Lemma~2]{van2013bernstein} with
$\psi(x) =
\exp\Big(\big(\sqrt{1+2 x / \sqrt{\log n}}-1 \big)^2
\log n \Big)-1$
now shows that
%
\begin{align*}
\Bigvvvert \sum_{i=1}^n u_i(f,f') \Bigvvvert_\psi
&\lesssim
\frac{\|w-w'\|_2}{\sqrt{n} h^{m/2+1}}
\end{align*}
%
so we take $L = \frac{1}{\sqrt{n} h^{m/2+1}}$.
Noting
$\psi^{-1}(t) = \sqrt{\log(1+t)} + \frac{\log(1+t)}{2\sqrt{\log n}}$
and $N_\delta \lesssim \delta^{-m}$,
%
\begin{align*}
J_\psi(\delta)
&=
\int_0^\delta
\psi^{-1}\big( N_\varepsilon \big)
\diff{\varepsilon}
+ \delta
\psi^{-1} \big( N_\delta \big)
\lesssim
\frac{\delta \log(1/\delta)}{\sqrt{\log n}}
+ \delta \sqrt{\log(1/\delta)}
\lesssim
\delta \sqrt{\log n}, \\
J_2(\delta)
&=
\int_0^\delta
\sqrt{\log N_\varepsilon}
\diff{\varepsilon}
\lesssim
\delta \sqrt{\log(1/\delta)}
\lesssim
\delta \sqrt{\log n}.
\end{align*}

\proofparagraph{strong approximation}

Recalling that
$\tilde\varepsilon_i = \varepsilon_i$
for all $i$ with high probability,
by Proposition~\ref{pro:yurinskii_emp_proc},
for all $t, \eta > 0$ there exists a
zero-mean Gaussian process $T(w)$ satisfying
%
\begin{align*}
\E\left[
\left(\sum_{i=1}^n f_w(W_i, \varepsilon_i)\right)
\left(\sum_{i=1}^n f_{w'}(W_i, \varepsilon_i)\right)
\right]
&= \E\big[ T(w) T(w')
\big]
\end{align*}
%
for all $w, w' \in \cW$ and
%
\begin{align*}
&\P\left(
\sup_{w \in \cW}
\left| \sum_{i=1}^n f_{w}(W_i, \varepsilon_i)
- T(w) \right|
\geq C_\psi(t + \eta)
\right) \\
&\quad\leq
C_\psi
\inf_{\delta > 0}
\inf_{\cF_\delta}
\Bigg\{
\frac{\beta_\delta^{1/3} (\log 2 |\cF_\delta|)^{1/3}}{\eta }
+ \left(\frac{\sqrt{\log 2 |\cF_\delta|}
\sqrt{\E\left[\|\Omega_\delta\|_2\right]}}{\eta }\right)^{2/3} \\
&\qquad+
\psi\left(\frac{t}{L J_\psi(\delta)}\right)^{-1}
+ \exp\left(\frac{-t^2}{L^2 J_2(\delta)^2}\right)
\Bigg\} \\
&\quad\leq
C_\psi
\Bigg\{
\frac{
\left(\frac{\log n} {n^2 h^{2m} \delta^{m}} \right)^{1/3}
(\log n)^{1/3}}{\eta }
+ \left(\frac{\sqrt{\log n}
\sqrt{n^{-3/2} h^{-7m/6} \delta^{-m}}
}{\eta }\right)^{2/3} \\
&\qquad+
\psi\left(\frac{t}{\frac{1}{\sqrt{n} h^{m/2+1}}
J_\psi(\delta)}\right)^{-1}
+ \exp\left(\frac{-t^2}{
\left( \frac{1}{\sqrt{n} h^{m/2+1}} \right)^2
J_2(\delta)^2}\right)
\Bigg\} \\
&\quad\leq
C_\psi
\Bigg\{
\frac{
(\log n)^{2/3}}{n^{2/3} h^{2m/3} \delta^{m/3} \eta}
+ \left(\frac{
n^{-3/4} h^{-7m/12} \delta^{-m/2} \sqrt{\log n}}
{\eta }\right)^{2/3} \\
&\qquad+
\psi\left(\frac{t\sqrt{n} h^{m/2+1}}
{\delta \sqrt{\log n}}\right)^{-1}
+ \exp\left(\frac{-t^2n h^{m+2}}
{\delta^2 \log n}\right)
\Bigg\}.
\end{align*}
%
Noting $\psi(x) \geq e^{x^2/4}$ for $x \leq 4 \sqrt{\log n}$,
any $R_n \to \infty$ gives the probability bound
%
\begin{align*}
\sup_{w \in \cW}
\left| \sum_{i=1}^n f_{w}(W_i, \varepsilon_i)
- T(w) \right|
&\lesssim_\P
\frac{(\log n)^{2/3}}{n^{2/3} h^{2m/3} \delta^{m/3}} R_n
+ \frac{\sqrt{\log n}}{n^{3/4} h^{7m/12} \delta^{m/2}} R_n
+ \frac{\delta \sqrt{\log n}} {\sqrt{n} h^{m/2+1}}.
\end{align*}
%
Optimizing over $\delta$ gives
$\delta \asymp \left(\frac{\log n}{n h^{m-6}}\right)^{\frac{1}{2m+6}}
= h \left( \frac{\log n}{n h^{3m}} \right)^{\frac{1}{2m+6}}$
and so
%
\begin{align*}
\sup_{w \in \cW}
\left| \sum_{i=1}^n f_{w}(W_i, \varepsilon_i)
- T(w) \right|
&\lesssim_\P
\left(
\frac{(\log n)^{m+4}}{n^{m+4}h^{m(m+6)}}
\right)^{\frac{1}{2m+6}} R_n.
\end{align*}

\proofparagraph{convergence of $\hat H(w)$}

For $1 \leq j,l \leq k$
define the zero-mean random variables
%
\begin{align*}
u_{i j l}(w)
&=
K_h(W_i-w) p_h(W_i-w)_j p_h(W_i-w)_l
- \E\big[K_h(W_i-w) p_h(W_i-w)_j p_h(W_i-w)_l \big]
\end{align*}
%
and note that
$|u_{i j l}(w)| \lesssim h^{-m}$.
By Lemma~\ref{lem:yurinskii_app_exponential_mixing}%
\ref{it:yurinskii_app_exponential_mixing_bounded}
for a constant $C_2 > 0$ and all $t > 0$,
%
\begin{align*}
\P\left(
\left|
\sum_{i=1}^n
u_{i j l}(w)
\right|
> C_2 h^{-m} \big( \sqrt{n t}
+ (\log n)(\log \log n) t \big)
\right)
&\leq
C_2 e^{-t}.
\end{align*}
%
Further, note that by Lipschitz properties,
%
\begin{align*}
\left|
\sum_{i=1}^n u_{i j l}(w)
- \sum_{i=1}^n u_{i j l}(w')
\right|
&\lesssim
h^{-m-1} \|w-w'\|_2
\end{align*}
%
so there is a $\delta$-cover of $(\cW, \|\cdot\|_2)$
with size at most $n^a \delta^{-a}$ for some $a > 0$.
Adjusting $C_2$,
%
\begin{align*}
\P\left(
\sup_{w \in \cW}
\left|
\sum_{i=1}^n
u_{i j l}(w)
\right|
> C_2 h^{-m} \big( \sqrt{n t}
+ (\log n)(\log \log n) t \big)
+ C_2 h^{-m-1} \delta
\right)
&\leq
C_2 n^a \delta^{-a}
e^{-t}
\end{align*}
%
and hence
%
\begin{align*}
\sup_{w \in \cW}
\left|
\sum_{i=1}^n
u_{i j l}(w)
\right|
&\lesssim_\P
h^{-m} \sqrt{n \log n}
+ h^{-m} (\log n)^3
\lesssim_\P
\sqrt{\frac{n \log n}{h^{2m}}}.
\end{align*}
%
Therefore
%
\begin{align*}
\sup_{w\in\cW} \|\hat H(w)-H(w)\|_2
&\lesssim_\P
\sqrt{\frac{n \log n}{h^{2m}}}.
\end{align*}

\proofparagraph{bounding the matrix term}

Firstly, note that
since $\sqrt{\frac{\log n}{n h^{2m}}} \to 0$,
we have that uniformly in $w \in \cW$
%
\begin{align*}
\|\hat H(w)^{-1}\|_2
\leq
\frac{\|H(w)^{-1}\|_2}
{1 - \|\hat H(w)-H(w)\|_2 \|H(w)^{-1}\|_2}
&\lesssim_\P
\frac{1/n}
{1 - \sqrt{\frac{n \log n}{h^{2m}}} \frac{1}{n}}
\lesssim_\P
\frac{1}{n}.
\end{align*}
%
Therefore
%
\begin{align*}
&\sup_{w \in \cW}
\big|
e_1^\T \big(\hat H(w)^{-1} - H(w)^{-1}\big)
S(w)
\big|
\leq
\sup_{w \in \cW}
\big\|\hat H(w)^{-1} - H(w)^{-1}\big\|_2
\|S(w)\|_2 \\
&\quad\leq
\sup_{w \in \cW}
\big\|\hat H(w)^{-1}\big\|_2
\big\|H(w)^{-1}\big\|_2
\big\|\hat H(w) - H(w)\big\|_2
\|S(w)\|_2
\lesssim_\P
\sqrt{\frac{\log n}{n^3 h^{2m}}}
\sup_{w \in \cW}
\|S(w)\|_2.
\end{align*}
%
Now for $1 \leq j \leq k$ write
$u_{i j}(w) = K_h(W_i-w) p_h(W_i-w)_j \tilde \varepsilon_i$
so that $S(w)_j = \sum_{i=1}^n u_{i j}(w)$ with high probability.
Note that $u_{i j}(w)$ are zero-mean with
$\Cov[u_{i j}(w), u_{i' j}(w)] = 0$ for $ i \neq i'$.
Also $|u_{i j}(w)| \lesssim h^{-m} \log n$
and $\Var[u_{i j}(w)] \lesssim h^{-m}$.
By Lemma~\ref{lem:yurinskii_app_exponential_mixing}%
\ref{it:yurinskii_app_exponential_mixing_bernstein}
for a constant $C_3>0$,
%
\begin{align*}
\P\left(
\Big| \sum_{i=1}^n u_{i j}(w) \Big|
\geq C_3 \big( (h^{-m/2} \sqrt n + h^{-m} \log n) \sqrt t
+ h^{-m} (\log n)^3 t \big)
\right)
&\leq
C_3 e^{-t}, \\
\P\left(
\Big| \sum_{i=1}^n u_{i j}(w) \Big|
>
C_3 \left(
\sqrt{\frac{tn}{h^{m}}}
+ \frac{t(\log n)^3}{h^{m}}
\right)
\right)
&\leq
C_3 e^{-t},
\end{align*}
%
where we used $n h^{m} \gtrsim (\log n)^2$
and adjusted the constant if necessary.
As before,
$u_{i j}(w)$ is Lipschitz in $w$ with a constant which is at most
polynomial in $n$,
so for some $a>0$
%
\begin{align*}
\P\left(
\sup_{w \in \cW}
\Big| \sum_{i=1}^n u_{i j}(w) \Big|
>
C_3 \left(
\sqrt{\frac{tn}{h^{m}}}
+ \frac{t(\log n)^3}{h^{m}}
\right)
\right)
&\leq
C_3 n^a e^{-t}, \\
\sup_{w \in \cW}
\|S(w)\|_2
\lesssim_\P
\sqrt{\frac{n \log n}{h^{m}}}
+ \frac{(\log n)^4}{h^{m}}
&\lesssim_\P
\sqrt{\frac{n \log n}{h^{m}}}
\end{align*}
%
as $n h^m \gtrsim (\log n)^7$.
Finally,
%
\begin{align*}
\sup_{w \in \cW}
\big|
e_1^\T \big(\hat H(w)^{-1} - H(w)^{-1}\big)
S(w)
\big|
&\lesssim_\P
\sqrt{\frac{\log n}{n^3 h^{2m}}}
\sqrt{\frac{n \log n}{h^{m}}}
\lesssim_\P
\frac{\log n}{\sqrt{n^2 h^{3m}}}.
\end{align*}

\proofparagraph{bounding the bias}

Since $\mu \in \cC^\gamma$, we have, by the multivariate version of Taylor's
theorem,
%
\begin{align*}
\mu(W_i)
&=
\sum_{|\kappa|=0}^{\gamma-1}
\frac{1}{\kappa!}
\partial^{\kappa} \mu(w)
(W_i-w)^\kappa
+ \sum_{|\kappa|=\gamma}
\frac{1}{\kappa!}
\partial^{\kappa} \mu(w')
(W_i-w)^\kappa
\end{align*}
%
for some $w'$ on the line segment connecting
$w$ and $W_i$.
Now since $p_h(W_i-w)_1 = 1$,
%
\begin{align*}
&e_1^\T \hat H(w)^{-1}
\sum_{i=1}^n K_h(W_i-w) p_h(W_i-w) \mu(w) \\
&\quad=
e_1^\T \hat H(w)^{-1}
\sum_{i=1}^n K_h(W_i-w) p_h(W_i-w) p_h(W_i-w)^\T e_1 \mu(w)
= e_1^\T e_1 \mu(w) = \mu(w).
\end{align*}
%
Therefore
%
\begin{align*}
\Bias(w)
&=
e_1^\T \hat H(w)^{-1}
\sum_{i=1}^n K_h(W_i-w) p_h(W_i-w) \mu(W_i)
- \mu(w) \\
&=
e_1^\T \hat H(w)^{-1}
\sum_{i=1}^n K_h(W_i-w) p_h(W_i-w) \\
&\quad\times
\Bigg(
\sum_{|\kappa|=0}^{\gamma-1}
\frac{1}{\kappa!}
\partial^{\kappa} \mu(w)
(W_i-w)^\kappa
+ \sum_{|\kappa|=\gamma}
\frac{1}{\kappa!}
\partial^{\kappa} \mu(w')
(W_i-w)^\kappa
- \mu(w)
\Bigg) \\
&=
\sum_{|\kappa|=1}^{\gamma-1}
\frac{1}{\kappa!}
\partial^{\kappa} \mu(w)
e_1^\T \hat H(w)^{-1}
\sum_{i=1}^n K_h(W_i-w) p_h(W_i-w)
(W_i-w)^\kappa \\
&\quad+
\sum_{|\kappa|=\gamma}
\frac{1}{\kappa!}
\partial^{\kappa} \mu(w')
e_1^\T \hat H(w)^{-1}
\sum_{i=1}^n K_h(W_i-w) p_h(W_i-w)
(W_i-w)^\kappa \\
&=
\sum_{|\kappa|=\gamma}
\frac{1}{\kappa!}
\partial^{\kappa} \mu(w')
e_1^\T \hat H(w)^{-1}
\sum_{i=1}^n K_h(W_i-w) p_h(W_i-w)
(W_i-w)^\kappa,
\end{align*}
%
where we used that
$p_h(W_i-w)$ is a vector containing monomials
in $W_i-w$ of order up to $\gamma$, so
$e_1^\T \hat H(w)^{-1}
\sum_{i=1}^n K_h(W_i-w) p_h(W_i-w)
(W_i-w)^\kappa = 0$
whenever $1 \leq |\kappa| \leq \gamma$.
Finally,
%
\begin{align*}
\sup_{w\in\cW}
|\Bias(w)|
&=
\sup_{w\in\cW}
\Bigg|
\sum_{|\kappa|=\gamma}
\frac{1}{\kappa!}
\partial^{\kappa} \mu(w')
e_1^\T \hat H(w)^{-1}
\sum_{i=1}^n K_h(W_i-w) p_h(W_i-w)
(W_i-w)^\kappa
\Bigg| \\
&\lesssim_\P
\sup_{w\in\cW}
\max_{|\kappa| = \gamma}
\left|
\partial^{\kappa} \mu(w')
\right|
\|\hat H(w)^{-1}\|_2
\Bigg\|
\sum_{i=1}^n K_h(W_i-w) p_h(W_i-w)
\Bigg\|_2
h^\gamma \\
&\lesssim_\P
\frac{h^\gamma}{n}
\sup_{w\in\cW}
\Bigg\|
\sum_{i=1}^n K_h(W_i-w) p_h(W_i-w)
\Bigg\|_2.
\end{align*}
%
Write
$\tilde u_{i j}(w) = K_h(W_i-w)p_h(W_i-w)_j$
and note $|\tilde u_{i j}(w)| \lesssim h^{-m}$
and $\E[\tilde u_{i j}(w)] \lesssim 1$, so
%
\begin{align*}
\P\left(
\left|
\sum_{i=1}^n \tilde u_{i j}(w)
- \E\left[
\sum_{i=1}^n \tilde u_{i j}(w)
\right]
\right|
> C_4 h^{-m} \big( \sqrt{n t}
+ (\log n)(\log \log n) t \big)
\right)
&\leq
C_4 e^{-t}
\end{align*}
%
by Lemma~\ref{lem:yurinskii_app_exponential_mixing}%
\ref{it:yurinskii_app_exponential_mixing_bounded} for a constant $C_4$,
By Lipschitz properties, this implies
%
\begin{align*}
\sup_{w \in \cW}
\left|
\sum_{i=1}^n \tilde u_{i j}(w)
\right|
&\lesssim_\P
n
\left(
1 + \sqrt{\frac{\log n}{n h^{2m}}}
\right)
\lesssim_\P
n.
\end{align*}
%
Therefore
$\sup_{w\in\cW} |\Bias(w)|
\lesssim_\P n h^\gamma / n
\lesssim_\P h^\gamma$.

\proofparagraph{conclusion}

By the previous parts,
%
\begin{align*}
\sup_{w \in \cW}
\left|\hat \mu(w) - \mu(w) - T(w) \right|
&\leq
\sup_{w \in \cW}
\left|e_1^\T H(w)^{-1} S(w) - T(w) \right| \\
&\quad+
\sup_{w \in \cW}
\left| e_1^\T \big(\hat H(w)^{-1} - H(w)^{-1}\big) S(w) \right|
+ \sup_{w \in \cW}
|\Bias(w)| \\
&\lesssim_\P
\left(
\frac{(\log n)^{m+4}}{n^{m+4}h^{m(m+6)}}
\right)^{\frac{1}{2m+6}} R_n
+ \frac{\log n}{\sqrt{n^2 h^{3m}}}
+ h^\gamma \\
&\lesssim_\P
\frac{R_n}{\sqrt{n h^m}}
\left(
\frac{(\log n)^{m+4}}{n h^{3m}}
\right)^{\frac{1}{2m+6}}
+ h^\gamma,
\end{align*}
%
where the last inequality follows because
$n h^{3m} \to \infty$
and $\frac{1}{2m+6} \leq \frac{1}{2}$.
Finally, we verify the upper and lower bounds
on the variance of the Gaussian process.
Since the spectrum of $H(w)^{-1}$
is bounded above and below by $1/n$,
%
\begin{align*}
\Var[T(w)]
&=
\Var\left[
e_1^\T H(w)^{-1}
\sum_{i=1}^n K_h(W_i-w) p_h(W_i-w) \varepsilon_i
\right] \\
&=
e_1^\T H(w)^{-1}
\Var\left[
\sum_{i=1}^n K_h(W_i-w) p_h(W_i-w) \varepsilon_i
\right]
H(w)^{-1} e_1^\T \\
&\lesssim
\|H(w)^{-1}\|_2^2
\max_{1 \leq j \leq k}
\sum_{i=1}^n
\Var\big[
K_h(W_i-w) p_h(W_i-w)_j \sigma(W_i)
\big] \\
&\lesssim
\frac{1}{n^2} n
\frac{1}{h^m}
\lesssim
\frac{1}{n h^m}.
\end{align*}
%
Similarly,
$\Var[T(w)] \gtrsim \frac{1}{n h^m}$
by the same argument used to bound eigenvalues of
$H(w)^{-1}$.
%
\end{proof}

\section{High-dimensional central limit theorems for martingales}%
\label{sec:yurinskii_app_high_dim_clt}

We present an application of our main results to
high-dimensional central limit theorems for martingales. Our main
contribution here is the generality of our results, which are broadly
applicable to martingale data and impose minimal extra assumptions. In exchange
for the scope and breadth of our results, we naturally do not necessarily
achieve state-of-the-art distributional approximation errors in certain special
cases, such as with independent data or when restricting the class of sets over
which the central limit theorem must hold. Extensions of our high-dimensional
central limit theorem results to mixingales and other approximate martingales,
along with third-order refinements and Gaussian mixture target distributions,
are possible through methods akin to those used to establish our main results
in Section~\ref{sec:yurinskii_main_results}, but we omit these for succinctness.

Our approach to deriving a high-dimensional martingale central limit theorem
proceeds as follows. Firstly, the upcoming
Proposition~\ref{pro:yurinskii_app_clt} uses our
main result on martingale coupling
(Corollary~\ref{cor:yurinskii_sa_martingale}) to
reduce the problem to that of providing anti-concentration results for
high-dimensional Gaussian vectors. We then demonstrate the utility of this
reduction by employing a few such anti-concentration methods from the existing
literature. Proposition~\ref{pro:yurinskii_app_bootstrap} gives a feasible
implementation via
the Gaussian multiplier bootstrap, enabling valid
resampling-based inference using
the resulting conditional Gaussian distribution. Finally, in
Section~\ref{sec:yurinskii_app_lp} we provide an example application:
distributional
approximation for $\ell^p$-norms of high-dimensional martingale vectors
in Kolmogorov--Smirnov distance, relying on some recent results
concerning Gaussian perimetric inequalities
\citep{nazarov2003maximal,kozbur2021dimension,
giessing2023anti,chernozhukov2017detailed}.

We begin this section with some notation. Assume the setup of
Corollary~\ref{cor:yurinskii_sa_martingale} and suppose $\Sigma$ is
non-random. Let $\cA$ be a class of measurable subsets of
$\R^d$ and take $T \sim \cN(0, \Sigma)$.
For $\eta>0$ and $p \in [1, \infty]$ define the Gaussian perimetric quantity
%
\begin{align*}
\Delta_p(\cA, \eta)
&=
\sup_{A\in \cA}
\big\{\P(T\in A_p^\eta\setminus A)
\vee \P(T\in A \setminus A_p^{-\eta})\big\},
\end{align*}
%
where $A_p^\eta = \{x \in \R^d : \|x - A\|_p \leq \eta\}$,
$A_p^{-\eta} = \R^d \setminus (\R^d \setminus A)_p^\eta$,
and $\|x - A\|_p = \inf_{x' \in A} \|x - x'\|_p$.
Using this perimetric term allows us to convert coupling results
to central limit theorems as follows.
Denote by $\Gamma_p(\eta)$ the rate of strong approximation attained in
Corollary~\ref{cor:yurinskii_sa_martingale}:
%
\begin{align*}
\Gamma_p(\eta)
&=
24 \left(
\frac{\beta_{p,2} \phi_p(d)^2}{\eta^3}
\right)^{1/3}
+ 17 \left(
\frac{\E \left[ \|\Omega\|_2 \right] \phi_p(d)^2}{\eta^2}
\right)^{1/3}.
\end{align*}

\begin{proposition}[High-dimensional central limit theorem for martingales]%
\label{pro:yurinskii_app_clt}

Take the setup of Corollary~\ref{cor:yurinskii_sa_martingale},
and $\Sigma$ non-random.
For a class $\cA$ of measurable sets in $\R^d$,
%
\begin{equation}%
\label{eq:yurinskii_app_high_dim_clt}
\sup_{A\in \cA}
\big|\P(S\in A) -\P(T\in A)\big|
\leq \inf_{p \in [1, \infty]} \inf_{\eta>0}
\big\{\Gamma_p(\eta) + \Delta_p(\cA, \eta) \big\}.
\end{equation}
\end{proposition}

\begin{proof}[Proposition~\ref{pro:yurinskii_app_clt}]

This follows from Strassen's theorem
(Lemma~\ref{lem:yurinskii_app_strassen}), but we
provide a proof for completeness.
%
\begin{align*}
\P(S \in A)
&\leq
\P(T \in A)
+ \P(T \in A_p^\eta \setminus A)
+ \P(\|S - T\| > \eta)
\end{align*}
%
and applying this to $\R^d \setminus A$ gives
%
\begin{align*}
\P(S\in A)
&=
1 - \P(S\in \R^d \setminus A) \\
&\geq
1 - \P(T \in \R^d \setminus A)
- \P(T \in (\R^d \setminus A)_p^\eta \setminus (\R^d \setminus A))
- \P(\|S - T\| > \eta) \\
&=
\P(T \in A)
- \P(T \in A \setminus A_p^{-\eta})
- \P(\|S - T\| > \eta).
\end{align*}
%
Since this holds for all $p \in [1, \infty]$,
%
\begin{align*}
\sup_{A\in \cA}
\big|\P(S\in A) -\P(T\in A)\big|
&\leq
\sup_{A \in \cA}
\big\{\P(T \in A_p^\eta\setminus A)
\vee \P(T \in A \setminus A_p^{-\eta})\big\}
+ \P(\|S - T\| > \eta) \\
&\leq
\inf_{p \in [1, \infty]} \inf_{\eta>0}
\big\{\Gamma_p(\eta) + \Delta_p(\cA, \eta) \big\}.
\end{align*}
%
\end{proof}

The term $\Delta_p(\cA, \eta)$
in \eqref{eq:yurinskii_app_high_dim_clt} is a Gaussian anti-concentration
quantity
so it depends on the law of $S$ only through the covariance matrix $\Sigma$.
A few results are available in the literature
for bounding this term.
For instance, with
$\cA = \cC = \{A \subseteq \R^d \text{ is convex}\}$,
\citet{nazarov2003maximal} showed
%
\begin{equation}%
\label{eq:yurinskii_app_convex_anticonc}
\Delta_2(\cC, \eta)
\asymp
\eta\sqrt{\|\Sigma^{-1}\|_{\rF}},
\end{equation}
%
whenever $\Sigma$ is invertible.
Proposition~\ref{pro:yurinskii_app_clt} with $p=2$
and \eqref{eq:yurinskii_app_convex_anticonc} yield for convex sets
%
\begin{align*}
\sup_{A\in \cC}
\big|\P(S\in A) -\P(T\in A)\big|
&\lesssim
\inf_{\eta > 0}
\left\{
\left(\frac{\beta_{p,2} d}{\eta^3}\right)^{1/3}
+ \left(\frac{\E[\|\Omega \|_2] d}{\eta^2}\right)^{1/3}
+ \eta \sqrt{\|\Sigma^{-1}\|_\rF}
\right\}.
\end{align*}

Alternatively, one can take $\cA = \cR$,
the class of axis-aligned rectangles in $\R^d$.
By Nazarov's Gaussian perimetric inequality
\citep{nazarov2003maximal,chernozhukov2017central},
%
\begin{align}%
\label{eq:yurinskii_app_rect_anticonc}
\Delta_\infty(\cR, \eta)
\leq \frac{\eta (\sqrt{2\log d} + 2)}{\sigma_{\min}}
\end{align}
%
whenever $\min_j \, \Sigma_{j j} \geq \sigma_{\min}^2$
for some $\sigma_{\min}>0$.
Proposition~\ref{pro:yurinskii_app_clt} with $p = \infty$
and \eqref{eq:yurinskii_app_rect_anticonc} yields
%
\begin{align*}%
&\sup_{A\in \cR}
\big|\P(S\in A) -\P(T\in A)\big|
\lesssim
\inf_{\eta > 0}
\left\{
\left(\frac{\beta_{\infty,2} \log 2d}{\eta^3}\right)^{1/3}
+ \left(\frac{\E[\|\Omega \|_2] \log 2d}{\eta^2}\right)^{1/3}
+ \frac{\eta \sqrt{\log 2d}}{\sigma_{\min}}
\right\}.
\end{align*}
%
In situations where
$\liminf_n \min_j \, \Sigma_{j j} = 0$,
it may be possible in certain cases to regularize
the minimum variance away from zero and then apply
a Gaussian--Gaussian rectangular approximation result
such as Lemma~2.1 from \citet{chernozhukov2023nearly}.

\begin{remark}[Comparisons with the literature]

The literature on high-dimensional central limit theorems
has developed rapidly in recent years
\citep[see][and references therein]{%
zhai2018high,%
koike2021notes,%
buzun2022strong,%
lopes2022central,%
chernozhukov2023nearly%
},
particularly for the special case of
sums of independent random vectors
on the rectangular sets $\cR$.
%
Our corresponding results are rather weaker in terms of
dependence on the dimension than for example
\citet[Theorem~2.1]{chernozhukov2023nearly}.
This is an inherent issue due to our approach of first
considering the class of all Borel sets
and only afterwards specializing to the smaller class $\cR$,
where sharper results in the literature directly target the
Kolmogorov--Smirnov distance via Stein's method and Slepian interpolation.
\end{remark}

Next, we present a version of Proposition~\ref{pro:yurinskii_app_clt} in which
the covariance
matrix $\Sigma$ is replaced by an estimator $\hat \Sigma$. This ensures that
the associated conditionally Gaussian vector is feasible and can be resampled,
allowing Monte Carlo quantile estimation via a Gaussian
multiplier bootstrap.

\begin{proposition}[Bootstrap central limit theorem for martingales]%
\label{pro:yurinskii_app_bootstrap}

Assume the setup of Corollary~\ref{cor:yurinskii_sa_martingale},
with $\Sigma$ non-random,
and let $\hat \Sigma$ be an $\bX$-measurable random
$d \times d$ positive semi-definite matrix,
where $\bX = (X_1, \ldots, X_n)$.
For a class $\cA$ of measurable subsets of $\R^d$,
%
\begin{align*}
&\sup_{A\in \cA}
\left|
\P\big(S \in A\big)
- \P\big(\hat \Sigma^{1/2} Z \in A \bigm| \bX \big)
\right| \\
&\quad\leq
\inf_{p \in [1,\infty]} \inf_{\eta>0}
\left\{ \Gamma_p(\eta) + 2 \Delta_p(\cA, \eta)
+ 2d \exp\left(\frac{-\eta^2}
{2d^{2/p}\big\|\hat \Sigma^{1/2} - \Sigma^{1/2}\big\|_2^2}
\right)
\right\},
\end{align*}
%
where $Z \sim \cN(0,I_d)$ is independent of $\bX$.
\end{proposition}

\begin{proof}[Proposition~\ref{pro:yurinskii_app_bootstrap}]

Since $T = \Sigma^{1/2} Z$ is independent of $\bX$,
%
\begin{align*}
&\left|
\P\big(S \in A\big)
- \P\left(\hat \Sigma^{1/2} Z \in A \bigm| \bX\right)
\right| \\
&\quad\leq
\left|
\P\big(S \in A\big)
- \P\big(T \in A\big)
\right|
+\left|
\P\big(\Sigma^{1/2} Z \in A\big)
- \P\left(\hat \Sigma^{1/2} Z \in A \bigm| \bX\right)
\right|.
\end{align*}
%
The first term is bounded by Proposition~\ref{pro:yurinskii_app_clt};
the second by Lemma~\ref{lem:yurinskii_app_feasible_gaussian}
conditional on $\bX$.
%
\begin{align*}
&\left|
\P\big(S \in A\big)
- \P\left(\hat \Sigma^{1/2} Z \in A \bigm| \bX\right)
\right| \\
&\quad\leq
\Gamma_p(\eta) + \Delta_p(\cA, \eta)
+ \Delta_{p'}(\cA, \eta')
+ 2 d \exp \left( \frac{-\eta'^2}
{2 d^{2/p'} \big\|\hat\Sigma^{1/2} - \Sigma^{1/2}\big\|_2^2}
\right)
\end{align*}
%
for all $A \in \cA$
and any $p, p' \in [1, \infty]$ and $\eta, \eta' > 0$.
Taking a supremum over $A$ and infima over
$p = p'$ and $\eta = \eta'$ yields the result.
We do not need
$p = p'$ and $\eta = \eta'$ in general.
%
\end{proof}

A natural choice for $\hat\Sigma$ in certain situations is the sample
covariance matrix $\sum_{i=1}^n X_i X_i^\T$, or a correlation-corrected variant
thereof. In general, whenever $\hat \Sigma$ does not depend on unknown
quantities, one can sample from the law of $\hat T = \hat\Sigma^{1/2} Z$
conditional on $\bX$ to approximate the distribution of $S$.
Proposition~\ref{pro:yurinskii_app_bootstrap} verifies that this Gaussian
multiplier
bootstrap approach is valid whenever $\hat\Sigma$ and $\Sigma$ are sufficiently
close. To this end, Theorem~X.1.1 in \citet{bhatia1997matrix} gives
$\big\|\hat\Sigma^{1/2} - \Sigma^{1/2}\big\|_2
\leq \big\|\hat\Sigma - \Sigma\big\|_2^{1/2}$
and Problem~X.5.5 in the same gives
$\big\|\hat\Sigma^{1/2} - \Sigma^{1/2}\big\|_2
\leq \big\|\Sigma^{-1/2}\big\|_2 \big\|\hat\Sigma - \Sigma\big\|_2$
when $\Sigma$ is invertible. The latter often gives a tighter bound when the
minimum eigenvalue of $\Sigma$ can be bounded away from zero, and consistency
of $\hat \Sigma$ can be established using a range of matrix concentration
inequalities.

In Section~\ref{sec:yurinskii_app_lp} we apply
Proposition~\ref{pro:yurinskii_app_clt} to the special case
of approximating the distribution of the $\ell^p$-norm of a high-dimensional
martingale. Proposition~\ref{pro:yurinskii_app_bootstrap} is then used to
ensure that
feasible distributional approximations are also available.

\subsection{Application: distributional approximation of martingale
\texorpdfstring{$\ell^p$}{lp}-norms}
\label{sec:yurinskii_app_lp}

In empirical applications,
including nonparametric significance tests
\citep{lopes2020bootstrapping}
and nearest neighbor search procedures
\citep{biau2015high},
an estimator or test statistic
can be expressed under the null hypothesis
as the $\ell^p$-norm of a zero-mean
martingale for some $p \in [1, \infty]$.
In the notation of Corollary~\ref{cor:yurinskii_sa_martingale},
it is of interest to bound Kolmogorov--Smirnov
quantities of the form
$\sup_{t \geq 0} \big| \P( \|S\|_p \leq t) - \P( \|T\|_p \leq t) \big|$.
Let $\cB_p$ be the class of closed $\ell^p$-balls in $\R^d$ centered at the
origin and set
$\Delta_p(\eta) \vcentcolon= \Delta_p(\cB_p, \eta)
= \sup_{t \geq 0} \P( t < \|T\|_p \leq t + \eta )$.

\begin{proposition}[Distributional approximation of
martingale $\ell^p$-norms]
\label{pro:yurinskii_app_application_lp}

Assume the setup of Corollary~\ref{cor:yurinskii_sa_martingale},
with $\Sigma$ non-random. Then for $T \sim \cN(0, \Sigma)$,
%
\begin{equation}%
\label{eq:yurinskii_app_application_lp}
\sup_{t \geq 0}
\big| \P( \|S\|_p \leq t )
- \P\left( \|T\|_p \leq t \right) \big|
\leq \inf_{\eta>0}
\big\{\Gamma_p(\eta) + \Delta_p(\eta) \big\}.
\end{equation}
%
\end{proposition}

\begin{proof}[Proposition~\ref{pro:yurinskii_app_application_lp}]

Applying Proposition~\ref{pro:yurinskii_app_clt}
with $\cA=\cB_p$ gives
%
\begin{align*}
\sup_{t \geq 0}
\big| \P( \|S\|_p \leq t )
- \P\left( \|T\|_p \leq t \right) \big|
&= \sup_{A\in \cB_p}
\big|\P(S\in A) -\P(T\in A)\big| \\
&\leq
\inf_{\eta>0}
\big\{\Gamma_p(\eta) + \Delta_p(\cB_p, \eta) \big\}
\leq
\inf_{\eta>0}
\big\{\Gamma_p(\eta) + \Delta_p(\eta) \big\}.
\end{align*}
%
\end{proof}

The right-hand side of
\eqref{eq:yurinskii_app_application_lp} can be controlled in various ways.
%
In the case of $p=\infty$,
note that $\ell^\infty$-balls are rectangles so
$\cB_\infty\subseteq \cR$
and \eqref{eq:yurinskii_app_rect_anticonc} applies, giving
$\Delta_\infty(\eta) \leq \eta (\sqrt{2\log d} + 2) / \sigma_{\min}$
whenever $\min_j \Sigma_{j j} \geq \sigma_{\min}^2$.
Alternatively, \citet[Theorem~1]{giessing2023anti} provides
$\Delta_\infty(\eta) \lesssim \eta / \sqrt{\Var[\|T\|_\infty] + \eta^2}$.
By H{\"o}lder duality of $\ell^p$-norms, we can write
$\|T\|_p = \sup_{\|u\|_q \leq 1} u^\T T$ where $1/p + 1/q = 1$.
Applying the Gaussian process anti-concentration result of
\citet[Theorem~2]{giessing2023anti} yields the more general
$\Delta_p(\eta) \lesssim \eta / \sqrt{\Var[\|T\|_p] + \eta^2}$.
Thus, the problem can be reduced to that of bounding
$\Var\left[\|T\|_p\right]$, with techniques for doing so
discussed in \citet[Section~4]{giessing2023anti}.
Alongside the $\ell^p$-norms, other functionals can be analyzed in this manner,
including the maximum and other order statistics
\citep{kozbur2021dimension,giessing2023anti}.

To conduct inference in this setting, we must feasibly
approximate the quantiles of $\|T\|_p$.
To that end, take a significance level $\tau\in(0,1)$ and set
%
$\hat q_p(\tau) =
\inf \big\{t \in \R: \P(\|\hat T\|_p \leq t \mid \bX) \geq \tau \}$
where $\hat T \mid \bX \sim \cN(0, \hat\Sigma)$,
%
with $\hat\Sigma$ any $\bX$-measurable positive semi-definite
estimator of $\Sigma$.
Note that for the canonical estimator $\hat\Sigma = \sum_{i=1}^n X_i X_i^\T$
we can write $\hat T =\sum_{i=1}^n X_i Z_i$ with
$Z_1,\dots,Z_n$ i.i.d.\ standard Gaussian independent of $\bX$,
yielding the Gaussian multiplier bootstrap.
Now assuming
the law of $\|\hat T\|_p \mid \bX$ has no atoms,
we can apply Proposition~\ref{pro:yurinskii_app_bootstrap}
to see
%
\begin{align*}
&\sup_{\tau\in(0,1)}
\big|\P\left(\|S\|_p \leq \hat q_p(\tau)\right) - \tau \big|
\leq
\E\left[
\sup_{t \geq 0}
\big|
\P(\|S\|_p \leq t)
- \P(\|\hat T\|_p \leq t \mid \bX)
\big|
\right] \\
&\qquad\leq
\inf_{\eta>0}
\left\{ \Gamma_p(\eta)
+ 2 \Delta_p(\eta)
+ 2d\, \E\left[
\exp\left(\frac{-\eta^2}
{2d^{2/p}\big\|\hat \Sigma^{1/2} - \Sigma^{1/2}\big\|_2^2}\right)
\right]
\right\},
\end{align*}
%
and hence the bootstrap is valid whenever
$\|\hat \Sigma^{1/2} - \Sigma^{1/2}\big\|_2^2$ is sufficiently small. See the
preceding discussion regarding methods for bounding this object.

\begin{remark}[One-dimensional distributional approximations]
In our application to distributional approximation of $\ell^p$-norms,
the object of interest $\|S\|_p$ is a
one-dimensional functional of the high-dimensional martingale;
contrast this with the more general Proposition~\ref{pro:yurinskii_app_clt}
which
directly considers the $d$-dimensional random vector $S$.
As such, our coupling-based approach may be improved in certain settings
by applying a more carefully tailored smoothing argument.
For example, \citet{belloni2018high}
employ a ``log sum exponential'' bound
\citep[see also][]{chernozhukov2013gaussian}
for the maximum statistic
$\max_{1 \leq j \leq d} S_j$
along with a coupling due to \citet{chernozhukov2014gaussian} to attain
an improved dependence on the dimension.
Naturally, their approach does not permit the formulation of
high-dimensional central limit theorems over arbitrary classes of
Borel sets as in our Proposition~\ref{pro:yurinskii_app_clt}.
\end{remark}

\clearpage
\addcontentsline{toc}{chapter}{Bibliography}
\bibliographystyle{phd_dissertation}
\bibliography{refs}

\end{document}
